\documentclass[10pt,a4paper,oneside,ngerman,numbers=noenddot]{scrartcl}
\usepackage[T1]{fontenc}
\usepackage[utf8]{inputenc}
\usepackage[ngerman]{babel}
\usepackage{amsmath}
\usepackage{amsfonts}
\usepackage{amssymb}
\usepackage{paralist}
\usepackage{gauss}
\usepackage{pgfplots}
\usepackage[locale=DE,exponent-product=\cdot,detect-all]{siunitx}
\usepackage{tikz}
\usetikzlibrary{matrix,fadings,calc,positioning,decorations.pathreplacing,arrows,decorations.markings}
\usepackage{polynom}
\polyset{style=C, div=:,vars=x}
\pgfplotsset{compat=1.8}
\pagenumbering{arabic}
% ensures that paragraphs are separated by empty lines
\parskip 12pt plus 1pt minus 1pt
\parindent 0pt
% define how the sections are rendered
\def\thesection{\arabic{section})}
\def\thesubsection{\alph{subsection})}
\def\thesubsubsection{(\roman{subsubsection})}
% some matrix magic
\makeatletter
\renewcommand*\env@matrix[1][*\c@MaxMatrixCols c]{%
  \hskip -\arraycolsep
  \let\@ifnextchar\new@ifnextchar
  \array{#1}}
\makeatother

\begin{document}
\author{Jan Branitz (6326955), Jim Martens (6420323),\\
Stephan Niendorf (6242417)}
\title{Hausaufgaben zum 28. Oktober}
\maketitle
\section{} %1
	\subsection{} %a
		\textbf{Aufgabe:} Lösen Sie das folgende LP-Problem mit dem Simplexverfahren:
		\begin{alignat*}{4}
			\text{maximiere}\; & x_{1} &+& 6x_{2} &-& 4x_{3} && \\
			\multicolumn{8}{l}{\text{unter den Nebenbedingungen}} && \\
			\;& 2x_{1} && &+& x_{3} &\leq & 5 \\
			\;-& x_{1} &+& 3x_{2} &-& 2x_{3} &\leq & 2 \\
			\;& && x_{2} &-& x_{3} &\leq & 2 \\
			\multicolumn{6}{r}{$x_{1}, x_{2}, x_{3}$} \,&\geq &\, 0
		\end{alignat*}
		
		\textbf{Lösung.}
		
		\underline{Starttableau}:
		\begin{alignat*}{5}
			x_{4} \,&=&\, 5 \,&-&\, 2x_{1} && &-&\, x_{3} \\
			x_{5} \,&=&\, 2 \,&+&\, x_{1} \,&-&\, 3x_{2} \,&+&\, 2x_{3} \\
			x_{6} \,&=&\, 2 && &-&\, x_{2} \,&+&\, x_{3} \\ \cline{1 - 9}
			z &=& && x_{1} \,&+&\, 6x_{2} \,&-&\, 4x_{3}
		\end{alignat*}
		
		\underline{1. Iteration}:
		
		Eingangsvariable: $x_{2}$\\
		Ausgangsvariable: $x_{5}$
		
		Es folgt
		\begin{alignat*}{2}
			3x_{2} \,&=&&\, 2 + x_{1} + 2x_{3} - x_{5} \\
			x_{2} \,&=&&\, \frac{2}{3} + \frac{1}{3}x_{1} + \frac{2}{3}x_{3} - \frac{1}{3}x_{5} \\
			x_{4} \,&=&&\, 5 - 2x_{1} - x_{3} \\
			x_{6} \,&=&&\, 2 - \left(\frac{2}{3} + \frac{1}{3}x_{1} + \frac{2}{3}x_{3} - \frac{1}{3}x_{5}\right) + x_{3} \\
			&=&&\, 2 - \frac{2}{3} - \frac{1}{3}x_{1} - \frac{2}{3}x_{3} + \frac{1}{3}x_{5} + x_{3} \\
			&=&&\, \frac{4}{3} - \frac{1}{3}x_{1} + \frac{1}{3}x_{3} + \frac{1}{3}x_{5} \\
			z \,&=&&\, x_{1} + 6\left(\frac{2}{3} + \frac{1}{3}x_{1} + \frac{2}{3}x_{3} - \frac{1}{3}x_{5}\right) - 4x_{3} \\
			&=&&\, x_{1} + 4 + 2x_{1} + 4x_{3} - 2x_{5} - 4x_{3} \\
			&=&&\, 4 + 3x_{1} - 2x_{5}
		\end{alignat*}
		
		\underline{Ergebnis der 1. Iteration}:
		\begin{alignat*}{5}
			x_{2} \,&=&\, \frac{2}{3} \,&+&\, \frac{1}{3}x_{1} \,&+&\, \frac{2}{3}x_{3} \,&-&\, \frac{1}{3}x_{5} \\
			x_{4} \,&=&\, 5 \,&-&\, 2x_{1} \,&-&\, x_{3} && \\
			x_{6} \,&=&\, \frac{4}{3} \,&-&\, \frac{1}{3}x_{1} \,&+&\, \frac{1}{3}x_{3} \,&+&\, \frac{1}{3}x_{5} \\ \cline{1 - 9}
			z &=& 4 \,&+&\, 3x_{1} && &-& 2x_{5}
		\end{alignat*}
		
		\underline{2. Iteration}:
		
		Eingangsvariable: $x_{1}$ \\
		Ausgangsvariable: $x_{4}$
		
		Es folgt
		\begin{alignat*}{2}
			2x_{1} &=&& 5 - x_{3} - x_{4} \\
			x_{1} &=&& \frac{5}{2} - \frac{1}{2}x_{3} - \frac{1}{2}x_{4} \\
			x_{2} &=&& \frac{2}{3} + \frac{1}{3}\left(\frac{5}{2} - \frac{1}{2}x_{3} - \frac{1}{2}x_{4}\right) + \frac{2}{3}x_{3} - \frac{1}{3}x_{5} \\
			&=&& \frac{3}{2} - \frac{1}{6}x_{3} - \frac{1}{6}x_{4} + \frac{2}{3}x_{3} - \frac{1}{3}x_{5} \\
			&=&& \frac{3}{2} + \frac{1}{2}x_{3} - \frac{1}{6}x_{4} - \frac{1}{3}x_{5} \\
			x_{6} &=&& \frac{4}{3} - \frac{1}{3}\left(\frac{5}{2} - \frac{1}{2}x_{3} - \frac{1}{2}x_{4}\right) + \frac{1}{3}x_{3} + \frac{1}{3}x_{5} \\
			&=&& \frac{1}{2} + \frac{1}{6}x_{3} + \frac{1}{6}x_{4} + \frac{1}{3}x_{3} + \frac{1}{3}x_{5} \\
			&=&& \frac{1}{2} + \frac{1}{2}x_{3} + \frac{1}{6}x_{4} + + \frac{1}{3}x_{5} \\
			z &=&& 4 + 3\left(\frac{5}{2} - \frac{1}{2}x_{3} - \frac{1}{2}x_{4}\right) - 2x_{5} \\
			&=&& \frac{23}{2} - \frac{3}{2}x_{3} - \frac{3}{2}x_{4} - 2x_{5} \\
			&=&& \frac{23}{2} - \frac{3}{2}x_{3} - \frac{3}{2}x_{4} - 2x_{5}
		\end{alignat*}
		
		\underline{Ergebnis der 2. Iteration}:
		\begin{alignat*}{5}
			x_{1} \,&=&\, \frac{5}{2} \,&-&\, \frac{1}{2}x_{3} \,&-&\, \frac{1}{2}x_{4} && \\
			x_{2} \,&=&\, \frac{3}{2} \,&+&\, \frac{1}{2}x_{3} \,&-&\, \frac{1}{6}x_{4} \,&-&\, \frac{1}{3}x_{5} \\
			x_{6} \,&=&\, \frac{1}{2} \,&+&\, \frac{1}{2}x_{3} \,&+&\, \frac{1}{6}x_{4} \,&+&\, \frac{1}{3}x_{5} \\ \cline{1 - 9}
			z &=& \frac{23}{2} \,&-&\, \frac{3}{2}x_{3} \,&-&\, \frac{3}{2}x_{4} \,&-&\, 2x_{5}
		\end{alignat*}
		Dieses Tableau liefert die optimale Lösung $x_{1} = \frac{5}{2}, x_{2} = \frac{3}{2}, x_{3} = 0$ mit $z = \frac{23}{2}$.
		
		\underline{Startlösung ("`zulässige Basislösung am Anfang"')}:
		\[
			x_{1} = 0, x_{2} = 0, x_{3} = 0, x_{4} = 5, x_{5} = 2, x_{6} = 2  \text{ mit } z = 0
		\]
		\underline{Zulässige Basislösung nach der 1. Iteration}:
		\[
			x_{1} = 0, x_{2} = \frac{2}{3}, x_{3} = 0, x_{4} = 5, x_{5} = 0, x_{6} = \frac{4}{3} \text{ mit } z = 4
		\]
		\underline{Zulässige Basislösung nach der 2. Iteration}:
		\[
			x_{1} = \frac{5}{2}, x_{2} = \frac{3}{2}, x_{3} = 0, x_{4} = 0, x_{5} = 0, x_{6} = \frac{1}{2} \text{ mit } z = \frac{23}{2}
		\]
	\subsection{} %b
		\textbf{Aufgabe:} Lösen Sie das folgende LP-Problem mit dem Simplexverfahren:
		\begin{alignat*}{4}
			\text{maximiere}\; -& 5x_{1} &+& 11x_{2} &-& 5x_{3} && \\
			\multicolumn{8}{l}{\text{unter den Nebenbedingungen}} && \\
			\;-& x_{1} &+& 3x_{2} &-& 4x_{3} &\leq & 2 \\
			\;& x_{1} &+& 5x_{2} &+& 3x_{3} &\leq & 6 \\
			\;-& x_{1} &+& 3x_{2} &+& 3x_{3} &\leq &\, 4 \\
			\;& x_{1} &-& x_{2} &+& 3x_{3} &\leq &\, 2 \\			
			\multicolumn{6}{r}{$x_{1}, x_{2}, x_{3}$} \,&\geq &\, 0
		\end{alignat*}
		
		\textbf{Lösung.}
		
		\underline{Starttableau}:
		\begin{alignat*}{5}
			x_{4} \,&=&\, 2 \,&+&\, x_{1} \,&-&\, 3x_{2} \,&+&\, 4x_{3} \\
			x_{5} \,&=&\, 6 \,&-&\, x_{1} \,&-&\, 5x_{2} \,&-&\, 3x_{3} \\
			x_{6} \,&=&\, 4 \,&+&\, x_{1} \,&-&\, 3x_{2} \,&-&\, 3x_{3} \\
			x_{7} \,&=&\, 2 \,&-&\, x_{1} \,&+&\, x_{2} \,&-&\, 3x_{3} \\ \cline{1 - 9}
			z &=& &-& 5x_{1} \,&+&\, 11x_{2} \,&-&\, 5x_{3}
		\end{alignat*}
		
		\underline{1. Iteration}:
		
		Eingangsvariable: $x_{2}$\\
		Ausgangsvariable: $x_{4}$
		
		Es folgt
		\begin{alignat*}{2}
			3x_{2} \,&=&&\, 2 + x_{1} + 4x_{3} - x_{4} \\
			x_{2} \,&=&&\, \frac{2}{3} + \frac{1}{3}x_{1} + \frac{4}{3}x_{3} - \frac{1}{3}x_{4} \\
			x_{5} \,&=&&\, 6 - x_{1} - 5\left(\frac{2}{3} + \frac{1}{3}x_{1} + \frac{4}{3}x_{3} - \frac{1}{3}x_{4}\right) - 3x_{3} \\
			&=&& 6 - x_{1} - \frac{10}{3} - \frac{5}{3}x_{1} - \frac{20}{3}x_{3} + \frac{5}{3}x_{4} - 3x_{3} \\
			&=&& \frac{8}{3} - \frac{8}{3}x_{1} - \frac{29}{3}x_{3} + \frac{5}{3}x_{4} \\
			x_{6} \,&=&&\, 4 + x_{1} - 3\left(\frac{2}{3} + \frac{1}{3}x_{1} + \frac{4}{3}x_{3} - \frac{1}{3}x_{4}\right) - 3x_{3} \\
			&=&&\, 4 + x_{1} - 2 - x_{1} - 4x_{3} + x_{4} - 3x_{3} \\
			&=&&\, 2 - 7x_{3} + x_{4} \\
			x_{7} &=&& 2 - x_{1} + \frac{2}{3} + \frac{1}{3}x_{1} + \frac{4}{3}x_{3} - \frac{1}{3}x_{4} - 3x_{3} \\
			&=&& \frac{8}{3} - \frac{2}{3}x_{1} - \frac{5}{3}x_{3} - \frac{1}{3}x_{4} \\
			z \,&=&&\, - 5x_{1} + 11\left(\frac{2}{3} + \frac{1}{3}x_{1} + \frac{4}{3}x_{3} - \frac{1}{3}x_{4}\right) - 5x_{3} \\
			&=&&\, -5x_{1} + \frac{22}{3} + \frac{11}{3}x_{1} + \frac{44}{3}x_{3} - \frac{11}{3}x_{4} - 5x_{3} \\
			&=&&\, \frac{22}{3} - \frac{4}{3}x_{1} + \frac{29}{3}x_{3} - \frac{11}{3}x_{4}
		\end{alignat*}
		
		\underline{Ergebnis der 1. Iteration}:
		\begin{alignat*}{5}
			x_{2} \,&=&\, \frac{2}{3} \,&+&\, \frac{1}{3}x_{1} \,&+&\, \frac{4}{3}x_{3} \,&-&\, \frac{1}{3}x_{4} \\
			x_{5} \,&=&\, \frac{8}{3} \,&-&\, \frac{8}{3}x_{1} \,&-&\, \frac{29}{3}x_{3} \,&+&\, \frac{5}{3}x_{4} \\
			x_{6} \,&=&\, 2 && \,&-&\, 7x_{3} \,&+&\, x_{4} \\
			x_{7} \,&=&\, \frac{8}{3} \,&-&\, \frac{2}{3}x_{1} \,&-&\, \frac{5}{3}x_{3} \,&-&\, \frac{1}{3}x_{4} \\ \cline{1 - 9}
			z &=& \frac{22}{3} \,&-&\, \frac{4}{3}x_{1} \,&+&\, \frac{29}{3}x_{3} \,&-&\, \frac{11}{3}x_{4}
		\end{alignat*}
		
		\underline{2. Iteration}:
		
		Eingangsvariable: $x_{3}$ \\
		Ausgangsvariable: $x_{5}$
		
		Es folgt
		\begin{alignat*}{2}
			\frac{29}{3}x_{3} &=&& \frac{8}{3} - \frac{8}{3}x_{1} + \frac{5}{3}x_{4} - x_{5} \\
			x_{3} &=&& \frac{8}{29} - \frac{8}{29}x_{1} + \frac{5}{29}x_{4} - \frac{3}{29}x_{5} \\
			x_{2} &=&& \frac{2}{3} + \frac{1}{3}x_{1} + \frac{4}{3}\left(\frac{8}{29} - \frac{8}{29}x_{1} + \frac{5}{29}x_{4} - \frac{3}{29}x_{5}\right) - \frac{1}{3}x_{4} \\
			&=&& \frac{2}{3} + \frac{1}{3}x_{1} + \frac{32}{87} - \frac{32}{87}x_{1} + \frac{20}{87}x_{4} - \frac{4}{29}x_{5} - \frac{1}{3}x_{4} \\
			&=&& \frac{30}{29} - \frac{1}{29}x_{1} - \frac{1}{29}x_{4} - \frac{4}{29}x_{5} \\
			x_{6} &=&& 2 - 7\left(\frac{8}{29} - \frac{8}{29}x_{1} + \frac{5}{29}x_{4} - \frac{3}{29}x_{5}\right) + x_{4} \\
			&=&& 2 - \frac{56}{29} + \frac{56}{29}x_{1} - \frac{35}{29}x_{4} + \frac{21}{29}x_{5} + x_{4} \\
			&=&& \frac{2}{29} + \frac{56}{29}x_{1} - \frac{6}{29}x_{4} + \frac{21}{29}x_{5} \\
			x_{7} &=&& \frac{8}{3} - \frac{2}{3}x_{1} - \frac{5}{3}\left(\frac{8}{29} - \frac{8}{29}x_{1} + \frac{5}{29}x_{4} - \frac{3}{29}x_{5}\right) - \frac{1}{3}x_{4} \\
			&=&& \frac{8}{3} - \frac{2}{3}x_{1} - \frac{40}{87} + \frac{40}{87}x_{1} - \frac{25}{87}x_{4} + \frac{5}{29}x_{5} - \frac{1}{3}x_{4} \\
			&=&& \frac{64}{29} - \frac{6}{29}x_{1} - \frac{18}{29}x_{4} + \frac{5}{29}x_{5} \\
			z &=&& \frac{22}{3} - \frac{4}{3}x_{1} + \frac{29}{3}\left(\frac{8}{29} - \frac{8}{29}x_{1} + \frac{5}{29}x_{4} - \frac{3}{29}x_{5}\right) - \frac{11}{3}x_{4} \\
			&=&& \frac{22}{3} - \frac{4}{3}x_{1} + \frac{8}{3} - \frac{8}{3}x_{1} + \frac{5}{3}x_{4} - x_{5} - \frac{11}{3}x_{4} \\
			&=&& 10 + \frac{4}{3}x_{1} - 2x_{4} - x_{5}
		\end{alignat*}
		
		\underline{Ergebnis der 2. Iteration}:
		\begin{alignat*}{5}
			x_{3} \,&=&\, \frac{8}{29} \,&-&\, \frac{8}{29}x_{1} \,&+&\, \frac{5}{29}x_{4} \,&-&\, \frac{3}{29}x_{5} \\
			x_{2} \,&=&\, \frac{30}{29} \,&-&\, \frac{1}{29}x_{1} \,&-&\, \frac{1}{29}x_{4} \,&-&\, \frac{4}{29}x_{5} \\
			x_{6} \,&=&\, \frac{2}{29} \,&+&\, \frac{56}{29}x_{1} \,&-&\, \frac{6}{29}x_{4} \,&+&\, \frac{21}{29}x_{5} \\
			x_{7} \,&=&\, \frac{64}{29} \,&-&\, \frac{6}{29}x_{1} \,&-&\, \frac{18}{29}x_{4} \,&+&\, \frac{5}{29}x_{5} \\ \cline{1 - 9}
			z &=& 10 \,&+&\, \frac{4}{3}x_{1} \,&-&\, 2x_{4} \,&-&\, x_{5}
		\end{alignat*}
		
		\underline{3. Iteration}:
		
		Eingangsvariable: $x_{1}$\\
		Ausgangsvariable: $x_{3}$
		
		Es folgt
		\begin{alignat*}{2}
			\frac{8}{29}x_{1} &=&& \frac{8}{29} + \frac{5}{29}x_{4} - \frac{3}{29}x_{5} - x_{3} \\
			x_{1} &=&& 1 + \frac{5}{8}x_{4} - \frac{3}{8}x_{5} - \frac{29}{8}x_{3} \\
			x_{2} &=&& \frac{30}{29} - \frac{1}{29}\left(1 + \frac{5}{8}x_{4} - \frac{3}{8}x_{5} - \frac{29}{8}x_{3}\right) - \frac{1}{29}x_{4} - \frac{4}{29}x_{5} \\
			&=&& \frac{30}{29} - \frac{1}{29} - \frac{5}{232}x_{4} +  \frac{3}{232}x_{5} + \frac{1}{8}x_{3} - \frac{1}{29}x_{4} - \frac{4}{29}x_{5} \\
			&=&& 1 - \frac{13}{232}x_{4} - \frac{1}{8}x_{5} + \frac{1}{8}x_{3} \\
			x_{6} &=&& \frac{2}{29} + \frac{56}{29}\left(1 + \frac{5}{8}x_{4} - \frac{3}{8}x_{5} - \frac{29}{8}x_{3}\right) - \frac{6}{29}x_{4} + \frac{21}{29}x_{5} \\
			&=&& \frac{2}{29} + \frac{56}{29} + \frac{35}{29}x_{4} - \frac{21}{29}x_{5} - 7x_{3} - \frac{6}{29}x_{4} + \frac{21}{29}x_{5} \\
			&=&& 2 + x_{4} - 7x_{3} \\
			x_{7} &=&& \frac{64}{29} - \frac{6}{29}\left(1 + \frac{5}{8}x_{4} - \frac{3}{8}x_{5} - \frac{29}{8}x_{3}\right) - \frac{18}{29}x_{4} + \frac{5}{29}x_{5} \\
			&=&& \frac{64}{29} - \frac{6}{29} - \frac{15}{116}x_{4} + \frac{18}{232}x_{5} + \frac{3}{4}x_{3} - \frac{18}{29}x_{4} + \frac{5}{29}x_{5} \\
			&=&& 2 - \frac{3}{4}x_{4} + \frac{1}{4}x_{5} + \frac{3}{4}x_{3} \\
			z &=&& 10 + \frac{4}{3}\left(1 + \frac{5}{8}x_{4} - \frac{3}{8}x_{5} - \frac{29}{8}x_{3}\right) - 2x_{4} - x_{5} \\
			&=&& 10 + \frac{4}{3} + \frac{5}{6}x_{4} - \frac{1}{2}x_{5} - \frac{29}{6}x_{3} - 2x_{4} - x_{5} \\
			&=&& \frac{34}{3} - \frac{7}{6}x_{4} - \frac{3}{2}x_{5} - \frac{29}{6}x_{3}
		\end{alignat*}
		
		\underline{Ergebnis der 3. Iteration}:
		\begin{alignat*}{5}
			x_{1} \,&=&\, 1 \,&+&\, \frac{5}{8}x_{4} \,&-&\, \frac{3}{8}x_{5} \,&-&\, \frac{29}{8}x_{3} \\
			x_{2} \,&=&\, 1 \,&-&\, \frac{13}{232}x_{4} \,&-&\, \frac{1}{8}x_{5} \,&+&\, \frac{1}{8}x_{3} \\
			x_{6} \,&=&\, 2 \,&+&\, x_{4} && \,&-&\, 7x_{3} \\
			x_{7} \,&=&\, 2 \,&-&\, \frac{3}{4}x_{4} \,&+&\, \frac{1}{4}x_{5} \,&+&\, \frac{3}{4}x_{3} \\ \cline{1 - 9}
			z &=& \frac{34}{3} \,&-&\, \frac{7}{6}x_{4} \,&-&\, \frac{3}{2}x_{5} \,&-&\, \frac{29}{6}x_{3}
		\end{alignat*}
		
		Dieses Tableau liefert die optimale Lösung $x_{1} = 1, x_{2} = 1, x_{3} = 0$ mit $z = \frac{34}{3}$.
		
		\underline{Startlösung ("`zulässige Basislösung am Anfang"')}:
		\[
			x_{1} = 0, x_{2} = 0, x_{3} = 0, x_{4} = 2, x_{5} = 6, x_{6} = 4, x_{7} = 2 \text{ mit } z = 0
		\]
		\underline{Zulässige Basislösung nach der 1. Iteration}:
		\[
			x_{1} = 0, x_{2} = \frac{2}{3}, x_{3} = 0, x_{4} = 0, x_{5} = \frac{8}{3}, x_{6} = 2, x_{7} = \frac{8}{3} \text{ mit } z = \frac{22}{3} = 7\frac{1}{3}
		\]
		\underline{Zulässige Basislösung nach der 2. Iteration}:
		\[
			x_{1} = 0, x_{2} = \frac{30}{29}, x_{3} = \frac{8}{29}, x_{4} = 0, x_{5} = 0, x_{6} = 2, x_{7} = \frac{64}{29} \text{ mit } z = 10
		\]
		\underline{Zulässige Basislösung nach der 3. Iteration}:
		\[
			x_{1} = 1, x_{2} = 1, x_{3} = 0, x_{4} = 0, x_{5} = 0, x_{6} = 2, x_{7} = 2 \text{ mit } z = \frac{34}{3} = 11\frac{1}{3}
		\]
\section{} %2
\end{document}
