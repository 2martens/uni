\documentclass[10pt,a4paper,oneside,ngerman,numbers=noenddot]{scrartcl}
\usepackage[T1]{fontenc}
\usepackage[utf8]{inputenc}
\usepackage[ngerman]{babel}
\usepackage{amsmath}
\usepackage{amsfonts}
\usepackage{amssymb}
\usepackage{paralist}
\usepackage{gauss}
\usepackage{pgfplots}
\usepackage[locale=DE,exponent-product=\cdot,detect-all]{siunitx}
\usepackage{tikz}
\usetikzlibrary{matrix,fadings,calc,positioning,decorations.pathreplacing,arrows,decorations.markings}
\usepackage{polynom}
\usepackage{multirow}
\polyset{style=C, div=:,vars=x}
\pgfplotsset{compat=1.8}
\pagenumbering{arabic}
% ensures that paragraphs are separated by empty lines
\parskip 12pt plus 1pt minus 1pt
\parindent 0pt
% define how the sections are rendered
\def\thesection{\arabic{section})}
\def\thesubsection{\alph{subsection})}
\def\thesubsubsection{(\roman{subsubsection})}
% some matrix magic
\makeatletter
\renewcommand*\env@matrix[1][*\c@MaxMatrixCols c]{%
  \hskip -\arraycolsep
  \let\@ifnextchar\new@ifnextchar
  \array{#1}}
\makeatother

\begin{document}
\author{Jan Branitz (6326955), Jim Martens (6420323),\\
Stephan Niendorf (6242417)}
\title{Hausaufgaben zum 27. Januar}
\maketitle

\section{} %1
	\begin{tabular}{|c||c|c|c|c|c|c|c|c|c|c|c|c|c|c|c|c|c|c|c|c|}
		\hline
		7 & 0 & 0 & 0 & 2 & 2 & 4 & 4 & 5 & 6 & 6 & 7 & 8 & 9 & 10 & 10 & 11 & 11 & 11 & 13 & \underline{13} \\
		\hline
		6 & 0 & 0 & 0 & 0 & 2 & 4 & 4 & 5 & 6 & 6 & 6 & 7 & 9 & 10 & 10 & 11 & 11 & 11 & 12 & \underline{13} \\
		\hline
		5 & 0 & 0 & 0 & 0 & 2 & 2 & 3 & 5 & 6 & 6 & 6 & 7 & 8 & 8 & 9 & 11 & 11 & 11 & 11 & \underline{13} \\
		\hline
		4 & 0 & 0 & 0 & 0 & 2 & 2 & 3 & 5 & 5 & 5 & 5 & \underline{7} & 7 & 8 & 8 & 8 & 8 & 9 & 11 & \underline{11} \\
		\hline
		3 & 0 & 0 & 0 & 0 & 2 & 2 & \underline{3} & 3 & 3 & 3 & 5 & \underline{6} & 6 & 6 & 6 & 8 & 8 & 9 & 9 & 9 \\
		\hline
		2 & 0 & 0 & 0 & 0 & 0 & 0 & \underline{3} & 3 & 3 & 3 & 3 & 6 & 6 & 6 & 6 & 6 & 6 & 9 & 9 & 9 \\
		\hline
		1 & 0 & 0 & 0 & 0 & 0 & 0 & \underline{3} & 3 & 3 & 3 & 3 & 3 & 3 & 3 & 3 & 3 & 3 & 3 & 3 & 3 \\
		\hline
		0 & 0 & 0 & 0 & 0 & 0 & 0 & 0 & 0 & 0 & 0 & 0 & 0 & 0 & 0 & 0 & 0 & 0 & 0 & 0 & 0 \\
		\hline
		\hline
		& 0 & 1 & 2 & 3 & 4 & 5 & 6 & 7 & 8 & 9 & 10 & 11 & 12 & 13 & 14 & 15 & 16 & 17 & 18 & 19 \\
		\hline
	\end{tabular}
	
	Der Rucksack ist mit den Items 1, 4 und 5 gefüllt.
\section{} %2
	\begin{tabular}{c|c|c|c|c|c}
		& s & a & b & c & d \\
		\hline
		0 & 0 - & $\infty$ - & $\infty$ - & $\infty$ - & $\infty$ - \\
		1 & 0 - & 6 s & 4 s & 2 b & 2 a \\
		2 & 0 - & 6 s & 3 d & 1 b & 2 a \\
		3 & 0 - & 6 s & 3 d & 1 b & 2 a \\
		4 & 0 - & 6 s & 3 d & 1 b & 2 a
	\end{tabular}
\end{document}
