\documentclass[10pt,a4paper,oneside,ngerman,numbers=noenddot]{scrartcl}
\usepackage[T1]{fontenc}
\usepackage[utf8]{inputenc}
\usepackage[ngerman]{babel}
\usepackage{amsmath}
\usepackage{amsfonts}
\usepackage{amssymb}
\usepackage{paralist}
\usepackage{gauss}
\usepackage{pgfplots}
\usepackage[locale=DE,exponent-product=\cdot,detect-all]{siunitx}
\usepackage{tikz}
\usetikzlibrary{matrix,fadings,calc,positioning,decorations.pathreplacing,arrows,decorations.markings}
\usepackage{polynom}
\usepackage{multirow}
\polyset{style=C, div=:,vars=x}
\pgfplotsset{compat=1.8}
\pagenumbering{arabic}
% ensures that paragraphs are separated by empty lines
\parskip 12pt plus 1pt minus 1pt
\parindent 0pt
% define how the sections are rendered
\def\thesection{\arabic{section})}
\def\thesubsection{\alph{subsection})}
\def\thesubsubsection{(\roman{subsubsection})}
% some matrix magic
\makeatletter
\renewcommand*\env@matrix[1][*\c@MaxMatrixCols c]{%
  \hskip -\arraycolsep
  \let\@ifnextchar\new@ifnextchar
  \array{#1}}
\makeatother

\begin{document}
\author{Jan Branitz (6326955), Jim Martens (6420323),\\
Stephan Niendorf (6242417)}
\title{Hausaufgaben zum 13. Januar}
\maketitle
\section{} %1
	\subsection{} %a
		\subsubsection{} %i
			Die Knoten werden in der Reihenfolge a, b, f, c, d, g, e und t markiert.
			
			\begin{tabular}{c|c}
			Knoten & Markierung \\
			\hline
			s & $(-, \infty)$ \\
			a & (s, +, 38) \\
			b & (s, +, 1) \\
			f & (s, +, 2) \\
			c & (a, +, 10) \\
			d & (a, +, 38) \\
			g & (f, +, 2) \\
			e & (c, +, 10) \\
			t & (c, +, 10)
			\end{tabular}
			
			Es gibt keine Knoten, die unmarkiert bleiben.
		\subsubsection{} %ii
			%s, a, b, f, d, c, g, e, t
			Der Pfad $f_{0}$ führt von s über a und c nach t. Der verbesserte Fluss $f_{1}$ weist die folgenden Markierungen auf.
			
			\begin{tabular}{c|c}
			Knoten & Markierung \\
			\hline
			s & $(-, \infty)$ \\
			a & (s, +, 28) \\
			b & (s, +, 1) \\
			f & (s, +, 2) \\
			d & (a, +, 28) \\
			c & (b, +, 1) \\
			g & (f, +, 2) \\
			e & (d, +, 1) \\
			t & (d, +, 7)
			\end{tabular}
	\subsection{} %b
		Die Knoten, die zu S gehören, sind markiert und die Knoten, die zu T gehören, sind nicht markiert.
	\subsection{} %c
		(i) ist wahr und (ii) ist falsch.
	\subsection{} %d
		In jedem Graphen gilt $m(G) \leq c(G)$ (siehe Skript Seite 121, 11.1), weswegen (ii) falsch ist. Solange mindestens zwei Knoten nicht zum Matching gehören (je mindestens einer in X und einer in Y), können diese im besten Fall durch eine neue Matchingkante dem Matching hinzugefügt werden. In diesem Fall ist die Matchingzahl $m(G)$ also mindestens um eins höher als ein unter den beschriebenen Voraussetzungen vorhandenes Matching.
\section{} %2
	\subsection{} %a
		\begin{tabular}{c|l}
			Durchgang & Matching \\
			\hline
			1 & $M = \{\{x_{1}, y_{1}\}\}$ \\
			2 & $M = \{\{x_{1}, y_{1}\}, \{x_{2}, y_{2}\}\}$ \\
			3 & $M = \{\{x_{1}, y_{4}\}, \{x_{2}, y_{2}\}, \{x_{3}, y_{1}\}\}$ \\
			4 & $M = \{\{x_{1}, y_{4}\}, \{x_{2}, y_{3}\}, \{x_{3}, y_{1}\}, \{x_{4}, y_{2}\}\}$ \\
			5 & $M = \{\{x_{1}, y_{4}\}, \{x_{2}, y_{3}\}, \{x_{3}, y_{1}\}, \{x_{4}, y_{2}\}, \{x_{5}, y_{5}\}\}$ \\
			6 & $M = \{\{x_{1}, y_{4}\}, \{x_{2}, y_{3}\}, \{x_{3}, y_{1}\}, \{x_{4}, y_{2}\}, \{x_{5}, y_{5}\}, \{x_{6}, y_{6}\}\}$ \\
		\end{tabular}
	\subsection{} %b
		\begin{tabular}{c|l}
			Durchgang & Matching \\
			\hline
			1 & $M = \{\{x_{1}, y_{1}\}\}$ \\
			2 & $M = \{\{x_{1}, y_{1}\}, \{x_{2}, y_{2}\}\}$ \\
			3 & $M = \{\{x_{1}, y_{3}\}, \{x_{2}, y_{2}\}, \{x_{3}, y_{1}\}\}$ \\
			4 & $M = \{\{x_{1}, y_{3}\}, \{x_{2}, y_{2}\}, \{x_{3}, y_{1}\}, \{x_{4}, y_{4}\}\}$ \\
			5 & $M = \{\{x_{1}, y_{3}\}, \{x_{2}, y_{5}\}, \{x_{3}, y_{1}\}, \{x_{4}, y_{4}\}, \{x_{5}, y_{2}\}\}$ \\
			6 & $M = \{\{x_{1}, y_{3}\}, \{x_{2}, y_{5}\}, \{x_{3}, y_{1}\}, \{x_{4}, y_{4}\}, \{x_{5}, y_{2}\}, \{x_{6}, y_{6}\}\}$ \\
		\end{tabular}
\end{document}
