\documentclass[10pt,a4paper,oneside,ngerman,numbers=noenddot]{scrartcl}
\usepackage[T1]{fontenc}
\usepackage[utf8]{inputenc}
\usepackage[ngerman]{babel}
\usepackage{amsmath}
\usepackage{amsfonts}
\usepackage{amssymb}
\usepackage{paralist}
\usepackage{gauss}
\usepackage{pgfplots}
\usepackage[locale=DE,exponent-product=\cdot,detect-all]{siunitx}
\usepackage{tikz}
\usetikzlibrary{matrix,fadings,calc,positioning,decorations.pathreplacing,arrows,decorations.markings}
\usepackage{polynom}
\usepackage{multirow}
\polyset{style=C, div=:,vars=x}
\pgfplotsset{compat=1.8}
\pagenumbering{arabic}
% ensures that paragraphs are separated by empty lines
\parskip 12pt plus 1pt minus 1pt
\parindent 0pt
% define how the sections are rendered
\def\thesection{\arabic{section})}
\def\thesubsection{\alph{subsection})}
\def\thesubsubsection{(\roman{subsubsection})}
% some matrix magic
\makeatletter
\renewcommand*\env@matrix[1][*\c@MaxMatrixCols c]{%
  \hskip -\arraycolsep
  \let\@ifnextchar\new@ifnextchar
  \array{#1}}
\makeatother

\begin{document}
\author{Jan Branitz (6326955), Jim Martens (6420323),\\
Stephan Niendorf (6242417)}
\title{Hausaufgaben zum 6. Januar}
\maketitle
\section{} %1
\section{} %2
	\subsection{} %a
		\begin{alignat*}{9}
			\text{minimiere}\; & 67w &+& 120k &+& 100h &+& 60f &+& 97b &+& 124n &+& 22s &+& 62m &&  \\
			\multicolumn{18}{l}{\text{unter den Nebenbedingungen}} && \\
			& 8w &+& 25k &+& 30h &+& 22f &+& 3b &+& 8n &+& 6s && &\geq &\, 75 \\
			& w &+& 35k &+& 8h &+& f && &+& 33n &+& 13s &+& 98m &\geq &\, 90 \\
			& 54w && && && &+& 42b &+& 4n &+& 63s && &\geq &\, 300 \\
			\multicolumn{16}{r}{$w, k, h, f, b, n, s, m$} \,&\geq &\, 0
		\end{alignat*}
	\subsection{} %b
		In das verlinkte Tool wird folgendes eingegeben:
\begin{verbatim}
Minimize p = 67w + 120k + 100h + 60f + 97b + 124n + 22s + 62m subject to
8w + 25k + 30h + 22f + 3b + 8n + 6s + 0m >= 75
w + 35k + 8h + f + 0b + 33n + 13s + 98m >= 90
54w + 0k + 0h + 0f + 42b + 4n + 63s + 0m >= 300
w >= 0
k >= 0
h >= 0
f >= 0
b >= 0
n >= 0
s >= 0
m >= 0
\end{verbatim}
		Die Ausgabe ist:
\begin{verbatim}
Optimal Solution: p = 6801/28; w = 0, k = 0, h = 0, f = 87/56, 
b = 0, n = 0, s = 381/56, m = 0
\end{verbatim}
	\subsection{} %c
		In das verlinkte Tool wird folgendes eingegeben:
\begin{verbatim}
Minimize p = 3x1 + 24x2 + 13x3 + 9x4 + 20x5 + 19x6 subject to
110x1 + 205x2 + 160x3 + 160x3 + 420x5 + 260x6 >= 2000
4x1 + 32x2 + 13x3 + 8x4 + 4x5 + 14x6 >= 55
2x1 + 12x2 + 54x3 + 285x4 + 22x5 + 80x6 >= 800
x1 <= 4
x2 <= 3
x3 <= 2
x4 <= 8
x5 <= 2
x6 <= 2
x1 >= 0
x2 >= 0
x3 >= 0
x4 >= 0
x5 >= 0
x6 >= 0
\end{verbatim}
		Die Ausgabe ist: 
\begin{verbatim}
Optimal Solution: p = 46621/380; x1 = 4, x2 = 0, x3 = 5/4, 
x4 = 347/190, x5 = 2, x6 = 2
\end{verbatim}
	\subsection{} %d
		In das verlinkte Tool wird folgendes eingegeben:
\begin{verbatim}
Maximize p = 8a1 + 3a2 + 6b1 + 3b2 + 9c1 + 5c2 subject to
a1 + a2 <= 400
b1 + b2 <= 480
c1 + c2 <= 230
a1 + b1 + c1 <= 420
a2 + b2 + c2 <= 250
a1 >= 0
a2 >= 0
b1 >= 0
b2 >= 0
c1 >= 0
c2 >= 0
\end{verbatim}
	Die Ausgabe ist:
\begin{verbatim}
Optimal Solution: p = 4550; a1 = 400, a2 = 0, b1 = 0, b2 = 40, 
c1 = 20, c2 = 210
\end{verbatim}
\end{document}
