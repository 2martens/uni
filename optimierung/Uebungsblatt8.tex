\documentclass[10pt,a4paper,oneside,ngerman,numbers=noenddot]{scrartcl}
\usepackage[T1]{fontenc}
\usepackage[utf8]{inputenc}
\usepackage[ngerman]{babel}
\usepackage{amsmath}
\usepackage{amsfonts}
\usepackage{amssymb}
\usepackage{paralist}
\usepackage{gauss}
\usepackage{pgfplots}
\usepackage[locale=DE,exponent-product=\cdot,detect-all]{siunitx}
\usepackage{tikz}
\usetikzlibrary{matrix,fadings,calc,positioning,decorations.pathreplacing,arrows,decorations.markings}
\usepackage{polynom}
\polyset{style=C, div=:,vars=x}
\pgfplotsset{compat=1.8}
\pagenumbering{arabic}
% ensures that paragraphs are separated by empty lines
\parskip 12pt plus 1pt minus 1pt
\parindent 0pt
% define how the sections are rendered
\def\thesection{\arabic{section})}
\def\thesubsection{\alph{subsection})}
\def\thesubsubsection{(\roman{subsubsection})}
% some matrix magic
\makeatletter
\renewcommand*\env@matrix[1][*\c@MaxMatrixCols c]{%
  \hskip -\arraycolsep
  \let\@ifnextchar\new@ifnextchar
  \array{#1}}
\makeatother

\begin{document}
\author{Jan Branitz (6326955), Jim Martens (6420323),\\
Stephan Niendorf (6242417)}
\title{Hausaufgaben zum 9. Dezember}
\maketitle
\section{} %1
	\subsection{} %a
		\begin{alignat*}{4}
			\text{minimiere}\; & y_{1} \,&+&\, 2y_{2} \,&+&\, 3y_{3} && \\
			\multicolumn{8}{l}{\text{unter den Nebenbedingungen}} && \\
			\;& 2y_{1} \,&+&\, 3y_{2} \,&+&\, y_{3} \,&\geq & 5 \\
			\;& 3y_{1} \,&+&\, y_{2} \,&+&\, y_{3} \,&\geq & -7 \\
			\;-& y_{1} \,&+&\, 4y_{2} \,&-&\, 2y_{3} \,&\geq & 3 \\
			\;& y_{1} \,&-&\, 2y_{2} \,&-&\, y_{3} \,&=& 1 \\
			\multicolumn{6}{r}{$y_{2}, y_{3}$} \,&\geq &\, 0
		\end{alignat*}
	\subsection{} %b
		\begin{alignat*}{9}
			\text{maximiere}\; -& y_{1} &+& 16y_{2} &+& 5y_{3} &+& 8y_{4} &+& y_{5} &-& 4y_{6} &-& 10y_{7} &+& 9y_{8} &&  \\
			\multicolumn{18}{l}{\text{unter den Nebenbedingungen}} && \\
			& 2y_{1} &+& y_{2} &+& y_{3} &+& 2y_{4} &+& y_{5} &+& 4y_{6} &-& 4y_{7} &+& y_{8} &\leq &\, -2 \\
			-& 4y_{1} &+& 5y_{2} && &+& 4y_{4} &-& 3y_{5} &-& 3y_{6} &+& 3y_{7} &+& 2y_{8} &=&\, 3 \\
			& y_{1} &+& y_{2} &+& y_{3} &-& y_{4} &+& y_{5} && &-& 5y_{7} &+& y_{8} &=&\, 22 \\
			\multicolumn{16}{r}{$y_{4}, y_{5}, y_{6}, y_{7}$} \,&\geq &\, 0
		\end{alignat*}
\section{} %2
	\subsection{} %a
		Das LP-Problem:
		\begin{alignat*}{3}
			\text{maximiere}\; & 40x_{1} \,&+&\, 70x_{2} && \\
			\multicolumn{6}{l}{\text{unter den Nebenbedingungen}} && \\
			\;& x_{1} \,&+&\, x_{2} \,&\leq & 100 \\
			\;& 10x_{1} \,&+&\, 50x_{2} \,&\leq & 4000 \\
			\multicolumn{4}{r}{$x_{1}, x_{2}$} \,&\geq &\, 0
		\end{alignat*}
		
		Das duale Problem:
		\begin{alignat*}{3}
			\text{minimiere}\; & 100y_{1} \,&+&\, 4000y_{2} && \\
			\multicolumn{6}{l}{\text{unter den Nebenbedingungen}} && \\
			\;& y_{1} \,&+&\, 10y_{2} \,&\geq & 40 \\
			\;& y_{1} \,&+&\, 50y_{2} \,&\geq & 70 \\
			\multicolumn{4}{r}{$y_{1}, y_{2}$} \,&\geq &\, 0
		\end{alignat*}

		\subsubsection{} %i
			\underline{Starttableau}:
			\begin{alignat*}{4}
				x_{3} \,&=&\, 100 \,&-&\, x_{1} \,&-&\, x_{2} \\
				x_{4} \,&=&\, 4000 \,&-&\, 10x_{1} \,&-&\, 50x_{2} \\ \cline{1 - 7}
				z &=& &&\, 40x_{1} \,&+&\, 70x_{2}
			\end{alignat*}
			
			\underline{1. Iteration}:
		
			Eingangsvariable: $x_{2}$\\
			Ausgangsvariable: $x_{4}$
		
			Es folgt
			\begin{alignat*}{2}
				50x_{2} \,&=&&\, 4000 - 10x_{1} - x_{4} \\
				x_{2} \,&=&&\, 80 - \frac{1}{5}x_{1} - \frac{1}{50}x_{4} \\
				x_{3} \,&=&&\, 100 - x_{1} - \left(80 - \frac{1}{5}x_{1} - \frac{1}{50}x_{4}\right) \\
				&=&&\, 100 - x_{1} - 80 + \frac{1}{5}x_{1} + \frac{1}{50}x_{4} \\
				&=&&\, 20 - \frac{4}{5}x_{1} + \frac{1}{50}x_{4} \\
				z \,&=&&\, 40x_{1} + 70\left(80 - \frac{1}{5}x_{1} - \frac{1}{50}x_{4}\right) \\
				&=&&\, 40x_{1} + 5600 - 14x_{1} - \frac{7}{5}x_{4} \\
				&=&&\, 5600 + 26x_{1} - \frac{7}{5}x_{4}
			\end{alignat*}
		
			\underline{Ergebnis der 1. Iteration}:
			\begin{alignat*}{4}
				x_{2} \,&=&\, 80 \,&-&\, \frac{1}{5}x_{1} \,&-&\, \frac{1}{50}x_{4} \\
				x_{3} \,&=&\, 20 \,&-&\, \frac{4}{5}x_{1} \,&+&\, \frac{1}{50}x_{4} \\ \cline{1 - 7}
				z &=& 5600 \,&+&\, 26x_{1} \,&-&\, \frac{7}{5}x_{4}
			\end{alignat*}
			
			\underline{2. Iteration}:
		
			Eingangsvariable: $x_{1}$\\
			Ausgangsvariable: $x_{3}$
		
			Es folgt
			\begin{alignat*}{2}
				\frac{4}{5}x_{1} \,&=&&\, 20 +  \frac{1}{50}x_{4} - x_{3} \\
				x_{1} \,&=&&\, 25 + \frac{1}{40}x_{4} - \frac{5}{4}x_{3} \\
				x_{2} \,&=&&\, 80 - \frac{1}{5}\left(25 + \frac{1}{40}x_{4} - \frac{5}{4}x_{3}\right) - \frac{1}{50}x_{4} \\
				&=&&\, 80 - 5 + \frac{1}{200}x_{4} - \frac{1}{4}x_{3} - \frac{1}{50}x_{4} \\
				&=&&\, 75 - \frac{3}{200}x_{4} - \frac{1}{4}x_{3} \\
				z \,&=&&\, 5600 + 26\left(25 + \frac{1}{40}x_{4} - \frac{5}{4}x_{3}\right) - \frac{7}{5}x_{4} \\
				&=&&\, 5600 + 650 + \frac{13}{20}x_{4} - \frac{65}{2}x_{3} - \frac{7}{5}x_{4} \\
				&=&&\, 6250 - \frac{3}{4}x_{4} - \frac{65}{2}x_{3}
			\end{alignat*}
		
			\underline{Ergebnis der 2. Iteration}:
			\begin{alignat*}{4}
				x_{1} \,&=&\, 25 \,&-&\, \frac{1}{40}x_{4} \,&-&\, \frac{5}{4}x_{3} \\
				x_{2} \,&=&\, 75 \,&-&\, \frac{3}{200}x_{4} \,&-&\, \frac{1}{4}x_{3} \\ \cline{1 - 7}
				z &=& 6250 \,&-&\, \frac{3}{4}x_{4} \,&-&\, \frac{65}{2}x_{3}
			\end{alignat*}
			
			Wie hier deutlich wird, ist $x_{1}^{*} = 25, x_{2}^{*} = 75$ eine optimale Lösung des primalen Problems.
			
			\underline{Startlösung ("`zulässige Basislösung am Anfang"')}:
			\[
				x_{1} = 0, x_{2} = 0, x_{3} = 100, x_{4} = 4000 \text{ mit } z = 0
			\]
			\underline{Zulässige Basislösung nach der 1. Iteration}:
			\[
				x_{1} = 0, x_{2} = 80, x_{3} = 20, x_{4} = 0 \text{ mit } z = 5600
			\]
			\underline{Zulässige Basislösung nach der 2. Iteration}:
			\[
				x_{1} = 25, x_{2} = 75, x_{3} = 0, x_{4} = 0 \text{ mit } z = 6250
			\]
			
			Durch Einsetzen von $y_{1}^{*} = 32.5$ und $y_{2}^{*} = 0.75$ in die Zielfunktion des dualen Problems ergibt sich folgendes:
			\[
				100 \cdot \frac{65}{2} + 4000 \cdot \frac{3}{4} = 3250 + 3000 = 6250
			\]
			
			Die beiden Zielfunktionswerte stimmen überein. Nach dem Dualitätssatz folgt daraus, dass $y_{1}^{*} = 32.5, y_{2}^{*} = 0.75$ tatsächlich eine optimale Lösung für das duale Problem darstellt.
		\subsubsection{} %ii
			Zum Überprüfen der vorgeschlagenen Lösung werden die Werte zunächst in die Ungleichungen des LP-Problems eingesetzt.
		
			Erste Ungleichung:
			\begin{alignat*}{2}
				25 + 75 &\leq & 100 \\
				100 &\leq & 100
			\end{alignat*}
		
			Zweite Ungleichung:
			\begin{alignat*}{2}
				10 \cdot 25 + 50 \cdot 75 &\leq & 4000 \\
				4000 &\leq & 4000
			\end{alignat*}
			
			Da beide Ungleichungen mit Gleichheit erfüllt sind, lassen sich keine Rückschlüsse auf $y$-Werte ziehen. Da beide $x$-Werte größer als $0$ sind, müssen beide Ungleichungen im dualen Problem mit Gleichheit erfüllt sein.
			
			\begin{alignat*}{2}
				I \;& y_{1} + 10y_{2} &=& 40 \\
				II \;& y_{1} + 50y_{2} &=& 70 \\
				II - I \;& 40y_{2} &=& 30 \\
				\;& y_{2} &=& \frac{3}{4} \\
				\intertext{Einsetzen von $y_{2}$ in $I$}
				I \;& y_{1} + 10 \cdot \frac{3}{4} &=& 40 \\
				\;& y_{1} + \frac{15}{2} &=& 40 \\
				\;& y_{1} &=& \frac{65}{2}
			\end{alignat*}
			
			Es ergeben sich somit die eindeutig bestimmten Zahlen $y_{1}^{*} = \frac{65}{2}, y_{2}^{*} = \frac{3}{4}$. Diese Zahlen erfüllen zusammen mit der vorgeschlagenen Lösung die komplementären Schlupfbedingungen.
			Schließlich muss noch geprüft werden, ob diese Zahlen auch eine zulässige Lösung des dualen Problems sind. Dafür werden diese eingesetzt:

			Erste Ungleichung:
			\begin{alignat*}{2}
				1 \cdot \frac{65}{2} + 10 \cdot \frac{3}{4} &\geq & 40 \\
				40 &\geq & 40
			\end{alignat*}
		
			Zweite Ungleichung:
			\begin{alignat*}{2}
				1 \cdot \frac{65}{2} + 50 \cdot \frac{3}{4} &\geq & 70 \\
				70 &\geq & 70
			\end{alignat*}
			
			Da alle zwei Ungleichungen mit den herausgefundenen Zahlen gültig sind, stellen die gefundenen Zahlen eine zulässige Lösung des dualen Problems dar.
	\subsection{} %b
		\underline{Starttableau}:
			\begin{alignat*}{4}
				x_{3} \,&=&\, 100 \,&-&\, x_{1} \,&-&\, x_{2} \\
				x_{4} \,&=&\, 4000 + t \,&-&\, 10x_{1} \,&-&\, 50x_{2} \\ \cline{1 - 7}
				z &=& &&\, 40x_{1} \,&+&\, 70x_{2}
			\end{alignat*}
			
			\underline{1. Iteration}:
			
			\textbf{Es wird vorausgesetzt, dass $0 \leq t \leq 1000$ gilt.} Für $t=0$ gilt im Folgenden genau das Gleiche wie in 2a i). Für $t=1000$ kann eine der Ausgangsvariablen nach Belieben gewählt werden, da beide potentiellen Ausgangsvariablen $x_{2}$ gleichermaßen beschränken. Da in den meisten Fällen jedoch $t$ kleiner als $1000$ ist, wird $x_{4}$ als Ausgangsvariable gewählt.
		
			Eingangsvariable: $x_{2}$\\
			Ausgangsvariable: $x_{4}$
		
			Es folgt
			\begin{alignat*}{2}
				50x_{2} \,&=&&\, 4000 + t - 10x_{1} - x_{4} \\
				x_{2} \,&=&&\, 80 + \frac{1}{50}t - \frac{1}{5}x_{1} - \frac{1}{50}x_{4} \\
				x_{3} \,&=&&\, 100 - x_{1} - \left(80  + \frac{1}{50}t - \frac{1}{5}x_{1} - \frac{1}{50}x_{4}\right) \\
				&=&&\, 100 - x_{1} - 80 - \frac{1}{50}t + \frac{1}{5}x_{1} + \frac{1}{50}x_{4} \\
				&=&&\, 20 - \frac{1}{50}t - \frac{4}{5}x_{1} + \frac{1}{50}x_{4} \\
				z \,&=&&\, 40x_{1} + 70\left(80 + \frac{1}{50}t - \frac{1}{5}x_{1} - \frac{1}{50}x_{4}\right) \\
				&=&&\, 40x_{1} + 5600 + \frac{7}{5}t - 14x_{1} - \frac{7}{5}x_{4} \\
				&=&&\, 5600 + \frac{7}{5}t + 26x_{1} - \frac{7}{5}x_{4}
			\end{alignat*}
		
			\underline{Ergebnis der 1. Iteration}:
			\begin{alignat*}{4}
				x_{2} \,&=&\, 80 + \frac{1}{50}t \,&-&\, \frac{1}{5}x_{1} \,&-&\, \frac{1}{50}x_{4} \\
				x_{3} \,&=&\, 20 - \frac{1}{50}t \,&-&\, \frac{4}{5}x_{1} \,&+&\, \frac{1}{50}x_{4} \\ \cline{1 - 7}
				z &=& 5600 + \frac{7}{5}t \,&+&\, 26x_{1} \,&-&\, \frac{7}{5}x_{4}
			\end{alignat*}
			
			\underline{2. Iteration}:
			
			\textbf{Es wird vorausgesetzt, dass $t \leq 1000$ gilt.} Könnte $t$ größer sein, dann würde die Möglichkeit bestehen, dass $x_{3}$ in der Basislösung nach der ersten Iteration einen negativen Wert hat.
		
			Eingangsvariable: $x_{1}$\\
			Ausgangsvariable: $x_{3}$
		
			Es folgt
			\begin{alignat*}{2}
				\frac{4}{5}x_{1} \,&=&&\, 20 - \frac{1}{50}t +  \frac{1}{50}x_{4} - x_{3} \\
				x_{1} \,&=&&\, 25 - \frac{1}{40}t + \frac{1}{40}x_{4} - \frac{5}{4}x_{3} \\
				x_{2} \,&=&&\, 80 + \frac{1}{50}t - \frac{1}{5}\left(25 - \frac{1}{40}t + \frac{1}{40}x_{4} - \frac{5}{4}x_{3}\right) - \frac{1}{50}x_{4} \\
				&=&&\, 80 + \frac{1}{50}t - 5 + \frac{1}{200}t - \frac{1}{200}x_{4} - \frac{1}{4}x_{3} - \frac{1}{50}x_{4} \\
				&=&&\, 75 + \frac{1}{40}t - \frac{1}{40}x_{4} - \frac{1}{4}x_{3} \\
				z \,&=&&\, 5600 + \frac{7}{5}t + 26\left(25 - \frac{1}{40}t + \frac{1}{40}x_{4} - \frac{5}{4}x_{3}\right) - \frac{7}{5}x_{4} \\
				&=&&\, 5600 + \frac{7}{5}t + 650 - \frac{13}{20}t + \frac{13}{20}x_{4} - \frac{65}{2}x_{3} - \frac{7}{5}x_{4} \\
				&=&&\, 6250 + \frac{3}{4}t - \frac{3}{4}x_{4} - \frac{65}{2}x_{3}
			\end{alignat*}
		
			\underline{Ergebnis der 2. Iteration}:
			\begin{alignat*}{4}
				x_{1} \,&=&\, 25 - \frac{1}{40}t \,&-&\, \frac{1}{40}x_{4} \,&-&\, \frac{5}{4}x_{3} \\
				x_{2} \,&=&\, 75 + \frac{1}{40}t \,&-&\, \frac{1}{40}x_{4} \,&-&\, \frac{1}{4}x_{3} \\ \cline{1 - 7}
				z &=& 6250 + \frac{3}{4}t \,&-&\, \frac{3}{4}x_{4} \,&-&\, \frac{65}{2}x_{3}
			\end{alignat*}
			
			Wie hier deutlich wird, ist $x_{1}^{*} = 25, x_{2}^{*} = 75$ eine optimale Lösung des primalen Problems.
			
			\underline{Startlösung ("`zulässige Basislösung am Anfang"')}:
			\[
				x_{1} = 0, x_{2} = 0, x_{3} = 100, x_{4} = 4000 + t \text{ mit } z = 0
			\]
			\underline{Zulässige Basislösung nach der 1. Iteration}:
			\[
				x_{1} = 0, x_{2} = 80 + \frac{1}{50}t, x_{3} = 20 - \frac{1}{50}t, x_{4} = 0 \text{ mit } z = 5600 + \frac{7}{5}t
			\]
			\underline{Zulässige Basislösung nach der 2. Iteration}:
			\[
				x_{1} = 25 - \frac{1}{40}t, x_{2} = 75 + \frac{1}{40}t, x_{3} = 0, x_{4} = 0 \text{ mit } z = 6250 + \frac{3}{4}t
			\]
			
			Wie im Folgenden zu sehen ist, entsprechen die Werte der optimalen Lösung den in (7.24) auf Skriptseite 67 angenommenen Werten.
			\[
				x_{1} = 25 - \frac{1}{40}t = 25 - \frac{25}{1000}t = 25 - 0.025t
			\]
			\[
				x_{2} = 75 + \frac{1}{40}t = 75 + \frac{25}{1000}t = 75 + 0.025t
			\]
			Im Folgenden ist zu sehen, dass tatsächlich ein zusätzlicher Gewinn von $0.75t$ erzielt wird.
			\[
				z = 6250 + \frac{3}{4}t = 6250 + \frac{75}{100}t = 6250 + 0.75t
			\]
\end{document}
