\documentclass[10pt,a4paper,oneside,ngerman,numbers=noenddot]{scrartcl}
\usepackage[T1]{fontenc}
\usepackage[utf8]{inputenc}
\usepackage[ngerman]{babel}
\usepackage{amsmath}
\usepackage{amsfonts}
\usepackage{amssymb}
\usepackage{paralist}
\usepackage{gauss}
\usepackage{pgfplots}
\usepackage[locale=DE,exponent-product=\cdot,detect-all]{siunitx}
\usepackage{tikz}
\usetikzlibrary{matrix,fadings,calc,positioning,decorations.pathreplacing,arrows,decorations.markings}
\usepackage{polynom}
\polyset{style=C, div=:,vars=x}
\pgfplotsset{compat=1.8}
\pagenumbering{arabic}
\def\thesection{\arabic{section})}
\def\thesubsection{\alph{subsection})}
\def\thesubsubsection{(\roman{subsubsection})}
\makeatletter
\renewcommand*\env@matrix[1][*\c@MaxMatrixCols c]{%
  \hskip -\arraycolsep
  \let\@ifnextchar\new@ifnextchar
  \array{#1}}
\makeatother

\begin{document}
\author{Jan Branitz (6326955), Jim Martens (6420323),\\
Stephan Niendorf (6242417)}
\title{Hausaufgaben zum 21. Oktober}
\maketitle
\section{} %1
	\subsection{} %a
		\subsubsection{} %i
		Die Zielfunktion muss in Standardform maximiert werden. Um dies zu erreichen, wird mit $-1$ multipliziert. Die erste Nebenbedingung wird auf gleiche Weise umgeformt. Die dritte Nebenbedingung wird durch zwei Bedingungen ersetzt.
		
		\begin{alignat*}{5}
		\text{maximiere} -& 2x_{1} &-& x_{2} &+& x_{3} &-& 2x_{4} && \\
		\multicolumn{10}{l}{\text{unter den Nebenbedingungen}} && \\
		& 3x_{1} &+& x_{2} &-& x_{3} && &\leq & -2 \\
		-& 7x_{1} &-& x_{2} && &+& x_{4} &\leq &\, 3 \\
		& && x_{2} &+& x_{3} &-& x_{4} &\leq &\, 7 \\
		& &-& x_{2} &-& x_{3} &+& x_{4} &\leq & -7 \\
		\multicolumn{8}{r}{$x_{1}, x_{2}, x_{3}, x_{4}$} \,&\geq &\, 0
		\end{alignat*}
		\subsubsection{} %ii
		Die erste Nebenbedingung wird mit $-1$ multipliziert. Auch wird die dritte Nebenbedingung durch zwei Bedingungen ersetzt. Da $x_{1}$ in der Nichtnegativitätsbedingung fehlt, werden zwei Variablen $x_{1}^{'}, x_{1}^{''}$ erzeugt, die je den positiven bzw. negativen Teil von $x_{1}$ darstellen.
		\begin{alignat*}{6}
		\text{maximiere}\; & 2x_{1}^{'} &-& 2x_{1}^{''} &+& x_{2} &-& x_{3} &+& 2x_{4} && \\
		\multicolumn{12}{l}{\text{unter den Nebenbedingungen}} && \\
		& 3x_{1}^{'} &-& 3x_{1}^{''} &+& x_{2} &-& x_{3} && &\leq & -2 \\
		-& 7x_{1}^{'} &+& 7x_{1}^{''} &-& x_{2} && &+& x_{4} &\leq &\, 3 \\
		& && && x_{2} &+& x_{3} &-& x_{4} &\leq &\, 7 \\
		& && &-& x_{2} &-& x_{3} &+& x_{4} &\leq & -7 \\
		& && && && && x_{4} &\leq & 9 \\
		\multicolumn{10}{r}{$x_{1}^{'}, x_{1}^{''}, x_{2}, x_{3}, x_{4}$} \,&\geq &\, 0
		\end{alignat*}
	\subsection{} %b
	Es gilt das folgende Problem mit der grafischen Methode zu lösen.
	\begin{alignat*}{3}
		\text{maximiere}\; & 2x_{1} &+& 5x_{2}&& \\
		\multicolumn{6}{l}{\text{unter den Nebenbedingungen}} && \\
		& 3x_{1} &-& 2x_{2} &\leq &\, 6 \\
		& x_{1} &+& x_{2} &\leq &\, 6 \\
		-& 2x_{1} &+& 6x_{2} &\leq &\, 18 \\
		\multicolumn{4}{r}{$x_{1}, x_{2}$} \,&\geq &\, 0
	\end{alignat*}
	Nach Umstellen der Nebenbedingungen nach $x_{2}$ ergibt sich dieses:
	\begin{alignat*}{3}
	x_{2} &\geq & \frac{3}{2}x_{1} &-& 3 \\
	x_{2} &\leq & -x_{1} &+& 6 \\
	x_{2} &\leq & \frac{1}{3}x_{1} &+& 3
	\end{alignat*}
	Daraus lässt sich die Fläche aller gültigen Werte zeichnen.	
	
	\begin{tikzpicture}[>=stealth]
	\begin{axis}[
		ymin=0,ymax=7,
		x=1cm,
		y=1cm,
		axis x line=middle,
		axis y line=middle,
		axis line style=->,
		xlabel={$x_{1}$},
		ylabel={$x_{2}$},
		xmin=0,xmax=7
	]

	\addplot[no marks, black, -] expression[domain=2:6,samples=100]{1.5*x - 3} node[pos=0.65,anchor=north]{};
	\addplot[no marks, black, -] expression[domain=0:6,samples=100]{-1*x + 6} node[pos=0.65,anchor=north]{};
	\addplot[no marks, black, -] expression[domain=0:6,samples=100]{0.3333333333333*x + 3} node[pos=0.65,anchor=north]{};
	\addplot[no marks, black, -] expression[domain=0:7,samples=100]{-0.4*x + 4.65} node[pos=0.65,anchor=north]{};
	\node at (axis cs: 2.5,4.5) {(2.2,3.75)};
	\node at (axis cs: 6,2) {z};
	%\draw[>=stealth] (axis cs:1,0) -- (axis cs:1,-6) node [pos=0.65,anchor=north]{};
	\end{axis}
	\end{tikzpicture}\\
	Das optimale Ergebnis kann folgendermaßen bestimmt werden:
	
	\begin{alignat*}{5}
		I &-&\; 2x_{1} &+& 6x_{2} &=& 18 && \;| + 2II\\
		II &&\; x_{1} &+& x_{2} &=& 6 && \\
		\overset{I+2II}{\Rightarrow} &&\; && 8x_{2} &=& 30 && \;| \cdot \frac{1}{8} \\
		\Leftrightarrow &&\; && x_{2} &=& \frac{30}{8} = \frac{15}{4} &&
		\intertext{Einsetzen in II}
		\overset{II}{\Rightarrow} &&\; x_{1} &+& \frac{15}{4} &=& 6 &&\;| - \frac{15}{4} \\
		&&\; x_{1} && &=& \frac{24}{4} - \frac{15}{4} = \frac{9}{4} &&
	\end{alignat*}
	Anhand der beiden $x$-Werte kann nun der Wert der Zielfunktion berechnet werden.
	
	\[
		2 \cdot \frac{9}{4} + 5 \cdot \frac{15}{4} = \frac{18}{4} + \frac{75}{4} = \frac{93}{4} = 23,25
	\]
	Damit ist $\frac{93}{4}$ das optimale Ergebnis für die Zielfunktion $2x_{1} + 5x_{2}$ unter den gegebenen Nebenbedingungen.
\section{} %2
	\subsection{} %a
	Pauls Diätproblem:\\
	\begin{alignat*}{7}
		\text{maximiere}\; -& 3x_{1} &-& 24x_{2} &-& 13x_{3} &-& 9x_{4} &-& 20x_{5} &-& 19x_{6} && \\
		\multicolumn{14}{l}{\text{unter den Nebenbedingungen}} && \\
		- & 110x_{1} &-& 205x_{2} &-& 160x_{3} &-& 160x_{4} &-& 420x_{5} &-& 260x_{6} &\leq & -2000 \\
		-& 4x_{1} &-& 32x_{2} &-& 13x_{3} &-& 8x_{4} &-& 4x_{5} &-& 14x_{6} &\leq & -55 \\
		-& 2x_{1} &-& 12x_{2} &-& 54x_{3} &-& 285x_{4} &-& 22x_{5} &-& 80x_{6} &\leq & -800 \\
		& x_{1} && && && && && &\leq &\, 4 \\
		& && x_{2} && && && && &\leq &\, 3 \\
		& && && x_{3} && && && &\leq &\, 2 \\
		& && && && x_{4} && && &\leq &\, 8 \\
		& && && && && x_{5} && &\leq &\, 2 \\
		& && && && && && x_{6} &\leq &\, 2 \\
		\multicolumn{12}{r}{$x_{1}, x_{2}, x_{3}, x_{4}, x_{5}, x_{6}$} \,&\geq &\, 0
		\end{alignat*}
		Problem (1.2) in Standardform:\\
		\begin{alignat*}{7}
		\text{maximiere}\; & 3x_{1}^{'} &-& 3x_{1}^{''} &+& x_{2} && && && \\
		\multicolumn{14}{l}{\text{unter den Nebenbedingungen}} && \\
		& x_{1}^{'} &-& x_{1}^{''} &-& 6x_{2} &+& x_{3} &-& x_{4}^{'} &+& x_{4}^{''} &\leq &\, 3 \\
		& && && 7x_{2} && &-& 2x_{4}^{'} &+& 2x_{4}^{''} &\leq &\, 5 \\
		& && &-& 7x_{2} && &+& 2x_{4}^{'} &-& 2x_{4}^{''} &\leq & -5 \\
		-& x_{1}^{'} &+& x_{1}^{''} &+& x_{2} &+& x_{3} && && &\leq &\, 1 \\
		& x_{1}^{'} &-& x_{1}^{''} &-& x_{2} &-& x_{3} && && &\leq & -1 \\
		& && && && x_{3} &-& x_{4}^{'} &+& x_{4}^{''} &\leq &\, 2 \\
		\multicolumn{12}{r}{$x_{1}^{'}, x_{1}^{''}, x_{2}, x_{3}, x_{4}^{'}, x_{4}^{''}$} \,&\geq &\, 0
		\end{alignat*}
	\subsection{} %b
		\begin{alignat*}{7}
		\text{maximiere}\; & 13a_{1} &+& 8a_{2} &+& 15b_{1} &+& 12b_{2} &+& 14c_{1} &+& 10c_{2} &&  \\
		\multicolumn{14}{l}{\text{unter den Nebenbedingungen}} && \\
		& a_{1} &+& a_{2} && && && && &\leq &\, 400 \\
		& && && b_{1} &+& b_{2} && && &\leq &\, 480 \\
		& && && && && c_{1} &+& c_{2} &\leq &\, 230 \\
		& a_{1} && &+& b_{1} && &+& c_{1} && &\leq &\, 420 \\
		& && a_{2} && &+& b_{2} && &+& c_{2} &\leq &\, 250 \\
		\multicolumn{12}{r}{$a_{1}, a_{2}, b_{1}, b_{2}, c_{1}, c_{2}$} \,&\geq &\, 0
		\end{alignat*}
\end{document}
