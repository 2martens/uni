\documentclass{beamer}
\usepackage[T1]{fontenc}
\usepackage[utf8]{inputenc}
\usepackage[ngerman]{babel}
%\usepackage{paralist}
%\useoutertheme{infolines} 
\usepackage{graphicx}
\usepackage{hyperref}
\usepackage{listings}
\usepackage{color}
\usetheme{Warsaw}
\usecolortheme{crane}
\pagenumbering{arabic}
\def\thesection{\arabic{section})}
\def\thesubsection{\alph{subsection})}
\def\thesubsubsection{(\roman{subsubsection})}
\setbeamertemplate{navigation symbols}{}
\graphicspath{ {src/} {/home/jim/Pictures/} }

\definecolor{mygreen}{rgb}{0,0.6,0}
\definecolor{mygray}{rgb}{0.5,0.5,0.5}
\definecolor{mymauve}{rgb}{0.58,0,0.82}

\lstset{ %
  backgroundcolor=\color{white},   % choose the background color; you must add \usepackage{color} or \usepackage{xcolor}
  basicstyle=\footnotesize,        % the size of the fonts that are used for the code
  breakatwhitespace=false,         % sets if automatic breaks should only happen at whitespace
  breaklines=true,                 % sets automatic line breaking
  captionpos=b,                    % sets the caption-position to bottom
  commentstyle=\color{mygray},    % comment style
  deletekeywords={},            % if you want to delete keywords from the given language
  escapeinside={\%*}{*)},          % if you want to add LaTeX within your code
  extendedchars=true,              % lets you use non-ASCII characters; for 8-bits encodings only, does not work with UTF-8
  keepspaces=true,                 % keeps spaces in text, useful for keeping indentation of code (possibly needs columns=flexible)
  keywordstyle=\color{blue},       % keyword style
  language=PHP,                 % the language of the code
  morekeywords={class, function, return, protected, public, private, const, static, new, extends, namespace, null},            % if you want to add more keywords to the set
  numbers=left,                    % where to put the line-numbers; possible values are (none, left, right)
  numbersep=5pt,                   % how far the line-numbers are from the code
  numberstyle=\tiny\color{mygray}, % the style that is used for the line-numbers
  rulecolor=\color{black},         % if not set, the frame-color may be changed on line-breaks within not-black text (e.g. comments (green here))
  showspaces=false,                % show spaces everywhere adding particular underscores; it overrides 'showstringspaces'
  showstringspaces=false,          % underline spaces within strings only
  showtabs=false,                  % show tabs within strings adding particular underscores
  stepnumber=2,                    % the step between two line-numbers. If it's 1, each line will be numbered
  stringstyle=\color{mygreen},     % string literal style
  tabsize=2,                       % sets default tabsize to 2 spaces
  title=\lstname                   % show the filename of files included with \lstinputlisting; also try caption instead of title
}

\hypersetup{
	pdfauthor=Jim Martens,
	pdfstartview=Fit
}

\expandafter\def\expandafter\insertshorttitle\expandafter{%
	\raggedleft \insertframenumber\,/\,\inserttotalframenumber\;}

\begin{document}
\author{Jim 2martens}
\title{Warum Informatik studieren?}
\date{\today}

\begin{frame}
    \titlepage\end{frame}

\begin{frame}
    \tableofcontents
\end{frame}

\section{Introduction}
\begin{frame}{About me}
    \begin{itemize}
      \item Abi 2012 am Gymnasium Ohmoor
      \item PGW-Profil
      \item Deutsch, Mathe, PGW und Informatik als Abifächer
      \item Informatik als PP
      \item Thema: "`Roboter für Senioren – sozialer Gewinn oder
verantwortungsloser Leichtsinn"'
      \item Studienbeginn 2012
      \item Studiengang: BSc. Informatik
      \item aktuelles Semester: 6
    \end{itemize}
\end{frame}

\begin{frame}{Formalia}
  \begin{itemize}
    \item Anmeldephase Juni bis Juli (zu passender Zeit auf Uniseite nachsehen)
    \item NC wird im Nachhinein festgelegt
    \item beim letzten Mal (2014) <insert NC here>
    \item weitere Infos siehe Uniseiten
  \end{itemize}
\end{frame}

\begin{frame}{Schocktherapie}
  \centering
  MAATHEEE!!!
\end{frame}
\begin{frame}{Mathematik}
  \begin{itemize}
    \item es gibt \textbf{ein} verpflichtendes Mathemodul
    \item Stoff baut teilweise auf Schulstoff auf (Analysis und Lineare Algebra)
    \item zum Teil mit weniger Umfang als im erhöhtem Niveau in der Oberstufe (Matrizen und Vektoren)
    \item sehr guter Dozent
    \item absolut \textbf{KEIN} Grund sich nicht zu bewerben, wenn einen Informatik sonst interessiert
  \end{itemize}
\end{frame}

\section{Informelles}
\begin{frame}{Fachschaft}
  \begin{itemize}
    \item nett
    \item hilfsbereit
    \item for i=0;i<100;i++: füge freundliche Attribute ein
    \item Kurzum: Super Fachschaft, wohl die beste in ganz Deutschland
    \item MafiA: Menge aller fachschaftsinteressierten Aktivisten
    \item alle fachschaftsaktiven Personen quasi wie erweiterte Familie
  \end{itemize}
\end{frame}
\begin{frame}{(Arbeits-)klima am Fachbereich}
  \begin{itemize}
    \item gutes Verhältnis zu Profs und WiMis
    \item super geiles Studienbüro
    \item viele Privilegien für Studis
    \begin{itemize}
      \item Zugang zu Fachschaftsräumen und Gelände 16/7
      \item unbeschränkter, unzensierter Internetzugang (vorbehaltlich gesetzlicher Regelungen)
      \item Grillpartys und Weihnachts"`seminar"'
      \item Getränkeverkauf im Fachschaftscafé
    \end{itemize}
    \item vernünftiges Verhalten vorausgesetzt
    \begin{itemize}
      \item Geschirr abwaschen
      \item Müll raustragen
      \item Don't be an asshole
    \end{itemize}
  \end{itemize}
\end{frame}
\begin{frame}{Fachschaftsaktivitäten}
  \begin{itemize}
    \item KunterBuntesSeminar
    \item Arbeitsgruppen
    \item Fazit: von Studierenden, mit Studierenden, für Studierende
  \end{itemize}
\end{frame}

\section{Informatik und Gesellschaft}
\subsection{31C3}
\begin{frame}{31C3}
  \begin{itemize}
    \item Hacker- und Nerdkultur
    \item letzten Dezember im CCH
    \item veranstaltet vom CCC Hamburg
    \item vier Tage lang geile Vorträge
    \item viele Informatiker gehen hin
  \end{itemize}
\end{frame}
\begin{frame}{31C3}% TODO: Foto von Appelbaum und Poitras einfügen
  \begin{columns}[T]
    \begin{column}{0.48\textwidth}
      \begin{itemize}
        \item live dabei sein, wenn Weltgeschichte geschrieben wird
        \item Enthüllungen im Geheimdienstskandal - ausgelöst durch Edward Snowden
        \item Screening von Citizenfour - Oscar for Best Documentary 2014
      \end{itemize}
    \end{column}
    \hfill
    \begin{column}{0.5\textwidth}
      \centering
      \includegraphics[scale=0.25]{Studium/31c3/appelbaum-poitras-640_360}

      Quelle: \url{https://media.ccc.de/browse/congress/2014/} \\
      Talk: Reconstructing narratives
    \end{column}
  \end{columns}  
\end{frame}
\subsection{Datenschutz \& Überwachung}
\begin{frame}{Datenschutz und Überwachung}
  \begin{itemize}
    \item ein Kernkonflikt des 21. Jahrhunderts
    \item Informatiker auf beiden Seiten des Konfliktes
    \item große Verantwortung der Informatik
    \item Einfluss auf Leben von Milliarden von Menschen
    \item jeder kann sich für digitale Bürgerrechte einsetzen
  \end{itemize}
\end{frame}
\begin{frame}{Datenschutz und Überwachung}
  \begin{columns}[T]
    \begin{column}{0.48\textwidth}
      \begin{itemize}
        \item "`Volkszählungsurteil"' von 1983
        \item Recht auf informationelle Selbstbestimmung
        \item Prof. Dr. Klaus Brunnstein war Mitkläger sowie Mitgründer der Hamburger Informatik
        \item hielt bis zuletzt Verantwortungsvortrag in unserer OE
      \end{itemize}
    \end{column}
    \hfill
    \begin{column}{0.5\textwidth}
      \centering
      \includegraphics[scale=1.25]{Studium/OE/brunnstein-scaled}
    \end{column}
  \end{columns}
\end{frame}
\begin{frame}{Freiheit statt Angst}
  \begin{columns}[T]
    \begin{column}{.48\textwidth}
      \begin{itemize}
        % Bild von Demo einfügen
        \item jährliche Demos gegen Massenüberwachung und für Datenschutz
        \item FSR Informatik Teil des Hamburger Bündnisses gegen Überwachung
        \item viele Informatiker gehen hin
        \item sei DU einer von ihnen
      \end{itemize}
    \end{column}
    \hfill
    \begin{column}{0.5\textwidth}
      \centering
      \includegraphics[scale=1.25]{Studium/Freiheit-statt-Angst-2015/freiheit-statt-angst-pic}
    \end{column}
  \end{columns}
\end{frame}
\begin{frame}{Andere relevante Themen}
  \begin{itemize}
    \item Künstliche Intelligenz - Wieviel wollen wir?
    \item Roboter/Drohnen im Militäreinsatz vs. Friedensforschung - In welche Richtung wollen wir?
    \item Rationalisierung durch Automation - Fortschritt um jeden Preis?
    \item Hochfrequenzhandel an Börsen - Wollen wir die Herrschaft des Geldes über die Vernunft?
    \item Dauererreichbarkeit durch Smartphones - Wieviel Burnout ist OK?
  \end{itemize}
\end{frame}
\section{Abschluss}
\begin{frame}{Fazit}
  Informatik...
  \begin{itemize}
    \item ...ist unglaublich spannend
    \item ...macht sehr viel Spaß
    \item ...wird dringend benötigt
  \end{itemize}
\end{frame}
\begin{frame}{Fragen}
  \centering
  Q \& A
\end{frame}

\end{document}

