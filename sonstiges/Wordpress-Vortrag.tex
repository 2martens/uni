\documentclass{beamer}
\usepackage[T1]{fontenc}
\usepackage[utf8]{inputenc}
\usepackage[ngerman]{babel}
%\usepackage{paralist}
\useoutertheme{infolines} 
\usepackage{graphicx}
\usepackage{hyperref}
\usepackage{listings}
\usetheme{Berkeley}
\pagenumbering{arabic}
\def\thesection{\arabic{section})}
\def\thesubsection{\alph{subsection})}
\def\thesubsubsection{(\roman{subsubsection})}
\setbeamertemplate{navigation symbols}{}
\graphicspath{ {src/} }
\lstset{language=PHP}

\begin{document}
\author{Jim Martens}
\title{Warum sich ein eigenes CMS lohnt}

\section{Situation}
\begin{frame}{Situation}
	\begin{itemize}
		\item	Wordpress erhielt 2009 den Overall Best Open Source CMS Award
		\item	2010 folgte Hall of Fame CMS
		\item	gilt als eines der beliebtesten CMS weltweit
	\end{itemize}
\end{frame}

\begin{frame}
	\centering
	Warum man dennoch ein eigenes CMS schreiben sollte?
\end{frame}

\begin{frame}[fragile]{Darum}
	aus wp-login.php Zeile 207 ff. (etwas HTML ausgelassen, gekennzeichnet durch Kommentar)
	\begin{lstlisting}
		function login_footer($input_id = '') {
		    global $interim_login;
		    if ( ! $interim_login ): ?>
			    <p id="backtoblog"><!-- something --></p>
		    <?php endif; ?>
	\end{lstlisting}
\end{frame}

\begin{frame}
	\centering	
	Warum ist der gezeigte Code nicht so toll?
\end{frame}

\begin{frame}{Deswegen}
	\begin{itemize}
		\item	\texttt{global} is Bad Practice
		\begin{itemize}		
			\item	stammt aus PHP 4 Zeiten
			\item	nicht objektorientiert
		\end{itemize}
		\item	keine Trennung zwischen Businesslogik und Ausgabe
	\end{itemize}
\end{frame}

\section{Verbesserung}
\begin{frame}
	\centering
	Wie kann man es besser machen?
\end{frame}	

\begin{frame}{So kann man es besser machen}
	\begin{itemize}
		\item	OOP-Framework wie Symfony 2 nutzen
		\begin{itemize}
			\item[$\rightarrow$] klare Trennung zwischen Businesslogik und Ausgabe
			\item[$\rightarrow$] Object-Relational Mapping (Doctrine)
			\item[$\rightarrow$] Templateengine (Twig)	
			\item[$\rightarrow$] Dependency Management (Composer)
			\item[$\rightarrow$] einfache Konfigurierbarkeit
			\item[$\rightarrow$] Unterstützung für Unit- und Integrationstests (PHPUnit)
			\item[$\rightarrow$] hohe Code Coverage der Unittests
		\end{itemize}
	\end{itemize}
\end{frame}

\section{Call for Action}
\begin{frame}
	\centering
	Warum erzähle ich euch das eigentlich alles?
\end{frame}

\begin{frame}{Planung}
	\begin{itemize}
		\item	habe bereits ein CMS auf Basis des WoltLab Community Framework (WCF) entwickelt
		\item	das wird vsl. Januar/Februar RC-Status erreichen (Betaseite: beta.plugins-zum-selberbauen.de)
		\item	nächstes Projekt: WCF-Funktionen mit Symfony mergen
		\begin{itemize}
			\item	ACP, Paketsystem, einfache Installation
			\item	Mehrsprachigkeit, User und Benutzergruppen
			\item	flexibles Stilsystem, Event Listener
		\end{itemize}
		\item	und dann, ja dann: CMS mit Symfony und WCF-Erweiterungen
	\end{itemize}
\end{frame}

\begin{frame}
	\centering
	Warum interessiert euch das?
\end{frame}

\begin{frame}{Anforderungsprofil}
	\begin{itemize}
		\item	praktische Erfahrung sammeln mit OOP-Entwicklung
		\item	Voraussetzungen: PHP-Kenntnis (>= 5.3), OOP-Kenntnis, Bereitschaft sich einzuarbeiten, Coding-Standards einhalten
		\item	Git als VCS (github.com/frmwrk123/symfony-wcf)
		\item	Lizenz: LGPL
		\item	keine Zeitvorgabe - it's done when it's done
	\end{itemize}
\end{frame}

\section{Fertig}
\begin{frame}{Ich habe fertig}
	\centering
	Vielen Dank für Eure Aufmerksamkeit!
\end{frame}


\section{Quellen}
\begin{frame}{Quellen}
	\begin{itemize}	
		\item	http://webdesign.about.com/b/2010/11/22/wordpress-wins-best-cms-award.html
		\item	http://wordpress.org/news/2009/11/wordpress-wins-cms-award/
		\item	http://symfony.com/doc/master
	\end{itemize}
\end{frame}

\end{document}
