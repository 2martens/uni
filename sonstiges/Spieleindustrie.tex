\documentclass{beamer}
\usepackage[T1]{fontenc}
\usepackage[utf8]{inputenc}
\usepackage[ngerman]{babel}
%\usepackage{paralist}
\useoutertheme{infolines} 
\usepackage{graphicx}
\usepackage{hyperref}
\usepackage{listings}
\usetheme{Berkeley}
\pagenumbering{arabic}
\def\thesection{\arabic{section})}
\def\thesubsection{\alph{subsection})}
\def\thesubsubsection{(\roman{subsubsection})}
\setbeamertemplate{navigation symbols}{}
\graphicspath{ {src/} }
\lstset{language=PHP}

\begin{document}
\author{Jim Martens}
\title{Verwandlung der Spieleindustrie}

\section{Wie es war}
\begin{frame}{Wie es war}
	\begin{itemize}
		\item	tolle Spiele(reihen), die teilweise noch immer gespielt werden
		\begin{itemize}
			\item<2->	Wing Commander
			\item<2->	Privateer
			\item<2->	Freelancer
			\item<3->	Independence War 2
			\item<4->	Ultima
			\item<4->	Baldurs Gate
			\item<4->	Neverwinter Nights
			\item<4->	Knights of the Old Republic
			\item<5->	Gothic (1-3)
			\item<5->	Morrowind, Oblivion, Nehrim (Total Conversion von Oblivion, fanmade)
		\end{itemize}
		\item<6->	viele kleine Studios mit motivierten Entwicklern (meistens selber leidenschaftliche Spieler)
	\end{itemize}
\end{frame}

\section{Wie es ist}
\begin{frame}{Wie es ist}
	\begin{itemize}
		\item	wenige große Studios mit vielen Entwicklern, für die es ein ganz normaler Beruf ist
		\item<2->	übermächtige Publisher (EA, Ubisoft, Microsoft, ...), für die allein der Profit zählt
		\item<3->	Spiele werden zu Gewinnbringern, der Inhalt gerät in den Hintergrund
		\item<4->	Release-Patches, Käufer sind eigentliche Betatester, vernachlässigte Qualitätskontrolle
		\item<5->	kurze Entwicklungszeiten
	\end{itemize}
\end{frame}

\section{Wie es sein sollte}
\begin{frame}{Wie es sein sollte (allgemein)}
	\begin{itemize}
		\item	ausreichende Betatests
		\item<2->	Entwickler, die ihr ganzes Herzblut in ein Spiel hineinstecken
		\item<3->	den Spielinhalt in den Mittelpunkt rücken, Gewinn ist Nebenprodukt
		\item<4->	keine Vercasualisierung
	\end{itemize}
\end{frame}

\begin{frame}{Wie es sein sollte (für Spiele mit Story)}
	\begin{itemize}
		\item	vor Erschaffung eines Spieles sollte ein glaubwürdiges Universum geschaffen werden
		\item<2->	dieses Universum enthält unterschiedliche Völker und deren Historie
		\item<3->	erst nach Schaffung eines Universums, das für sich alleine existieren kann, wird mit der Planung einer Spielereihe begonnen
		\item<4->	die Spielereihe fügt sich in das Universum ein, nicht umgekehrt
		\item<5->	Spiele dürfen keine Storys haben, die das Universum an sich beschädigen
		\item<6->	Ergebnis: ein Universum, dass nach einer Spielereihe nicht ausgelaugt ist
		\item<7->	Nachteil: es erfordert eine Menge an initialer Arbeit
		\item<7->	Vorteil: großes Potenzial unterschiedlichster Storys
	\end{itemize}
\end{frame}

\begin{frame}{Schlussfolgerungen}
	\begin{itemize}
		\item	Jene Spiele unterstützen, die eine reichhaltige Lore aufweisen
		\item	Jene Spiele unterstützen, bei denen der Inhalt im Mittelpunkt steht
		\item	Jene Spiele nicht unterstützen, die reine Abzocke sind
		\item	Jene Spiele nicht unterstützen, die den Spieler zu einem Konsumenten degradieren
	\end{itemize}
\end{frame}

\end{document}