\documentclass{beamer}
\usepackage[T1]{fontenc}
\usepackage[utf8]{inputenc}
\usepackage[ngerman]{babel}
%\usepackage{paralist}
\useoutertheme{infolines} 
\usepackage{graphicx}
\usepackage{hyperref}
%\usepackage{listings}
%\usetheme{Berkeley}
\pagenumbering{arabic}
\def\thesection{\arabic{section})}
\def\thesubsection{\alph{subsection})}
\def\thesubsubsection{(\roman{subsubsection})}
\setbeamertemplate{navigation symbols}{}
\graphicspath{ {src/} }
%\lstset{language=PHP}
\setbeamertemplate{frametitle}[default][center]

\begin{document}
\author{Jim Martens}
\title{Menschliche Rasse(n) -- politisch oder biologisch?}

	% Introduction
	{
	\setbeamertemplate{footline}{} 
	\begin{frame}
		\titlepage
	\end{frame}
	}
	\addtocounter{framenumber}{-1}
	
	\begin{frame}{Disclaimer}
		\begin{quotation}
			Dieser Vortrag enthält persönliche Meinung, unwissenschaftliche Quellen, sowie potentiell Ironie. Einige Personen reagieren allergisch darauf. Wer einen allergischen Anfall erwartet sollte nicht weiter zuhören.
		\end{quotation}
	
	\end{frame}
	
	%\begin{frame}{Agenda}
	%	\tableofcontents
	%\end{frame}	
	
	\section{Motivation}
	% Motivation
	\begin{frame}{Rechtliches}
		Artikel 3 Absatz 3 GG (verfasst 1948):
		\begin{quotation}
		"`Niemand darf wegen seines Geschlechts, seiner Abstammung, seiner Rasse, seiner Sprache, seiner Heimat und Herkunft, seines Glaubens, seiner religiösen oder politischen Anschauungen benachteiligt oder bevorzugt werden. Niemand darf wegen seiner Behinderung benachteiligt werden."'
		\end{quotation}
		
		15th Amendment to US constitution, first paragraph (ratified 1870):
		\begin{quotation}
		``The right of citizens of the United States to vote shall not be denied or abridged by the United States or by any State on account of race, color, or previous condition of servitude.''
		\end{quotation}
	\end{frame}
	
	\begin{frame}{Definitions}
		Oxford Dictionary:
		\begin{quotation}
			``The fact or condition of belonging to a racial division or group; the qualities or characteristics associated with this.''
		\end{quotation}
		\begin{quotation}
			``A group of people sharing the same culture, history, language, etc.; an ethnic group.''
		\end{quotation}
		\begin{quotation}
			``A group or set of people or things with a common feature or features.''
		\end{quotation}
		\begin{quotation}
			``A population within a species that is distinct in some way, especially a subspecies.''
		\end{quotation}
	\end{frame}
	
	\section{Menschen}
	\begin{frame}{Timeline of Humanity}
		\begin{itemize}
			\item Homo erectus/Homo ergaster
			\item Homo habilis
			\item Homo antecessor
			\item Homo heidelbergensis
			\item Homo sapiens
			\item Y-chromosomal Adam			
			\item Mitochondrial Eve
			\item Homo sapiens sapiens
		\end{itemize}
	\end{frame}
	
	\begin{frame}{Andere ausgestorbene "`Homo"'s}
		\begin{itemize}
			\item Homo floresiensis (Nickname: "`Hobbits"'\footnote{No copyright infringement intended})
			\begin{itemize}
				\item ausgestorben vor ca. 12000 Jahren
			\end{itemize}			
			\item Homo neanderthalensis
			\item Homo rhodesiensis
			\item Homo sapiens idaltu
			\item Homo gautengensis
			\item Denisova Hominin
			\item Red Deer Cave people
		\end{itemize}
	\end{frame}
	
	\begin{frame}{Wir}
		\begin{itemize}
			\item einzige überlebende Vertreter des Genus Homo
			\item alle heutigen Menschen sind "`Homo sapiens sapiens"'
			\item Subspezies von "`Homo sapiens"'
		\end{itemize}
	\end{frame}
	
	\section{Quellen}
	\begin{frame}{Quellen}
		\begin{itemize}
			\item US Constitution: http://www.usconstitution.net/xconst\_Am15.html
			\item Oxford Dictionary: http://www.oxforddictionaries.com/definition/english/race\#race-2
			\item Wikipedia: https://en.wikipedia.org/wiki/Homo, https://en.wikipedia.org/wiki/Human, https://en.wikipedia.org/wiki/Homo\_floresiensis, https://en.wikipedia.org/wiki/Neanderthal, https://en.wikipedia.org/wiki/Timeline\_of\_human\_evolution, https://en.wikipedia.org/wiki/Y-chromosomal\_Adam, https://en.wikipedia.org/wiki/Homo\_sapiens\_idaltu
		\end{itemize}
	\end{frame}
\end{document}

