\documentclass{beamer}
\usepackage[T1]{fontenc}
\usepackage[utf8]{inputenc}
\usepackage[ngerman]{babel}
%\usepackage{paralist}
%\useoutertheme{infolines}
\usepackage{graphicx}
\usepackage{hyperref}
\usepackage{listings}
\usepackage{color}
\usepackage{textcomp}
\usepackage[german=quotes]{csquotes}
\usetheme{Warsaw}
\usecolortheme{crane}
\pagenumbering{arabic}
\def\thesection{\arabic{section})}
\def\thesubsection{\alph{subsection})}
\def\thesubsubsection{(\roman{subsubsection})}
\setbeamertemplate{navigation symbols}{}
\graphicspath{ {src/} {/home/jim/Pictures/} }

\definecolor{mygreen}{rgb}{0,0.6,0}
\definecolor{mygray}{rgb}{0.5,0.5,0.5}
\definecolor{mymauve}{rgb}{0.58,0,0.82}

\MakeOuterQuote{"}

%\definecolor{craneorange}{RGB}{61,61,61}
%\definecolor{craneblue}{RGB}{255,255,255}

\lstset{ %
  backgroundcolor=\color{white},   % choose the background color; you must add \usepackage{color} or \usepackage{xcolor}
  basicstyle=\footnotesize,        % the size of the fonts that are used for the code
  breakatwhitespace=false,         % sets if automatic breaks should only happen at whitespace
  breaklines=true,                 % sets automatic line breaking
  captionpos=b,                    % sets the caption-position to bottom
  commentstyle=\color{mygray},    % comment style
  deletekeywords={},            % if you want to delete keywords from the given language
  escapeinside={\%*}{*)},          % if you want to add LaTeX within your code
  extendedchars=true,              % lets you use non-ASCII characters; for 8-bits encodings only, does not work with UTF-8
  keepspaces=true,                 % keeps spaces in text, useful for keeping indentation of code (possibly needs columns=flexible)
  keywordstyle=\color{blue},       % keyword style
  language=PHP,                 % the language of the code
  morekeywords={class, function, return, protected, public, private, const, static, new, extends, namespace, null},            % if you want to add more keywords to the set
  numbers=left,                    % where to put the line-numbers; possible values are (none, left, right)
  numbersep=5pt,                   % how far the line-numbers are from the code
  numberstyle=\tiny\color{mygray}, % the style that is used for the line-numbers
  rulecolor=\color{black},         % if not set, the frame-color may be changed on line-breaks within not-black text (e.g. comments (green here))
  showspaces=false,                % show spaces everywhere adding particular underscores; it overrides 'showstringspaces'
  showstringspaces=false,          % underline spaces within strings only
  showtabs=false,                  % show tabs within strings adding particular underscores
  stepnumber=2,                    % the step between two line-numbers. If it's 1, each line will be numbered
  stringstyle=\color{mygreen},     % string literal style
  tabsize=2,                       % sets default tabsize to 2 spaces
  title=\lstname                   % show the filename of files included with \lstinputlisting; also try caption instead of title
}

\hypersetup{
	pdfauthor=Jim Martens,
	pdfstartview=Fit
}

\expandafter\def\expandafter\insertshorttitle\expandafter{%
	\raggedleft \insertframenumber\,/\,\inserttotalframenumber\;}

\begin{document}
\author{Jim 2martens}
\title{Umgang mit Daten}
\date{\today}

\begin{frame}
    \titlepage
\end{frame}

\begin{frame}
    \tableofcontents
\end{frame}

\section{Umgang mit Daten}
\begin{frame}{Drei Perspektiven}
    \begin{itemize}
        \item Anbieterseite
        \vfill
        \item Staat
        \vfill
        \item Anwender*innen-Seite
    \end{itemize}
\end{frame}

\begin{frame}{Anbieterseite}
    \begin{itemize}
        \item Prinzipien des Datenschutzes verinnerlichen
        \vfill
        \item keine unnötigen Daten sammeln (Datensparsamkeit)
        \vfill
        \item Erhebung von unnötigen Daten nicht zur Voraussetzung für Nutzung
              machen
        \vfill
        \item Daten gehören den Personen
        \vfill
        \item bei Nutzung von Daten für Werbezwecke: Beteiligung der Personen
              an Werbeeinnahmen
        \vfill
        \item alternativ: Nutzer*innen wählen lassen zwischen bezahlen mit Geld
              oder mit Daten (sorgt für höheres Bewusstsein, dass Daten
              Zahlungsmittel sind)
        \vfill
        \item Website nur auf HTTPS anbieten (z.B. via Lets Encrypt)
    \end{itemize}
\end{frame}

\begin{frame}{Staat}
    \begin{itemize}
        \item Verpflichtung zu Interoperabilität gesetzlich festhalten
        \vfill
        \item u.a. auch für soziale Netzwerke
        \vfill
        \item ermöglicht fairen Wettbewerb durch Features
        \vfill
        \item verhindert Vendor Lock-in und damit Monopolbildung durch Datensilos
        \vfill
        \item bspw. könnten Nutzer zwischen Facebook und Diaspora kommunizieren
        \vfill
        \item nur Verwendung von offenen Dateiformaten in staatlichen
              Einrichtungen (z.B. an Schulen, die keine Microsoft-Werbepartner
              sein sollten)
    \end{itemize}
\end{frame}

\begin{frame}{Anwender*innen}
    \begin{itemize}
        \item Privatsphäre- und Datenschutzeinstellungen NUTZEN!
        \vfill
        \item Ende-zu-Ende-Verschlüsselung bei Emails und Messengern benutzen
        \vfill
        \item HTTPS Everywhere von EFF benutzen
        \vfill
        \item auf sichere Messenger wie Signal oder Threema umsteigen
        \vfill
        \item von Windows auf Linux umsteigen
    \end{itemize}
\end{frame}

\begin{frame}{Anwender*innen}
    \begin{itemize}
        \item keine Daten ins Internet stellen, die nicht öffentlich sind
        \vfill
        \item Freie Software nutzen, wo immer möglich (z.B. LibreOffice statt
              MS Office)
        \vfill
        \item offene Dateiformate nutzen
    \end{itemize}
\end{frame}

\section{Eigenschaften und Fähigkeiten}

\begin{frame}{Eigenschaften}
    \begin{itemize}
        \item Lernbereitschaft
        \vfill
        \item Neugierde
        \vfill
        \item Affinität zu Technologie
        \vfill
        \item Logisches Denkmervögen
        \vfill
        \item Ethikverständnis
        \vfill
        \item Kreativität
    \end{itemize}
\end{frame}

\begin{frame}{Fähigkeiten}
    \begin{itemize}
        \item selbstständig sich eine Anwendung aneignen können
        \vfill
        \item hängt darüber hinaus stark von dem Job ab
    \end{itemize}
\end{frame}

\begin{frame}{Abschlussthesen}
    \begin{itemize}
        \item Digitalisierung bietet eine Chance, wenn Menschen nicht länger als
              auszubeutende Ressourcen oder "Kunden" angesehen werden, deren
              Daten einfach genutzt werden können.
        \vfill
        \item Soziale Netzwerke müssen unter Einhaltung eines strengen
              Datenschutzes benutzt werden können.
    \end{itemize}
\end{frame}
\end{document}
