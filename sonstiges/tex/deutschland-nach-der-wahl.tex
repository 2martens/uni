\documentclass{beamer}
\usepackage[T1]{fontenc}
\usepackage[utf8]{inputenc}
\usepackage[ngerman]{babel}
%\usepackage{paralist}
%\useoutertheme{infolines}
\usepackage{graphicx}
\usepackage{hyperref}
\usepackage{listings}
\usepackage{color}
\usepackage{textcomp}
\usepackage[german=quotes]{csquotes}
\usetheme{Warsaw}
\usecolortheme{crane}
\pagenumbering{arabic}
\def\thesection{\arabic{section})}
\def\thesubsection{\alph{subsection})}
\def\thesubsubsection{(\roman{subsubsection})}
\setbeamertemplate{navigation symbols}{}
\graphicspath{ {src/} {/home/jim/Pictures/} }

\definecolor{mygreen}{rgb}{0,0.6,0}
\definecolor{mygray}{rgb}{0.5,0.5,0.5}
\definecolor{mymauve}{rgb}{0.58,0,0.82}

\MakeOuterQuote{"}

%\definecolor{craneorange}{RGB}{61,61,61}
%\definecolor{craneblue}{RGB}{255,255,255}

\lstset{ %
  backgroundcolor=\color{white},   % choose the background color; you must add \usepackage{color} or \usepackage{xcolor}
  basicstyle=\footnotesize,        % the size of the fonts that are used for the code
  breakatwhitespace=false,         % sets if automatic breaks should only happen at whitespace
  breaklines=true,                 % sets automatic line breaking
  captionpos=b,                    % sets the caption-position to bottom
  commentstyle=\color{mygray},    % comment style
  deletekeywords={},            % if you want to delete keywords from the given language
  escapeinside={\%*}{*)},          % if you want to add LaTeX within your code
  extendedchars=true,              % lets you use non-ASCII characters; for 8-bits encodings only, does not work with UTF-8
  keepspaces=true,                 % keeps spaces in text, useful for keeping indentation of code (possibly needs columns=flexible)
  keywordstyle=\color{blue},       % keyword style
  language=PHP,                 % the language of the code
  morekeywords={class, function, return, protected, public, private, const, static, new, extends, namespace, null},            % if you want to add more keywords to the set
  numbers=left,                    % where to put the line-numbers; possible values are (none, left, right)
  numbersep=5pt,                   % how far the line-numbers are from the code
  numberstyle=\tiny\color{mygray}, % the style that is used for the line-numbers
  rulecolor=\color{black},         % if not set, the frame-color may be changed on line-breaks within not-black text (e.g. comments (green here))
  showspaces=false,                % show spaces everywhere adding particular underscores; it overrides 'showstringspaces'
  showstringspaces=false,          % underline spaces within strings only
  showtabs=false,                  % show tabs within strings adding particular underscores
  stepnumber=2,                    % the step between two line-numbers. If it's 1, each line will be numbered
  stringstyle=\color{mygreen},     % string literal style
  tabsize=2,                       % sets default tabsize to 2 spaces
  title=\lstname                   % show the filename of files included with \lstinputlisting; also try caption instead of title
}

\hypersetup{
	pdfauthor=Jim Martens,
	pdfstartview=Fit
}

\expandafter\def\expandafter\insertshorttitle\expandafter{%
	\raggedleft \insertframenumber\,/\,\inserttotalframenumber\;}

\begin{document}
\author{Jim 2martens}
\title{Deutschland nach der Wahl}
\date{\today}

\begin{frame}
    \titlepage
\end{frame}

\begin{frame}{Ergebnisse}
    \begin{itemize}
        \item GRÜNE haben 0,5\% gewonnen, 8,9\% insgesamt
        \vfill
        \item SPD mit historisch schlechtem Ergebnis und 20,5\%
        \vfill
        \item Die LINKE auf 9,2\%
        \vfill
        \item CDU weiterhin stärkste Kraft mit 32,9\%
        \vfill
        \item FDP wieder drin mit 10,7\%
        \vfill
        \item AfD drittstärkste Kraft mit 12,6\%
    \end{itemize}
\end{frame}

\begin{frame}{"Trump"-Effekt 2.0}
    \begin{itemize}
        \item über 3800 neue SPD-Mitglieder
        \vfill
        \item über 1100 neue GRÜNE-Mitglieder
        \vfill
        \item ca. 2000 neue FDP-Mitglieder
        \vfill
        \item über 1500 neue Linke-Mitglieder
        \vfill
        \item kein überproportionaler Anstieg an Mitgliedern nach Wahl bei der CDU
    \end{itemize}
\end{frame}

\begin{frame}{Engagement}
    \begin{itemize}
        \item bspw. mittwochs, 18.30 Uhr: Aktiventreffen der GRÜNEN JUGEND Hamburg
        in der Landesgeschäftsstelle der GRÜNEN (Burchardstraße 21, Nähe Gerhart-Hauptmann-Platz)
        \vfill
        \item macht Spaß
        \vfill
        \item ihr lernt viele nützliche Dinge
        \vfill
        \item ihr könnt die Welt verändern
    \end{itemize}
\end{frame}

\begin{frame}{Call to Action}
    \centering
    Werdet aktiv! Nicht meckern, machen!
\end{frame}
\end{document}
