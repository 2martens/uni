\documentclass{beamer}
\usepackage[T1]{fontenc}
\usepackage[utf8]{inputenc}
\usepackage[ngerman]{babel}
%\usepackage{paralist}
%\useoutertheme{infolines} 
\usepackage{graphicx}
\usepackage{hyperref}
\usepackage{listings}
\usepackage{color}
\usetheme{Warsaw}
\usecolortheme{crane}
\pagenumbering{arabic}
\def\thesection{\arabic{section})}
\def\thesubsection{\alph{subsection})}
\def\thesubsubsection{(\roman{subsubsection})}
\setbeamertemplate{navigation symbols}{}
\graphicspath{ {src/} {/home/jim/Pictures/} }

\definecolor{mygreen}{rgb}{0,0.6,0}
\definecolor{mygray}{rgb}{0.5,0.5,0.5}
\definecolor{mymauve}{rgb}{0.58,0,0.82}

%\definecolor{craneorange}{RGB}{61,61,61}
%\definecolor{craneblue}{RGB}{255,255,255}

\lstset{ %
  backgroundcolor=\color{white},   % choose the background color; you must add \usepackage{color} or \usepackage{xcolor}
  basicstyle=\footnotesize,        % the size of the fonts that are used for the code
  breakatwhitespace=false,         % sets if automatic breaks should only happen at whitespace
  breaklines=true,                 % sets automatic line breaking
  captionpos=b,                    % sets the caption-position to bottom
  commentstyle=\color{mygray},    % comment style
  deletekeywords={},            % if you want to delete keywords from the given language
  escapeinside={\%*}{*)},          % if you want to add LaTeX within your code
  extendedchars=true,              % lets you use non-ASCII characters; for 8-bits encodings only, does not work with UTF-8
  keepspaces=true,                 % keeps spaces in text, useful for keeping indentation of code (possibly needs columns=flexible)
  keywordstyle=\color{blue},       % keyword style
  language=PHP,                 % the language of the code
  morekeywords={class, function, return, protected, public, private, const, static, new, extends, namespace, null},            % if you want to add more keywords to the set
  numbers=left,                    % where to put the line-numbers; possible values are (none, left, right)
  numbersep=5pt,                   % how far the line-numbers are from the code
  numberstyle=\tiny\color{mygray}, % the style that is used for the line-numbers
  rulecolor=\color{black},         % if not set, the frame-color may be changed on line-breaks within not-black text (e.g. comments (green here))
  showspaces=false,                % show spaces everywhere adding particular underscores; it overrides 'showstringspaces'
  showstringspaces=false,          % underline spaces within strings only
  showtabs=false,                  % show tabs within strings adding particular underscores
  stepnumber=2,                    % the step between two line-numbers. If it's 1, each line will be numbered
  stringstyle=\color{mygreen},     % string literal style
  tabsize=2,                       % sets default tabsize to 2 spaces
  title=\lstname                   % show the filename of files included with \lstinputlisting; also try caption instead of title
}

\hypersetup{
    pdfauthor=Jim Martens,
    pdfstartview=Fit
}

\expandafter\def\expandafter\insertshorttitle\expandafter{%
    \raggedleft \insertframenumber\,/\,\inserttotalframenumber\;}

\begin{document}
\author{Jim 2martens}
\title{Wie finde ich ein Vortragsthema?}
\date{\today}

\begin{frame}
    \titlepage
\end{frame}

\begin{frame}{Motivation}
    \centering
    Warum ein Metatalk über das Finden eines Vortragsthemas?
\end{frame}

\begin{frame}{Motivation}
    Viele Personen haben keine Ahnung...
    \begin{itemize}
        \item ...über welches Thema sie Vortrag halten sollen
        \item ...ob Thema überhaupt für Publikum interessant ist
        \item ...wie Vortrag gestaltet sein sollte
    \end{itemize}
\end{frame}

\begin{frame}{Agenda} 
    \tableofcontents
\end{frame}

\begin{frame}{Organisatorisches}
    \begin{itemize}
        \item Folieninhalte für den theoretischen Teil und zum Nachlesen
        \item praktischer Teil zum Üben
        \item Vortrag beschäftigt sich NICHT mit wissenschaftlichen Vorträgen
    \end{itemize}
\end{frame}

\section{Themenfindung}
    \subsection{Brainstorming}

    \begin{frame}{Brainstorming} 
        \begin{itemize}
            \item alle (wirklich alle) Begriffe aufschreiben, die einem einfallen
            \item noch keine Bewertung vornehmen
        \end{itemize}
    \end{frame}

    \begin{frame}{Brainstorming}
        \centering
        Für 5 Minuten Begriffe aufschreiben (Zettel Papier oder Computer)
    \end{frame}

    \subsection{Gruppierung}

    \begin{frame}{Gruppierung}
        \begin{itemize}
            \item Gruppieren
            \item Kategorisieren
            \item Kurzum: Überblick verschaffen
        \end{itemize}
    \end{frame}

    \subsection{Bewertung}

    \begin{frame}{Bewertung}
        \begin{enumerate}
            \item alle Themen wegstreichen, die einen selbst nicht interessieren
            \item alle Themen wegstreichen, die unpassend für Publikum sind (z.B. Thema "`Sex"' bei Publikum aus Grundschulkindern)
            \item übrige Themen, wo nötig, verallgemeinern (z.B. statt "`Syntax von Java 8"' einfach "`Java"')
            \item aus übrigen Themen eines auswählen
        \end{enumerate}
    \end{frame}

    \subsection{Niveau}

    \begin{frame}{Niveau}
        \begin{itemize}
            \item Publikum bestimmt auf welchem Niveau Vortrag gehalten werden kann (voraussetzbares Vorwissen)
            \item eigenes Wissen bestimmt auf welchem Niveau Vortrag gehalten werden kann
            \item sollte eigenes Wissen unter dem voraussetzbaren Vorwissen des Publikums liegen: anderes Thema auswählen
            \item Mehrwert für Publikum muss bestehen (weder zu geringes Niveau noch zu hohes)
        \end{itemize}
    \end{frame}

    \begin{frame}{Unterthema}
        \centering
        Entsprechend dem Niveau ein geeignetes Unterthema wählen.
    \end{frame}

\section{Vortragsgestaltung}
    
    \subsection{Kernbotschaft}

    \begin{frame}{Take-away Message}
        \centering
        Was ist das, was das Publikum mit nach Hause nehmen soll?
    \end{frame}

    \begin{frame}{Roter Faden}
        \begin{itemize}
            \item muss vorhanden und erkennbar sein
            \item muss auf Kernbotschaft hinarbeiten
            \item am Ende muss klar sein, was die Essenz des Vortrags ist
            \item alles, was Vortrag unnötig aufbläht ohne auf Kernbotschaft hinzuarbeiten, ist wegzulassen
        \end{itemize}
    \end{frame}
    
    \subsection{Software}

    \begin{frame}{Software}
        \begin{itemize}
            \item passende Software für Ziel des Vortrags auswählen
            \item persönliche Vorliebe ebenso wichtig wie Eignung für Vortragsthema
        \end{itemize}
    \end{frame}

    \subsection{Sinnabschnitte}

    \begin{frame}{Gliederung}
        \begin{itemize}
            \item Vortrag in sinnvolle Abschnitte aufteilen
            \item zu jeder Zeit sollte klar sein, worum es geht
        \end{itemize}
    \end{frame}

\section{Abschluss}

\begin{frame}{Zusammenfassung}
    \begin{itemize}
        \item Thema finden ist einfach
        \item passendes Niveau für Publikum zu finden ist einfach
        \item Unterthema finden ist einfach
        \item Kernbotschaft finden ist einfach
        \item Roten Faden finden ist einfach
        \item Vortrag gestalten ist einfach
        \item Kurzum: alles bis auf den Vortrag zu halten ist einfach
        \item für das Halten des Vortrags hilft nur üben, üben und nochmals üben
    \end{itemize}
\end{frame}

\end{document}

