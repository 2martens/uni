\RequirePackage{pdf14}
\documentclass{beamer}
\usepackage[T1]{fontenc}
\usepackage[utf8]{inputenc}
\usepackage[ngerman]{babel}
%\usepackage{paralist}
%\useoutertheme{infolines} 
\usepackage{graphicx}
\usepackage{hyperref}
\usepackage{listings}
\usepackage{color}
\usetheme{Warsaw}
\usecolortheme{crane}
\pagenumbering{arabic}
\def\thesection{\arabic{section})}
\def\thesubsection{\alph{subsection})}
\def\thesubsubsection{(\roman{subsubsection})}
\setbeamertemplate{navigation symbols}{}
\graphicspath{ {src/} {/home/jim/Pictures/Studium/KBS/} }

\definecolor{mygreen}{rgb}{0,0.6,0}
\definecolor{mygray}{rgb}{0.5,0.5,0.5}
\definecolor{mymauve}{rgb}{0.58,0,0.82}

\lstset{ %
  backgroundcolor=\color{white},   % choose the background color; you must add \usepackage{color} or \usepackage{xcolor}
  basicstyle=\footnotesize,        % the size of the fonts that are used for the code
  breakatwhitespace=false,         % sets if automatic breaks should only happen at whitespace
  breaklines=true,                 % sets automatic line breaking
  captionpos=b,                    % sets the caption-position to bottom
  commentstyle=\color{mygray},    % comment style
  deletekeywords={},            % if you want to delete keywords from the given language
  escapeinside={\%*}{*)},          % if you want to add LaTeX within your code
  extendedchars=true,              % lets you use non-ASCII characters; for 8-bits encodings only, does not work with UTF-8
  keepspaces=true,                 % keeps spaces in text, useful for keeping indentation of code (possibly needs columns=flexible)
  keywordstyle=\color{blue},       % keyword style
  language=PHP,                 % the language of the code
  morekeywords={class, function, return, protected, public, private, const, static, new, extends, namespace, null},            % if you want to add more keywords to the set
  numbers=left,                    % where to put the line-numbers; possible values are (none, left, right)
  numbersep=5pt,                   % how far the line-numbers are from the code
  numberstyle=\tiny\color{mygray}, % the style that is used for the line-numbers
  rulecolor=\color{black},         % if not set, the frame-color may be changed on line-breaks within not-black text (e.g. comments (green here))
  showspaces=false,                % show spaces everywhere adding particular underscores; it overrides 'showstringspaces'
  showstringspaces=false,          % underline spaces within strings only
  showtabs=false,                  % show tabs within strings adding particular underscores
  stepnumber=2,                    % the step between two line-numbers. If it's 1, each line will be numbered
  stringstyle=\color{mygreen},     % string literal style
  tabsize=2,                       % sets default tabsize to 2 spaces
  title=\lstname                   % show the filename of files included with \lstinputlisting; also try caption instead of title
}

\hypersetup{
    pdfauthor=Jim Martens,
    pdfstartview=Fit
}

\expandafter\def\expandafter\insertshorttitle\expandafter{%
    \raggedleft \insertframenumber\,/\,\inserttotalframenumber\;}

\begin{document}
\author{Jim 2martens}
\title{RahmenPrüfungsOrdnung}
\date{11. Oktober 2016}

\begin{frame}
  \titlepage
\end{frame}

\begin{frame}{Was ist das?}
  \begin{itemize}
    \item wie bisherige Prüfungsordnungen der Fakultäten, nur uniweit
    \item Prüfungsordnungen der Fakultäten wären einer solchen RPO untergeordnet
  \end{itemize}
\end{frame}

\begin{frame}{Warum gibts das?}
  \begin{itemize}
    \item RPO ermöglicht und gefordert durch letzte Gesetzesänderung vom HambHG
    \item Akademischer Senat beauftragte uniweiten Ausschuss für Lehre und Studium mit Entwurfserarbeitung
    \item Entwurf nun fertig
    \item jetzt umfangreicher Beteiligungsprozess der Fakultäten und Fachbereiche
  \end{itemize}
\end{frame}

\begin{frame}{Warum ist das (für euch) interessant?}
  \begin{itemize}
    \item Erhöhung der Prüfungsversuche von bisher 3 auf unbegrenzt
    \item uniweite Festschreibung, dass es keine Fristen geben darf
    \item Reduzierung der in die Endnote eingehenden Prüfungsleistungen auf 14 (rund 2 pro Semester + 2)
    \item keine Anwesenheitspflicht mehr außer in hochschuldidaktisch begründeten Einzelfällen
    \item u.v.m.
  \end{itemize}
\end{frame}

\begin{frame}{Wo gibts weitere Informationen?}
  \centering
  Infos und Möglichkeit der Unterstützung auf:
  \url{https://rpo-uhh.de}

  Informations-/Aktiventreffen am 17.10. (nächster Montag) um 18 Uhr (Raum 1536b im Geomatikum)
\end{frame}

\end{document}