\documentclass{beamer}
\usepackage[T1]{fontenc}
\usepackage[utf8]{inputenc}
\usepackage[ngerman]{babel}
%\usepackage{paralist}
%\useoutertheme{infolines}
\usepackage{graphicx}
\usepackage{hyperref}
\usepackage{listings}
\usepackage{color}
\usepackage{textcomp}
\usetheme{Warsaw}
\usecolortheme{crane}
\pagenumbering{arabic}
\def\thesection{\arabic{section})}
\def\thesubsection{\alph{subsection})}
\def\thesubsubsection{(\roman{subsubsection})}
\setbeamertemplate{navigation symbols}{}
\graphicspath{ {src/} {/home/jim/Pictures/} }

\definecolor{mygreen}{rgb}{0,0.6,0}
\definecolor{mygray}{rgb}{0.5,0.5,0.5}
\definecolor{mymauve}{rgb}{0.58,0,0.82}

%\definecolor{craneorange}{RGB}{61,61,61}
%\definecolor{craneblue}{RGB}{255,255,255}

\lstset{ %
  backgroundcolor=\color{white},   % choose the background color; you must add \usepackage{color} or \usepackage{xcolor}
  basicstyle=\footnotesize,        % the size of the fonts that are used for the code
  breakatwhitespace=false,         % sets if automatic breaks should only happen at whitespace
  breaklines=true,                 % sets automatic line breaking
  captionpos=b,                    % sets the caption-position to bottom
  commentstyle=\color{mygray},    % comment style
  deletekeywords={},            % if you want to delete keywords from the given language
  escapeinside={\%*}{*)},          % if you want to add LaTeX within your code
  extendedchars=true,              % lets you use non-ASCII characters; for 8-bits encodings only, does not work with UTF-8
  keepspaces=true,                 % keeps spaces in text, useful for keeping indentation of code (possibly needs columns=flexible)
  keywordstyle=\color{blue},       % keyword style
  language=PHP,                 % the language of the code
  morekeywords={class, function, return, protected, public, private, const, static, new, extends, namespace, null},            % if you want to add more keywords to the set
  numbers=left,                    % where to put the line-numbers; possible values are (none, left, right)
  numbersep=5pt,                   % how far the line-numbers are from the code
  numberstyle=\tiny\color{mygray}, % the style that is used for the line-numbers
  rulecolor=\color{black},         % if not set, the frame-color may be changed on line-breaks within not-black text (e.g. comments (green here))
  showspaces=false,                % show spaces everywhere adding particular underscores; it overrides 'showstringspaces'
  showstringspaces=false,          % underline spaces within strings only
  showtabs=false,                  % show tabs within strings adding particular underscores
  stepnumber=2,                    % the step between two line-numbers. If it's 1, each line will be numbered
  stringstyle=\color{mygreen},     % string literal style
  tabsize=2,                       % sets default tabsize to 2 spaces
  title=\lstname                   % show the filename of files included with \lstinputlisting; also try caption instead of title
}

\hypersetup{
	pdfauthor=Jim Martens,
	pdfstartview=Fit
}

\expandafter\def\expandafter\insertshorttitle\expandafter{%
	\raggedleft \insertframenumber\,/\,\inserttotalframenumber\;}

\begin{document}
\author{Jim 2martens}
\title{StuPa}
\date{\today}

\begin{frame}
    \titlepage
\end{frame}

\begin{frame}
    \tableofcontents
\end{frame}

\section{Allgemeines}
\begin{frame}{Erwartungen/Befürchtungen}
    \centering
    Was erwartet ihr? Was befürchtet ihr?
\end{frame}
\begin{frame}{Trivia}
    \begin{itemize}
        \item Wahl im Dezember/Januar
        \item 47 Parlamentarier*innen
        \item tagt i.d.R. alle 2 Wochen am Donnerstag in der Vorlesungszeit
        \item CG hat 11 Sitze in der Legislatur 17/18
        \item höchstes Gremium der studentischen Selbstverwaltung
    \end{itemize}
\end{frame}

\section{Struktur}
\subsection{Übersicht}

\begin{frame}{Beziehungen}
    \begin{itemize}
        \item StuPa wählt Präsidium
        \item StuPa wählt AStA-Vorsitzende
        \item StuPa bestätigt AStA-Referent*innen
        \item StuPa bestätigt Referent*innen und Wahlprotokolle der teilautonomen
              Referate des AStA
        \item StuPa wählt Ausschussmitglieder
    \end{itemize}
\end{frame}

\begin{frame}{Aufgaben}
    \begin{itemize}
        \item Beschluss von Satzung, Wahlordnung, Wirtschaftsordnung,
              Fachschaftsrahmenordnung, Beitragsordnung, Geschäftsordnung
              (vergleichbar mit Gesetzen)
        \item Beschluss vom Haushalt
        \item Beschluss von einfachen Anträgen (fast alle Beschlüsse)
    \end{itemize}
\end{frame}

\subsection{Präsidium}
\begin{frame}{Aufgaben}
    \begin{itemize}
        \item Geschäftsführung des StuPa (Einladungen, Sitzungsleitung, Protokolle)
        \item Wahlorganisation
        \item Dienstleistungen (Hilfe bei FSR-Wahlen, Konstituierung der Ausschüsse)
        \item Öffentlichkeitsarbeit
        \item ...
    \end{itemize}
\end{frame}

\begin{frame}{Arbeitsaufwand (SoSe)}
    \begin{itemize}
        \item 4h Präsidiumssitzung/Woche in Vorlesungszeit
        \item Anwesenheit in allen StuPa-Sitzungen von Anfang bis Ende
        \item ca. 2h/Woche für normale ToDos pro Woche
        \item zweiwöchentliche Treffen in der vorlesungsfreien Zeit
    \end{itemize}
\end{frame}

\begin{frame}{Arbeitsaufwand (WiSe)}
    \begin{itemize}
        \item 5h Präsidiumssitzung/Woche in Vorlesungszeit
        \item Anwesenheit in allen StuPa-Sitzungen von Anfang bis Ende
        \item ca. 2-4h/Woche für normale ToDos pro Woche
        \item häufigere und längere Treffen ab November
        \item Urnenwahlwoche mit 40h/Woche + Auszählung einplanen
    \end{itemize}
\end{frame}

\begin{frame}{Privilegien}
    \begin{itemize}
        \item Gehalt + Aufwandsentschädigung = 636€/Monat
        \item eigenes Büro am Hauptcampus
        \item Erlernen von Zeitmanagement und Kooperationsfähigkeit
        \item guter Punkt in Lebenslauf
    \end{itemize}
\end{frame}

\begin{frame}{Voraussetzungen}
    \begin{itemize}
        \item StuPa-Mitglied
        \item Zuverlässigkeit
        \item Regelmäßige Anwesenheit in vorlesungsfreier Zeit (Urlaub ist möglich)
        \item Interesse
        \item Bereitschaft neue Sachen zu lernen (u.a. Websitenadministration)
    \end{itemize}
\end{frame}

\begin{frame}{NICHT-Voraussetzungen}
    \begin{itemize}
        \item 3 Jahre Berufserfahrung
        \item Diplom in Zeitmanagement
        \item Informatikkenntnisse
        \item männlich
        \item Moderationserfahrung
    \end{itemize}
\end{frame}

\begin{frame}{Chancen}
    \begin{itemize}
        \item Arbeitsabläufe verbessern
        \item Öffentlichkeitsarbeit verbessern
        \item Inhaltliche Konsenslinien herausarbeiten
        \item Vorbild sein
        \item junge Menschen für Politik begeistern
    \end{itemize}
\end{frame}

\subsection{Ausschüsse}

\begin{frame}{Voraussetzungen}
    \begin{itemize}
        \item Interesse
        \item Zuverlässigkeit
    \end{itemize}
\end{frame}

\begin{frame}{Wahlämter}
    \begin{itemize}
        \item Ausschussvorsitzende*r
        \item Stellvertretende*r Ausschussvorsitzende*r
    \end{itemize}
\end{frame}

\begin{frame}{SWOGA}
    \begin{itemize}
        \item berät Satzungs-, Wahlordnungs- und Geschäftsordnungsvorlagen
        \item tagt so häufig, wie es nötig ist
        \item über aktive Mitarbeit kann viel umgesetzt werden
        \item variabler Arbeitsaufwand
        \item mind. 7 Mitglieder
    \end{itemize}
\end{frame}

\begin{frame}{Haushalt}
    \begin{itemize}
        \item tagt nur im SoSe
        \item berät die Haushaltsvorlagen vom AStA
        \item über kurzen Zeitraum hoher Arbeitsaufwand
        \item mind. 7 Mitglieder
        \item Mitglieder dürfen nicht AStA-Mitglieder sein
    \end{itemize}
\end{frame}

\begin{frame}{gegen Rechts}
    \begin{itemize}
        \item tagt regelmäßig
        \item macht gute Arbeit
        \item kontinuierlicher Arbeitsaufwand
        \item leidet tendenziell unter Beschlussunfähigkeit
        \item mind. 7 Mitglieder
    \end{itemize}
\end{frame}

\begin{frame}{Ältestenrat}
    \begin{itemize}
        \item Schiedsgericht der Verfassten Studierendenschaft
        \item Streitigkeit über Zusammensetzung und Aufgabenspektrum
        \item unregelmäßiger Arbeitsaufwand (abhängig von Fällen)
    \end{itemize}
\end{frame}

\begin{frame}{Wirtschaftsrat}
    \begin{itemize}
        \item genehmigt Haushalt der Verfassten Studierendenschaft
        \item Mitglieder dürfen nicht dem AStA angehören
        \item tagt im SoSe
        \item 6 stud. Mitglieder (HV/SV), ein Paar von CG
    \end{itemize}
\end{frame}

\subsection{Fraktion}

\begin{frame}{Grundsätzliches}
    \begin{itemize}
        \item CGs Vertretung im Parlament
        \item 11 Sitze in kommender Legislatur (Meike und Jim direkt)
        \item für Profilbildung von CG in der Verfassten Studierendenschaft zuständig
        \item AStA-tragende Fraktion
    \end{itemize}
\end{frame}

\begin{frame}{Arbeit}
    \begin{itemize}
        \item (Änderungs-)Anträge schreiben
        \item Anträge vorbesprechen
        \item Debatten inhaltlich vorbereiten
        \item taktische Verhandlungsstragie vorbesprechen (Nutzung von Fraktionspausen)
        \item aktive Teilnahme im Parlament (bitte auch von ¬AStA- und ¬Präsidiumsmenschen)
    \end{itemize}
\end{frame}

\begin{frame}{Struktur (Vorschlag)}
    \begin{itemize}
        \item Person zuständig für Kommunikation nach "rechts"
        \item Person zuständig für Kommunikation nach "links"
        \item Person zuständig für Kommunikation nach innen
        \item Nutzung von Fraktions-Mailingliste für Terminankündigungen
    \end{itemize}
\end{frame}

\section{Strategische Planung}

\begin{frame}{Präsidiumsbesetzung}
    \begin{itemize}
        \item Gunhild wird antreten (garantiert 7 Stimmen)
        \item N.N. (Opposition) wird antreten (garantiert 15 Stimmen)
        \item 14 ungebundene Stimmen (Medi*, UC, LISTE, AL, CC, Jusos)
        \item N.N. (CG-unterstützt) wird antreten (garantiert 11 Stimmen)
    \end{itemize}
\end{frame}

\begin{frame}{Präsidiumsbesetzung}
    \begin{itemize}
        \item garantiert 3 Sitze/Liste (>= 38 Stimmen), nicht zu schaffen
        \item garantiert 2 Sitze/Liste (>= 24 Stimmen)
        \item garantiert 1 Sitz/Liste (>= 12 Stimmen), RCDS bekommt garantiert 1 Sitz
        \item wir kriegen maximal 2 Leute rein
        \item mit 14 ungebundenen Stimmen für unsere Liste bräuchten wir Gunhild
              nicht
    \end{itemize}
\end{frame}

\begin{frame}{Inhaltlicher Fahrplan}
    \begin{itemize}
        \item neue GO für das StuPa auf den Weg bringen (erste Aktion im SWOGA)
        \item evtl. Satzung ändern und Präsidium gleichberechtigt machen
              (mögliche Initiative vom BAE)
        \item Fachschaftsrahmenordnung neufassen
        \item Eigene Projekte über das StuPa in den AStA einbringen
        \item Anträge zu anstehenden Themen
    \end{itemize}
\end{frame}

\begin{frame}{Abschluss}
    \begin{itemize}
        \item Donnerstags ab 18 Uhr freihalten
        \item Donnerstags von 11 - 13:30 evtl. Vorbereitung der Vorbesprechung
        \item Donnerstags von 17-18 Uhr Vorbesprechung
        \item Ausschusssitzungen idealerweise donnerstags in sitzungsfreien Wochen
    \end{itemize}
\end{frame}
\end{document}
