\RequirePackage{pdf14}
\documentclass{beamer}
\usepackage[T1]{fontenc}
\usepackage[utf8]{inputenc}
\usepackage[ngerman]{babel}
%\usepackage{paralist}
%\useoutertheme{infolines} 
\usepackage{graphicx}
\usepackage{hyperref}
\usepackage{listings}
\usepackage{color}
\usetheme{Warsaw}
\usecolortheme{crane}
\pagenumbering{arabic}
\def\thesection{\arabic{section})}
\def\thesubsection{\alph{subsection})}
\def\thesubsubsection{(\roman{subsubsection})}
\setbeamertemplate{navigation symbols}{}
\graphicspath{ {src/} {/home/jim/Pictures/} }

\definecolor{mygreen}{rgb}{0,0.6,0}
\definecolor{mygray}{rgb}{0.5,0.5,0.5}
\definecolor{mymauve}{rgb}{0.58,0,0.82}

\lstset{ %
  backgroundcolor=\color{white},   % choose the background color; you must add \usepackage{color} or \usepackage{xcolor}
  basicstyle=\footnotesize,        % the size of the fonts that are used for the code
  breakatwhitespace=false,         % sets if automatic breaks should only happen at whitespace
  breaklines=true,                 % sets automatic line breaking
  captionpos=b,                    % sets the caption-position to bottom
  commentstyle=\color{mygray},    % comment style
  deletekeywords={},            % if you want to delete keywords from the given language
  escapeinside={\%*}{*)},          % if you want to add LaTeX within your code
  extendedchars=true,              % lets you use non-ASCII characters; for 8-bits encodings only, does not work with UTF-8
  keepspaces=true,                 % keeps spaces in text, useful for keeping indentation of code (possibly needs columns=flexible)
  keywordstyle=\color{blue},       % keyword style
  language=PHP,                 % the language of the code
  morekeywords={class, function, return, protected, public, private, const, static, new, extends, namespace, null},            % if you want to add more keywords to the set
  numbers=left,                    % where to put the line-numbers; possible values are (none, left, right)
  numbersep=5pt,                   % how far the line-numbers are from the code
  numberstyle=\tiny\color{mygray}, % the style that is used for the line-numbers
  rulecolor=\color{black},         % if not set, the frame-color may be changed on line-breaks within not-black text (e.g. comments (green here))
  showspaces=false,                % show spaces everywhere adding particular underscores; it overrides 'showstringspaces'
  showstringspaces=false,          % underline spaces within strings only
  showtabs=false,                  % show tabs within strings adding particular underscores
  stepnumber=2,                    % the step between two line-numbers. If it's 1, each line will be numbered
  stringstyle=\color{mygreen},     % string literal style
  tabsize=2,                       % sets default tabsize to 2 spaces
  title=\lstname                   % show the filename of files included with \lstinputlisting; also try caption instead of title
}

\hypersetup{
	pdfauthor=Jim Martens,
	pdfstartview=Fit
}

\expandafter\def\expandafter\insertshorttitle\expandafter{%
	\raggedleft \insertframenumber\,/\,\inserttotalframenumber\;}

\begin{document}
\author{Jim Martens}
\title{Emails, Deadlines, Productivity}
\date{14. November 2014}

	% Introduction
	\begin{frame}
		\titlepage
	\end{frame}
	
	% Contents
	\begin{frame}{Contents}
		\tableofcontents
	\end{frame}
	
	\section{Basics}
	% Terms	
	\begin{frame}{Terms}
		\begin{itemize}
			\item IMAP (preferred)
			\item POP
			\item SMTP
			\item Ports
		\end{itemize}
	\end{frame}
	
	% Applications
	\begin{frame}{Applications}
		\begin{itemize}
			\item Mozilla Thunderbird
			\item Microsoft Office Outlook
			\item Mozilla Firefox
			\item Google Chrome
		\end{itemize}
	\end{frame}
	
	\section{Preparation}
	\subsection{Installation}
	\begin{frame}[fragile]{Linux}
		\begin{verbatim}
			sudo apt-get install thunderbird
			sudo yum install thunderbird
			sudo pacman -S thunderbird			
		\end{verbatim}
	\end{frame}
	\begin{frame}{Rest}
		https://www.mozilla.org/de/thunderbird/
	\end{frame}
	\begin{frame}{All}
		Install Lightning addon
	\end{frame}
	
	\subsection{Configuration}
	\begin{frame}{Add email address}
		Add @inf email address
	\end{frame}
	
	\section{Filter}
	\subsection{Basics}
	\begin{frame}{Filter basics}
		\begin{itemize}
			\item client-side
			\begin{itemize}
				\item must be repeated on each client
				\item useful for (additional) junk filter
			\end{itemize}
			\item server-side
			\begin{itemize}
				\item setup once, use on each client
				\item preferred way
			\end{itemize}
		\end{itemize}
	\end{frame}
	
	\subsection{Setup}
	\begin{frame}{Webmail}
		\begin{enumerate}
			\item Open browser
			\item Navigate to https://webmail.informatik.uni-hamburg.de
			\item Login with @inf credentials
			\item Navigate to ``Ordner''
			\item Create one folder for each mailing list (e.g. mafia@inf, stud@inf, bacc@inf, fs-inf@inf)
			\item Navigate to ``Filter''
			\item Create one rule for each mailing list
			\item Repeat above steps whenever you join a new mailing list
		\end{enumerate}
	\end{frame}
	
	\begin{frame}{Thunderbird}
		\begin{enumerate}
			\item Click on header of mail account
			\item Click on ``Manage message filters''
			\item Create the filters you want
		\end{enumerate}
	\end{frame}
	
	\section{Calendar}
	\subsection{Basics}
	\begin{frame}{Basics}
		\begin{itemize}
			\item CALDav
			\item iCal
			\item Calendars are extremely useful
		\end{itemize}
	\end{frame}	
	
	\subsection{Setup}
	\begin{frame}{Mafiasi calendars}
		\begin{itemize}
			\item KBS\footnote{\url{ https://sogo.mafiasi.de/SOGo/dav/mafiasi/Calendar/51B8-5425D480-5-7ED00B00/ }}
			\item Fachschaft\footnote{\url{https://sogo.mafiasi.de/SOGo/dav/mafiasi/Calendar/51B8-5425D400-3-7ED00B00/ }}
			\item Spieleabend\footnote{\url{https://sogo.mafiasi.de/SOGo/dav/mafiasi/Calendar/51B8-5425D380-1-7ED00B00/ }}
		\end{itemize}
	\end{frame}	
	
	\begin{frame}{Lightning}
		\begin{itemize}
			\item Thunderbird addon
			\item understands CALDav
			\item tasks
			\item events (invite others)
		\end{itemize}
	\end{frame}
	
	\section{Productivity}
	\subsection{Settings}
	\begin{frame}{Thunderbird}
		\begin{itemize}
			\item text-only mails
			\item begin answer below quote
			\item don't show alert if new message
			\item don't play sound if new message
			\item check for missing attachment
			\item automatically add outgoing mails to my ``Collected Addresses''
			\item tell sites you don't want to be tracked
			\item mark messages marked as junk as read
			\item enable adaptive junk filter logging
			\item move junk messages to ``Junk'' folder
			\item don't mark messages automatically as read
			\item don't only show display name for people in your address book
		\end{itemize}
	\end{frame}
	
	\subsection{Workflow}		
	\begin{frame}{Workflow}
		\begin{enumerate}
			\item have mail client open all the time
			\item don't look at it all the time
			\item whenever you have time, look at it
			\item scan new/unread mails (info mails, action mails and general mails, personal mails)
			\item answer all action mails that require less than 5 minutes time
			\item once answered, mark them as read
			\item leave mails, you still have to respond to, marked as unread
		\end{enumerate}
	\end{frame}	
	
	\section*{Final Words}
	\begin{frame}{Conclusion}
		\begin{enumerate}
			\item<1-> use a mail client (Thunderbird)
			\item<2-> use a local calendar (e.g. Lightning addon for Thunderbird)
			\item<3-> read your mails at least daily (part of workflow)
			\item<4-> be productive
			\item<5> be better
		\end{enumerate}
	\end{frame}
	\begin{frame}{Contact}
		\begin{itemize}
			\item Email: 2martens@inf
			\item Jabber: 12martens
			\item twitter.com/2martens
			\item github.com/frmwrk123 (personal)
			\item github.com/2martens (organization)
			\item www.2martens.de/web-platform
		\end{itemize}
	\end{frame}
\end{document}

