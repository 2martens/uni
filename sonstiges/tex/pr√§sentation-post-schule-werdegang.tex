\documentclass{beamer}
\usepackage[T1]{fontenc}
\usepackage[utf8]{inputenc}
\usepackage[ngerman]{babel}
%\usepackage{paralist}
%\useoutertheme{infolines}
\usepackage{graphicx}
\usepackage{hyperref}
\usepackage{listings}
\usepackage{color}
\usepackage{textcomp}
\usepackage[german=quotes]{csquotes}
\usetheme{Warsaw}
\usecolortheme{crane}
\pagenumbering{arabic}
\def\thesection{\arabic{section})}
\def\thesubsection{\alph{subsection})}
\def\thesubsubsection{(\roman{subsubsection})}
\setbeamertemplate{navigation symbols}{}
\graphicspath{ {src/} {/home/jim/Pictures/} }

\definecolor{mygreen}{rgb}{0,0.6,0}
\definecolor{mygray}{rgb}{0.5,0.5,0.5}
\definecolor{mymauve}{rgb}{0.58,0,0.82}

\MakeOuterQuote{"}

%\definecolor{craneorange}{RGB}{61,61,61}
%\definecolor{craneblue}{RGB}{255,255,255}

\lstset{ %
  backgroundcolor=\color{white},   % choose the background color; you must add \usepackage{color} or \usepackage{xcolor}
  basicstyle=\footnotesize,        % the size of the fonts that are used for the code
  breakatwhitespace=false,         % sets if automatic breaks should only happen at whitespace
  breaklines=true,                 % sets automatic line breaking
  captionpos=b,                    % sets the caption-position to bottom
  commentstyle=\color{mygray},    % comment style
  deletekeywords={},            % if you want to delete keywords from the given language
  escapeinside={\%*}{*)},          % if you want to add LaTeX within your code
  extendedchars=true,              % lets you use non-ASCII characters; for 8-bits encodings only, does not work with UTF-8
  keepspaces=true,                 % keeps spaces in text, useful for keeping indentation of code (possibly needs columns=flexible)
  keywordstyle=\color{blue},       % keyword style
  language=PHP,                 % the language of the code
  morekeywords={class, function, return, protected, public, private, const, static, new, extends, namespace, null},            % if you want to add more keywords to the set
  numbers=left,                    % where to put the line-numbers; possible values are (none, left, right)
  numbersep=5pt,                   % how far the line-numbers are from the code
  numberstyle=\tiny\color{mygray}, % the style that is used for the line-numbers
  rulecolor=\color{black},         % if not set, the frame-color may be changed on line-breaks within not-black text (e.g. comments (green here))
  showspaces=false,                % show spaces everywhere adding particular underscores; it overrides 'showstringspaces'
  showstringspaces=false,          % underline spaces within strings only
  showtabs=false,                  % show tabs within strings adding particular underscores
  stepnumber=2,                    % the step between two line-numbers. If it's 1, each line will be numbered
  stringstyle=\color{mygreen},     % string literal style
  tabsize=2,                       % sets default tabsize to 2 spaces
  title=\lstname                   % show the filename of files included with \lstinputlisting; also try caption instead of title
}

\hypersetup{
	pdfauthor=Jim Martens,
	pdfstartview=Fit
}

\expandafter\def\expandafter\insertshorttitle\expandafter{%
	\raggedleft \insertframenumber\,/\,\inserttotalframenumber\;}

\begin{document}
\author{Jim 2martens}
\title{Ändern wir die Welt, bevor es andere tun!}
\date{\today}

\begin{frame}
    \titlepage
\end{frame}

\begin{frame}
    \tableofcontents
\end{frame}

\section{Übersicht}
\begin{frame}{Highway to University}
    \begin{itemize}
        \item Schnupperpraktikum Informatik (Oktober 2010)
        \vfill
        \item Abiturzeugnis bekommen (Juni 2012)
        \vfill
        \item auf BSc. Informatik bewerben (Juli 2012)
        \vfill
        \item Mathevorkurs besuchen (September 2012)
        \vfill
        \item Orientierungseinheit und Studienbeginn (Oktober 2012)
    \end{itemize}
\end{frame}

\begin{frame}{Hochschulpolitik}
    \begin{itemize}
        \item Fachschaftsrat Informatik (WiSe 2012/13, SoSe14-SoSe16)
        \vfill
        \item Studierendenparlament für CampusGrün (ab April 2015)
        \vfill
        \item Präsidium des Studierendenparlaments (April 2016-April 2017)
    \end{itemize}
\end{frame}

\begin{frame}{Highway to Master}
    \begin{itemize}
        \item Bachelorarbeit schreiben (November 2015-August 2016)
        \vfill
        \item für Master bewerben (August 2016)
        \vfill
        \item Bachelorzeugnis bekommen (Herbst 2016)
        \vfill
        \item Studienbeginn Master (Oktober 2016)
    \end{itemize}
\end{frame}

\begin{frame}{Politisches Engagement}
    \begin{itemize}
        \item Mitglied von Diem25 (seit Februar 2016)
        \vfill
        \item Landesvorstand GRÜNE JUGEND Hamburg (seit Oktober 2016)
        \vfill
        \item Mitglied bei den GRÜNEN (seit Januar/Februar 2017)
        \vfill
        \item Delegierter für Bundesparteitag der GRÜNEN (Juni 2017, bis Ende 2017)
    \end{itemize}
\end{frame}

\section{Herausforderungen}
\begin{frame}{Zeit!}
    \begin{itemize}
        \item Grundsätzlich: Mehr zu tun als Zeit vorhanden ist.
        \vfill
        \item gutes Zeitmanagement nötig
        \vfill
        \item gute Terminplanung nötig
    \end{itemize}
\end{frame}

\begin{frame}{Spaß oder der Mangel daran}
    \begin{itemize}
        \item kein Bock auf Bachelorarbeit, Politik viel spannender
        \vfill
        \item schlechte Profs
        \vfill
        \item Null Bock Tage
    \end{itemize}
\end{frame}

\begin{frame}{ungleichmäßiger Arbeitsaufwand}
    \begin{itemize}
        \item stellenweise extrem hohe Belastung (Klausurenzeit)
        \vfill
        \item gute Planung nötig, damit später noch Zeit bleibt
        \vfill
        \item Prinzip: Was du heute kannst besorgen, verschiebe nicht auf morgen,
                       denn morgen können hundert neue Sachen dazu kommen, die eine
                       höhere Priorität haben.
    \end{itemize}
\end{frame}

\section{Vision}
\begin{frame}{Für bessere Welt kämpfen}
    \begin{itemize}
        \item Verbesserungen von heute basieren auf Kämpfen von gestern
        \vfill
        \item gute Welt von morgen benötigt unseren Kampf heute
        \vfill
        \item Nicht meckern, machen!
    \end{itemize}
\end{frame}

\section{Zukunft}

\begin{frame}{Perspektive Arbeit}
    \begin{itemize}
        \item gute Masterarbeit schreiben (WiSe 2018/19, SoSe 2019)
        \vfill
        \item gut bezahlten max. 30-Stunden-Job finden
        \vfill
        \item Rest der Zeit in Politik, Freizeit, Gesellschaft stecken
    \end{itemize}
\end{frame}

\begin{frame}{Perspektive Politik}
    \begin{itemize}
        \item weiter als Schatzmeister der GRÜNEN JUGEND Hamburg bis Herbst 2018
        \vfill
        \item evtl. über Koalitionsverhandlungen und -vertrag abstimmen
        \vfill
        \item Mitgliederwachstum für GRÜNE JUGEND Hamburg:
              1/3 der GRÜNEN Hamburg bis 2020
        \vfill
        \item 20\% + x bei unter 30-jährigen Menschen in 2020
        \vfill
        \item evtl. Kandidatur für Bürgerschaft in 2020
    \end{itemize}
\end{frame}

\begin{frame}{Qual der Wahl}
    Worüber möchtet ihr mehr hören?
    \begin{itemize}
        \item Studium
        \vfill
        \item Digitalisierung/Informatik
        \vfill
        \item Hochschulpolitik
        \vfill
        \item Politisches Engagement
    \end{itemize}
\end{frame}

\end{document}
