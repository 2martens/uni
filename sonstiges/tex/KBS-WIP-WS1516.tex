\RequirePackage{pdf14}
\documentclass{beamer}
\usepackage[T1]{fontenc}
\usepackage[utf8]{inputenc}
\usepackage[ngerman]{babel}
%\usepackage{paralist}
%\useoutertheme{infolines} 
\usepackage{graphicx}
\usepackage{hyperref}
\usepackage{listings}
\usepackage{color}
\usetheme{Warsaw}
\usecolortheme{crane}
\pagenumbering{arabic}
\def\thesection{\arabic{section})}
\def\thesubsection{\alph{subsection})}
\def\thesubsubsection{(\roman{subsubsection})}
\setbeamertemplate{navigation symbols}{}
\graphicspath{ {src/} {/home/jim/Pictures/Studium/KBS/} }

\definecolor{mygreen}{rgb}{0,0.6,0}
\definecolor{mygray}{rgb}{0.5,0.5,0.5}
\definecolor{mymauve}{rgb}{0.58,0,0.82}

\lstset{ %
  backgroundcolor=\color{white},   % choose the background color; you must add \usepackage{color} or \usepackage{xcolor}
  basicstyle=\footnotesize,        % the size of the fonts that are used for the code
  breakatwhitespace=false,         % sets if automatic breaks should only happen at whitespace
  breaklines=true,                 % sets automatic line breaking
  captionpos=b,                    % sets the caption-position to bottom
  commentstyle=\color{mygray},    % comment style
  deletekeywords={},            % if you want to delete keywords from the given language
  escapeinside={\%*}{*)},          % if you want to add LaTeX within your code
  extendedchars=true,              % lets you use non-ASCII characters; for 8-bits encodings only, does not work with UTF-8
  keepspaces=true,                 % keeps spaces in text, useful for keeping indentation of code (possibly needs columns=flexible)
  keywordstyle=\color{blue},       % keyword style
  language=PHP,                 % the language of the code
  morekeywords={class, function, return, protected, public, private, const, static, new, extends, namespace, null},            % if you want to add more keywords to the set
  numbers=left,                    % where to put the line-numbers; possible values are (none, left, right)
  numbersep=5pt,                   % how far the line-numbers are from the code
  numberstyle=\tiny\color{mygray}, % the style that is used for the line-numbers
  rulecolor=\color{black},         % if not set, the frame-color may be changed on line-breaks within not-black text (e.g. comments (green here))
  showspaces=false,                % show spaces everywhere adding particular underscores; it overrides 'showstringspaces'
  showstringspaces=false,          % underline spaces within strings only
  showtabs=false,                  % show tabs within strings adding particular underscores
  stepnumber=2,                    % the step between two line-numbers. If it's 1, each line will be numbered
  stringstyle=\color{mygreen},     % string literal style
  tabsize=2,                       % sets default tabsize to 2 spaces
  title=\lstname                   % show the filename of files included with \lstinputlisting; also try caption instead of title
}

\hypersetup{
    pdfauthor=Jim Martens,
    pdfstartview=Fit
}

\expandafter\def\expandafter\insertshorttitle\expandafter{%
    \raggedleft \insertframenumber\,/\,\inserttotalframenumber\;}

\begin{document}
\author{Jim 2martens}
\title{WIP}
\date{26. Januar 2016}

\begin{frame}
  \titlepage
\end{frame}

\begin{frame}{Motivation}
  \begin{itemize}
    \item Fachschaft hat ein Wiki
    \item super Sache
    \item viele nützliche Dinge stehen drin
    \item kaum genutzt
  \end{itemize}
\end{frame}

\begin{frame}{Motivation}
  \begin{itemize}
    \item veraltete Seiten
    \item falsche Fakten
    \item fehlende Infos
    \item mangelhafte inhaltliche Wartung
  \end{itemize}
\end{frame}

\begin{frame}[fragile]{Wiki Improvement Program}
  \begin{itemize}
    \item veraltete/zu überarbeitende Seiten taggen
      \begin{verbatim}
        {{Vorlage:Überarbeitung}}
      \end{verbatim}
      \begin{itemize}
        \item Diskussionsseite anlegen, falls noch nicht geschehen
        \item zu erledigende Arbeitsschritte mit Aufwandsschätzung (siehe nächste Folie) angeben
      \end{itemize}
    \item Seiten in der Kategorie Überarbeitung überarbeiten
  \end{itemize} 
\end{frame}

\begin{frame}{Wie überarbeiten?}
  \begin{itemize}
    \item auf Diskussionseite einer getaggten Seite gehen und Tickets abarbeiten
    \item nach Abarbeitung einen Kommentar hinterlassen mit Signatur
    \item vier Stufen von Aufwand:
    \begin{description}
      \item[1 Ticket] wenig Aufwand, schnell machbar
      \item[2 Tickets] etwas Aufwand, keine bis wenig Kreativität gefordert
      \item[3 Tickets] größerer Aufwand, meist komplette Aktualisierung eines Abschnittes
      \item[4 Tickets] größerer Aufwand, meist komplette Aktualisierung einer Seite oder Neufassung einer umfangreichen Seite
    \end{description}
  \end{itemize}
\end{frame}

\begin{frame}{Wettbewerb}
  \begin{itemize}
    \item läuft ab jetzt bis zur nächsten VV
    \item frühere Editierungen zählen nicht
    \item zwei Preise
    \begin{itemize}
      \item Person mit meisten bearbeiteten Tickets
      \item Person, die am meisten Seiten getagged hat
    \end{itemize}
    \item Voraussetzung: bearbeitender Wiki-Account kann euch zugeordnet werden
    \item Belohnung ist je Preis eine Kekspackung
  \end{itemize}
\end{frame}

\begin{frame}{Ende}
  \centering
  Frohes Editieren!
\end{frame}
\end{document}