\RequirePackage{pdf14}
\documentclass{beamer}
\usepackage[T1]{fontenc}
\usepackage[utf8]{inputenc}
\usepackage[ngerman]{babel}
%\usepackage{paralist}
%\useoutertheme{infolines} 
\usepackage{graphicx}
\usepackage{hyperref}
\usepackage{listings}
\usepackage{color}
\usetheme{Warsaw}
\usecolortheme{crane}
\pagenumbering{arabic}
\def\thesection{\arabic{section})}
\def\thesubsection{\alph{subsection})}
\def\thesubsubsection{(\roman{subsubsection})}
\setbeamertemplate{navigation symbols}{}
\graphicspath{ {src/} {/home/jim/Pictures/Studium/KBS/} }

\definecolor{mygreen}{rgb}{0,0.6,0}
\definecolor{mygray}{rgb}{0.5,0.5,0.5}
\definecolor{mymauve}{rgb}{0.58,0,0.82}

\lstset{ %
  backgroundcolor=\color{white},   % choose the background color; you must add \usepackage{color} or \usepackage{xcolor}
  basicstyle=\footnotesize,        % the size of the fonts that are used for the code
  breakatwhitespace=false,         % sets if automatic breaks should only happen at whitespace
  breaklines=true,                 % sets automatic line breaking
  captionpos=b,                    % sets the caption-position to bottom
  commentstyle=\color{mygray},    % comment style
  deletekeywords={},            % if you want to delete keywords from the given language
  escapeinside={\%*}{*)},          % if you want to add LaTeX within your code
  extendedchars=true,              % lets you use non-ASCII characters; for 8-bits encodings only, does not work with UTF-8
  keepspaces=true,                 % keeps spaces in text, useful for keeping indentation of code (possibly needs columns=flexible)
  keywordstyle=\color{blue},       % keyword style
  language=PHP,                 % the language of the code
  morekeywords={class, function, return, protected, public, private, const, static, new, extends, namespace, null},            % if you want to add more keywords to the set
  numbers=left,                    % where to put the line-numbers; possible values are (none, left, right)
  numbersep=5pt,                   % how far the line-numbers are from the code
  numberstyle=\tiny\color{mygray}, % the style that is used for the line-numbers
  rulecolor=\color{black},         % if not set, the frame-color may be changed on line-breaks within not-black text (e.g. comments (green here))
  showspaces=false,                % show spaces everywhere adding particular underscores; it overrides 'showstringspaces'
  showstringspaces=false,          % underline spaces within strings only
  showtabs=false,                  % show tabs within strings adding particular underscores
  stepnumber=2,                    % the step between two line-numbers. If it's 1, each line will be numbered
  stringstyle=\color{mygreen},     % string literal style
  tabsize=2,                       % sets default tabsize to 2 spaces
  title=\lstname                   % show the filename of files included with \lstinputlisting; also try caption instead of title
}

\hypersetup{
    pdfauthor=Jim Martens,
    pdfstartview=Fit
}

\expandafter\def\expandafter\insertshorttitle\expandafter{%
    \raggedleft \insertframenumber\,/\,\inserttotalframenumber\;}

\begin{document}
\author{Jim 2martens}
\title{Eine Hochbahnminute}
\date{30. Mai 2016}

\begin{frame}
  \titlepage
\end{frame}

\begin{frame}{US-Vorwahlkampf: Republikaner}
  \centering
  Donald "`I'm a fascist"' Trump
\end{frame}

\begin{frame}{US-Vorwahlkampf: Demokraten}
  Anzahl pledged delegates (as of now)
  \begin{itemize}
    \item Sanders: 1448
    \item Clinton: 1750
    \item Differenz: 302
  \end{itemize}

  Verbleibende pledged delegates (noch nicht verteilt)
  \begin{itemize}
    \item Washington D.C.: 20
    \item California: 475
    \item Montana: 21
    \item New Jersey: 126
    \item North Dakota: 18
    \item New Mexico: 34
    \item South Dakota: 20
    \item Puerto Rico: 60
    \item Virgin Islands: 7
    \item Summe: 781
  \end{itemize}

  Nötige Delegierte für Nominierung: 2382
\end{frame}

\begin{frame}{Konsequenzen}
  \begin{itemize}
    \item Sanders kann nicht genügend pledged delegates für dieses Quorum bekommen
    \item Clinton braucht dafür noch 632 pledged delegates, was ebenso unwahrscheinlich ist
    \item ergo: contested convention, wo Superdelegates entscheiden
  \end{itemize}
\end{frame}

\begin{frame}{Superdelegates}
  \begin{itemize}
    \item insgesamt 712 Delegierte
    \item sind völlig frei in ihrer Wahl auf der Convention
    \item wenn sie wichtig sind, hängt es davon ab, wer mehr pledged delegates hat
    \item dies kann noch immer Bernie Sanders werden
    \item hat Sanders mehr und wird nicht gewählt ist die Partei am Ende
  \end{itemize}
\end{frame}

\begin{frame}{Fazit}
  \centering
  Eine Hochbahnminute dauert oft länger als man denkt.
\end{frame}
\end{document}