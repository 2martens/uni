\RequirePackage{pdf14}
\documentclass{beamer}
\usepackage[T1]{fontenc}
\usepackage[utf8]{inputenc}
\usepackage[ngerman]{babel}
%\usepackage{paralist}
%\useoutertheme{infolines} 
\usepackage{graphicx}
\usepackage{hyperref}
\usepackage{listings}
\usepackage{color}
\usetheme{Warsaw}
\usecolortheme{crane}
\pagenumbering{arabic}
\def\thesection{\arabic{section})}
\def\thesubsection{\alph{subsection})}
\def\thesubsubsection{(\roman{subsubsection})}
\setbeamertemplate{navigation symbols}{}
\graphicspath{ {src/} {/home/jim/Pictures/Studium/KBS/} }

\definecolor{mygreen}{rgb}{0,0.6,0}
\definecolor{mygray}{rgb}{0.5,0.5,0.5}
\definecolor{mymauve}{rgb}{0.58,0,0.82}

\lstset{ %
  backgroundcolor=\color{white},   % choose the background color; you must add \usepackage{color} or \usepackage{xcolor}
  basicstyle=\footnotesize,        % the size of the fonts that are used for the code
  breakatwhitespace=false,         % sets if automatic breaks should only happen at whitespace
  breaklines=true,                 % sets automatic line breaking
  captionpos=b,                    % sets the caption-position to bottom
  commentstyle=\color{mygray},    % comment style
  deletekeywords={},            % if you want to delete keywords from the given language
  escapeinside={\%*}{*)},          % if you want to add LaTeX within your code
  extendedchars=true,              % lets you use non-ASCII characters; for 8-bits encodings only, does not work with UTF-8
  keepspaces=true,                 % keeps spaces in text, useful for keeping indentation of code (possibly needs columns=flexible)
  keywordstyle=\color{blue},       % keyword style
  language=PHP,                 % the language of the code
  morekeywords={class, function, return, protected, public, private, const, static, new, extends, namespace, null},            % if you want to add more keywords to the set
  numbers=left,                    % where to put the line-numbers; possible values are (none, left, right)
  numbersep=5pt,                   % how far the line-numbers are from the code
  numberstyle=\tiny\color{mygray}, % the style that is used for the line-numbers
  rulecolor=\color{black},         % if not set, the frame-color may be changed on line-breaks within not-black text (e.g. comments (green here))
  showspaces=false,                % show spaces everywhere adding particular underscores; it overrides 'showstringspaces'
  showstringspaces=false,          % underline spaces within strings only
  showtabs=false,                  % show tabs within strings adding particular underscores
  stepnumber=2,                    % the step between two line-numbers. If it's 1, each line will be numbered
  stringstyle=\color{mygreen},     % string literal style
  tabsize=2,                       % sets default tabsize to 2 spaces
  title=\lstname                   % show the filename of files included with \lstinputlisting; also try caption instead of title
}

\hypersetup{
	pdfauthor=Jim Martens,
	pdfstartview=Fit
}

\expandafter\def\expandafter\insertshorttitle\expandafter{%
	\raggedleft \insertframenumber\,/\,\inserttotalframenumber\;}

\begin{document}
\author{Jim 2martens}
\title{Panem - EU - Nazi-Deutschland}
\date{28. November 2014}

	% Introduction
	\begin{frame}
		\titlepage
	\end{frame}
	
	\section{Panem}
	\begin{frame}{Panem Struktur}
		\begin{itemize}
			\item Capitol
			\item Distrikte
			\item Diktatur
			\item Peacekeepers
			\item jährliche Hunger Games
			\item alle 25 Jahre Quarter Quells
			\item Dystopie
		\end{itemize}
	\end{frame}
	
	\begin{frame}{Panem Ideologie}
		\begin{itemize}
			\item reiche Oberschicht im Capitol ("`bessere"' Menschen)
			\item verteidigt durch riesige Schutzmaßnahmen
			\item arme Leute in Distrikten ("`schlechtere"' Menschen)
			\item dem Capitol schutzlos ausgeliefert und eingezäunt
			\item "`Distrikte versorgen das Capitol, im Gegenzug bietet dieses Ordnung und Sicherheit"' (Präsident Snow)
		\end{itemize}
	\end{frame}
	
	\section{EU}
	\begin{frame}{EU Struktur}
		\begin{itemize}
			\item EU Kommission
			\item nicht demokratisch gewählt
			\item Frontex (EU-Grenzschutzagentur)
			\item Schutzzäune um Festung Europa
			\item bisher sind 20.000 Menschen an der EU-Außengrenze gestorben
			\item in 40 Jahren sind 800 Menschen an der DDR-Grenze gestorben
			\item DDR := Unrechtsregime
			\item EU := Friedensnobelpreisträger
		\end{itemize}
		
		Quelle: ZDF Die Anstalt 18.11.2014
	\end{frame}
	\begin{frame}{EU Struktur}
		\centering
		\includegraphics[scale=0.9]{eu-wall-spain-africa}
		
		Grenzzaun in spanischer Exklave in Afrika
	\end{frame}
	\begin{frame}{EU Struktur}
		\centering
		\includegraphics[scale=0.4]{eu-wall-greece}
		
		Grenzzaun in Griechenland
	\end{frame}
	\begin{frame}{EU Struktur}
		\centering
		\includegraphics[scale=0.6]{eu-wall-bulgaria}
		
		Grenzzaun in Bulgarien
	\end{frame}
	\begin{frame}{EU Struktur}
		\centering
		\includegraphics[scale=0.7]{eu-wall-bulgaria-turkey}
		
		Grenzzaun zwischen Bulgarien und Türkei
	\end{frame}
	\begin{frame}{EU Ideologie}
		\begin{itemize}
			\item in Europa geborene Menschen (offenbar "`bessere"' Menschen)
			\item können sich frei bewegen
			\item als Flüchtlinge nach Europa kommende Menschen (offenbar "`schlechtere"' Menschen)
			\item sterben oder werden gefangen genommen und der "`Rückführung zugeführt"'
		\end{itemize}
		
		Quelle: ZDF Die Anstalt 18.11.2014
	\end{frame}
	
	\section{Nazi-Deutschland}
	\begin{frame}{Nazi-DEU Struktur}
		\begin{itemize}
			\item Diktatur
			\item Konzentrations- und Vernichtungslager
			\item industrielle Vernichtung und Verarbeitung von Menschen
			\item zentralisiert
			\item Gaue als Verwaltungseinheiten
		\end{itemize}
	\end{frame}
	\begin{frame}{Nazi-DEU Ideologie}
		\begin{itemize}
			\item linientreue "`Arier"' ("`bessere"' Menschen)
			\item keine Einschränkungen
			\item Andere: z.B. politische Gegner, Juden, Homosexuelle, Behinderte, Roma ("`schlechtere"' Menschen)
			\item politisch verfolgt und vernichtet
		\end{itemize}
	\end{frame}
	
	\section{Fazit}
	\begin{frame}{Fazit}
		\begin{itemize}
			\item alle drei Systeme haben im Grunde die gleiche Ideologie
			\item "`bessere"' und "`schlechtere"' Menschen
			\item Rhetorik und Mittel sind unterschiedlich
		\end{itemize}
	\end{frame}
	\begin{frame}{Konsequenzen}
		\begin{itemize}
			\item Geht wählen!
			\item Beteiligt euch an politischen Diskussionen!
			\item Verändert die Welt zum Besseren, damit Panem niemals Realität werden und die Vergangenheit nicht zurückkommen kann!
			\item Werdet ein mündiger Bürger und entsagt der Politikverdrossenheit, denn sie führt unweigerlich zur Diktatur.
		\end{itemize}
	\end{frame}
\end{document}

