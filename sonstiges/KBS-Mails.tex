\documentclass{beamer}
\usepackage[T1]{fontenc}
\usepackage[utf8]{inputenc}
\usepackage[ngerman]{babel}
%\usepackage{paralist}
\useoutertheme{infolines} 
\usepackage{graphicx}
\usepackage{hyperref}
\usepackage{listings}
%\usetheme{Berkeley}
\pagenumbering{arabic}
\def\thesection{\arabic{section})}
\def\thesubsection{\alph{subsection})}
\def\thesubsubsection{(\roman{subsubsection})}
\setbeamertemplate{navigation symbols}{}
\graphicspath{ {src/} {/home/jim/Pictures/} }
%\lstset{language=PHP}
\setbeamertemplate{frametitle}[default][center]


\hypersetup{
	pdfauthor=Jim Martens
}

\begin{document}
\author{Jim Martens}
\title{E-Mails vernünftig einrichten}
\date{14. April 2014}

	% Introduction
	\begin{frame}
		\titlepage
	\end{frame}
	
	\begin{frame}{Motivation}
		Studentin A:
		\begin{quotation}
			Das KBS findet jetzt montags statt?
		\end{quotation}
		
		Student B:
		\begin{quotation}
			In den Semesterferien lese ich meine Mails nicht. Ich bin momentan immer noch ein bisschen in dem Modus.
		\end{quotation}
		
		Student C:
		\begin{quotation}
			Ich komme gar nicht dazu die Mails zu lesen.
		\end{quotation}
	\end{frame}
	
	\begin{frame}{Motivation}
		\centering
		\only<1>{\small Wie kann das sein?}
		\only<2>{\Large Wie kann das sein?}
		\only<3>{\Huge Wie kann das sein?}
	\end{frame}
	
	{
	\setbeamertemplate{footline}{}
	\begin{frame}
		\includegraphics[scale=0.25]{Thunderbird-KBS.png}
	\end{frame}
	}
	
	\begin{frame}{Lösung}
		\begin{enumerate}[1.]
			\item Mailclient installieren (falls noch nicht geschehen)
			\item @inf-Adresse hinzufügen (falls noch nicht geschehen)
			\item auf webmail.informatik.uni-hamburg.de 1 Ordner/Mailingliste anlegen
			\item dort Filterregeln für Mailinglisten anlegen (falls noch nicht geschehen)
			\item Mailclient mind. einmal täglich (sofern praktisch möglich + Internet vorhanden) öffnen
			\item Mails checken \underline{und} ungelesene Mails lesen
		\end{enumerate}
	\end{frame}
	
	\begin{frame}{Lösung}
		\begin{itemize}
			\item nützliches Tool: wmctrl (\texttt{sudo apt-get install wmctrl})
			\item listet Desktops auf (aktueller mit Stern versehen): \texttt{wmctrl -d}
			\item Skript schreiben, das benötigte Anwendungen automatisch startet und Desktops zuordnet (mithilfe von wmctrl)
		\end{itemize}		
	\end{frame}
	
	\begin{frame}[fragile]
		\frametitle{Beispiel}
		\texttt{wmctrl -d} Output
		\begin{verbatim}
			0  - DG: 1366x768  VP: N/A  WA: 0,0 1366x743  MAIN
1  - DG: 1366x768  VP: N/A  WA: 0,0 1366x743  MAIL
2  * DG: 1366x768  VP: 0,0  WA: 0,0 1366x743  TERMINAL
3  - DG: 1366x768  VP: N/A  WA: 0,0 1366x743  CHAT
4  - DG: 1366x768  VP: N/A  WA: 0,0 1366x743  OTHER
		\end{verbatim}
	\end{frame}
	
	\begin{frame}[fragile]
		\frametitle{Beispiel - 2}
		\begin{verbatim}			
firefox &
thunderbird &
pidgin &
sleep 20
gnome-terminal &
sleep 15
wmctrl -r "Terminal" -N "TermLeft"
wmctrl -r "TermLeft" -e 0,6,48,677,716
wmctrl -r "TermLeft" -t 2
wmctrl -r "Mozilla Thunderbird" -e 0,0,48,1366,718
wmctrl -r "Buddy List" -e 0,689,48,677,716
wmctrl -r "Star Citizen" -e 0,0,48,677,716
wmctrl -r "Mozilla Thunderbird" -t 1
wmctrl -r "Buddy List" -t 3
		\end{verbatim}
	\end{frame}
	
	\begin{frame}[fragile]
		\frametitle{Beispiel - 3}
		\begin{verbatim}
wmctrl -s 1	
sleep 2
wmctrl -r "Star Citizen" -t 3
wmctrl -r "Terminal" -N "TermRight"
wmctrl -r "TermRight" -e 0,689,48,677,716
wmctrl -r "TermRight" -t 2
echo "Done"
		\end{verbatim}
	\end{frame}
\end{document}

