\documentclass{beamer}
\usepackage[T1]{fontenc}
\usepackage[utf8]{inputenc}
\usepackage[ngerman]{babel}
%\usepackage{paralist}
\useoutertheme{infolines} 
\usepackage{graphicx}
\usepackage{hyperref}
%\usepackage{listings}
%\usetheme{Berkeley}
\pagenumbering{arabic}
\def\thesection{\arabic{section})}
\def\thesubsection{\alph{subsection})}
\def\thesubsubsection{(\roman{subsubsection})}
\setbeamertemplate{navigation symbols}{}
\graphicspath{ {src/} {/home/jim/Pictures/} }
%\lstset{language=PHP}
\setbeamertemplate{frametitle}[default][center]


\hypersetup{
	pdfauthor=Jim Martens
}

\begin{document}
\author{Jim Martens}
\title{E-Mails vernünftig einrichten}
\date{14. April 2014}

	% Introduction
	\begin{frame}
		\titlepage
	\end{frame}
	
	\begin{frame}{Motivation}
		Studentin A:
		\begin{quotation}
			Das KBS findet jetzt montags statt?
		\end{quotation}
		
		Student B:
		\begin{quotation}
			In den Semesterferien lese ich meine Mails nicht. Ich bin momentan immer noch ein bisschen in dem Modus.
		\end{quotation}
		
		Student C:
		\begin{quotation}
			Ich komme gar nicht dazu die Mails zu lesen.
		\end{quotation}
	\end{frame}
	
	\begin{frame}{Motivation}
		\centering
		\only<1>{\small Wie kann das sein?}
		\only<2>{\Large Wie kann das sein?}
		\only<3>{\Huge Wie kann das sein?}
	\end{frame}
	
	{
	\setbeamertemplate{footline}{}
	\begin{frame}
		\includegraphics[scale=0.25]{Thunderbird-KBS.png}
	\end{frame}
	}
	
	
\end{document}

