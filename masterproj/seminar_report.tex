\documentclass[12pt]{scrartcl}

%%%%%%%%%%%%%%%%%%%%%%%%%%%%%%%%%%%%%%%%%%%%%%%%%%%%%%%%%%%%%
% Languages:

% Falls die Ausarbeitung in Deutsch erfolgt:
% \usepackage[german]{babel}
% \usepackage[T1]{fontenc}
% \usepackage[latin1]{inputenc}
% \usepackage[latin9]{inputenc}
% \selectlanguage{german}

% If the thesis is written in English:
\usepackage[spanish,english]{babel}
\selectlanguage{english}

%%%%%%%%%%%%%%%%%%%%%%%%%%%%%%%%%%%%%%%%%%%%%%%%%%%%%%%%%%%%%
% Bind packages:
\usepackage[utf8]{inputenc} % Unicode funktioniert unter Windows, Linux und Mac
\usepackage[T1]{fontenc}
\usepackage{acronym}                    % Acronyms
\usepackage{algorithmic}								% Algorithms and Pseudocode
\usepackage{algorithm}									% Algorithms and Pseudocode
\usepackage{amsfonts}                   % AMS Math Packet (Fonts)
\usepackage{amsmath}                    % AMS Math Packet
\usepackage{amssymb}                    % Additional mathematical symbols
\usepackage{amsthm}
\usepackage{booktabs}                   % Nicer tables
%\usepackage[font=small,labelfont=bf]{caption} % Numbered captions for figures
\usepackage{color}                      % Enables defining of colors via \definecolor
\definecolor{uhhRed}{RGB}{254,0,0}		  % Official Uni Hamburg Red
\definecolor{uhhGrey}{RGB}{122,122,120} % Official Uni Hamburg Grey
\usepackage{fancybox}                   % Gleichungen einrahmen
%\usepackage{fancyhdr}										% Packet for nicer headers
\usepackage[automark]{scrlayer-scrpage}
\usepackage[hidelinks]{hyperref}\urlstyle{rm}
%\usepackage{fancyheadings}             % Nicer numbering of headlines

%\usepackage[outer=3.35cm]{geometry} 	  % Type area (size, margins...) !!!Release version
%\usepackage[outer=2.5cm]{geometry} 		% Type area (size, margins...) !!!Print version
%\usepackage{geometry} 									% Type area (size, margins...) !!!Proofread version
\usepackage[outer=3.15cm]{geometry} 	  % Type area (size, margins...) !!!Draft version
\geometry{a4paper,body={5.8in,9in}}

\usepackage{graphicx}                   % Inclusion of graphics
%\usepackage{latexsym}                  % Special symbols
\usepackage{longtable}									% Allow tables over several parges
\usepackage{listings}                   % Nicer source code listings
\usepackage{multicol}										% Content of a table over several columns
\usepackage{multirow}										% Content of a table over several rows
\usepackage{rotating}										% Alows to rotate text and objects
\usepackage[hang]{subfigure}            % Allows to use multiple (partial) figures in a fig
%\usepackage[font=footnotesize,labelfont=rm]{subfig}	% Pictures in a floating environment
\usepackage{tabularx}										% Tables with fixed width but variable rows
\usepackage{url,xspace,boxedminipage}   % Accurate display of URLs

\usepackage{csquotes}
\usepackage[
backend=biber,
bibstyle=ieee,
citestyle=ieee,
minnames=1,
maxnames=2
]{biblatex}
%%%%%%%%%%%%%%%%%%%%%%%%%%%%%%%%%%%%%%%%%%%%%%%%%%%%%%%%%%%%%
% Configurationen:

\hyphenation{whe-ther} 									% Manually use: "\-" in a word: Staats\-ver\-trag

%\lstloadlanguages{C}                   % Set the default language for listings
\DeclareGraphicsExtensions{.pdf,.svg,.jpg,.png,.eps} % first try pdf, then eps, png and jpg
\graphicspath{{./src/}} 								% Path to a folder where all pictures are located
%\pagestyle{fancy} 											% Use nicer header and footer
\pagestyle{scrheadings}

\addbibresource{bib.bib}

\begin{document}

\title{Master project: seminar report template}
\author{Jim Martens}

\maketitle
\section*{Abstract}

The short abstract (100-150 words) is intended to give the reader an overview of the paper and your general opinion about the paper.


% Lists:
\setcounter{tocdepth}{2} 					% depth of the table of contents (for Seminars 2 is recommented)
\tableofcontents
\pagenumbering{arabic}
\clearpage

\section{Introduction}
Use this template as a starting point for preparing your seminar report.
For more information on \LaTeX, please consult, e.g., the online book at \url{https://en.wikibooks.org/wiki/LaTeX}.
Refer also to material on scientific writing.
The length of the report should not exceed 10 pages (excluding the reference list).

This part contains the introduction to the topic.
It introduces the general problem area of the paper, and leads the reader to the next section that provides more details.
This part should also cite other related work (not only the seminar paper you are working on) and compare the approaches on a high level.

\section{Method description}
This section describes the proposed approach in the paper in more detail.
Do not take sections directly from the paper, provide your own understanding and description.


\section{Experimental result and evaluation}
In this section, the evaluation and experimental results of proposed method should be described.
Also provide some discussion, answering questions such as: when does the method work well, when not? How does it compare to other state-of-the-art works?

\section{Discussion} % (fold)
\label{sec:discussion}
After providing the details of the paper, this secion contains your persnal opinion regarding the mothd that was proposed in this paper.
\subsection{Paper Strengths} % (fold)
\label{sub:paper_strengths}
Please discuss, justifying your comments with the appropriate level of details, the strengths of the paper
% subsection positive_aspect (end)

\subsection{Paper Weaknesses} % (fold)
\label{sub:paper_weaknesses}
Please discuss, justifying your comments with the appropriate level of details, the weaknesses of the paper
% subsection negitive (end)

% section review (end)

\section{Conclusion}
Summarize your report.
Provide some concluding discussion about the paper, along with, e.g., suggestions for future work.


\newpage
\printbibliography
\addcontentsline{toc}{section}{Bibliography}% Add to the TOC

\end{document}
