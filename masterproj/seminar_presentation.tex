\RequirePackage{pdf14}
\documentclass{beamer}
\usepackage[T1]{fontenc}
\usepackage[utf8]{inputenc}
\usepackage[english]{babel}
%\usepackage{paralist}
%\useoutertheme{infolines}
\usepackage{graphicx}
\usepackage{hyperref}
\usepackage{listings}
\usepackage{color}
\usepackage{textcomp}
\usepackage{csquotes}
\usetheme{Warsaw}
\usecolortheme{crane}
\pagenumbering{arabic}
\def\thesection{\arabic{section})}
\def\thesubsection{\alph{subsection})}
\def\thesubsubsection{(\roman{subsubsection})}
\setbeamertemplate{navigation symbols}{}
\graphicspath{ {src/} {/home/jim/Pictures/} }

\definecolor{mygreen}{rgb}{0,0.6,0}
\definecolor{mygray}{rgb}{0.5,0.5,0.5}
\definecolor{mymauve}{rgb}{0.58,0,0.82}

\usepackage[
backend=biber,
bibstyle=ieee,
citestyle=ieee,
minnames=1,
maxnames=2
]{biblatex}

\addbibresource{bib.bib}

\MakeOuterQuote{"}

%\definecolor{craneorange}{RGB}{61,61,61}
%\definecolor{craneblue}{RGB}{255,255,255}

\lstset{ %
  backgroundcolor=\color{white},   % choose the background color; you must add \usepackage{color} or \usepackage{xcolor}
  basicstyle=\footnotesize,        % the size of the fonts that are used for the code
  breakatwhitespace=false,         % sets if automatic breaks should only happen at whitespace
  breaklines=true,                 % sets automatic line breaking
  captionpos=b,                    % sets the caption-position to bottom
  commentstyle=\color{mygray},    % comment style
  deletekeywords={},            % if you want to delete keywords from the given language
  escapeinside={\%*}{*)},          % if you want to add LaTeX within your code
  extendedchars=true,              % lets you use non-ASCII characters; for 8-bits encodings only, does not work with UTF-8
  keepspaces=true,                 % keeps spaces in text, useful for keeping indentation of code (possibly needs columns=flexible)
  keywordstyle=\color{blue},       % keyword style
  language=PHP,                 % the language of the code
  morekeywords={class, function, return, protected, public, private, const, static, new, extends, namespace, null},            % if you want to add more keywords to the set
  numbers=left,                    % where to put the line-numbers; possible values are (none, left, right)
  numbersep=5pt,                   % how far the line-numbers are from the code
  numberstyle=\tiny\color{mygray}, % the style that is used for the line-numbers
  rulecolor=\color{black},         % if not set, the frame-color may be changed on line-breaks within not-black text (e.g. comments (green here))
  showspaces=false,                % show spaces everywhere adding particular underscores; it overrides 'showstringspaces'
  showstringspaces=false,          % underline spaces within strings only
  showtabs=false,                  % show tabs within strings adding particular underscores
  stepnumber=2,                    % the step between two line-numbers. If it's 1, each line will be numbered
  stringstyle=\color{mygreen},     % string literal style
  tabsize=2,                       % sets default tabsize to 2 spaces
  title=\lstname                   % show the filename of files included with \lstinputlisting; also try caption instead of title
}

\hypersetup{
	pdfauthor=Jim Martens,
	pdfstartview=Fit
}

\expandafter\def\expandafter\insertshorttitle\expandafter{%
	\raggedleft \insertframenumber\,/\,\inserttotalframenumber\;}

\begin{document}
\author{Jim 2martens}
\title{Deep Sliding Shapes: A Review}
\date{July 4th, 2018}

\begin{frame}
    \titlepage
\end{frame}

\begin{frame}{Contents}
    \tableofcontents
\end{frame}

\section{Motivation}
\begin{frame}{Task}
    \begin{itemize}
        \item object detection is central task for neural networks
        \vfill
        \item combination of classification and localization tasks
        \vfill
        \item output are usually bounding boxes and classifications
    \end{itemize}
\end{frame}

\begin{frame}{Field}
    \begin{itemize}
        \item 2D object detection very mature with Single Shot MultiBox Detector\cite{Liu2016}
        \vfill
        \item with more availability of depth data, usage of depth becomes more
              important
        \vfill
        \item early approaches use depth as fourth channel in 2D object detection,
              for example Depth RCNN\cite{Gupta2015}
        \vfill
        \item Deep Sliding Shapes\cite{Song2016} uses 3D data for actual 3D deep
              learning and uses 2D object detectors
    \end{itemize}
\end{frame}

\section{Method}
\begin{frame}{Method}
    \begin{enumerate}
        \item encoding 3D representation and normalization
        \vfill
        \item multi-scale 3D region proposal network
        \vfill
        \item joint amodal object recognition network
    \end{enumerate}
\end{frame}

\begin{frame}{Representation and Normalization}
    \begin{itemize}
        \item raw 3D space divided into equally spaced 3D voxel grid
        \vfill
        \item data encoded by Truncated Signed Distance Function
        \vfill
        \item each voxel stores distance from its center to surface of input depth
              map and direction of each surface point
        \vfill
        \item every scene is rotated to align with gravity direction
        \vfill
        \item major room directions are used for proposal orientations
    \end{itemize}
\end{frame}

\begin{frame}{Region Proposal Network}
    \begin{itemize}
        \item proposes a few interesting regions for the object recognition network
        \vfill
        \item each region proposal corresponds to one anchor box
        \vfill
        \item two scales are used since anchor box size varies a lot (from 0.3
              to 2 meters)
        \vfill
        \item a full 3D convolutional architecture is used
        \vfill
        \item after the calculation of the region proposals multiple bars have
              to be met for regions for them to be proposed
        \vfill
        \item in the end only the top 2000 regions move on (after the convolution
              with only dropping all regions with point density lower than 0.005
              points per cubic centimeter a total of 107674 regions remain on average)
    \end{itemize}
\end{frame}

\begin{frame}{Object Recognition Network}
    \begin{itemize}
        \item starts with both 3D and 2D object recognition networks
        \vfill
        \item VGGnet pretrained on ImageNet is used for extracting colour features
        \vfill
        \item resulting feature vectors of both networks are concatenated
        \vfill
        \item at the end two separate fully connected layers predict object label
              and 3D bounding box
        \vfill
        \item some outlier protection measures are applied
    \end{itemize}
\end{frame}

\section{Experimental Results}
\begin{frame}{Evaluation}
    \begin{itemize}
        \item evaluated on NYUv2\cite{Silberman2012} and SUN RGB-D\cite{Song2015}
        \vfill
        \item threshold of 0.25 used for average recall of proposal generation and
              average precision of detection
        \vfill
        \item ground truth bounding boxes obtained from SUN RGB-D
        \vfill
        \item single-scale RPN, multi-scale RPN and multi-scale RPN with RGB colour
              usage (RGB colour encoded in 3D TSDF) were compared against each
              other and the baselines using the NYU data set
        \vfill
        \item 3D selective search and naive 2D to 3D conversion used as baselines
        \vfill
        \item second experiment tested ORN with different region proposals
    \end{itemize}
\end{frame}

\begin{frame}{Results}
    \begin{itemize}
        \item works well on non-planar objects with depth information
        \vfill
        \item 2D component helps in distinguishing similar shaped objects
        \vfill
        \item 3D Deep Sliding Shapes outperforms chosen state-of-the-art methods
    \end{itemize}
\end{frame}

\section{Review}
\begin{frame}{Review}
    \begin{itemize}
        \item idea to use 3D data directly very intruiging
        \vfill
        \item high-level structure of region proposal followed by object recognition
              is visible in more recent approaches like Frustum Pointnet\cite{Qi2017}
              as well
        \vfill
        \item motivations for used data sets NYUv2 and SUN RGB-D unclear
        \vfill
        \item no information on process of "obtaining" ground truth bounding boxes
              from SUN RGB-D data set
        \vfill
        \item no implementation details provided
    \end{itemize}
\end{frame}

\section{Conclusion}
\begin{frame}{Conclusion}
    \begin{itemize}
        \item 3D Deep Sliding Shapes should be compared to other 3D centric approaches
              like Frustum Pointnet
        \vfill
        \item structural comparison with other 3D approaches is interesting
    \end{itemize}
\end{frame}

\begin{frame}[allowframebreaks]{References}
    \printbibliography
\end{frame}
\end{document}
