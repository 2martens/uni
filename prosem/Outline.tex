\documentclass[10pt,a4paper,oneside,english,numbers=noenddot,titlepage]{scrartcl}
\usepackage[T1]{fontenc}
\usepackage[utf8]{inputenc}
\usepackage[english]{babel}
\usepackage{amsmath}
\usepackage{amsfonts}
\usepackage{amssymb}
\usepackage{paralist}
\usepackage{gauss}
\usepackage{pgfplots}
\usepackage[locale=DE,exponent-product=\cdot,detect-all]{siunitx}
\usepackage{tikz}
\usetikzlibrary{matrix,fadings,calc,positioning,decorations.pathreplacing,arrows,decorations.markings}
\usepackage{polynom}
\polyset{style=C, div=:,vars=x}
\pgfplotsset{compat=1.8}
\pagenumbering{arabic}
% ensures that paragraphs are separated by empty lines
\parskip 12pt plus 1pt minus 1pt
\parindent 0pt
% define how the sections are rendered
%\def\thesection{\arabic{section})}
%\def\thesubsection{\alph{subsection})}
%\def\thesubsubsection{(\roman{subsubsection})}
% some matrix magic
\makeatletter
\renewcommand*\env@matrix[1][*\c@MaxMatrixCols c]{%
  \hskip -\arraycolsep
  \let\@ifnextchar\new@ifnextchar
  \array{#1}}
\makeatother
\addto{\captionsenglish}{\renewcommand{\refname}{Bibliography}}

\begin{document}
\author{Jim Martens}
\title{Outline about ``With what methods can we understand natural language to build dialog systems?''}
%\title{Outline about "Mit welchen Methoden können wir natürliche Sprache verstehen um Dialogsysteme aufzubauen?"}
\maketitle
\section*{Abstract}
	This is a placeholder for the abstract.
\tableofcontents
\clearpage

\section{Introduction}
	\begin{itemize}
		\item	two kinds of natural language: spoken language and written language
		\item	will concentrate on written language
		\item	important method for written language: parsing
		\item	different approaches for the kind of grammar being used
	\end{itemize}
\section{Evaluation of approaches}
	\subsection{CYK, PCFG, lexicalized PCFG, DCG}
		\begin{itemize}
			\item	presents the context-free approach explained by Norvig and Russel\cite{Russel2010}
		\end{itemize}
	\subsection{Link Grammar}
		\begin{itemize}
			\item	presents an alternative to PCFGs; referencing Sleator here\cite{Sleator1993}
		\end{itemize}
	\subsection{Dependency grammar}
		\begin{itemize}
			\item	presents dependency grammar here, referencing Paskin\cite{Paskin2001}
		\end{itemize}
	\subsection{Categorial grammar}
		\begin{itemize}
			\item	presents categorial grammars, using Clark\cite{Clark2004} here
		\end{itemize}
		
\section{Critical discussion}
	\begin{itemize}
		\item	compares the presented grammar approaches with each other
	\end{itemize}
	
\section{Conclusion}
	 \begin{itemize}
	 	\item	summarizes the results of the critical discussion
	 	\item	depending on the results: may give an advice which approach is more useful/easier etc.
	 \end{itemize}

\clearpage

\bibliography{prosem-ki}
\bibliographystyle{ieeetr}
\addcontentsline{toc}{section}{Bibliography}
\end{document}