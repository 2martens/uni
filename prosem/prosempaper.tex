\documentclass[12pt,twoside]{scrartcl}
%%%%%%%%%%%%%%%%%%%%%%%%%%%%%%%%%%%%%%%%%%%%%%%%%%%%%%%%%%%%%
% Meta informations:
\newcommand{\trauthor}{Jim Martens}
\newcommand{\trtype}{Proseminar Paper} %{Seminararbeit} %{Proseminararbeit}
\newcommand{\trcourse}{Proseminar Artificial Intelligence}
\newcommand{\trtitle}{Methods for understanding natural language}
\newcommand{\trmatrikelnummer}{6420323}
\newcommand{\tremail}{2martens@informatik.uni-hamburg.de}
\newcommand{\trarbeitsbereich}{Knowledge Technology, WTM}
\newcommand{\trdate}{10.02.2014}

%%%%%%%%%%%%%%%%%%%%%%%%%%%%%%%%%%%%%%%%%%%%%%%%%%%%%%%%%%%%%
% Languages:

% Falls die Ausarbeitung in Deutsch erfolgt:
% \usepackage[german]{babel}
% \usepackage[T1]{fontenc}
% \usepackage[latin1]{inputenc}
% \usepackage[latin9]{inputenc}	 				
% \selectlanguage{german}

% If the thesis is written in English:
\usepackage[english]{babel} 						
\selectlanguage{english}
\addto{\captionsenglish}{\renewcommand{\refname}{Bibliography}}
%%%%%%%%%%%%%%%%%%%%%%%%%%%%%%%%%%%%%%%%%%%%%%%%%%%%%%%%%%%%%
% Bind packages:
\usepackage{acronym}                    % Acronyms
%\usepackage{algorithmic}								% Algorithms and Pseudocode
\usepackage{algpseudocode}
\usepackage{algorithm}									% Algorithms and Pseudocode
\usepackage{amsfonts}                   % AMS Math Packet (Fonts)
\usepackage{amsmath}                    % AMS Math Packet
\usepackage{amssymb}                    % Additional mathematical symbols
\usepackage{amsthm}
\usepackage{booktabs}                   % Nicer tables
%\usepackage[font=small,labelfont=bf]{caption} % Numbered captions for figures
\usepackage{color}                      % Enables defining of colors via \definecolor
\definecolor{uhhRed}{RGB}{254,0,0}		  % Official Uni Hamburg Red
\definecolor{uhhGrey}{RGB}{122,122,120} % Official Uni Hamburg Grey
\usepackage{fancybox}                   % Gleichungen einrahmen
\usepackage{fancyhdr}										% Packet for nicer headers
%\usepackage{fancyheadings}             % Nicer numbering of headlines

%\usepackage[outer=3.35cm]{geometry} 	  % Type area (size, margins...) !!!Release version
%\usepackage[outer=2.5cm]{geometry} 		% Type area (size, margins...) !!!Print version
%\usepackage{geometry} 									% Type area (size, margins...) !!!Proofread version
\usepackage[outer=3.15cm]{geometry} 	  % Type area (size, margins...) !!!Draft version
\geometry{a4paper,body={5.8in,9in}}

\usepackage{graphicx}                   % Inclusion of graphics
%\usepackage{latexsym}                  % Special symbols
\usepackage{longtable}									% Allow tables over several parges
\usepackage{listings}                   % Nicer source code listings
\usepackage{multicol}										% Content of a table over several columns
\usepackage{multirow}										% Content of a table over several rows
\usepackage{rotating}										% Alows to rotate text and objects
\usepackage[hang]{subfigure}            % Allows to use multiple (partial) figures in a fig
%\usepackage[font=footnotesize,labelfont=rm]{subfig}	% Pictures in a floating environment
\usepackage{tabularx}										% Tables with fixed width but variable rows
\usepackage{url,xspace,boxedminipage}   % Accurate display of URLs

\usepackage{float}
\floatstyle{boxed}
\restylefloat{figure}

%%%%%%%%%%%%%%%%%%%%%%%%%%%%%%%%%%%%%%%%%%%%%%%%%%%%%%%%%%%%%
% Configurationen:

\hyphenation{whe-ther} 									% Manually use: "\-" in a word: Staats\-ver\-trag
\hyphenation{spe-ci-fies}
\hyphenation{spe-ci-fi-ca-tion}

%\lstloadlanguages{C}                   % Set the default language for listings
\DeclareGraphicsExtensions{.pdf,.svg,.jpg,.png,.eps} % first try pdf, then eps, png and jpg
\graphicspath{{./src/}} 								% Path to a folder where all pictures are located
\pagestyle{fancy} 											% Use nicer header and footer

% Redefine the environments for floating objects:
\setcounter{topnumber}{3}
\setcounter{bottomnumber}{2}
\setcounter{totalnumber}{4}
\renewcommand{\topfraction}{0.9} 			  %Standard: 0.7
\renewcommand{\bottomfraction}{0.5}		  %Standard: 0.3
\renewcommand{\textfraction}{0.1}		  	%Standard: 0.2
\renewcommand{\floatpagefraction}{0.8} 	%Standard: 0.5

% Tables with a nicer padding:
\renewcommand{\arraystretch}{1.2}

%%%%%%%%%%%%%%%%%%%%%%%%%%%%
% Additional 'theorem' and 'definition' blocks:
\theoremstyle{plain}
\newtheorem{theorem}{Theorem}[section]
%\newtheorem{theorem}{Satz}[section]		% Wenn in Deutsch geschrieben wird.
\newtheorem{axiom}{Axiom}[section] 	
%\newtheorem{axiom}{Fakt}[chapter]			% Wenn in Deutsch geschrieben wird.
%Usage:%\begin{axiom}[optional description]%Main part%\end{fakt}

\theoremstyle{definition}
\newtheorem{definition}{Definition}[section]

%Additional types of axioms:
\newtheorem{lemma}[axiom]{Lemma}
\newtheorem{observation}[axiom]{Observation}

%Additional types of definitions:
\theoremstyle{remark}
%\newtheorem{remark}[definition]{Bemerkung} % Wenn in Deutsch geschrieben wird.
\newtheorem{remark}[definition]{Remark} 

%%%%%%%%%%%%%%%%%%%%%%%%%%%%
% Provides TODOs within the margin:
\newcommand{\TODO}[1]{\marginpar{\emph{\small{{\bf TODO: } #1}}}}

%%%%%%%%%%%%%%%%%%%%%%%%%%%%
% Abbreviations and mathematical symbols
\newcommand{\modd}{\text{ mod }}
\newcommand{\RS}{\mathbb{R}}
\newcommand{\NS}{\mathbb{N}}
\newcommand{\ZS}{\mathbb{Z}}
\newcommand{\dnormal}{\mathit{N}}
\newcommand{\duniform}{\mathit{U}}

\newcommand{\erdos}{Erd\H{o}s}
\newcommand{\renyi}{-R\'{e}nyi}
%%%%%%%%%%%%%%%%%%%%%%%%%%%%%%%%%%%%%%%%%%%%%%%%%%%%%%%%%%%%%
% Document:
\begin{document}
\renewcommand{\headheight}{14.5pt}

\fancyhead{}
\fancyhead[LE]{ \slshape \trauthor}
\fancyhead[LO]{}
\fancyhead[RE]{}
\fancyhead[RO]{ \slshape \trtitle}

%%%%%%%%%%%%%%%%%%%%%%%%%%%%
% Cover Header:
\begin{titlepage}
	\begin{flushleft}
		Universit\"at Hamburg\\
		Department Informatik\\
		\trarbeitsbereich\\
	\end{flushleft}
	\vspace{3.5cm}
	\begin{center}
		\huge \trtitle\\
	\end{center}
	\vspace{3.5cm}
	\begin{center}
		\normalsize\trtype\\
		[0.2cm]
		\Large\trcourse\\
		[1.5cm]
		\Large \trauthor\\
		[0.2cm]
		\normalsize Matr.Nr. \trmatrikelnummer\\
		[0.2cm]
		\normalsize\tremail\\
		[1.5cm]
		\Large \trdate
	\end{center}
	\vfill
\end{titlepage}

	%backsite of cover sheet is empty!
\thispagestyle{empty}
\hspace{1cm}
\newpage

%%%%%%%%%%%%%%%%%%%%%%%%%%%%
% Abstract:

% Abstract gives a brief summary of the main points of a paper:
\section*{Abstract}
  Syntactic parsing and semantic analysis are two important methods for understanding natural language. Each of them has their individual strengths and weaknesses. But both of them have major issues with ambiguity once a restricted environment is left. Understanding unrestricted natural language is therefore far from being reached.

% Lists:
\setcounter{tocdepth}{2} 					% depth of the table of contents (for Seminars 2 is recommended)
\tableofcontents
\pagenumbering{arabic}
\clearpage

%%%%%%%%%%%%%%%%%%%%%%%%%%%%
% Content:

% the actual content, usually separated over a number of sections
% each section is assigned a label, in order to be able to put a
% crossreference to it

\section{Introduction}
\label{sec:introduction}

	It's the dream of many Science-Fiction fans: A fully sentient AI. Let's ignore for a moment all the odds that are against it (morality, physics, etc.) and concentrate on one aspect that is mandatory for even much less ambitious dreams. Imagine a computer game in which you can talk natural language to the NPC counterparts so that they react appropriately to it. Well maybe that is still too ambitious. What about writing what you want to say? In that case the computer needs to understand what you are writing so that it can react to it.
	
	The input in this case is plain text, following the grammar of a natural language like English. Without loss of generality it is assumed that the input is syntactically correct and follows the grammar of the natural language. The computer therefore gets a certain amount of text that follows a specified grammar. The grammar of modern English is assumed for the scope of this paper. With this information available, the computer still knows nothing about the meaning of the text. You could ask for a hot chocolate or you could write nasty things, it won't make a difference at this point.
	
	In order to make the computer react properly to your input, it needs to understand and therefore process the input in the first place. This can be achieved by the usage of some methods for natural language understanding. For the scope of this paper ``natural language understanding'' contains all the methods used for understanding natural language. These include both methods to understand written natural language and methods used to understand spoken natural language. This paper looks closer at two of the methods used to understand written language. The first one is the syntactic parsing, the second one the semantic analysis. To understand how these methods work, you need to know the basic terminology of the subject matter. In the following paragraphs the terms syntax, semantics and pragmatics are explained with respect to the two mentioned methods.
	
	The first method syntactic parsing relies on a grammar that describes the set of possible input, also called syntax. The syntax specifies what are allowed sentence structures and how these are built.
	
	The semantic analysis relies on the semantics of a given input. That means what the given input means. An example: ``You run around the bush''. The semantic meaning of this sentence is that you are running around a bush. 
	The pragmatics though define what is the intended meaning of an input. In this example it's not that you run around the bush but actually that you take a long time to get to the point in a discussion. It's a so called idiom. This difference between semantic meaning, where just the sentence as it is written is considered, and pragmatic meaning, where the intended meaning is considered, generates ambiguity that is easy for humans to resolve but difficult for computers. But even the pragmatics in this example are ambiguous, because it depends on the context what it actually means. If two persons are walking around in a forest and one starts running around the bush, the pragmatic meaning of the sentence in this example would be the previously mentioned semantic meaning.
	
	On top of that the semantic meaning itself isn't always clear either. Sometimes words have multiple meanings, so that even the semantic meaning can have different possible interpretations.
	
	The basic terminology should be clear by now. Whenever there are additional prerequisites to understand a method, these are explained in the section of that method.
	
	Before the actual evaluation of the methods starts, the usage of the result of both methods is shortly described. After both syntactic parsing and semantic analysis have been executed, in this order, you have a semantic representation of the input. This representation could be used for example for an interface to a knowledge database where the user just inserts the question and gets an appropriate answer.
	
	But there are other possible use cases as well. The two described methods could be used in a chatbot.
	
	%TODO how can semantic representations be used (add references)
	
	In this paper both syntactic parsing and semantic analysis are presented. After the presentation of the methods, they are critically discussed to finally come to a conclusion.

\section{Evaluation of methods}
\label{sec:evalMethods}
	
	%TODO add more detailed and concrete examples for each method
	
	Syntactic parsing and semantic analysis offer each a broad range of approaches. In this paper the ``syntax-driven semantic analysis''\cite[p.~617]{Jurafsky2009} is evaluated. It's especially interesting because it utilizes the output of the syntactic parsing to analyze the meaning. Therefore the two methods can be lined up in chronological order. First comes the syntactic parsing and then the semantic analysis. The methods are presented here in the same order. 
	
	They will be explained with the help of an example. Let's take the sentence ``The tree is very high''. For every method the theory is introduced first and the practical application with the example comes after it.
	
	\subsection{Syntactic Parsing}
	\label{subSec:syntacticParsing}
		Syntactic Parsing is used to create parse trees. These can be used for grammar checks in a text editor: ``A sentence that cannot be parsed may have grammatical errors''\cite[p.~461]{Jurafsky2009b}. But they more likely ``serve as an important intermediate stage of representation for semantic analysis''\cite[p.~461]{Jurafsky2009b}. There are different algorithms available to create such trees. The CYK\footnote{named after inventors John Cocke, Daniel Younger and Tadeo Kasami\cite[p.~893]{Russel2010}} algorithm will be explained further. But before the CYK algorithm is explained, the reason for its existance is presented.
		
		\begin{figure}
			\begin{alignat*}{2}
				Noun &\rightarrow && \text{tree [1.00]} \\
				Verb &\rightarrow && \text{is [1.00]} \\
				Adjective &\rightarrow && \text{high [0.50]} \;|\; \text{very [0.50]} \\
				Article &\rightarrow && \text{the [1.00]} \\
			\end{alignat*}
			\caption{The lexicon for $\varepsilon_{0}$. The sum of the probabilities for each category is 1.}
			\label{fig:lexicon}
		\end{figure}
		
		\begin{figure}
			\begin{alignat*}{3}
				\varepsilon_{0}:& S \;&\rightarrow &\; NP\;\;VP \;&[1.00]&\; \text{The tree + is very high} \\
				& NP \;&\rightarrow &\; A\;N \;&[1.00]&\; \text{The + tree}\\
				& A \;&\rightarrow &\; Article\;&[1.00]&\; \text{the}\\
				& N \;&\rightarrow &\; Noun\;&[1.00]&\; \text{tree}\\
				& VP \;&\rightarrow &\; Verb \;&[0.40]&\; \text{is} \\
				& \;&|&\; VP\;Adjs \;&[0.60]&\; \text{is + very high} \\
				& Adjs \;&\rightarrow &\; Adjective \;&[0.80]&\; \text{very} \\
				& \;&|&\; Adjs\;Adjs \;&[0.20]&\; \text{very + high}
			\end{alignat*}
		
			\caption{The grammar for $\varepsilon_{0}$ with example phrases for each rule. The syntactic categories are sentence (S), noun phrase (NP), verb phrase (VP), article (A), noun (N) and list of adjectives (Adjs). The categories article and noun have been added to allow a CNF grammar.}
			\label{fig:grammar}
		\end{figure}
		
		\subsubsection*{Bottom-up and Top-down}
		\label{subSubSec:bottomUpTopDown}
		
		There are two classical ways of parsing a sentence. The one is bottom-up and the other one is top-down. Both approaches have their own advantages and disadvantages. In addition the ambiguity creates problems. To implement bottom-up and top-down search algorithms in the face of ambiguity, ``an agenda-based backtracking strategy''\cite[p.~468]{Jurafsky2009b} is used. The problem here is that every time the parser recognizes that the current parse tree is wrong, it has to backtrack and explore other parts of the sentence. This creates a huge amount of work duplication and is therefore inefficient.
		
		\subsubsection*{CYK algorithm}
		\label{subSubSec:cykAlgorithm}		
		
		A solution to these problems is offered by ``dynamic programming parsing methods''\cite[p.~469]{Jurafsky2009b}. The CYK algorithm is one of multiple algorithms based on dynamic programming.
		
		The CYK does only work with grammars in the Chomsky Normal Form (CNF). Every context-free grammar can be converted to CNF without loss in expressiveness. Therefore this restriction does no harm but simplifies the parsing. For information on how context-free grammars can be converted to CNF, refer to Jurafsky\cite{Jurafsky2009b}.
		
		CYK requires $\mathcal{O}(n^{2}m)$ space for the $P$ table (a table with probabilities), where ``$m$ is the number of nonterminal symbols in the grammar''\cite[p.~893]{Russel2010}, and uses $\mathcal{O}(n^{3}m)$ time. ``$m$ is constant for a particular grammar, [so it] is commonly described as $\mathcal{O}(n^{3})$''\cite[p.~893]{Russel2010}. But these values are of no value if there is no benchmark. How good is $\mathcal{O}(n^{3})$ in comparison? To give a better idea of the relations, here a small comparison to the ``Earley Algorithm''\cite[p.~477]{Jurafsky2009b}. The Earley algorithm performs better with all unambiguous grammars.\cite{Li} It has the same upper bound in time but in most cases it is quicker. Furthermore it has a space complexity of $\mathcal{O}(n)$ which is definitely better than CYK.\cite{Li} For ambiguous grammars though the Earley algorithm uses more space than CYK and the real space used is dependent on the length of the input.\cite{Li} In time complexity the CYK algorithm can only compete with Earley if ambiguous grammars are used.\cite{Li} But CYK is still of use for parsing of natural language, because natural language grammars are always ambiguous. Therefore there is no algorithm that is better than CYK for general context-free grammars.\cite{Russel2010}
		
		But how does CYK work? CYK doesn't examine all parse trees. It just examines the most probable one and computes the probability of that tree. All the other parse trees are present in the $P$ table and could be enumerated with a little work (in exponential time). But the strength and beauty of CYK is, that they don't have to be enumerated. CYK defines ``the complete state space defined by the `apply grammar rule' operator''\cite[p.~894]{Russel2010}. You can search just a part of this space with $A^{*}$ search.\cite{Russel2010} ``With the $A^{*}$ algorithm [...] the first parse found will be the most probable''\cite[p.~895]{Russel2010}. The actual pseudo code can be found in figure 23.5 in Russel\cite[p.~894]{Russel2010}.
		
		\subsubsection*{Treebank}
		\label{subSubSec:treebank}
		
		But these probabilities need to be learned from somewhere. This somewhere is usually a ``treebank''\cite[p.~895]{Russel2010}, which contains a corpus of correctly parsed sentences. The best known is the Penn Treebank\cite{Russel2010}, which ``consists of 3 million words which have been annotated with part of speech and parse-tree structure, using human labor assisted by some automated tools''\cite[p.~895]{Russel2010}. The probabilities are then computed by counting and smoothing in the given data.\cite{Russel2010} There are other ways to learn the probabilities that are more difficult. For more information refer to Russel\cite{Russel2010}.
		
		\subsubsection*{Application}
		\label{subSubSec:application}
		
		Now it is time to use the CYK algorithm with our example. For this case a restricted language called $\varepsilon_{0}$ is defined that is suitable to form one sentence about a tree. Next a lexicon (figure \ref{fig:lexicon}), ``or list of allowable words''\cite[p.~890]{Russel2010}, is defined. Furthermore a grammar (figure \ref{fig:grammar}) for $\varepsilon_{0}$ is defined. The lexicon and the grammar are based upon the lexicon and grammar in figures 23.1 and 23.2 of \cite{Russel2010} respectively.
		
		The CYK algorithm is given the words and the grammar and returns the table $P$ containing the probabilities for the whole sentence and it's subsequences.\cite{Russel2010} The pseudo code can bee seen in algorithm \ref{alg:cyk}.
		
		\begin{algorithm}
			\caption{Application of CYK for our problem}
			\label{alg:cyk}
			\begin{algorithmic}[1]
				\Procedure{CYK-Parse}{$words, grammar$}
					\State $N \gets \Call{Length}{Words}$\Comment{N = 5}
					\State $M \gets$ the number of nonterminal symbols in $grammar$\Comment{M = 6}
					\State $P \gets$ an array of size [M, N, N], initially all 0
					\For{$i = 1$ to $N$}
						\ForAll{rules of form ($X \rightarrow words_{i}[p]$)}
							\State $P[X, i, 1] \gets p$
						\EndFor
					\EndFor
					\For{$length = 2$ to $N$}
						\For{$start = 1$ to $N - length + 1$}
							\For{$len1 = 1$ to $N - 1$}
								\State $len2 \gets length - len1$
								\ForAll{rules of the form ($X \rightarrow$ $Y$ $Z$ [$p$])}
									\State $P[X, start, length] \gets \Call{Max}{P[X, start, length], P[Y, start, len1] \times P[Z, start + len1, len2] \times p}$
								
								\EndFor
							\EndFor
						\EndFor
					\EndFor
					\State \Return $P$
				\EndProcedure
			\end{algorithmic}
		\end{algorithm}
		
		The resulting P table is depicted in table \ref{tab:p}. As you can see in the table there is just one possible parse for the whole sentence. In linear form the sentence can be parsed as [$S$ [$NP$ [$A$ the] [$N$ tree]][$VP$ [$VP$ is][$Adjs$ [$Adjs$ very][$Adjs$ high]]]. With this information given a parse tree could be easily constructed.
		
		\begin{table}
			\caption{Table of probabilities from the CYK parse. The entries with probability 0 have been left out.}
			\label{tab:p}
			\centering
			\begin{tabular}{|c|c|c|c|}
				\hline
				X & start & length & p \\
				\hline
				A & 1 & 1 & 1.00 \\
				\hline
				N & 2 & 1 & 1.00 \\
				\hline				
				VP & 3 & 1 & 0.40 \\
				\hline
				Adjs & 4 & 1 & 0.80 \\
				\hline
				Adjs & 5 & 1 & 0.80 \\
				\hline
				NP & 1 & 2 & 1.00 \\
				\hline
				VP & 3 & 2 & 0.192 \\
				\hline
				Adjs & 4 & 2 & 0.128 \\
				\hline
				S & 1 & 3 & 0.40 \\
				\hline
				VP & 3 & 3 & 0.09216 \\
				\hline
				S & 1 & 4 & 0.192 \\
				\hline
				S & 1 & 5 & 0.09216 \\
				\hline
			\end{tabular}
		\end{table}
	\subsection{Semantic Analysis}
	\label{subSec:semanticAnalysis}
	
		Semantic analysis provides multiple approaches. In this paper the approach of ``syntax-driven semantic analysis''\cite[p.~617]{Jurafsky2009} is explained further. In this approach the output of a parser, the syntactic analysis, ``is passed as input to a semantic analyzer to produce a meaning representation''\cite[p.~618]{Jurafsky2009}.

		Therefore context-free grammar rules are augmented with ``semantic attachments''\cite[p.~618]{Jurafsky2009}. Every word and syntactic structure in a sentence gets such a semantic attachment. The tree with syntactic components is now traversed in a bottom-up manner. On the way the semantic attachments are combined to finally produce ``First-Order Logic''\cite[p.~589]{Jurafsky2009a} that can be interpreted in a meaningful way. This procedure has some prerequisites that will be explained first.
		
		\subsubsection*{First-Order Logic}
		\label{subSubSec:firstOrderLogic}
		
		The mentioned \textit{First-Order Logic} can be represented by a context-free grammar specification. It is beyond this paper to describe this specification completely. Jurafsky\cite{Jurafsky2009a} provides a detailed picture of the specification with all elements in figure 17.3. The most important aspects of this specification are explained here. The logic provides terms which can be functions, constants and variables. Functions have a term as argument. Syntactically they are the same as single-argument predicates. But functions represent one unique object.
		Predicates can have multiple terms as arguments. In addition the logic provides quantifiers ($\forall, \exists$) and connectives ($\wedge, \vee, \Rightarrow$).
		
		\subsubsection*{Lambda notation}
		\label{subSubSec:lambdaNotation}
		
		Another prerequisite is the ``lambda notation''\cite[p.~593]{Jurafsky2009a}. A simple example of this notation is an expression of the following form\footnote{examples taken from Jurafsky\cite[pp.~593-594]{Jurafsky2009a}}:		
		\[
			\lambda x.P(x)
		\]
		
		The $\lambda$ can be reduced in a so called ``$\lambda$-reduction''\cite[p.~593]{Jurafsky2009a}. The expression above could be reduced in the following way:
		
		\begin{alignat*}{2}
			\lambda x.&P(x)&(A) \\
			&P(A)&
		\end{alignat*}
		
		Those expressions can be extended to $n$ such $\lambda$s. An example is this expression (where $x$ and $y$ denote things from which a distance to each other can be calculated):
		\[
			\lambda x.\lambda y.Near(x,y)
		\]
				
		This expression can be reduced in multiple steps.
		\begin{alignat*}{1}
			\lambda x.\lambda y.&Near(x,y)(Bacaro) \\
			\lambda y.&Near(Bacaro, y)(Centro) \\
			&Near(Bacaro, Centro)
		\end{alignat*}
				
		This technique is called ``currying''\cite[p.~594]{Jurafsky2009a} and is used to convert ``a predicate with multiple arguments into a sequence of single-argument predicates''\cite[p.~594]{Jurafsky2009a}.
		
		\subsubsection*{Syntax-driven semantic analysis}
		\label{subSubSec:syntaxDrivenSemanticAnalysis}
		
		After the prerequisites are now explained, it is time to start with the actual syntax-driven semantic analysis. It will be shown with an example provided by Jurafsky. In the following expressions $x$ symbolizes a thing and $e$ symbolizes a state or attribute. Assume the sentence \textit{Every restaurant closed}. ``The target representation for this example should be the following''\cite[p.~621]{Jurafsky2009}.
		
		\begin{equation}
		\label{eq:tarRep}
			\forall x \,Restaurant(x) \Rightarrow \exists e \,Closed(e) \wedge ClosedThing(e,x)
		\end{equation}
		
		The first step is to determine what the meaning representation of \textit{Every restaurant} should be. \textit{Every} is responsible for the $\forall$ quantifier and \textit{restaurant} specifies the category over which is quantified. This is called the ``restriction''\cite[p.~622]{Jurafsky2009} of the noun phrase. The meaning representation could be $\forall x\,Restaurant(x)$. It is a valid logical formula but it doesn't make much sense. ``It says that everything is a restaurant.''\cite[p.~622]{Jurafsky2009} But our goal is to say \textit{something} about all the restaurants. That something is that every restaurant closed. ``This notion is traditionally referred to as the [noun phrase's] nuclear scope''\cite[p.~622]{Jurafsky2009}. In our example the nuclear scope is \textit{closed}. To represent this in the target representation, a dummy predicate $Q$ is added, which results in this expression:
		\[
			\forall x\,Restaurant(x) \Rightarrow Q(x)
		\]
		To replace $Q$ with something meaningful, the $\lambda$ notation is needed.
		\[
			\lambda Q.\forall x\,Restaurant(x) \Rightarrow Q(x)
		\] 
		After more generalization, this is the result:
		\[
			\lambda P.\lambda Q.\forall x\,P(x) \Rightarrow Q(x)
		\]
		What happened? $Restaurant$ was replaced with $P$ and $\lambda P$ was added to the beginning of the previous expression. The verb is still missing though. Therefore the verb \textit{closed} gets the following expression.
		\[
			\lambda x.\exists e\,Closed(e) \wedge ClosedThing(e,x)
		\]
		After combining the formulas of the verb and the noun phrase, the previously shown target representation\eqref{eq:tarRep} is the result. This combination is a three step process:
		\[
			\lambda P.\lambda Q.\forall x\,P(x) \Rightarrow Q(x)(Restaurant)
		\]
		$P$ gets replaced by $Restaurant$:
		\[
			\lambda Q.\forall x\,Restaurant(x) \Rightarrow Q(x)(\lambda x.\exists e\,Closed(e) \wedge ClosedThing(e,x))
		\]
		$Q$ gets replaced by the verb formula:
		\[
			\forall x\,Restaurant(x) \Rightarrow \lambda x.\exists e\,Closed(e) \wedge ClosedThing(e,x)(x)
		\]
		$x$ gets replaced by $x$:
		\[
			\forall x\,Restaurant(x) \Rightarrow \exists e\,Closed(e) \wedge ClosedThing(e,x)
		\]
		After these steps we end up with the target representation, as announced earlier.
		
		This example is just one of many, but it shows how semantic meaning can be attached to syntactic components. Furthermore it should be clear now, how semantic analysis in a syntax-driven approach works.
\section{Critical discussion}
\label{sec:critDiscussion}

	%TODO back up every claim (reference after first sentence)

	Now that both methods have been presented with one selected approach each, it is time to discuss them critically. The CYK algorithm solves many problems like ambiguity; at least to a certain degree. But it also is problematic, because of the restriction to CNF. While in theory every context-free grammar can be converted to CNF, in practice it poses ``some non-trivial problems''\cite[p.~475]{Jurafsky2009b}. One of this problems can be explored in conjunction with the second presented method (semantic analysis). ``[T]he conversion to CNF will complicate any syntax-driven approach to semantic analysis''\cite[p.~475]{Jurafsky2009b}. A solution to this problem is some kind of post-processing in which the trees are converted back to the original grammar.\cite{Jurafsky2009b} Another option is to use a more complex dynamic programming algorithm that accepts any kind of context-free grammar. Such an algorithm is the ``Earley Algorithm''\cite[p.~477]{Jurafsky2009b}.
	
	The syntax-driven semantic analysis, as it has been presented, is a powerful method that is easy to understand. But it has one essential problem. It relies upon an existing set of grammar rules with semantic attachments to them. In a real world example such a table would contain thousands of grammar rules.\cite{Russel2010} While it is relatively easy to compute the final meaning representation with such a given table, it is very hard to create the table in the first place. The difficulty to create this table is split into two main issues. The first one being that you must find a grammar specification that fits all your use cases. This problem applies for the syntactic parsing as well. The second issue is that one has to find out the semantic attachments to the grammar rules.
	
	This initial workload to create a state, in which the semantic analysis works, is a unique effort.\cite{Jurafsky2009} A restricted environment has a limited set of words and topics compared to an unrestricted environment. An example is a flight check-in automaton that only needs to process a subset of the full English grammar. Therefore this workload is of low importance in such an environment. Even if it takes one month to create such a table by hand or by computing it, the subsequent analysis of input based on this table is rather quick and the initial workload is therefore acceptable. But this is only true for restricted environments. If someone tried to use syntax-driven semantic analysis for the complete language of modern English, the creation of such a table would outweigh any possible usage.
	
	%TODO three options: add reference to claim, introduce necessary knowledge prior to this point or drop it
	
	Comparing the complexity of the two methods it shows a mirror-like image. For the parsing the creation of the grammar is comparatively easy. The presented CYK algorithm works with context-free grammars which are a very restricted set compared to natural languages. But even within these context-free grammars there are ambiguities inside the texts themselves. The creation of the parse trees is therefore more of a problem.
	
	Syntax-driven semantic analysis on the other hand requires a decent amount of work to add semantic attachments to grammar rules.\cite{Jurafsky2009} But once this has been done, it works very fast.
	
	Both methods require an initial workload for every usage domain. This unique workload is the grammar creation for the parsing and the extension of the grammar with semantic attachments for the semantic analysis. The less restricted the usage environment, the more complex the initial workload becomes. The same is true for the recurring workload for every actual usage inside one usage domain.
	
	Judging by the state-of-the-art of computer technology, parsing does still pose a significant challenge once the restricted field of programming languages is left. The semantic analysis as the second method in the chain has therefore even more problems to date. As the presented syntax-driven approach does only work with syntactic representations\cite{Jurafsky2009}, a semantic analysis can only be undertaken once the syntactic parsing succeeds.
	
	The ambiguity remains one of the bigges issues for both methods. Especially the syntax-driven semantic analysis does only consider the semantic meaning alone. It's not it's fault as the analysis doesn't know the context. The presented approach looks at each sentence in a sandbox. The generated meaning representations are therefore only of limited use for a less restricted grammar.
	
\section{Conclusion}
\label{sec:concl}

	Syntactic parsing is an important method on the way to understand natural language. The usage of dynamic programming algorithms circumvents many of the issues that classical top-down or bottom-up parsing algorithms face. Ambiguity is the most prominent of those issues. The best algorithm for context-free grammars is the CYK algorithm, which is a dynamic programming algorithm. But in practice it is very restricted, because it only works with grammars in CNF. But there are more complex dynamic programming algorithms that allow any kind of context-free grammar. Such an algorithm is the ``Earley Algorithm''\cite[p.~477]{Jurafsky2009b} which was already introduced in the critical discussion.
	
	Semantic analysis is the second method in the chain to understand natural language, as it is presented here, and therefore important as well. There are different approaches to the analysis. One of them is the syntax-driven approach that depends on parse trees. This dependency creates a delay effect: As long as a certain peace of text cannot be parsed, it definitely can't be analyzed for it's semantic meaning either. This is not an issue for restricted environments like programming languages or a very restricted subset of a natural language's grammar. But it is a major issue for real natural language, because there already the parsing does pose significant challenges.
	
	Looking into the future both methods require substantial improvements on the algorithm side to reach a point where understanding non-restricted natural languages becomes possible. But as it is right now it is not possible to create dialog systems that interact fully natural with humans. To make any kind of language interaction, the set of possible words and sentence structures must be restricted. But even if that is given (like in a flight check-in automaton), the computer has only a finite set of possible cases. The programmer can add tons of if-clauses or comparable statements to check for different cases but in the end it's all finite so that many of the user inputs must lead to the same output or no output at all. This fact has led to the current situation in which the most interaction with a computer happens via a restricted interface in which the user can only choose from a limited set of options (by clicking on a button, selecting an item of a list, etc.).
	
	Furthermore the ambiguity of natural language is a major issue. The solution to it could lie in the understanding of the context. Even though natural language is full of ambiguity, we manage to communicate very successfully. Therefore the solution to ambiguity lies probably somewhere in our brain functionality. Cognitively-inspired methods that don't use traditional AI and First-Order logic but instead are inspired by our brain and try to understand and model natural language based on the context, might as well be the solution to ambiguity altogether. The method presented by Gnjatovic\cite{Gnjatovic2012} could be such a method.
	
	In a mission critical environment this ambiguity could lead to catastrophic results, because the computer, simply put, ``didn't get it''. This risk limits the usability of natural language communication with a computer for propably a long time to a very restricted set of use cases.
	
%%%%%%%%%%%%%%%%%%%%%%%%%%%%%%%%%%%%%%
% hier werden - zum Ende des Textes - die bibliographischen Referenzen
% eingebunden
%
% Insbesondere stehen die eigentlichen Informationen in der Datei
% ``bib.bib''
%
\clearpage
\bibliography{prosem-ki}
\bibliographystyle{plain}
\addcontentsline{toc}{section}{Bibliography}% Add to the TOC

\end{document}


