%!TEX encoding = UTF-8 Unicode
\documentclass[12pt]{scrartcl}
%\usepackage[applemac]{inputenc} % Mac-Umlaute direkt verwenden öäüß
%\usepackage[isolatin]{inputenc} % PC-Umlaute direkt verwenden 
\usepackage[utf8]{inputenc} % Unicode funktioniert unter Windows, Linux und Mac
\usepackage[T1]{fontenc}
\usepackage{times}
\usepackage[ngerman]{babel}
\usepackage[numbers]{natbib}
\usepackage[fixlanguage]{babelbib}
\selectbiblanguage{ngerman}
%\usepackage{ngerman}
\usepackage{graphicx}
\usepackage[hidelinks]{hyperref}\urlstyle{rm}
\usepackage{times}
\usepackage[scaled]{helvet}
\usepackage{a4wide}
\usepackage{rotating}
\usepackage{listings}\lstset{breaklines=true,breakatwhitespace=true,frame=leftline,captionpos=b,xleftmargin=6ex,tabsize=4,numbers=left,numberstyle=\ttfamily\footnotesize,basicstyle=\ttfamily\footnotesize}
\sloppy
\setlength{\parindent}{0em}
\setlength{\parskip}{1.2ex plus 0.5ex minus 0.5ex}
\pagestyle{plain}

\begin{document}

\newpage
\thispagestyle{empty}
\begin{center}\Large
Universität Hamburg \par
Fachbereich Informatik
\vfill
Seminararbeit
\vfill
{\Large\textsf{\textbf{Vergleich von IPSec und OpenVPN}}\par}
\vfill
vorgelegt von 
\par\bigskip
Mustafa Eris, Jim Martens, Benjamin Scholz \par
%Matrikelnummern 6420323 \par
Studiengang BSc. Informatik
\end{center}

\newpage
\section*{Zusammenfassung}


\newpage
\tableofcontents

\newpage
\section{Vorbemerkung}
Problem: VPN (Virtual Private Network) aufsetzen
Relevanz: sichere Kommunikation zwischen zwei privaten Netzwerken

\section{Grundlagen}
Was sind VPNs? Warum braucht man sie? Wozu werden sie verwendet?
Was ist das OSI-Referenzmodell? Wie ist es aufgebaut?
\section{IPSec}
Was ist IPSec? Wer ist dafür verantwortlich? Wie funktioniert es? 
\section{OpenVPN}
Was ist OpenVPN? Wer ist dafür verantwortlich? Wie funktioniert es? 
\section{Vergleich von IPSec mit OpenVPN}
Standard vs. Implementation, Netzwerkschicht vs. Anwendungsschicht,
IP vs. SSL/TLS, Firewalldurchlässigkeit, Geschwindigkeit, Sicherheit

\begin{itemize}
	\item IPSec unterstützt Secret Key\cite{Alshamsi2005}, Open VPN (SSL) nicht
	\item IPSec unterstützt eine Authentifizierungsmethode, SSL mehrere\cite{Alshamsi2005}
	\item beide nutzen MACs (Message Authentication Codes) für die Authentifizierung von Nachrichten nach initialem Kontakt\cite{Alshamsi2005}
	\item beide erfordern die Implementation von HMAC-SHA-1 und HMAC-MD5\cite{Alshamsi2005}
	\end{itemize}
\section{Schlussbemerkungen}
Ausblick: (weitere) Vereinfachung von IPSec?, unterschiedliche Ansätze, Vor- und Nachteile bei beiden

%%%%%%%%%%%%%%%%%%%%%%%%%%%%%%%%%%%%%%%%%%%%%%%%%%%%%%%%%%%%%%%%%%%%%%
\newpage

\bibliography{sem}
\bibliographystyle{plainnat}
\addcontentsline{toc}{section}{Literaturverzeichnis}


%%%%%%%%%%%%%%%%%%%%%%%%%%%%%%%%%%%%%%%%%%%%%%%%%%%%%%%%%%%%%%%%%%%%%%
\newpage
\addcontentsline{toc}{section}{Anhang}

\end{document}
