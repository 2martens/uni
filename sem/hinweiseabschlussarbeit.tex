%!TEX encoding = UTF-8 Unicode
\documentclass[12pt]{scrartcl}
%\usepackage[applemac]{inputenc} % Mac-Umlaute direkt verwenden öäüß
%\usepackage[isolatin]{inputenc} % PC-Umlaute direkt verwenden 
\usepackage[utf8]{inputenc} % Unicode funktioniert unter Windows, Linux und Mac
\usepackage[T1]{fontenc}
\usepackage{ngerman}
\usepackage{graphicx}
\usepackage{hyperref}\urlstyle{rm}
\usepackage{times}
\usepackage[scaled]{helvet}
\usepackage{a4wide}
\usepackage{rotating}
\usepackage{listings}\lstset{breaklines=true,breakatwhitespace=true,frame=leftline,captionpos=b,xleftmargin=6ex,tabsize=4,numbers=left,numberstyle=\ttfamily\footnotesize,basicstyle=\ttfamily\footnotesize}
\sloppy
\setlength{\parindent}{0em}
\setlength{\parskip}{1.2ex plus 0.5ex minus 0.5ex}
\pagestyle{plain}

\begin{document}

\newpage
\thispagestyle{empty}
\begin{center}\Large
Universität Hamburg \par
Fachbereich Informatik
\vfill
{\Large\textsf{\textbf{Hinweise für das Erscheinungsbild von Seminar-, Studien- und Bachelor-, Master- und Diplomarbeiten}}\par} 
\bigskip
am Arbeitsbereich Sicherheit in Verteilten Systemen (SVS) \par
\bigskip
Prof. Dr. Hannes Federrath \par
\bigskip
\today
\vfill
\vfill 
(Muster für das Deckblatt: siehe letzte Seite dieser Hinweise)
\end{center}

\newpage
\section*{Aufgabenstellung}

Nur Studien-, Bachelor-, Master- und Diplomarbeiten: Soweit eine ausformulierte Aufgabenstellung mit dem Betreuer vereinbart wurde, diese bitte hier einfügen.

\newpage
\section*{Zusammenfassung}

Für den eiligen Leser sollen auf etwa einer halben, maximal einer Seite die wichtigsten Inhalte, Erkenntnisse, Neuerungen bzw. Ergebnisse der Arbeit beschrieben werden. 

Durch eine solche Zusammenfassung (im engl. auch Abstract genannt) am Anfang der Arbeit wird die Arbeit deutlich aufgewertet. Hier sollte vermittelt werden, warum der Leser die Arbeit lesen sollte.

\newpage
\tableofcontents

\newpage
\section{Vorbemerkung}

Um auf die wiederholten Fragen von Studenten nach dem Umfang, formalen Aufbau und Erscheinungsbild, das bei Seminar-, Studien-, Bachelor-, Master- und Diplomarbeiten erwartet wird, einheitlich zu antworten, wird dieses Dokument bereitgestellt.

Diese Hinweise haben Empfehlungscharakter. Bei Unklarheiten stehen die Mitarbeiter der Arbeitsgruppe für weitere Auskünfte zur Verfügung. Als Muster steht auch eine große Anzahl abgeschlossener Arbeiten zur Ansicht zur Verfügung.

\section{Inhalt}

Eine Seminar-, Studien-, Bachelor-, Master- und Diplomarbeit ist eine längere wissenschaftliche Abhandlung, mit der der Student zeigen soll, dass er in einem vorgegebenen Zeitrahmen in der Lage ist, wissenschaftlich zu arbeiten. 

\subsection{Anforderungen an eine Arbeit}

Eine Seminar-, Studien-, Bachelor-, Master- und Diplomarbeit trägt inhaltlich normalerweise zu einem aktuell am Arbeitsbereich bearbeiteten Forschungsthema bzw. -projekt bei oder leistet einen Beitrag in der Lehre (z.B. Erstellung eines Lehrmittels).

Normalerweise besteht eine Arbeit aus einem darstellenden und einem konstruktiven Teil. Im darstellenden Teil zeigt der Student, dass er mit wissenschaftlicher Literatur umgehen kann, Wichtiges von Unwichtigem trennen und die relevanten Aspekte etwaiger Vorarbeiten erfassen und darstellen kann. Im konstruktiven Teil werden dann eigene Lösungen erarbeitet und bewertet.

Um den inhaltlichen und sprachlichen Stil einer wissenschaftlichen Arbeit zu treffen, ist es sehr empfehlenswert, beim Lesen wissenschaftlicher Publikationen auf deren \glqq Klang\grqq\/\footnote{Robert Tolksdorf: Wie halte ich ein Referat und wie schreibe ich ein Papier. Präsentation, FU Berlin, 2003.} zu achten. Die Ich-Form sollte im Übrigen vermieden werden.

\subsection{Aufbau der Arbeit}
 
Eine wissenschaftliche Arbeit sollte -- wie nahezu jede schriftliche Arbeit -- aus einer Einleitung, einem Hauptteil und einem Schluss bestehen. Der Einleitung ist ein Deckblatt, eine Zusammenfassung und ein Inhaltsverzeichnis voranzustellen. Tabellen- und Abbildungsverzeichnisse sind optional.

Als Muster kann dieses Dokument herangezogen werden.

In der Einleitung wird die Problemstellung und deren Relevanz geschildert. Außerdem werden die Methoden genannt, mit der die Problemstellung bearbeitet wird. 

Der Hauptteil sollte mit einem Kapitel zum Stand der Wissenschaft bzgl. des zu bearbeitenden Problems beginnen und das eigene Problem einordnen. Soweit erforderlich, können in einem weiteren Kapitel Grundlagen gelegt werden, z.B. Grundverfahren beschrieben werden, die in den folgenden Kapiteln benutzt, ausgebaut oder verändert werden.

Der Schluss fasst die Ergebnisse noch einmal zusammen, bewertet die eigenen Ergebnisse kritisch und benennt die offenen Fragen. Es ist völlig normal, dass im Verlauf der Bearbeitung neue Problemstellungen und Forschungsfragen entstehen, die dann wieder der Ausgangspunkt für weitere Arbeiten sein können.

Ein Literaturverzeichnis am Ende ist obligatorisch. Man sollte sich stets auf die besten Quellen stützen. In abnehmender Qualität: 
\begin{enumerate}
\item Fachbücher, Standards, 
\item Wiss. Zeitschriftenartikel, Survey-Artikel,
\item Konferenzbeiträge,
\item Technical Reports, graue Literatur,
\item Online-Material, Arbeitspapiere, Firmenmaterial, Ausarbeitungen.
\end{enumerate}
Im Internet können zur Feststellung der Qualität und Recherche von Publikationen auch die
\begin{itemize}
\item Computer Science Bibliography (\url{http://dblp.uni-trier.de/}) und die
\item Scientific Literature Digital Library (\url{http://citeseer.nj.nec.com/})
\end{itemize}
herangezogen werden.

Bei Bedarf kann auch ein Index und Abkürzungsverzeichnis beigefügt werden. Bei Seminar-, Studien-, Bachelor-, Master- und Diplomarbeiten ist dies jedoch normalerweise wegen des überschaubaren Umfangs nicht sinnvoll.

Bei umfangreichen Quelltexten (mehr als 2 Seiten) sollten diese nicht im Fließtext wiedergegeben werden, sondern im Anhang erscheinen. Dies gilt auch für andere den Lesefluss störende Informationen größeren Umfangs. 

Für Studien-, Bachelor-, Master- und Diplomarbeiten ist wichtig: Eigenhändig unterschriebene Selbständigkeitserklärung am Anfang oder Ende des Textes nicht vergessen (siehe Muster am Ende dieser Hinweise). Bei Seminararbeiten kann diese entfallen.

\section{Form}

\subsection{Umfang der schriftlichen Ausarbeitung}

Generell gilt: Je weniger Text, umso besser. Auf klare Formulierungen ist in jedem Fall zu achten. Für Studien-, Bachelor-, Master- und Diplomarbeiten ist der Richtwert 30--50 Seiten. 20 Seiten sind zu wenig, 100 sind zuviel. Bei Seminararbeiten genügen 5--15 Seiten.

Insbesondere für Bachelor-, Master- und Diplomarbeiten gilt: Wo immer möglich, sollte auf andere relevante Veröffentlichungen verwiesen werden, anstatt deren Inhalt noch einmal wiederzugeben. Für alle Aussagen und Darstellungen, die aus Veröffentlichungen stammen, sind Quellenangaben zu machen. Bei Inhalten aus fremden Quellen, die paraphrasiert oder wörtlich übernommen werden, ist die Quellenangabe an der Textstelle zu machen. Es genügt nicht, die Quelle ins Literaturverzeichnis aufzunehmen.

Viele Studenten haben zu Beginn der Bearbeitung Sorge, dass sie womöglich nicht auf die \glqq übliche\grqq\/ Seitenzahl kommen. Diese Sorge ist meist unbegründet. Es sollte möglichst früh mit dem Schreiben begonnen werden: Dokumentieren Sie, was Sie gelesen und neu erarbeitet haben.

\subsection{Gestaltung}

Es darf jedes Textverarbeitungssystem verwendet werden. 

Als Brotschrift (= Hauptschrift) sollte eine mit Serifen verwendet werden, z.B. Times (wie dieser Text) oder Times New Roman. In LaTeX kann auch die Computer Modern Roman (cmr, also die TeX-Standardschrift) verwendet werden oder besser, falls möglich, die Postscript-Schrift Times (\verb|\|usepackage\{times\} bzw. \verb|\|usepackage\{mathptm\}). Bitte verwenden Sie keine \textsf{Helvetica} oder \textsf{Arial}, da diese Schrift bei längeren Texten schwerer lesbar ist. In Überschriften ist diese Schrift jedoch in \textsf{\textbf{Bold}} erlaubt, wie in diesem Beispiel.

Die Schriftgröße sollte 12~pt (wie dieser Text), zur Not auch 11~pt sein. Eine Größe von 10~pt ist zu klein!
%
Als Zeilenabstand sollte möglichst 15~pt oder 14~pt verwendet werden. 1,5-zeilig entspricht etwa 18~pt und ist zuviel. Bei LaTeX sind keine benutzerdefinierten Abstände nötig. 
%
Der Text ist im Blocksatz zu setzen. Ränder bei A4-Papier: ca. 2,5--3cm rundherum. In LaTeX erzeugt beispielsweise \verb|\|usepackage\{a4wide\} einen geeigneten Satzspiegel (wie dieses Dokument).

Dieses Dokument wurde mit LaTeX erstellt und steht übrigens auch im Quelltext (.tex-File) zur Verfügung und kann für eigene Zwecke weiterverwendet werden.

Es sollten möglichst nicht mehr als drei Gliederungsebenen verwendet werden.
%
Eine Kopfzeile kann verwendet werden, muss aber nicht. Hier wird oft unnötig Zeit verschwendet!
%
Bitte benutzen Sie nur einen Absatztyp (wie in diesem Dokument; wird in LaTeX durch mindestens eine Leerzeile zwischen den Absätzen erzeugt). Es ist weit verbreitet, Gedanken, die irgendwie zusammenhängen, aber aus Sicht des Autors noch keinen neuen Absatz rechtfertigen, auf einer neuen Zeile zu beginnen --- in LaTeX meist durch 
%\verb|\\| 
{\textbackslash\textbackslash} 
erzeugt.\\
Man soll zwar keine Negativbeispiele bringen, aber der Zeilenwechsel vor diesem Satz ist eines. Dies ist zu vermeiden, weil es das Textbild uneinheitlich und unruhig macht.

Weniger kann übrigens manchmal mehr und Besseres bewirken. Spiegel Online berichtete in \cite{textwahrnehmung} beispielsweise, dass einfache, klare Sprache und eine gut lesbare Standardschrift die Textwahrnehmung verbessern kann: „Schreib so einfach und deutlich wie möglich, dann hält man dich eher für intelligent.“  

\subsection{Abbildungen, Tabellen und Listings}

Gleitobjekte wie Abbildungen und Tabellen müssen eine Unterschrift erhalten. Auf diese muss zudem im Text eindeutig verwiesen werden, da durch das automatische Setzen unter Umständen nicht ersichtlich ist, zu welchem Textabschnitt eine Abbildung gehört. Wie das aussehen kann, ist anhand von Abbildung \ref{fig:bsp} ersichtlich.

\begin{figure}
\centering
\sffamily\footnotesize
%\includegraphics[width=0.6\textwidth]{vanetsim_staumeldung.pdf}
\unitlength=0.75mm
\special{em:linewidth 0.4pt}
\linethickness{0.4pt}
\begin{picture}(111,73)(0,0)
\put(21,59){\makebox(0,0)[cc]{}}
\put(1,55){\line(1,0){40}}
\put(1,55){\line(0,1){8}}
\put(41,55){\line(0,1){8}}
\put(1,63){\line(1,0){40}}
\put(91,59){\makebox(0,0)[cc]{}}
\put(71,55){\line(1,0){40}}
\put(71,55){\line(0,1){8}}
\put(111,55){\line(0,1){8}}
\put(71,63){\line(1,0){40}}
\put(21,19){\makebox(0,0)[cc]{}}
\put(1,15){\line(1,0){40}}
\put(1,15){\line(0,1){8}}
\put(41,15){\line(0,1){8}}
\put(1,23){\line(1,0){40}}
\put(91,19){\makebox(0,0)[cc]{}}
\put(71,15){\line(1,0){40}}
\put(71,15){\line(0,1){8}}
\put(111,15){\line(0,1){8}}
\put(71,23){\line(1,0){40}}
\put(31,43){\makebox(0,0)[cc]{F}}
\put(24,38){\line(1,0){14}}
\put(24,38){\line(0,1){10}}
\put(38,38){\line(0,1){10}}
\put(24,48){\line(1,0){14}}
\put(81,43){\makebox(0,0)[cc]{F}}
\put(74,38){\line(1,0){14}}
\put(74,38){\line(0,1){10}}
\put(88,38){\line(0,1){10}}
\put(74,48){\line(1,0){14}}
\put(31,30){\circle{6}}
\put(81,30){\circle{6}}
\put(79,30){\line(1,0){4}}
\put(81,28){\line(0,1){4}}
\put(29,30){\line(1,0){4}}
\put(31,28){\line(0,1){4}}
\put(21,63){\line(0,1){7}}
\put(21,63){\vector(0,-1){0.12}}
\put(31,48){\line(0,1){7}}
\put(31,48){\vector(0,-1){0.12}}
\put(31,33){\line(0,1){5}}
\put(31,33){\vector(0,-1){0.12}}
\put(31,23){\line(0,1){4}}
\put(31,23){\vector(0,-1){0.12}}
\put(21,8){\line(0,1){7}}
\put(21,8){\vector(0,-1){0.12}}
\put(11,30){\line(1,0){17}}
\put(28,30){\vector(1,0){0.12}}
\put(91,63){\line(0,1){7}}
\put(91,63){\vector(0,-1){0.12}}
\put(81,48){\line(0,1){7}}
\put(81,48){\vector(0,-1){0.12}}
\put(81,33){\line(0,1){5}}
\put(81,33){\vector(0,-1){0.12}}
\put(81,23){\line(0,1){4}}
\put(81,23){\vector(0,-1){0.12}}
\put(91,8){\line(0,1){7}}
\put(91,8){\vector(0,-1){0.12}}
\put(84,30){\line(1,0){17}}
\put(84,30){\vector(-1,0){0.12}}
\put(11,30){\line(0,1){25}}
\put(101,30){\line(0,1){25}}
\put(21,73){\makebox(0,0)[cc]{Klartext M}}
\put(91,73){\makebox(0,0)[cc]{Chiffretext C}}
\put(21,3){\makebox(0,0)[cc]{Chiffretext C}}
\put(91,3){\makebox(0,0)[cc]{Klartext M}}
\put(56,43){\makebox(0,0)[cc]{K}}
\put(56,48){\makebox(0,0)[cc]{Schlüssel}}
\put(38,43){\line(1,0){13}}
\put(38,43){\vector(-1,0){0.12}}
\put(61,43){\line(1,0){13}}
\put(74,43){\vector(1,0){0.12}}
\put(21,55){\line(0,1){8}}
\put(21,15){\line(0,1){8}}
\put(91,55){\line(0,1){8}}
\put(91,15){\line(0,1){8}}
\put(11,59){\makebox(0,0)[cc]{L}}
\put(31,59){\makebox(0,0)[cc]{R}}
\put(81,59){\makebox(0,0)[cc]{R}}
\put(11,19){\makebox(0,0)[cc]{R}}
\put(81,19){\makebox(0,0)[cc]{L}}
\put(101,19){\makebox(0,0)[cc]{R}}
\put(13,52){\line(1,0){18}}
\put(81,52){\line(1,0){18}}
\put(99,23){\line(0,1){29}}
\put(99,23){\vector(0,-1){0.12}}
\put(13,23){\line(0,1){29}}
\put(13,23){\vector(0,-1){0.12}}
\put(31,19){\makebox(0,0)[cc]{L'}}
\put(101,59){\makebox(0,0)[cc]{L'}}
\end{picture}
\caption{Ein Beispiel für eine Abbildung}
\label{fig:bsp}
\end{figure}

Abbildungen sollten möglichst schlicht, in schwarzweiß und als Strichzeichnungen gestaltet sein. Wenn schon Farben verwendet werden, dann bitte in allen Abbildungen das gleiche Farbschema verwenden. Farben sind nur dann sinnvoll, wenn sie einen Sachverhalt deutlich unterstreichen oder veranschaulichen. Es ist zu beachten, dass die Aussagekraft auch in einem Schwarzweiß-Ausdruck erhalten bleiben muss!

Die Auflösung muss ausreichend groß gewählt werden, damit im fertigen Dokument weder Pixel noch Treppen oder Unschärfe erkennbar sind. Deshalb möglichst Vektorgrafiken verwenden.

Ein Abbildungsverzeichnis ist nicht unbedingt erforderlich, kann aber bei einer Vielzahl von verwendeten Abbildungen für Übersichtlichkeit sorgen.

Längere Listings sollten wie Abbildungen in einer Float-Umgebung untergebracht werden, d.h. eine Über- bzw. Unterschrift haben. Ein Beispiel zeigt Listing \ref{lst:ggt}.

\begin{lstlisting}[float,caption={Berechnung des größten gemeinsamen Teilers zweier ganzer Zahlen a und b},label={lst:ggt}]
int getGGTOf(int a, int b) {
	// requires ((a > 0) && (b > 0)); ensures return > 0;
	int h;
	while (b != 0) {
		h = b;
		b = a % b; // % is the modulo operator. This line long enough to show how line breaks in lstlisting are handled.
		a = h;
	}
	return a;
}
\end{lstlisting}

\subsection{Literaturverzeichnis}

Im Folgenden werden Beispiele für die Referenz von Quellen gegeben. Es ist wichtig, dass alle für den jeweiligen Quellentyp relevanten Informationen angegeben werden. Darüber hinaus sollte darauf geachtet werden, dass die Referenz der Quellen einheitlich erfolgt, also beispielsweise innerhalb eines Dokuments die Autoren konsequent zuerst mit Vorname und dann mit Nachname genannt werden.

\subsubsection{Fachbücher}

Typische Zitierweise:

Vorname Nachname: Buchtitel. {[}Auflage,{]} Verlag, Erscheinungsort Jahr.

Beispiele für solche Literaturstellen sind \cite{Beut2009, ScWe2007}.

\subsubsection{Zeitschriften}

Typische Zitierweise:

Vorname Nachname: Artikeltitel. Zeitschrift Jahrgang/Volume (Jahr) Seiten.

Beispiele sind \cite{Kili2006,Lamp1981}.

\subsubsection{Konferenzbeiträge}

Typische Zitierweise:

Vorname Nachname: Beitragstitel. Konferenz, Ort, Datum, Seiten.

Beispiele: \cite{InBr2009,WWPK2010,HSFN2009}.

\subsubsection{Onlinequellen}

Typische Zitierweise:

Vorname Nachname: Titel. {[}Quelle.{]} URL (Zugriffszeitpunkt).

Beispiele: \cite{CCC2009,Heise2011,textwahrnehmung}.

\subsubsection{Wikipedia}

Grundsätzlich sollte bei Quellenangaben darauf geachtet werden, dass die Originalquelle für eine Information oder einen Sachverhalt referenziert wird. Referenzierungen auf Wikipedia sollten daher tunlichst vermieden werden. Im wissenschaftlichen Kontext kann in seltenen Fällen aus didaktischen Gründen eine Referenz auf Inhalte aus Wikipedia trotzdem sinnvoll sein. In diesem Zusammenhang ist es wichtig, auf eine spezielle Version des Dokuments (in der Regel die zum Abrufzeitpunkt aktuellste) zu verweisen. Dies wird innerhalb von Wikipedia mittels der sogenannten \emph{oldid} realisiert. Ein Beispiel hierfür ist \cite{Wiki}.

\subsection{Vor der Abgabe}

Vor der Abgabe sollten die Funktionen zur Rechtschreibprüfung und Silbentrennung genutzt werden, soweit sie im jeweiligen Textverarbeitungssystem vorhanden sind. Zusätzlich lohnt es sich, den Text vor Abgabe von jemanden lesen zu lassen (Freund, Freundin, Bekannter, Haustier), damit er auch sprachlich noch einmal überprüft wurde.

Bachelor-, Master- und Diplomarbeiten müssen gebunden sein. Eine einfache Bindung (für ca. 3~EUR pro Exemplar aus dem Copyshop) genügt. Seminar- und Studienarbeiten können als lose Blätter mit einer Klarsichtfolie vorne und hinten abgegeben werden und werden am Arbeitsbereich zusammengeklammert. Bitte \emph{nicht} lochen!

Es sollte möglichst doppelseitig gedruckt werden, um Papier zu sparen. Die Umwelt dankt es. 
Schwarzweiß-Druck genügt in den meisten Fällen völlig.

Zusätzlich muss die Arbeit noch einmal als \emph{PDF-Datei per Mail an den Betreuer}\footnote{Eine etwa im Prüfungssekretariat abgegebene CD erreicht uns gewöhnlich nicht.} geschickt werden. Falls Quellcodes und Programme erstellt wurden, sollte vor Abgabe mit dem Betreuer besprochen werden, in welcher Weise diese abzugeben sind. 

\section{Betreuung und Bewertung der Arbeit}

Für die Betreuung der Arbeit steht der mit der Ausgabe der Aufgabenstellung oder im Seminar genannte Betreuer zur Verfügung. Der Betreuer steht im Rahmen der Sprechstunden und nach Vereinbarung für regelmäßige Gespräche (mindestens etwa alle 2-3 Wochen) und für Fragen zur Verfügung. Sinnvollerweise sollte man jeweils darauf vorbereitet sein, einen kurzen mündlichen Bericht über den Stand der Bearbeitung zu geben. Während der Vorlesungszeit finden möglicherweise regelmäßige Treffen aller Bearbeiter von Abschlussarbeiten statt, zu denen ggf. kurzfristig eingeladen wird. Jedes Gesprächsangebot sollte wahrgenommen werden!

\subsection{Schriftlicher Teil}

Folgende \textbf{Meilensteine} sollten bereits zu Beginn der Bearbeitung des Themas im Kalender vermerkt werden:

Bei \textbf{Arbeiten mit etwa 3-monatiger Bearbeitungszeit nach 1,5 Monaten die Abgabe eines ersten Textentwurfs} beim Betreuer. Wenn mit dem Zweitbetreuer (soweit vorhanden) nichts anderes vereinbart ist, sollte ihm zu diesem Zeitpunkt ein Zwischenbericht geliefert werden und ihm ggf. das Angebot gemacht werden, den Text zur Kommentierung zu überlassen.

Bei \textbf{Arbeiten mit etwa 6-monatiger Bearbeitungszeit} soll \textbf{nach 2 Monaten} ein etwa 12-seitiger Textentwurf inkl. Gliederungsentwurf vorliegen und \textbf{nach weiteren 2 Monaten} ein erster vollständiger Textentwurf. 

Die Textentwürfe (bitte auf Papier, nicht per E-Mail) werden von uns gelesen, kommentiert und zurückgegeben. Die Meilensteine dienen der Fortschrittskontrolle und sind für die endgültige Bewertung der Arbeit bedeutungslos; Fehler dürfen sorgenfrei gemacht werden.

Typische Kontrollfragen zur Beurteilung einer Arbeit sind:
\begin{itemize}
\item Wurde die Fragestellung auf hohem Niveau bearbeitet?
\item Handelt es sich um eine kreative Herangehensweise bzw. Lösung?
\item Sind die Annahmen und getroffenen Voraussetzungen realistisch, oder wurden unzulässige Vereinfachungen vorgenommen?
\item Sind alle Aussagen klar und verständlich formuliert?
\item Wurde die Literatur zur Kenntnis genommen?
\item Falls Programme entwickelt wurden: Sind die Quellcodes dokumentiert, die Module und Schnittstellen beschrieben? Gibt es eine Programmbeschreibung? 
\item Wie ist die äußere Form (Layout, Rechtschreibung, Grammatik)?
\item Ist der Umfang angemessen?
\end{itemize}

Bei der Bewertung der schriftlichen Ausarbeitung wird ein Punkteschema verwendet, dass sich an \cite{faui2} orientiert, welches am Lehrstuhl für Informatik 2 (Programmiersysteme) der Friedrich-Alexander Universität Erlangen-Nürnberg entwickelt wurde. Eine gekürzte und angepasste Übernahme des Punkteschemas ist im Anhang enthalten. Das Punkteschema nach \cite{faui2} wird in modifizierter Form beispielsweise auch vom Zentrum für Bioinformatik (Prof. Dr. Rarey) angewendet. 

\subsection{Referat}

Oft müssen die Ergebnisse der Arbeit in einem Referat vorgestellt werden. Generell gilt: Ein Referat soll die Zuhörerschaft gezielt informieren. Bei der Vorbereitung des Referats sollte deshalb Klarheit darüber bestehen, wieviele Zuhörer voraussichtlich teilnehmen werden, welches Vorwissen sie haben und mit welchen Erwartungen sie zu dem Referat gekommen sind.

Übersichtliche Folien sind für den Vortragenden und die Zuhörer eine große Unterstützung. Eine Folie sollte nicht mehr als 4--8 Stichpunkte enthalten, keinen Fließtext und aussagekräftige Abbildungen. Bei Farbfolien sollte man sich auf drei bis vier Farben beschränken, die durchgehend durch die Präsentation verwendet werden. Ansonsten wirken die Folien bunt und unruhig. Schriften ohne oder mit unauffälligen Serifen (z.B. Verdana oder Helvetica) in 18--20~pt eignen sich sehr gut für Vortragsfolien.

Daumenregel: Folienanzahl $\approx$ Vortragszeit$\,/\,$3 min.

Der Vortragende sollte während des gesamten Vortrags ins Publikum schauen und nicht zur Wand oder in den Laptop. 

Auch beim Referat wird nach festgelegten Kriterien beurteilt, die dem Formular im Anhang entnommen werden können.

\section{Schlussbemerkungen}

Im Internet sind zahlreiche Erfahrungsberichte von (renommierten) Wissenschaftlern zu finden, die auch bei der Bearbeitung einer Seminar- oder Abschlussarbeit hilfreich sein können. Hier einige wenige Empfehlungen:

\begin{itemize}
	\item Randy Pausch Lecture: Time Management. \\\url{http://www.youtube.com/watch?v=oTugjssqOT0}
	\item Richard Hamming: You and Your Research. \\\url{http://www.cs.virginia.edu/~robins/YouAndYourResearch.html}
	\item Nick Feamster: Writing Tips for Academics. \\\url{http://greatresearch.org/2013/10/11/storytelling-101-writing-tips-for-academics/}
\end{itemize}

Wissenschaftliches Arbeiten und Schreiben will gelernt sein. Dafür dienen während des Studiums u.a. die Seminararbeiten. Die Abschlussarbeit soll dann zeigen, welche methodischen und fachlichen Fähigkeiten während des Studiums erworben wurden. Neben einem guten Zeitmanagement, Disziplin und Bereitschaft zur Literaturrecherche ist die Kommunikation mit dem Betreuer ein Schlüssel zur erfolgreichen Bearbeitung des Themas. 

%%%%%%%%%%%%%%%%%%%%%%%%%%%%%%%%%%%%%%%%%%%%%%%%%%%%%%%%%%%%%%%%%%%%%%
\newpage
\addcontentsline{toc}{section}{Literaturverzeichnis}
\begin{raggedright}%schaltet Blocksatz ab, erzeugt ein stimmigeres Schriftbild im Literaturverzeichnis
\begin{thebibliography}{XXXXXXXX}
	\bibitem[Beut2009]{Beut2009} Albrecht Beutelsbacher: Kryptologie -- Eine Einführung in die Wissenschaft vom Verschlüsseln, Verbergen und Verheimlichen. 9. akt. Auflage, Vieweg+Teubner, Wiesbaden 2009.
	\bibitem[CCC2009]{CCC2009} Constanze Kurz, Frank Rieger: Chaos Computer Club veröffentlicht Stellungnahme zur Vorratsdatenspeicherung. \url{http://www.ccc.de/updates/2009/vds-gutachten} (Zugriff am 06.05.2011).
	\bibitem[FAUI2011]{faui2} Beurteilung von wissenschaftlichen Arbeiten am Lehrstuhl für Informatik 2 (Programmiersysteme). Friedrich-Alexander Universität Erlangen-Nürnberg. \url{https://www2.informatik.uni-erlangen.de/teaching/thesis/review.html} (Zugriff am 18.10.2011).
	\bibitem[Heise2011]{Heise2011} Heise-News: US-Professor wirft Sony Mitschuld am PSN-Hack vor. 6. Mai 2011. \url{http://www.heise.de/newsticker/meldung/US-Professor-wirft-Sony-Mitschuld-am-PSN-Hack-vor-1238676.html} (Zugriff am 17.05.2011).
	\bibitem[HSFN2009]{HSFN2009} Dominik Herrmann, Florian Scheuer, Philipp Feustel, Thomas Nowey, Hannes Federrath: A Privacy-Preserving Platform for User-Centric Quantitative Benchmarking. Proc. TrustBus 2009, Lecture Notes in Computer Science, Vol. 5695, Springer, Berlin 2009, 32-41.
	\bibitem[InBr2009]{InBr2009} Frank Innerhofer-Oberperfler, Ruth Breu: An empirically derived loss taxonomy based on publicly known security incidents. Proc. Fourth International Conference on Availability, Reliability and Security - ARES/CISIS 2009, Fukuoka, Japan, March 2009.
	\bibitem[Kili2006]{Kili2006} Detlef Kilian: Einführung in Informationssicherheitsmanagementsysteme (I) - Begriffsbestimmung und Standards. Datenschutz und Datensicherheit DuD 30/10 (2006) 651-654.
	\bibitem[Lamp1981]{Lamp1981} Leslie Lamport: Password authentication with insecure communication. Communications of the ACM 24/11 (1981) 770-772.
	\bibitem[ScWe2007]{ScWe2007} Uwe Schneider, Dieter Werner: Taschenbuch der Informatik. 6. neu bearb. Auflage, Hanser, München 2007.
	\bibitem[Spiegel2005]{textwahrnehmung} Spiegel Online: Textwahrnehmung -- Simple Sprache wirkt intelligenter. 1. November 2005. \url{http://www.spiegel.de/wissenschaft/mensch/0,1518,382730,00.html} (Zugriff am 01.11.2005)
	\bibitem[Wiki2011]{Wiki} Wikipedia: Enigma (Maschine). \url{http://de.wikipedia.org/w/index.php?title=Enigma_(Maschine)&oldid=88241310} (Bearbeitungsstand 29. April 2011, 09:09 UTC, Zugriff am 06.05.2011).
	\bibitem[WWPK2010]{WWPK2010} Benedikt Westermann, Rolf Wendolsky, Lexi Pimenidis, Dogan Kesdogan: Cryptographic Protocol Analysis of AN.ON. Financial Cryptography and Data Security 2010, Canary Islands, Spain, Jan 2010.
\end{thebibliography}
\end{raggedright}

%%%%%%%%%%%%%%%%%%%%%%%%%%%%%%%%%%%%%%%%%%%%%%%%%%%%%%%%%%%%%%%%%%%%%%
\newpage
\addcontentsline{toc}{section}{Anhang}
\addcontentsline{toc}{subsection}{Punktesystem zur Beurteilung von wissenschaftlichen Arbeiten}

\section*{Punktesystem zur Beurteilung von wissenschaftlichen Arbeiten}

Bei der Bewertung der schriftlichen Ausarbeitung wird ein Punkteschema verwendet, das am Lehrstuhl für Informatik 2 (Programmiersysteme) der Friedrich-Alexander Universität Erlangen-Nürnberg entwickelt wurde. Die folgende Übersicht ist eine gekürzte und angepasste Übernahme von \cite{faui2}.

\subsection*{Allgemeine Hinweise}

Die Arbeit wird unter fünf Aspekten einzeln bewertet, die jedoch nicht gleichgewichtig sind. Das verschiedene Gewicht wird dadurch berücksichtigt, dass für die einzelnen Aspekte verschieden hohe Punktzahlen zur Verfügung stehen:
\begin{center}
	\begin{tabular}{lp{8cm}} 
		\hline
	Punktzahl & Aspekt\\ \hline
	0 - 6 &  Schwierigkeitsgrad\\
	0 - 8 &  Originalität\\
	0 - 10 & wissenschaftliche Arbeitstechnik\\
	0 - 4 &  Stil\\
	0 - 3 &  Form\\
	0 - 31 & Summe\\
	\hline
	\end{tabular}	
\end{center}


\subsection*{Notenstufen}

Die Note wird in folgender Weise festgesetzt:
\begin{enumerate}
	\item  Arbeiten, bei denen für wissenschaftliche Arbeitstechnik, weniger als 4 Punkte oder für die wissenschaftliche Arbeitstechnik, den Stil und die Form zusammen weniger als 8 Punkte vergeben wurden, erhalten die Note 5.0 (nicht ausreichend, nicht bestanden).

	\item 	Alle anderen Arbeiten werden nach der folgenden Tabelle benotet.
\end{enumerate}

\begin{center}
	\begin{tabular}{lp{8cm}}
	\hline
	Punktzahl  & Note \\
	\hline
	31-29 &   1.0 \quad  sehr gut\\
	28-27 &   1.3\\
	\hline
	26-25 &   1.7\\
	24-23 &   2.0 \quad  gut\\
	22-21 &   2.3\\
	\hline
	20-19 &   2.7 \\
	18-17 &   3.0 \quad  befriedigend\\
	16-15 &   3.3\\
	\hline
	14-13 &   3.7\\
	12-11 &   4.0 \quad  ausreichend\\
	\hline
	\end{tabular}
\end{center}

\subsection*{1. Schwierigkeitsgrad (0-6)}

Bei der Beurteilung des Schwierigkeitsgrades ist davon auszugehen, ob die Problemstellung mit der durchschnittlichen Ausgangsqualifikation der Bearbeitungsgruppe gelöst werden kann (4 Punkte). Die Beurteilung des Schwierigkeitsgrades einer Arbeit kann erst nach Abschluss erfolgen und umfasst die Prüfung, ob die vorgelegte Fassung die genannten Merkmale auch tatsächlich enthält.

\subsection*{2. Schöpferische Originalität (0-8)}

Bei der Beurteilung der schöpferischen Originalität ist nicht nur davon auszugehen, inwieweit der Bearbeiter der Anleitung und Führung durch den Betreuer bedarf. Es ist vielmehr selbstverständlich, dass der Bearbeiter Initiative entwickelt, d.h. aus eigenem Antrieb Schwierigkeiten aufgreift und mit dem Betreuer diskutiert (4 Punkte).

\subsection*{3. Wissenschaftliche Arbeitstechnik (0-10)}

Bei der Beurteilung der wissenschaftlichen Arbeitstechnik ist nicht nur vom Grad der Fehlerfreiheit (formale Richtigkeit der Aussagen und eventueller Programme) auszugehen, die vielmehr als selbstverständlich vorausgesetzt werden muss. Daneben fällt sehr stark das Ausmaß der Selbstkontrolle ins Gewicht, das sich bei formalen Aussagen in der Beweisgründlichkeit, bei Programmen in ausführlichen Tests zeigt. Bezüglich der Programmrichtigkeit darf davon ausgegangen werden, dass bei hinreichend modularem Programmaufbau eine durchdachte (Begründung!) Menge von Testprogrammen genügt (5 Punkte).

\subsection*{4. Stil (0-4)}

Bei der Beurteilung des Stils ist von der sprachlichen Ausdrucksfähigkeit auszugehen, die sich dem Leser in der vorgelegten Arbeit bietet. Diese zeigt sich insbesondere in der Klarheit und Kürze des Ausdrucks: auch schwierige Probleme müssen verständlich dargelegt, triviale Zusammenhänge nicht hinter einem formalen Apparat verborgen sein. Die Gedankenführung muss eindeutig sein (2 Punkte).

\subsection*{5. Äußere Form (0-3)}

Bei der Beurteilung der äußeren Form fällt neben der Sorgfalt der Ausführung, insbesondere der Zeichnungen und Tabellen, die Klarheit der Gliederung und des Inhaltsverzeichnisses ins Gewicht (2 Punkte).



%%%%%%%%%%%%%%%%%%%%%%%%%%%%%%%%%%%%%%%%%%%%%%%%%%%%%%%%%%%%%%%%%%%%%%
\newpage
\addcontentsline{toc}{subsection}{Formular Präsentationsbewertung}
\thispagestyle{empty}
\begin{sideways}
	\begin{tabular}{|l|p{1.8cm}|p{1.8cm}|p{1.8cm}|p{1.8cm}|p{1.8cm}|p{1.8cm}|p{1.8cm}|p{1.8cm}|}
	\multicolumn{6}{l}{\textbf{Präsentationsbewertung}} \\ 
	\multicolumn{6}{l}{} \\ 
	\hline 
	Datum                                    & & & & & & & \\ \hline
	Name                                     & & & & & & & \\ \hline
	Thema                                    & & & & & & & \\ \hline 
	\multicolumn{6}{l}{} \\ 
	\multicolumn{6}{l}{\textbf{Stil}} \\ 
	\hline 
	Sicheres Auftreten                       & & & & & & & \\ \hline
	Kontakt zum Zuhörer                      & & & & & & & \\ \hline
	Deutliche Sprechweise                    & & & & & & & \\ \hline
	Angemessenes Tempo                       & & & & & & & \\ \hline
	Freies Sprechen                          & & & & & & & \\ \hline
	Einhalten der Zeit                       & & & & & & & \\ \hline 
	\multicolumn{6}{l}{} \\ 
	\multicolumn{6}{l}{\textbf{Inhalt}} \\ 
	\hline 
	Verständlichkeit des Inhalts             & & & & & & & \\ \hline
	Prägnanz                                 & & & & & & & \\ \hline
	Konzept/Gliederung                       & & & & & & & \\ \hline
	Beispiele                                & & & & & & & \\ \hline
	Angemessene Detailtiefe                  & & & & & & & \\ \hline 
	\multicolumn{6}{l}{} \\ 
	\multicolumn{6}{l}{\textbf{Folien/Demo}} \\ 
	\hline 
	Vertrautheit mit Folien/Demo             & & & & & & & \\ \hline
	Verständlichkeit der Folien/Demo         & & & & & & & \\ \hline
	Qualität von Abbildungen                 & & & & & & & \\ \hline
	Nachvollziehbarkeit der Demo             & & & & & & & \\ \hline 
	\multicolumn{6}{l}{} \\ 
	\hline
	Subjektiver Gesamteindruck               & & & & & & & \\ \hline
	\end{tabular}
\end{sideways}





%%%%%%%%%%%%%%%%%%%%%%%%%%%%%%%%%%%%%%%%%%%%%%%%%%%%%%%%%%%%%%%%%%%%%%
\newpage
\addcontentsline{toc}{subsection}{Muster der Selbständigkeitserklärung}
\section*{Erklärung}
Ich versichere, dass ich die Arbeit selbstständig verfasst und keine anderen als die angegebenen Hilfsmittel -- insbesondere keine im Quellenverzeichnis nicht benannten Internetquellen -- benutzt habe, die Arbeit vorher nicht in einem anderen Prüfungsverfahren eingereicht habe und die eingereichte schriftliche Fassung der auf dem elektronischen Speichermedium entspricht.

Ich bin mit der Einstellung der Arbeit in den Bestand der Bibliothek des Departments Informatik einverstanden.\footnote{ggf. streichen} 

Hamburg, den \today

\bigskip
(Unterschrift)\\
Hannes Federrath


%%%%%%%%%%%%%%%%%%%%%%%%%%%%%%%%%%%%%%%%%%%%%%%%%%%%%%%%%%%%%%%%%%%%%%
\newpage
\thispagestyle{empty}
\addcontentsline{toc}{subsection}{Muster des Deckblatts}
\begin{center}\Large
Universität Hamburg \par
Fachbereich Informatik
\vfill
Masterarbeit
\vfill
{\Large\textsf{\textbf{Analyse einer symmetrischen Blockchiffre}}\par}
\vfill
vorgelegt von 
\par\bigskip
Hannes Federrath \par
geb. am 8. Juni 1969 in Sonneberg \par
Matrikelnummer 1234567 \par
Studiengang Informatik
\vfill
eingereicht am \today
\vfill 
Betreuer: Dipl.-Inform. Heinz Mustermann \par
Erstgutachter: Prof. Dr.-Ing. Hannes Federrath \par
Zweitgutachter: N.N.
\end{center}



\end{document}
