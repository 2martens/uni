\documentclass[12pt,twoside]{scrartcl}
%%%%%%%%%%%%%%%%%%%%%%%%%%%%%%%%%%%%%%%%%%%%%%%%%%%%%%%%%%%%%
% Meta informations:
\newcommand{\trauthor}{Jim Martens}
\newcommand{\trtype}{Seminar Paper} %{Seminararbeit} %{Proseminararbeit}
\newcommand{\trcourse}{Neural Networks}
\newcommand{\trtitle}{Outline - Second Environmental Feedback Loop}
\newcommand{\trmatrikelnummer}{6420323}
\newcommand{\tremail}{2martens@informatik.uni-hamburg.de}
\newcommand{\trarbeitsbereich}{Knowledge Technology, WTM}
\newcommand{\trdate}{26.04.2018}

%%%%%%%%%%%%%%%%%%%%%%%%%%%%%%%%%%%%%%%%%%%%%%%%%%%%%%%%%%%%%
% Languages:

% Falls die Ausarbeitung in Deutsch erfolgt:
% \usepackage[german]{babel}
% \usepackage[T1]{fontenc}
% \usepackage[latin1]{inputenc}
% \usepackage[latin9]{inputenc}
% \selectlanguage{german}

% If the thesis is written in English:
\usepackage[spanish,english]{babel}
\selectlanguage{english}

%%%%%%%%%%%%%%%%%%%%%%%%%%%%%%%%%%%%%%%%%%%%%%%%%%%%%%%%%%%%%
% Bind packages:
\usepackage[utf8]{inputenc} % Unicode funktioniert unter Windows, Linux und Mac
\usepackage[T1]{fontenc}
\usepackage{acronym}                    % Acronyms
\usepackage{algorithmic}								% Algorithms and Pseudocode
\usepackage{algorithm}									% Algorithms and Pseudocode
\usepackage{amsfonts}                   % AMS Math Packet (Fonts)
\usepackage{amsmath}                    % AMS Math Packet
\usepackage{amssymb}                    % Additional mathematical symbols
\usepackage{amsthm}
\usepackage{booktabs}                   % Nicer tables
%\usepackage[font=small,labelfont=bf]{caption} % Numbered captions for figures
\usepackage{color}                      % Enables defining of colors via \definecolor
\definecolor{uhhRed}{RGB}{254,0,0}		  % Official Uni Hamburg Red
\definecolor{uhhGrey}{RGB}{122,122,120} % Official Uni Hamburg Grey
\usepackage{fancybox}                   % Gleichungen einrahmen
%\usepackage{fancyhdr}										% Packet for nicer headers
\usepackage[automark]{scrlayer-scrpage}
\usepackage[hidelinks]{hyperref}\urlstyle{rm}
%\usepackage{fancyheadings}             % Nicer numbering of headlines

%\usepackage[outer=3.35cm]{geometry} 	  % Type area (size, margins...) !!!Release version
%\usepackage[outer=2.5cm]{geometry} 		% Type area (size, margins...) !!!Print version
%\usepackage{geometry} 									% Type area (size, margins...) !!!Proofread version
\usepackage[outer=3.15cm]{geometry} 	  % Type area (size, margins...) !!!Draft version
\geometry{a4paper,body={5.8in,9in}}

\usepackage{graphicx}                   % Inclusion of graphics
%\usepackage{latexsym}                  % Special symbols
\usepackage{longtable}									% Allow tables over several parges
\usepackage{listings}                   % Nicer source code listings
\usepackage{multicol}										% Content of a table over several columns
\usepackage{multirow}										% Content of a table over several rows
\usepackage{rotating}										% Alows to rotate text and objects
\usepackage[hang]{subfigure}            % Allows to use multiple (partial) figures in a fig
%\usepackage[font=footnotesize,labelfont=rm]{subfig}	% Pictures in a floating environment
\usepackage{tabularx}										% Tables with fixed width but variable rows
\usepackage{url,xspace,boxedminipage}   % Accurate display of URLs

\usepackage{csquotes}
\usepackage[
backend=biber,
bibstyle=ieee,
citestyle=ieee,
minnames=1,
maxnames=2
]{biblatex}

\addbibresource{bib.bib}

%%%%%%%%%%%%%%%%%%%%%%%%%%%%%%%%%%%%%%%%%%%%%%%%%%%%%%%%%%%%%
% Configurationen:

\hyphenation{whe-ther} 									% Manually use: "\-" in a word: Staats\-ver\-trag

%\lstloadlanguages{C}                   % Set the default language for listings
\DeclareGraphicsExtensions{.pdf,.svg,.jpg,.png,.eps} % first try pdf, then eps, png and jpg
\graphicspath{{./src/}} 								% Path to a folder where all pictures are located
%\pagestyle{fancy} 											% Use nicer header and footer
\pagestyle{scrheadings}

% Redefine the environments for floating objects:
\setcounter{topnumber}{3}
\setcounter{bottomnumber}{2}
\setcounter{totalnumber}{4}
\renewcommand{\topfraction}{0.9} 			  %Standard: 0.7
\renewcommand{\bottomfraction}{0.5}		  %Standard: 0.3
\renewcommand{\textfraction}{0.1}		  	%Standard: 0.2
\renewcommand{\floatpagefraction}{0.8} 	%Standard: 0.5

% Tables with a nicer padding:
\renewcommand{\arraystretch}{1.2}

%%%%%%%%%%%%%%%%%%%%%%%%%%%%
% Additional 'theorem' and 'definition' blocks:
\theoremstyle{plain}
\newtheorem{theorem}{Theorem}[section]
%\newtheorem{theorem}{Satz}[section]		% Wenn in Deutsch geschrieben wird.
\newtheorem{axiom}{Axiom}[section]
%\newtheorem{axiom}{Fakt}[chapter]			% Wenn in Deutsch geschrieben wird.
%Usage:%\begin{axiom}[optional description]%Main part%\end{fakt}

\theoremstyle{definition}
\newtheorem{definition}{Definition}[section]

%Additional types of axioms:
\newtheorem{lemma}[axiom]{Lemma}
\newtheorem{observation}[axiom]{Observation}

%Additional types of definitions:
\theoremstyle{remark}
%\newtheorem{remark}[definition]{Bemerkung} % Wenn in Deutsch geschrieben wird.
\newtheorem{remark}[definition]{Remark}

%%%%%%%%%%%%%%%%%%%%%%%%%%%%
% Provides TODOs within the margin:
\newcommand{\TODO}[1]{\marginpar{\emph{\small{{\bf TODO: } #1}}}}

%%%%%%%%%%%%%%%%%%%%%%%%%%%%
% Abbreviations and mathematical symbols
\newcommand{\modd}{\text{ mod }}
\newcommand{\RS}{\mathbb{R}}
\newcommand{\NS}{\mathbb{N}}
\newcommand{\ZS}{\mathbb{Z}}
\newcommand{\dnormal}{\mathit{N}}
\newcommand{\duniform}{\mathit{U}}

\newcommand{\erdos}{Erd\H{o}s}
\newcommand{\renyi}{-R\'{e}nyi}
%%%%%%%%%%%%%%%%%%%%%%%%%%%%%%%%%%%%%%%%%%%%%%%%%%%%%%%%%%%%%
% Document:
\begin{document}
\renewcommand{\headheight}{14.5pt}

%\fancyhead{}
%\fancyhead[LE]{ \slshape \trauthor}
%\fancyhead[LO]{}
%\fancyhead[RE]{}
%\fancyhead[RO]{ \slshape \trtitle}
\lehead{\slshape \trauthor}
\rohead{\slshape \trtitle}

%%%%%%%%%%%%%%%%%%%%%%%%%%%%
% Cover Header:
\begin{titlepage}
	\begin{flushleft}
		Universit\"at Hamburg\\
		Department Informatik\\
		\trarbeitsbereich\\
	\end{flushleft}
	\vspace{3.5cm}
	\begin{center}
		\huge \trtitle\\
	\end{center}
	\vspace{3.5cm}
	\begin{center}
		\normalsize\trtype\\
		[0.2cm]
		\Large\trcourse\\
		[1.5cm]
		\Large \trauthor\\
		[0.2cm]
		\normalsize Matr.Nr. \trmatrikelnummer\\
		[0.2cm]
		\normalsize\tremail\\
		[1.5cm]
		\Large \trdate
	\end{center}
	\vfill
\end{titlepage}

	%backsite of cover sheet is empty!
\thispagestyle{empty}
\hspace{1cm}
\newpage

%%%%%%%%%%%%%%%%%%%%%%%%%%%%
% Abstract:

% Abstract gives a brief summary of the main points of a paper:
\section*{Abstract}
  Your text here...

% Lists:
\setcounter{tocdepth}{2} 					% depth of the table of contents (for Seminars 2 is recommented)
\tableofcontents
\pagenumbering{arabic}
\clearpage

%%%%%%%%%%%%%%%%%%%%%%%%%%%%
% Content:

% the actual content, usually separated over a number of sections
% each section is assigned a label, in order to be able to put a
% crossreference to it

\section{Introduction}
\label{sec:introduction}

Autonomous robots need to adapt to new situations. They have a need to learn
for an entire life. In order to do this they need a second environmental feedback
loop that tells them when to learn.\cite{Toutounji2016}

The learning poses another problem as well. The previously learned weights
are usually largely forgotten, which is known as catastrophic forgetting.\cite{French1999}

Since catastrophic forgetting is a key problem for autonomous learning, it is
crucial to overcome it. In this paper I will present some approaches for
learning in an autonomous setup to analyse which of them if any can overcome
catastrophic forgetting. Attempts to overcome that were made by
Kirkpatrick\cite{Kirkpatrick2017}, Velez\cite{Velez2017} and Shmelkov\cite{Shmelkov2017}.

In the context of this paper plasticity refers to synaptic plasticity as described
in Citri\cite{Citri2008}. The process of learning itself, changing the weights,
is already considered plasticity. It is important however that the weights can
be changed during the lifetime of the neural network. Ordinary neural networks
trained with backpropagation are not involving plasticity.

\section{Catastrophic Forgetting}
\label{sec:catastrophicforgetting}

Catastrophic Forgetting is a key issue for multi-task learning which itself
is a requirement for any neural network that is capable of solving more than
one type of problem.

\section{Plasticity}
\label{sec:plasticity}

Plasticity can be realized by various approaches. Here three approaches are
presented, Modulated Random Search, Modulated Gaussian Walk and Diffusion based
neuromodulation.

\subsection{Modulated Random Search}
\label{subsec:mrs}

Does things.

\subsection{Modulated Gaussian Walk}
\label{subsec:mgw}

Does things more efficient.

\subsection{Localized learning}
\label{subsec:diffusion}

Velez describe another approach that employs
modularity for the learning. Essentially this results in task-specific localized
learning and functional modules for each subtask.

\section{Comparison regarding catastrophic forgetting}
\label{sec:comparison}

Modulated Random Search is not at all useful for overcoming catastrophic forgetting.
Modulated Gaussian Walk improves to that end.
Localized learning overcomes catastrophic forgetting for small
networks.

\section{Conclusion}
\label{sec:concl}

A second environmental feedback loop is important to tell autonomous systems
when to learn. But the method to learn is important as well to be any use
in a practical environment. A comparison has shown that localized learning
can overcome catastrophic forgetting for small networks.

%%%%%%%%%%%%%%%%%%%%%%%%%%%%%%%%%%%%%%
% hier werden - zum Ende des Textes - die bibliographischen Referenzen
% eingebunden
%
% Insbesondere stehen die eigentlichen Informationen in der Datei
% ``bib.bib''
%
\newpage
\printbibliography
\addcontentsline{toc}{section}{Bibliography}% Add to the TOC

\end{document}
