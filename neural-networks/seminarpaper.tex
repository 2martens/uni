\documentclass[12pt,twoside]{scrartcl}
%%%%%%%%%%%%%%%%%%%%%%%%%%%%%%%%%%%%%%%%%%%%%%%%%%%%%%%%%%%%%
% Meta informations:
\newcommand{\trauthor}{Jim Martens}
\newcommand{\trtype}{Seminar Paper} %{Seminararbeit} %{Proseminararbeit}
\newcommand{\trcourse}{Neural Networks}
\newcommand{\trtitle}{Outline - Second Environmental Feedback Loop}
\newcommand{\trmatrikelnummer}{6420323}
\newcommand{\tremail}{2martens@informatik.uni-hamburg.de}
\newcommand{\trarbeitsbereich}{Knowledge Technology, WTM}
\newcommand{\trdate}{26.04.2018}

%%%%%%%%%%%%%%%%%%%%%%%%%%%%%%%%%%%%%%%%%%%%%%%%%%%%%%%%%%%%%
% Languages:

% Falls die Ausarbeitung in Deutsch erfolgt:
% \usepackage[german]{babel}
% \usepackage[T1]{fontenc}
% \usepackage[latin1]{inputenc}
% \usepackage[latin9]{inputenc}
% \selectlanguage{german}

% If the thesis is written in English:
\usepackage[spanish,english]{babel}
\selectlanguage{english}

%%%%%%%%%%%%%%%%%%%%%%%%%%%%%%%%%%%%%%%%%%%%%%%%%%%%%%%%%%%%%
% Bind packages:
\usepackage[utf8]{inputenc} % Unicode funktioniert unter Windows, Linux und Mac
\usepackage[T1]{fontenc}
\usepackage{acronym}                    % Acronyms
\usepackage{algorithmic}								% Algorithms and Pseudocode
\usepackage{algorithm}									% Algorithms and Pseudocode
\usepackage{amsfonts}                   % AMS Math Packet (Fonts)
\usepackage{amsmath}                    % AMS Math Packet
\usepackage{amssymb}                    % Additional mathematical symbols
\usepackage{amsthm}
\usepackage{booktabs}                   % Nicer tables
%\usepackage[font=small,labelfont=bf]{caption} % Numbered captions for figures
\usepackage{color}                      % Enables defining of colors via \definecolor
\definecolor{uhhRed}{RGB}{254,0,0}		  % Official Uni Hamburg Red
\definecolor{uhhGrey}{RGB}{122,122,120} % Official Uni Hamburg Grey
\usepackage{fancybox}                   % Gleichungen einrahmen
%\usepackage{fancyhdr}										% Packet for nicer headers
\usepackage[automark]{scrlayer-scrpage}
\usepackage[hidelinks]{hyperref}\urlstyle{rm}
%\usepackage{fancyheadings}             % Nicer numbering of headlines

%\usepackage[outer=3.35cm]{geometry} 	  % Type area (size, margins...) !!!Release version
%\usepackage[outer=2.5cm]{geometry} 		% Type area (size, margins...) !!!Print version
%\usepackage{geometry} 									% Type area (size, margins...) !!!Proofread version
\usepackage[outer=3.15cm]{geometry} 	  % Type area (size, margins...) !!!Draft version
\geometry{a4paper,body={5.8in,9in}}

\usepackage{graphicx}                   % Inclusion of graphics
%\usepackage{latexsym}                  % Special symbols
\usepackage{longtable}									% Allow tables over several parges
\usepackage{listings}                   % Nicer source code listings
\usepackage{multicol}										% Content of a table over several columns
\usepackage{multirow}										% Content of a table over several rows
\usepackage{rotating}										% Alows to rotate text and objects
\usepackage[hang]{subfigure}            % Allows to use multiple (partial) figures in a fig
%\usepackage[font=footnotesize,labelfont=rm]{subfig}	% Pictures in a floating environment
\usepackage{tabularx}										% Tables with fixed width but variable rows
\usepackage{url,xspace,boxedminipage}   % Accurate display of URLs

\usepackage{csquotes}
\usepackage[
backend=biber,
bibstyle=ieee,
citestyle=ieee,
minnames=1,
maxnames=2
]{biblatex}

\addbibresource{bib.bib}

%%%%%%%%%%%%%%%%%%%%%%%%%%%%%%%%%%%%%%%%%%%%%%%%%%%%%%%%%%%%%
% Configurationen:

\hyphenation{whe-ther} 									% Manually use: "\-" in a word: Staats\-ver\-trag

%\lstloadlanguages{C}                   % Set the default language for listings
\DeclareGraphicsExtensions{.pdf,.svg,.jpg,.png,.eps} % first try pdf, then eps, png and jpg
\graphicspath{{./src/}} 								% Path to a folder where all pictures are located
%\pagestyle{fancy} 											% Use nicer header and footer
\pagestyle{scrheadings}

% Redefine the environments for floating objects:
\setcounter{topnumber}{3}
\setcounter{bottomnumber}{2}
\setcounter{totalnumber}{4}
\renewcommand{\topfraction}{0.9} 			  %Standard: 0.7
\renewcommand{\bottomfraction}{0.5}		  %Standard: 0.3
\renewcommand{\textfraction}{0.1}		  	%Standard: 0.2
\renewcommand{\floatpagefraction}{0.8} 	%Standard: 0.5

% Tables with a nicer padding:
\renewcommand{\arraystretch}{1.2}
\MakeOuterQuote{"}

%%%%%%%%%%%%%%%%%%%%%%%%%%%%
% Additional 'theorem' and 'definition' blocks:
\theoremstyle{plain}
\newtheorem{theorem}{Theorem}[section]
%\newtheorem{theorem}{Satz}[section]		% Wenn in Deutsch geschrieben wird.
\newtheorem{axiom}{Axiom}[section]
%\newtheorem{axiom}{Fakt}[chapter]			% Wenn in Deutsch geschrieben wird.
%Usage:%\begin{axiom}[optional description]%Main part%\end{fakt}

\theoremstyle{definition}
\newtheorem{definition}{Definition}[section]

%Additional types of axioms:
\newtheorem{lemma}[axiom]{Lemma}
\newtheorem{observation}[axiom]{Observation}

%Additional types of definitions:
\theoremstyle{remark}
%\newtheorem{remark}[definition]{Bemerkung} % Wenn in Deutsch geschrieben wird.
\newtheorem{remark}[definition]{Remark}

%%%%%%%%%%%%%%%%%%%%%%%%%%%%
% Provides TODOs within the margin:
\newcommand{\TODO}[1]{\marginpar{\emph{\small{{\bf TODO: } #1}}}}

%%%%%%%%%%%%%%%%%%%%%%%%%%%%
% Abbreviations and mathematical symbols
\newcommand{\modd}{\text{ mod }}
\newcommand{\RS}{\mathbb{R}}
\newcommand{\NS}{\mathbb{N}}
\newcommand{\ZS}{\mathbb{Z}}
\newcommand{\dnormal}{\mathit{N}}
\newcommand{\duniform}{\mathit{U}}

\newcommand{\erdos}{Erd\H{o}s}
\newcommand{\renyi}{-R\'{e}nyi}
%%%%%%%%%%%%%%%%%%%%%%%%%%%%%%%%%%%%%%%%%%%%%%%%%%%%%%%%%%%%%
% Document:
\begin{document}
\renewcommand{\headheight}{14.5pt}

%\fancyhead{}
%\fancyhead[LE]{ \slshape \trauthor}
%\fancyhead[LO]{}
%\fancyhead[RE]{}
%\fancyhead[RO]{ \slshape \trtitle}
\lehead{\slshape \trauthor}
\rohead{\slshape \trtitle}

%%%%%%%%%%%%%%%%%%%%%%%%%%%%
% Cover Header:
\begin{titlepage}
	\begin{flushleft}
		Universit\"at Hamburg\\
		Department Informatik\\
		\trarbeitsbereich\\
	\end{flushleft}
	\vspace{3.5cm}
	\begin{center}
		\huge \trtitle\\
	\end{center}
	\vspace{3.5cm}
	\begin{center}
		\normalsize\trtype\\
		[0.2cm]
		\Large\trcourse\\
		[1.5cm]
		\Large \trauthor\\
		[0.2cm]
		\normalsize Matr.Nr. \trmatrikelnummer\\
		[0.2cm]
		\normalsize\tremail\\
		[1.5cm]
		\Large \trdate
	\end{center}
	\vfill
\end{titlepage}

	%backsite of cover sheet is empty!
\thispagestyle{empty}
\hspace{1cm}
\newpage

%%%%%%%%%%%%%%%%%%%%%%%%%%%%
% Abstract:

% Abstract gives a brief summary of the main points of a paper:
\section*{Abstract}
  Catastrophic forgetting is a huge problem for neural networks, in particular
  for autonomous systems. This paper will showcase three approaches using
  diffusion-based neuromodulation and compare them with respect to catastrophic
  forgetting. The results of the comparison being that modulated random search
  is not useful to combat catastrophic forgetting, modulated gaussian walk is
  significantly better on that front and that the localized learning approach of
  Velez and Clune overcomes catastrophic forgetting for small networks.

% Lists:
\setcounter{tocdepth}{2} 					% depth of the table of contents (for Seminars 2 is recommented)
\tableofcontents
\pagenumbering{arabic}
\clearpage

%%%%%%%%%%%%%%%%%%%%%%%%%%%%
% Content:

% the actual content, usually separated over a number of sections
% each section is assigned a label, in order to be able to put a
% crossreference to it

\section{Introduction}
\label{sec:introduction}

Autonomous robots need to adapt to new situations. They have a need to learn
for an entire life. In order to do this they need a second environmental feedback
loop that tells them when to learn.\cite{Toutounji2016}

The learning itself is also described as plasticity. In the context of this paper
the definition of synaptic plasticity given by Citri\cite{Citri2008} will be used.
In short the process of learning itself, changing the weights, is already
considered plasticity. This can occur throughout the lifetime of a network or
during the training phase of networks using for example supervised learning
and backpropagation.

When a network has to adapt to new situations, it has to learn new tasks. Usually
the previously learned weights are largely forgotten. This phenomenon is called
catastrophic forgetting.\cite{French1999,McCloskey1989}

Since catastrophic forgetting is a key problem for autonomous learning, it is
crucial to overcome it. In this paper I will present some approaches for
learning in an autonomous setup to analyse which of them if any can overcome
catastrophic forgetting.

\section{Catastrophic Forgetting}
\label{sec:catastrophicforgetting}

This section presents the major research developments related to catastrophic
forgetting and explains what it actually is. It follows the review of French\cite{French1999}.

McCloskey and Cohen\cite{McCloskey1989} originally discovered the problem of
catastrophic forgetting, which was referred to as catastrophic interference. This
discovery of a fundamental limitation of the classic neural network was as
important as the work of Minsky and Papert\cite{Minsky1969} who described the
limitations of a perceptron twenty years prior. The key discovery of McCloskey
and Cohen was that previously learned patterns were completely forgotten after
a few training cycles of learning a new pattern. The reason behind this
behaviour was the real problem. They identified the single set of shared
weights as responsible for it.

If one thinks a few minutes about it this makes absolute sense. The classic
backpropagation algorithm works by modifying the weights that have contributed
the most to a bad outcome. When the set of targets is changing then the network
will perform badly for the new pattern. In order to rectify this the backpropagation
algorithm will change many of the weights so that the network is delivering a
good result for the new pattern. This on the other hand results in an increasingly
worse performance on previously learned patterns. If this worsening was gradual
it would still be unfortunate but understandable. It is called catastrophic
because this performance change is not gradual but rather abrupt. Even small
changes in the weights can have a huge impact on the output.

In fact catastrophic forgetting is only a very radical example of a more general
problem for all models of memory, the so called "stability-plasticity" problem\cite{Grossberg1982}.
This problem, sometimes called dilemma, basically puts up the question how to
design a system in such a way that it is both sensitive to new input and not
radically disrupted by it. In other words: That it can learn new things
without largely or completely forgetting already learned things.

Early attempts to alleviate or overcome catastrophic forgetting required a more
sparse representation. This means that not every weight is responsible for all
possible inputs. The downside is a worse ability to generalize to new input
and overall a worse ability to discriminate. In an extreme form this can
lead to catastrophic remembering\cite{Sharkey1995}.
The idea here is that a
network learns the function describing the inputs too well and therefore
loses its ability to differentiate between new and already learned input.
This can be understood well with the example given by French\cite{French1999},
where a network has the task to reproduce the input at the output. It can detect
a new input if the output is diverging by a large margin. It has learned too well
if it learned the identity function and is therefore able to reproduce any
input perfectly at the output and hence loses the ability to detect new input.

Significant improvements were made by rehearsing previously learned input.
Robins\cite{Robins1995} found a way to rehearse prior input if it is no longer
available and called it "pseudo-patterns". The idea being that the weights
of the trained network resemble a function. A random input and the predicted
output together somewhat describe this function and are such a pattern. Robins
used many of them interleaved with new input and the results were promising
as the forgetting became more gradual. This insight together with
the findings of McClelland\cite{McClelland1995} resulted in the development
of dual-network models.
In short one network would model the hippocampus and be able to quickly learn new
information without disrupting previously learned regularities. This network
would then serve as teacher for the second network which models the neocortex
and is responsible for generalizing.

In the time between 1999 and 2018 more work was done with regards to catastrophic
forgetting. Most recently the work of Kirkpatrick\cite{Kirkpatrick2017},
Velez\cite{Velez2017} and Shmelkov\cite{Shmelkov2017} has to be named. The main
focus here was a brief introduction of catastrophic forgetting and therefore
these newest approaches will not be described here.

\section{Plasticity}
\label{sec:plasticity}

Every neural network involves a learning aspect and hence plasticity, given our
definition of it. In this section three approaches for plasticity using diffusion-based
neuromodulation are presented in more detail. Modulated Random Search and
Modulated Gaussian Walk are using linearly modulated neural networks. They
are taken from Toutounji and Pasemann\cite{Toutounji2016}. The third approach
was introduced by Velez and Clune\cite{Velez2017} uses diffusion-based neuromodulation
for localized learning hence the name of the subsection here.

\subsection{Modulated Random Search}
\label{subsec:mrs}

\subsubsection*{Modulated Neural Network}

Since both approaches from Toutounji and Pasemann are using linearly-modulated
neural networks the structure of these networks is described first. Linearly-modulated
neural networks (LMNN) are a specific variant of modulated neural
networks (MNN). Any artifical neural network (ANN) or simply neural network
in the context of Computer Science can become a modulated neural network by
adding a neuromodulator layer.

Toutounji and Pasemann describe a variant of this layer that uses neuromodulator
cells (NMCs). Each NMC produces a specific type of neuromodulator (NM) and
saves its own concentration level of it. The network wide concentration level at
a certain point in space and time can be obtained by summing up all concentration
levels saved in NMCs at the point in space. Produced neuromodulators usually
impact nearby network parts. This type of spatial impact requires a spatial
representation in the network where all network elements have a location in
the space.

There are a production and a reduction mode for the NMCs. During the production
mode the concentration of neuromodulator can be increased and during reduction
mode it can be decreased. A cell can enter production mode if it was stimulated
for some time while it falls back to reduction mode when this stimulation
does not happen for some time.

\subsubsection*{Linearly-Modulated Neural Network}

A linearly-modulated neural network uses discrete time and stimulates NMCs
with a simple linear model. Each NMC is connected to a carrier cell or neuron
which itself is part of a modulatory subnetwork. The NMC is stimulated if the
output of the carrier neuron is within a specified range
(\(\text{S}^{\text{min}}\), \(\text{S}^{\text{max}}\)). In every time step
is checked if the output of the carrier cell is high enough to stimulate the
NMC. If it is the stimulation level of the NMC increases. Otherwise it decreases.
Once the stimulation level reaches the threshold \(\text{T}^{\text{prod}}\)
the cell goes into the production mode. If it falls below \(\text{T}^{\text{red}}\)
the cell goes back into reduction mode.
Over time the neuromodulator diffuses to the surrounding control subnetwork
where it initiates plasticity that is dependent on the concentration of it at
the respective synapse.

\subsubsection*{Modulated Random Search}

\begin{table}
    \begin{tabular}{l|l}
        \textbf{Parameter} & \textbf{Description} \\
        \(Type\) & The neuromodulator type the synapse is sensitive to \\
        \(W\) & Weight change probability \\
        \(D\) & Disable / enable probability \\
        \(W^{min}, W^{max}\) & Minimum and maximum weight of synapse \\
        \(M\) & Maximum neuromodulator sensitivity limit of the synapse
    \end{tabular}
    \caption{Parameters stored for each synapse.
    Replication of Table 1 in Toutunji and Pasemann\cite{Toutounji2016}.}
    \label{tab:mrs-synapse}
\end{table}

Modulated random search means essentially random weight changes. Each synapse
has some parameters that are used (see table \ref{tab:mrs-synapse}). The weight
change probability is the product of the intrinsic weight change probability
and the concentration of the neuromodulator the synapse is sensitive to
at its location. Additionally the maximum neuromodulator sensitivity is
the ceiling for the second part of that product. This means there is a maximum
weight change probability for each synapse. Should a weight change occur a new
weight is chosen randomly from the range of values described by the minimum and
maximum weight of the synapse.

Moreover a synapse can disable or enable itself. The actual disable/enable
probability is the product of the intrinsic value saved as parameter and
the neuromodulator concentration. The concentration is again ceiled by the
maximum sensitivity limit given as parameter. This means there is a maximum
disable/enable probability as well. A disabled synapse is treated as having
weight 0 but the actual value is stored so that it can be restored when the
synapse is enabled again.

Given a so called neural network structure or substrate this makes it easier
to find different network topologies (structure and weights combined).

\subsection{Modulated Gaussian Walk}
\label{subsec:mgw}

Toutounji and Pasemann introduce the modulated gaussian walk. The key differences
start with the parameters. There is no maximum sensitivity for the neuromodulator
concentration. When a weight change occurs the new weight is not chosen randomly
but rather the difference to be added to the current weight is sampled from a
normal distribution with a mean of zero and \(\sigma^2\)-variance. The sampled
value could be infinitely large and hence the new weight outside of the given
bounds for it. Therefore the value is sampled until the sum of the
current weight and the sampled value are within the range.

Toutunji and Pasemann implemented a mechanism for disabling synapses
in the modulated gaussian walk as well but did not make use of it later and
therefore they did not describe how it works.

\subsection{Localized learning}
\label{subsec:diffusion}

Velez and Clune are using a small network to solve the foraging task. The network
represents an agent that has a lifetime of three years. Each year consists of
the seasons summer and winter. During each season the agent is presented with
food and has to either eat the food or not. Half of the food is nutritious
and the other poisonous. The target is a fitness value which is best if the
agent eats all nutritious food and none of the poisonous. The associations of
nutritious and poisonous are different between summer and winter but within
a season remain the same during the lifetime. Therefore a nutritious food in
summer will always be nutritious.
This setup makes it easy to measure if the agent is able to remember the learned
associations from the previous seasons.

The initial weights of the network are derived from an evolutionary algorithm.
All later learning uses neuromodulation. The neurons of the network are spatially
located and there are two sources of neuromodulators in the network - one on either
side. The sources are only active in their respective season and encode whether
the previously eaten food was nutritious (1) or poisonous (-1). If they are
not active their value is zero. As soon as the sources are activated the neuromodulators
fill a space within a radius of 1.5 units of distance from the source and potentially
trigger weight changes of neurons inside the radius. The strength of the neuromodulators
is decreasing with further distance from the source.

This explanation should suffice for the general understanding of their method.
The neurons within the vicinity of these sources only update their weights
in one of the seasons. Therefore they only learn for one season and are unaffected
by the other season. This results in a localized learning.

\section{Comparison regarding catastrophic forgetting}
\label{sec:comparison}

In this section the three presented approaches for plasticity are compared with
regard to their ability to mitigate or overcome catastrophic forgetting. For both
the modulated random search and the modulated gaussian walk this aspect was
analyzed in the experiments conducted by Toutunji and Pasemann\cite{Toutounji2016}.
Therefore the results of their work will be utilized for this comparison.
Velez and Clune\cite{Velez2017} introduced the presented approach of localized
learning to analyze its capability with respect to overcoming catastrophic
forgetting. Hence their results will be used for the comparison in this section.

Over multiple experiments of increasing difficulty the performance of modulated
random search and modulated gaussian walk were tested. The difficulty ranged
from a positive light-tropism task in the first experiment(E1) over an
obstacle-avoidance task in the second experiment (E2), a combination of E1 and E2
in the third experiment (E3) to a more difficult variant of E3 in the fourth
experiment (E4). The fifth experiment (E5) was a pendulum experiment.
In each experiment a robot had to learn the task from scratch. A pre-designed
LMNN was given in each case and defined the boundaries in which the learning
took place. If a temporary solution was discarded the learning started again.

Modulated random search was able to find successful behaviours in almost all
cases in E1 despite a short training time of only two hours. The slightly
longer training time for E2 of four hours however was apparently far too short
to find consistently good solutions. Both in E3 and E4 the number of intermediate
temporary solutions is significantly higher than the final number of solutions.
The pendulum experiment was an easier task and therefore many successful
behaviours were found.

Toutounji and Pasemann note that even almost stable networks are destroyed
if they have the slightest weakness. Therefore modulated random search
does not help at all against catastrophic forgetting.

Modulated gaussian walk contrary to the random search tends to improve temporary
solutions when they have weaknesses. For E3 the random search resulted in 34
temporary solutions which lasted longer than five minutes, averaging at \(5.7\)
minutes per solution. The gaussian walk found roughly twice that many temporary
solutions and averaged at \(12.5\) minutes per solution. This indicates that
gaussian walk mitigates catastrophic forgetting although it does not completely
remove it.

The experiment setup for the localized learning approach was already mentioned.
After performing some tests Velez and Clune discovered that two functional
modules formed. One set of connections is learning during sommer and the other
during winter. The connections learning in summer do not change in winter and
vice versa. This completely removes catastrophic forgetting.

If catastrophic forgetting is the only measurement then localized learning
seems to be the supreme solution to the problem. But Velez and Clune only
showed that it works in a very bespoke setup which a priori information about
the linear separability of the learning areas and correct solution. It has yet
to be shown that localized learning can be generalized to larger problems.
Modulated random search can be completely discarded as a potential solution.
Modulated gaussian walk is a clear improvement compared to the random search
in the analyzed experiments.

While all three approaches used diffusion-based neuromodulation the first two and
the third are quite different in their setup. For the future it would be interesting
to combine localized learning with gaussian walk on the experiments of Toutunji
and Pasemann. In particular the combined experiment might benefit from this as
localized learning could separate the learning for one task from the other
and gaussian walk could then improve the particular part that was problematic.

\section{Conclusion}
\label{sec:concl}

A second environmental feedback loop is important to tell autonomous systems
when to learn. But the method to learn is important as well to be any use
in a practical environment. The comparison has shown that localized learning
can overcome catastrophic forgetting for small networks in a very restricted
setup. Furthermore the comparison revealed that modulated random search is
not part of a solution to catastrophic forgetting and modulated gaussian
walk is significantly better in that regard.

Future work should look into the effects of localized learning on the kind of
autonomous robot experiments that were conducted by Toutunji and Pasemann and
in general research the applicability to bigger problems for example in the area
of deep neural networks.

%%%%%%%%%%%%%%%%%%%%%%%%%%%%%%%%%%%%%%
% hier werden - zum Ende des Textes - die bibliographischen Referenzen
% eingebunden
%
% Insbesondere stehen die eigentlichen Informationen in der Datei
% ``bib.bib''
%
\newpage
\printbibliography
\addcontentsline{toc}{section}{Bibliography}% Add to the TOC

\end{document}
