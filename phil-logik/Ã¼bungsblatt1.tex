\documentclass[10pt,a4paper,oneside,ngerman,numbers=noenddot]{scrartcl}
\usepackage[T1]{fontenc}
\usepackage[utf8]{inputenc}
\usepackage[ngerman]{babel}
\usepackage{amsmath}
\usepackage{amsfonts}
\usepackage{amssymb}
\usepackage{paralist}
\usepackage{gauss}
\usepackage{stmaryrd}
\usepackage{enumitem}
\usepackage[locale=DE,exponent-product=\cdot,detect-all]{siunitx}
\usepackage{tikz}
\usetikzlibrary{automata,matrix,fadings,calc,positioning,decorations.pathreplacing,arrows,decorations.markings}
\usepackage{polynom}
\polyset{style=C, div=:,vars=x}
\pagenumbering{arabic}
\def\thesection{\arabic{section}.}
\def\thesubsection{\thesection\arabic{subsection}}
\def\thesubsubsection{(\alph{subsubsection})}
\makeatletter
\renewcommand*\env@matrix[1][*\c@MaxMatrixCols c]{%
  \hskip -\arraycolsep
  \let\@ifnextchar\new@ifnextchar
  \array{#1}}
\makeatother
\addtolength{\parskip}{\baselineskip}

\begin{document}
\author{Jim Martens (6420323)}
\title{Prüfungsblatt Nr. 1}
\maketitle

\section{Argumente} %1.

\begin{enumerate}[label=\thesection\arabic*]
    \item

Kein Argument entsprechend der Definition. Inhaltlich geht es um eine
mathematische Wahrheit, die aber ebenfalls nicht vernünftig argumentiert
wird.

In Standardform sähe es dennoch so aus:

\begin{alignat*}{1}
    \text{(P1) } 1=1. \\
    \text{(P2) } 2=2. \\
    \text{(P3) } 3=3. \\
    \hline
    \text{Also (K) } 4=4.
\end{alignat*}

    \item

Kein Argument entsprechend der Definition, da es keine Konklusion gibt. Es kann
aber ein argumentativer Inhalt erkannt werden, wenn der letzte Satz zu einer
Konklusion gemacht wird.

\begin{alignat*}{1}
    \text{(P1) Die Wirtschaft stagniert.} \\
    \text{(P2) Die Arbeitslosigkeit steigt stetig.} \\
    \text{(P3) Das Angebot an Lehrstellen sinkt.} \\
    \hline
    \text{Also (K) Die Linke ist nicht mehr, was sie mal war.}
\end{alignat*}

    \item

%Bernd stellt sich an wie das letzte Rindvieh. Also wirklich, sein Verhalten geht
% mir mächtig auf den Keks.

Auch wenn die Formulierung sehr allgemeinsprachlich ist, so lässt sich ein
Argument erkennen.

\begin{alignat*}{1}
    \text{(P1) Bernd stellt sich an wie das letzte Rindvieh.} \\
    \hline
    \text{Also (K) Sein Verhalten geht mir mächtig auf den Keks.}
\end{alignat*}

    \item

Dieser Dialog enthält ein Argument, auch wenn die Reihenfolge nicht der
Definition entspricht. In Standardform sieht dies wie folgt aus:

\begin{alignat*}{1}
    \text{(P1) Die Dame hält eine Bahnkarte in der Innenfläche ihres linken Handschuhs.} \\
    \text{(P2) Das Datum der Fahrkarte weist den heutigen Tag aus.} \\
    \hline
    \text{Also (K) Die Dame ist heute morgen mit der Bahn angereist.}
\end{alignat*}

    \item

Auch hier findet sich ein Argument, obgleich die Form nicht der Definition
entspricht.

\begin{alignat*}{1}
    \text{(P1) Das Zusammengesetzte ist eine Anhäufung oder ein Aggregat vom Einfachen.} \\
    \text{(P2) Es gibt zusammengesetzte Substanzen.} \\
    \hline
    \text{Also (K) Es gibt einfache Substanzen.}
\end{alignat*}

\end{enumerate}

\section{Modalität} %2.

\begin{enumerate}[label=\thesection\arabic*]
    \item Notwendig falsch.
    \item Notwendig wahr.
    \item Kontingent wahr.
    \item Notwendig falsch.
    \item Notwendig wahr, da ein Bild niemals das echte Objekt ist.
    \item Kontingent falsch, da eine gemalte Pfeife auch ein fiktives Objekt beschreiben kann,
    zu welchem dann keine Abbildrelation besteht. Dennoch ist eine Welt vorstellbar,
    wo dieser Satz wahr ist.
    \item Diese Aussage ist zu unspezifisch, um sie klar zu beantworten. Im Vakuum ist
    das Licht die absolut schnellste Sache. Dort wäre die Aussage kontingent falsch.
    Kontingent falsch, weil auch ein Universum vorstellbar ist (siehe Star Trek),
    wo es Dinge gibt, die schneller als Licht im Vakuum sind.

    In anderen Situationen als dem Vakuum kann es auch auf unserer Welt sein,
    dass manche Dinge schneller als Licht sind. Dort wäre der Satz also kontingent wahr.

    Insgesamt lässt sich dies nicht eindeutig beantworten. Unter der Annahme, dass
    auf die allgemeine Regel "Nichts ist schneller als Licht" abgehoben wird,
    wäre dieser Satz kontingent falsch.
\end{enumerate}

\section{Wahr oder Falsch?} %3.

\begin{enumerate}[label=\thesection\arabic*]
    \item In dieser Pauschalität ist das falsch, da die Gültigkeit nur aussagt,
    dass die Konklusion wahr ist, wenn alle Prämissen wahr sind.
    \item Falsch, da alle Prämissen wahr sein könnten und die Konklusion falsch.
    In dem Fall wäre das Argument nicht gültig und somit auch nicht schlüssig.
    Dennoch wären alle Prämissen wahr.
    \item Falsch, da die Definition zur Gültigkeit nur etwas über die Konklusion
    sagt, wenn alle Prämissen wahr sind. Nämlich, dass in diesem Fall auch die
    Konklusion wahr ist. Daraus folgt aber nicht der Umkehrschluss, dass die
    Konklusion falsch sein muss, wenn nicht alle Prämissen wahr sind.
    \item Wahr, da es Argumente mit mehr als einer Prämisse gibt, wo mindestens
    eine Prämisse falsch ist, sodass auch die Konklusion falsch sein kann, ohne
    die Gültigkeit des Arguments zu gefährden.
    \item Wahr, da das Argument nicht schlüssig sein könnte, wenn die Konklusion
    falsch wäre.
\end{enumerate}

\section{Schlüssigkeit und Gültigkeit erkennen} % 4.

\begin{enumerate}[label=\thesection\arabic*]
    \item Schlüssig und somit gültig, da Kant auf jeden Fall einmal Junggeselle
    war und somit auch in seinem Leben einmal unverheiratet war.
    \item Schlüssig und somit gültig. Begründung analog zu 4.1
    \item Nicht gültig und somit nicht schlüssig, da die Prämisse wahr ist,
    die Konklusion jedoch nicht.
    \item Gültig und schlüssig, da beide Prämissen wahr sind und die Konklusion
    aus den Prämissen folgt.
\end{enumerate}

\section{Wohlgeformtheit} % 5.

\begin{enumerate}[label=\thesection\arabic*]
    \item Wohlgeformt, da:
    \begin{alignat*}{1}
        P \text{ und } Q \text{ sind wohlgeformt.} \\
        (P \& Q) \text{ ist wohlgeformt.} \\
        R \text{ ist wohlgeformt.} \\
        ((P \& Q) \& R) \text{ ist wohlgeformt.}
    \end{alignat*}
    \item Nicht wohlgeformt, da bei der Und-Verknüpfung Klammern entstehen
    und nur die äußeren Klammern weggelassen werden dürfen nach Konvention.
    In diesem Fall sind gar keine Klammern gesetzt, weswegen die Formel nicht
    wohlgeformt ist.
    \item Nicht wohlgeformt, da auch hier die inneren Klammern weggelassen wurden.
    Aufgrunddessen ist nicht ersichtlich, ob das \(\vee\) oder \(\rightarrow\)
    Hauptjunktor ist.
    \item Nicht wohlgeformt, da bei alleiniger Anwendung der Negation keine
    Klammern entstehen können.
\end{enumerate}

\end{document}
