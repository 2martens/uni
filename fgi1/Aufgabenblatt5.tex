\documentclass[10pt,a4paper,oneside,ngerman,numbers=noenddot]{scrartcl}
\usepackage[T1]{fontenc}
\usepackage[utf8]{inputenc}
\usepackage[ngerman]{babel}
\usepackage{amsmath}
\usepackage{amsfonts}
\usepackage{amssymb}
\usepackage{paralist}
\usepackage{gauss}
\usepackage{stmaryrd}
\usepackage[locale=DE,exponent-product=\cdot,detect-all]{siunitx}
\usepackage{tikz}
\usetikzlibrary{automata,matrix,fadings,calc,positioning,decorations.pathreplacing,arrows,decorations.markings}
\usepackage{polynom}
\polyset{style=C, div=:,vars=x}
\pagenumbering{arabic}
\def\thesection{5.\arabic{section})}
\def\thesubsection{\arabic{subsection}.}
\def\thesubsubsection{(\roman{subsubsection})}
\setcounter{section}{3}
\makeatletter
\renewcommand*\env@matrix[1][*\c@MaxMatrixCols c]{%
  \hskip -\arraycolsep
  \let\@ifnextchar\new@ifnextchar
  \array{#1}}
\makeatother

\begin{document}
\author{Jim Martens}
\title{Hausaufgaben zum 7. Mai}
\maketitle
\section{} %5.4
\subsection{} %1.
$G_{1}$:\\
\begin{alignat*}{2}
S &\rightarrow & aSb | A \\
A &\rightarrow & bAa | \lambda
\end{alignat*}
Behauptung: Es gilt $L(G_{1}) = L_{1}$.\\
Zunächst zeige ich $L(G_{1}) \subseteq L_{1}$. Sei dazu $w \in L(G_{1})$, d.h. es gilt $S \overset{*}{\Longrightarrow} w$. Durch $n$-malige Anwendung der Produktion $S \rightarrow aSb$ erhält man zunächst $S \overset{*}{\Longrightarrow} a^{n}Sb^{n}$. Anschließend muss die Produktion $S \rightarrow A$ gewählt werden. Durch $m$-malige Anwendung der Regel $A \rightarrow bAa$ erhält man $S \overset{*}{\Longrightarrow} a^{n}Sb^{n} \overset{*}{\Longrightarrow} a^{n}b^{m}Aa^{m}b^{n}$. Abschließend wird die Produktion $A \rightarrow \lambda$ gewählt. Insgesamt muss $w$ damit die Form $a^{n}b^{m}a^{m}b^{n}$ haben. Es gilt also $w \in L_{1}$.\\
\\
Sei umgekehrt $w \in L_{1}$. Hat $w$ die Form $a^{n}b^{m}a^{m}b^{n}$ für $n,m \geq 0$, so kann $w$ durch $n$-malige Anwendung der Produktion $S \rightarrow aSb$, Anwendung der Regel $S \rightarrow A$, $m$-malige Anwendung von $A \rightarrow bAa$ und abschließende Anwendung von $A \rightarrow \lambda$ abgeleitet werden.\\
Damit gilt für jedes $w \in L_{1}$ demnach auch $w \in L(G_{1})$ und die Behauptung ist gezeigt.\\
\\
Bei Anwendung des Pumping-Lemmas kommt es bei der Wahl des Wortes zu Problemen. Bei einer kontextfreien Sprache gilt das Pumping-Lemma für alle Wörter der Sprache. Möchte man dies nun zeigen, so müsste man unendlich viele Wörter zeigen, was nicht machbar ist.\\
Sollte man einen Widerspruch zeigen wollen, so wird man kein Wort der Sprache finden, bei dem es einen gibt, denn bei einer kontextfreien Sprache gilt das Pumping-Lemma für jedes Wort.
\subsection{} %2.
\begin{alignat*}{2}
L_{2} &=& \{a^{n}b^{m}a^{n}b^{m} | n,m \geq 0\}
\end{alignat*}\\
Angenommen $L_{2}$ wäre kontextfrei.\\
Sei $n$ die Zahl aus dem Pumping Lemma. Wähle $z = a^{n}b^{n}a^{n}b^{n}$ mit $n \in \mathbb{N}$.\\
Die ersten $a$s seien durch $a_{1}$ markiert und die ersten $b$s durch $b_{1}$. Die jeweils zweiten Blöcke durch den Index $2$.\\
Also $z \in L_{2}, |z| = 4n$.\\
Also existiert eine Zerlegung $z = uvwxy$ mit $(i)$ und $(ii)$. Ich führe nun jede dieser Zerlegungen zu einem Widerspruch mit der dritten Eigenschaft des Pumping Lemmas.\\

\begin{enumerate}[i)]
	\item $|vx| \geq 1$
	\item $|vwx| \leq n$
	\item $\forall i \geq 0:uv^{i}wx^{i}y \in L$
\end{enumerate}
Zunächst gibt es wegen $|vwx| \leq n$ und der Form von $z$ sieben Fälle zu unterscheiden: $vwx \in \{a_{1}\}^{*}, vwx \in \{a_{1}\}^{*}\{b_{1}\}^{*}, vwx \in \{b_{1}\}^{*}, vwx \in \{b_{1}\}^{*}\{a_{2}\}^{*}, vwx \in \{a_{2}\}^{*}, vwx \in \{a_{2}\}^{*}\{b_{2}\}^{*} \text{ und } vwx \in \{b_{2}\}^{*}$. In den Fällen $vwx \in \{a_{1}\}^{*}, vwx \in \{b_{1}\}^{*}, vwx \in \{a_{2}\}^{*}$ und $vwx \in \{b_{2}\}^{*}$ führt wegen $|vx| \geq 1$ die Betrachtung von $uv^{2}wx^{2}y$ bereits zum Widerspruch. Es ist dann $uv^{2}wx^{2}y = a^{n+j}b^{n}a^{n}b^{n}$, $uv^{2}wx^{2}y = a^{n}b^{n+j}a^{n}b^{n}$, $uv^{2}wx^{2}y = a^{n}b^{n}a^{n+j}b^{n}$ bzw. $uv^{2}wx^{2}y = a^{n}b^{n}a^{n}b^{n+j}$ mit $j > 0$ und folglich gilt nicht mehr, dass der Exponent von den beiden $a$- bzw. $b$-Blöcken gleich sein muss.\\
\\
Gilt in $vwx \in \{a_{1}\}^{*}\{b_{1}\}^{*}, vwx \in \{b_{1}\}^{*}\{a_{2}\}^{*}$ bzw. $vwx \in \{a_{2}\}^{*}\{b_{2}\}^{*}$, dass $v$ oder $x$ bereits aus mehr als einem Symbol aufgebaut sind (also selbst in $\{a\}^{+}\{b\}^{+}$ bzw. $\{b\}^{+}\{a\}^{+}$ sind), so führt $uv^{2}wx^{2}y$ sofort zu einem Widerspruch, da dieses Wort dann nicht einmal mehr in $\{a\}^{*}\{b\}^{*}\{a\}^{*}\{b\}^{*}$ ist (und damit ganz bestimmt nicht in $L_{2}$).\\
\\
Ist jedoch im Fall $vwx \in \{a_{1}\}^{*}\{b_{1}\}^{*}$ das $v$ aus den $a$s und das $x$ in den $b$s, so führt ähnlich wie bei den zuerst diskutierten Fällen die Betrachtung von $uv^{2}wx^{2}y = a^{n+j}b^{n+k}a^{n}b^{n}$ zu einem Widerspruch ($n+j = n$ bzw. $n+k = n$ gilt nicht, da wegen $|vx| \geq 1$ auch $j+k \geq 1$ sein muss). Der Fall $vwx \in \{a_{2}\}^{*}\{b_{2}\}^{*}$ führt analog zu einem Widerspruch (lediglich mit den beiden hinteren Blöcken).\\
Ist im Fall von $vwx \in \{b_{1}\}^{*}\{a_{2}\}^{*}$ das $v$ in den $b$s und das $x$ in den $a$s, so führt ebenfalls ähnlich zu den zuerst diskutierten Fällen die Betrachtung von $uv^{2}wx^{2}y = a^{n}b^{n+j}a^{n+k}b^{n}$ zu einem Widerspruch (abermals gilt $n+j = n$ bzw. $n+k = n$ nicht, da $j+k \geq 1$ sein muss).\\
\\
Damit sind alle Fälle zum Widerspruch geführt und es gibt also keine Zerlegung von $z$ in $uvwxy$ derart, dass die drei Bedingungen des Pumping Lemmas erfüllt sind. Folglich ist die ursprüngliche Annahme, $L_{2}$ wäre kontextfrei nicht haltbar und $L_{2}$ ist also nicht kontextfrei.
\section{} %5.5
\subsection{} %1.
$G_{1}$:\\
\begin{alignat*}{2}
S &\rightarrow & aSc | A \\
A &\rightarrow & bc | \lambda
\end{alignat*}
Behauptung: Es gilt $L(G_{1}) = L_{1}$.\\
Zunächst zeige ich $L(G_{1}) \subseteq L_{1}$. Sei dazu $w \in L(G_{1})$, d.h. es gilt $S \overset{*}{\Longrightarrow} w$. Durch $i$-malige Anwendung der Produktion $S \rightarrow aSc$ erhält man zunächst $S \overset{*}{\Longrightarrow} a^{i}Sc^{i}$. Anschließend muss die Produktion $S \rightarrow A$ gewählt werden. Durch $j$-malige Anwendung der Regel $A \rightarrow bAc$ erhält man $S \overset{*}{\Longrightarrow} a^{i}Sc^{i} \overset{*}{\Longrightarrow} a^{i}b^{j}Ac^{i+j}$. Abschließend wird die Produktion $A \rightarrow \lambda$ gewählt. Insgesamt muss $w$ damit die Form $a^{i}b^{j}c^{i+j} = a^{i}b^{j}c^{k}$ für $k = i+j$ haben. Es gilt also $w \in L_{1}$.\\
\\
Sei umgekehrt $w \in L_{1}$. Hat $w$ die Form $a^{i}b^{j}c^{k} = a^{i}b^{j}c^{i+j}$ für $k = i+j$, so kann $w$ durch $i$-malige Anwendung der Produktion $S \rightarrow aSc$, Anwendung der Regel $S \rightarrow A$, $j$-malige Anwendung von $A \rightarrow bAc$ und abschließende Anwendung von $A \rightarrow \lambda$ abgeleitet werden.\\
Damit gilt für jedes $w \in L_{1}$ demnach auch $w \in L(G_{1})$ und die Behauptung ist gezeigt.
\subsection{} %2.

\section{} %5.6
\subsection{} %1.
$z_{0}$ ist der Start- und $z_{4}$ der einzige Endzustand. Im Zustand $z_{0}$ können beliebig viele $a$s gelesen werden (auch keine). Anschließend kann bei leerer Eingabe in den Endzustand gewechselt werden oder bei einem $b$ in der Eingabe in den Zustand $z_{1}$. Der Zustand kann nur wieder verlassen werden, wenn mindestens ein weiteres $b$ von der Eingabe gelesen wird. Im Zustand $z_{2}$ befindet man sich demnach nur, wenn zuvor jeweils 2 $b$s gelesen wurden. Entweder direkt von $z_{0}$ nach $z_{1}$ und von $z_{1}$ nach $z_{2}$ oder von $z_{2}$ nach $z_{1}$ und von $z_{1}$ nach $z_{2}$.\\
Nur beim Wechsel von Zustand $z_{1}$ nach $z_{2}$ wird ein $B$ auf den Keller gelegt. Sind alle $b$s gelesen, liegen demnach also halb so viele $B$s auf dem Keller, wie $b$s gelesen wurden.\\
In den Zustand $z_{3}$ kann nur gewechselt werden, wenn ein $a$ gelesen wird und ein $B$ auf dem Keller liegt. $z_{3}$ kann erst wieder verlassen werden, wenn alle $B$s wieder vom Keller gelöscht worden sind.\\
Zum Schluss kann in den Endzustand gewechselt werden. Befindet man sich im Endzustand wurden also beliebig viele $a$s (auch keine), gefolgt von keinen weiteren Zeichen oder $2n$ $b$s und $n$ $a$s gelesen.\\
Ist $w \in L$, so enthält das Wort zunächst beliebig viele $a$s, dann keine weiteren Zeichen oder aber $2n$ $b$s und $n$ $a$s. Dies ist gleichbedeutend mit dem Endzustand des Automaten.\\
Ist umgekehrt $w \in L(A)$, so muss das Lesen von $w$ den Automaten $A$ in den Zustand $z_{4}$ überführen, da dies der einzige Endzustand ist. Dies ist aber gleichbedeutend damit, dass nach Konstruktion von $A$ auch $w = a^{m}b^{2n}a^{n}$, also $w \in L$ gilt.
\subsection{} %2.
\end{document}
