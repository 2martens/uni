\textit{Behauptung}\\
Für alle Formeln F$\, \in \mathcal{L}_{AL}$ gilt, [Behauptung formuliert mit F].\\
\\
\textit{Induktionsanfang}\\
Teilbeweis für die auf atomare Formeln eingeschränkte Behauptung: Für jedes Aussagensymbol
A$\, \in \mathcal{A}s_{AL}$ gilt: [Behauptung formuliert mit A].\\
\\
\textit{Induktionsannahme}\\
Es seien G$_{1}, \,$G$_{2} \in \mathcal{L}_{AL}$ Formeln, für die gilt: [Behauptung formuliert mit G$_{1}$] und [Behauptung
formuliert mit G$_{2}$].\\
\\
\textit{Induktionsschritt}\\
Fall: $\neg \,$G$_{1}$\\
Teilbeweis für [Behauptung formuliert mit $\neg \,$G$_{1}$].\\
(Dieser Teilbeweis darf auf die Induktionsannahme zurückgreifen.)\\
\\
Fall: (G$_{1} \circ \,$G$_{2}$) für $\circ \in \{\vee, \wedge, \Rightarrow, \Leftrightarrow\}$\\
Teilbeweis für [Behauptung formuliert mit (G$_{1} \circ \,$G$_{2}$)].\\
(Dieser Teilbeweis darf auf die Induktionsannahme zurückgreifen. Dabei kann es sein, dass
man alle Operatoren gleich behandeln kann, oder man muss eine Fallunterscheidung nach
Operator machen. Dann kann es hier bis zu 4 Teilbeweise geben.)\\
\\
\textit{Resumé}\\
Nach dem Prinzip der strukturellen Induktion ergibt sich damit: Für alle Formeln F$\, \in \mathcal{L}_{AL}$
gilt, [Behauptung formuliert mit F].