\documentclass[10pt,a4paper,oneside,ngerman,numbers=noenddot]{scrartcl}
\usepackage[T1]{fontenc}
\usepackage[utf8]{inputenc}
\usepackage[ngerman]{babel}
\usepackage{amsmath}
\usepackage{amsfonts}
\usepackage{amssymb}
\usepackage{paralist}
\usepackage{gauss}
\usepackage{stmaryrd}
\usepackage[locale=DE,exponent-product=\cdot,detect-all]{siunitx}
\usepackage{tikz}
\usetikzlibrary{automata,matrix,fadings,calc,positioning,decorations.pathreplacing,arrows,decorations.markings}
\usepackage{polynom}
\polyset{style=C, div=:,vars=x}
\pagenumbering{arabic}
\def\thesection{10.\arabic{section})}
\def\thesubsection{\arabic{subsection}.}
\def\thesubsubsection{(\alph{subsubsection})}
\setcounter{section}{2}
\makeatletter
\renewcommand*\env@matrix[1][*\c@MaxMatrixCols c]{%
  \hskip -\arraycolsep
  \let\@ifnextchar\new@ifnextchar
  \array{#1}}
\makeatother
\addtolength{\parskip}{\baselineskip}

\begin{document}
\author{Jim Martens}
\title{Hausaufgaben zum 11. Juni}
\maketitle

\section{} %10.3
\subsection{} %1.
\subsubsection{} %a
Es ist zu zeigen, dass F aus der angegebenen Formelmenge M folgt. Nach Definition 5.1 folgt F genau dann aus M, wenn alle Modelle von M auch Modelle von F sind. Nach Satz 5.11 sind die Modelle einer Formelmenge identisch mit den Modellen einer Konjunktion aller Mengenglieder.\\
Daraus ergibt sich, dass eine Belegung diese Formel erfüllen muss, um ein Modell von M zu sein:\\
$G = ((B \Rightarrow D) \Rightarrow (A \vee C)) \wedge ((A \vee C) \Rightarrow E) \wedge (B \Rightarrow (D \vee E)) \wedge (E \Rightarrow F)$\\
Daraus folgt wiederum, dass eine Belegung M nur erfüllt, wenn $(E \Rightarrow F)$ erfüllt ist. Diese Teilformel ist wiederum erfüllt, wenn E falsch oder F wahr ist. 
M kann nur erfüllt sein, wenn E wahr ist, denn sonst müsste auch $(A \vee C)$ und damit $(B \Rightarrow D)$ falsch sein. Letzteres würde voraussetzen, dass B wahr und D falsch sind. Daraus würde folgen, dass $(B \Rightarrow (D \vee E))$ falsch ergäbe, da hier nun B wahr und sowohl D als auch E falsch wären. Damit wiederum wäre die Formel G nicht erfüllt.\\
Daraus folgt also, dass E und damit auch F wahr sein müssen, damit eine Belegung ein Modell von M sein kann. Daher ist immer wahr, wenn eine Belegung ein Modell von M ist, womit diese Belegung auch ein Modell von F ist.
\subsubsection{} %b
In diesem Fall kann E falsch sein. Damit müssten auch $(A \vee C)$ und $\neg(B \Rightarrow D)$ falsch sein. $(B \Rightarrow D)$ wiederum müsste wahr sein, was durch den Wahrheitswert falsch für B erreicht werden kann. Damit kann eine Belegung M erfüllen, wenn E falsch ist. Wenn E falsch ist, dann muss F nicht wahr sein, damit $(E \Rightarrow F)$ wahr ist, womit nicht alle Modelle von M auch Modelle von F sind.
\subsection{} %2.
F folgt genau dann aus M, wenn jede Belegung, die M wahr macht, auch F wahr macht. Damit folgt F genau dann aus M, wenn jede Belegung, die alle Formeln aus M wahr macht, $\neg F$ falsch macht. Daher kann F nicht aus $M \cup \{\neg F\}$ folgen, da jede Belegung, die diese Menge erfüllt, F falsifiziert. Wenn F nicht aus $M \cup \{\neg F\}$, dann darf F auch nicht aus M folgen. Dies ist nur möglich, wenn M unerfüllbar ist.
\section{} %10.4
\subsection{} %1.
\setcounter{subsubsection}{0}
\subsubsection{} %a
\begin{alignat*}{2}
\text{sub}_{1a}(A) &=& (A \Rightarrow \neg B) \\
\text{sub}_{1a}(B) &=& (D \wedge A) \\
\text{sub}_{1a}(C) &=& (C \vee D) \\
\intertext{Für alle anderen Aussagensymbole $A_{i}$ sei $sub_{1a}(A_{i}) = A_{i}$}
sub_{1a}(F_{a}) &=& (((A \Rightarrow \neg B) \Rightarrow \neg(D \wedge A)) \wedge (C \vee D)) = G_{a}
\end{alignat*}
\subsubsection{} %b
\begin{alignat*}{2}
\text{sub}_{1b}(A) &=& (\neg(B \vee C) \wedge E) \\
\text{sub}_{1b}(D) &=& \neg(B \vee C) \\
\intertext{Für alle anderen Aussagensymbole $A_{i}$ sei $sub_{1b}(A_{i}) = A_{i}$}
sub_{1b}(F_{b}) &=& ((\neg(B \vee C) \wedge E) \Leftrightarrow \neg(B \vee C)) \\ 
&=& ((\neg(B \vee C) \wedge E) \Leftrightarrow \neg(B \vee C)) = sub_{1b}(G_{b})
\end{alignat*}
\subsection{} %2.
\subsubsection{} %a
$R_{a} = \frac{\neg A}{A \Rightarrow B}$\\
\begin{tabular}{c|cc|cc}
 & F & G & $\neg$F & $(F \Rightarrow G)$ \\
 \hline
$\mathcal{A}_{0}$ & 0 & 0 & 1 & 1 \\
$\mathcal{A}_{1}$ & 0 & 1 & 1 & 1 \\
$\mathcal{A}_{2}$ & 1 & 0 & 0 & 0 \\
$\mathcal{A}_{3}$ & 1 & 1 & 0 & 1
\end{tabular}\\
Für alle Belegungen mit $\mathcal{A}(\neg F) = 1$ gilt auch $\mathcal{A}((F \Rightarrow G)) = 1$.\\
Falls M eine Formelmenge ist und $M \vdash_{R_{a}} G$, dann $M \models G$. \\
Daher ist die Regel korrekt.
\subsubsection{} %b
$R_{b} = \frac{A \Leftrightarrow B}{A}$\\
\begin{tabular}{c|cc|c}
 & F & G & $(F \Leftrightarrow G)$ \\
 \hline
$\mathcal{A}_{0}$ & 0 & 0 & 1 \\
$\mathcal{A}_{1}$ & 0 & 1 & 0 \\
$\mathcal{A}_{2}$ & 1 & 0 & 0 \\
$\mathcal{A}_{3}$ & 1 & 1 & 1
\end{tabular}\\
Für die Belegung $\mathcal{A}_{0}$ mit $\mathcal{A}((F \Leftrightarrow G)) = 1$ gilt nicht $\mathcal{A}(F) = 1$. Demzufolge ist nicht jedes Modell von $(F \Leftrightarrow G)$ auch eines von $F$. Daher ist die Regel nicht korrekt.
\subsubsection{} %c
$R_{c} = \frac{A \vee B, B \vee C}{A \vee C}$\\
\begin{tabular}{c|ccc|ccc}
 & F & G & H & $(F \vee G)$ & $(G \vee H)$ & $(F \vee H)$ \\
 \hline
$\mathcal{A}_{0}$ & 0 & 0 & 0 & 0 & 0 & 0\\
$\mathcal{A}_{1}$ & 0 & 0 & 1 & 0 & 1 & 1\\
$\mathcal{A}_{2}$ & 0 & 1 & 0 & 1 & 1 & 0\\
$\mathcal{A}_{3}$ & 0 & 1 & 1 & 1 & 1 & 1\\
$\mathcal{A}_{4}$ & 1 & 0 & 0 & 1 & 0 & 1\\
$\mathcal{A}_{5}$ & 1 & 0 & 1 & 1 & 1 & 1\\
$\mathcal{A}_{6}$ & 1 & 1 & 0 & 1 & 1 & 1\\
$\mathcal{A}_{7}$ & 1 & 1 & 1 & 1 & 1 & 1\\
\end{tabular}\\
Für die Belegung $\mathcal{A}_{2}$ mit $\mathcal{A}((F \vee G)) = 1$ und $\mathcal{A}((G \vee H)) = 1$ gilt nicht $\mathcal{A}((F \vee H)) = 1$.\\
Demzufolge erfüllt nicht jede Belegung, die $(F \vee G)$ und $(G \vee H)$ erfüllt, auch $(F \vee H)$. Daher ist die Regel nicht korrekt.
\subsection{} %3.
\def\thesubsubsection{(\arabic{subsubsection})}
\subsubsection{} %1
Annahme aus M
\subsubsection{} %2
Axiom $H_{6b}$ \\
$sub_{2}(A) = (G \Rightarrow H)$ \\
$sub_{2}(B) = (H \Rightarrow F)$
\subsubsection{} %3
Modus ponens\\
$sub_{3}(A) = (G \Rightarrow H) \wedge (H \Rightarrow F)$ \\
$sub_{3}(B) = (H \Rightarrow F)$
\subsubsection{} %4
Axiom $H_{1}$ \\
$sub_{4}(A) = (H \Rightarrow F)$\\
$sub_{4}(B) = G$
\subsubsection{} %5
Modus ponens\\
$sub_{5}(A) = (H \Rightarrow F)$\\
$sub_{5}(B) = (G \Rightarrow (H \Rightarrow F))$
\subsubsection{} %6
Axiom $H_{6a}$\\
$sub_{6}(A) = (G \Rightarrow H)$\\
$sub_{6}(B) = (H \Rightarrow F)$
\subsubsection{} %7
Modus ponens\\
$sub_{7}(A) = (G \Rightarrow (H \Rightarrow F))$ \\
$sub_{7}(B) = (G \Rightarrow H)$
\subsubsection{} %8
Axiom $H_{2}$\\
$sub_{8}(A) = G$\\
$sub_{8}(B) = H$\\
$sub_{8}(C) = F$
\subsubsection{} %9
Modus ponens\\
$sub_{9}(A) = (G \Rightarrow H)$\\
$sub_{9}(B) = ((G \Rightarrow (H \Rightarrow F)) \Rightarrow (G \Rightarrow F))$
\subsubsection{} %10
Modus ponens\\
$sub_{10}(A) = (G \Rightarrow (H \Rightarrow F))$\\
$sub_{10}(B) = (G \Rightarrow F)$
\end{document}
