\documentclass[10pt,a4paper,oneside,ngerman,numbers=noenddot]{scrartcl}
\usepackage[T1]{fontenc}
\usepackage[utf8]{inputenc}
\usepackage[ngerman]{babel}
\usepackage{amsmath}
\usepackage{amsfonts}
\usepackage{amssymb}
\usepackage{paralist}
\usepackage{gauss}
\usepackage{stmaryrd}
\usepackage[locale=DE,exponent-product=\cdot,detect-all]{siunitx}
\usepackage{tikz}
\usetikzlibrary{automata,matrix,fadings,calc,positioning,decorations.pathreplacing,arrows,decorations.markings}
\usepackage{polynom}
\polyset{style=C, div=:,vars=x}
\pagenumbering{arabic}
\def\thesection{3.\arabic{section})}
\def\thesubsection{\arabic{subsection}.}
\def\thesubsubsection{(\roman{subsubsection})}
\setcounter{section}{3}
\makeatletter
\renewcommand*\env@matrix[1][*\c@MaxMatrixCols c]{%
  \hskip -\arraycolsep
  \let\@ifnextchar\new@ifnextchar
  \array{#1}}
\makeatother

\begin{document}
\author{Jim Martens}
\title{Hausaufgaben zum 23. April}
\maketitle
\section{} %3.4
\subsection{} %1.
\begin{alignat*}{2}
L_{1} &=& \{a^{2n}bcba^{2n} | n \geq 0 \}
\end{alignat*}
Angenommen $L_{1} \in REG$. Sei $n$ die PL-Zahl. Betrachte $z = a^{2n}bcba^{2n} \in L_{1}$. Also $|z| = 4n + 3 \geq n$. Also existiert eine Zerlegung $z = uvw$ mit (i), (ii) und (iii).\\
\begin{enumerate}[i)]
	\item $|uv| \leq n$ 
	\item $|v| \geq 1$
	\item $\forall i \in \mathbb{N}: uv^{i}w \in L$
\end{enumerate}
\begin{alignat*}{2}
\overset{i)}{\Rightarrow} uv \in \{a\}^{*} &\Rightarrow & u,v \in \{a\}^{*} \\
&\overset{ii)}{\Rightarrow} & v \in \{a\}^{+} \text{ d.h. } v=a^{m} \text{ für } 1 \leq m \leq n
\end{alignat*}\\
Mit iii) folgt für $i = 0$, dass $uv^{i}w = a^{2n-m}bcba^{2n} \not\in L_{1} \lightning$.\\
Also $L_{1} \not\in REG$.
\subsection{} %2.
\begin{alignat*}{2}
L_{2} &=& \{w \in \{0,1\}^{*} | |w|_{1} = 2 \cdot |w|_{0} \}
\end{alignat*}
Angenommen $L_{2} \in REG$. Sei $n$ die PL-Zahl. Betrachte $z = 0^{n}1^{2n} \in L_{2}$. Also $|z| = 3n \geq n$. Also existiert eine Zerlegung $z = uvw$ mit (i), (ii) und (iii).\\
\begin{enumerate}[i)]
	\item $|uv| \leq n$ 
	\item $|v| \geq 1$
	\item $\forall i \in \mathbb{N}: uv^{i}w \in L$
\end{enumerate}
\begin{alignat*}{2}
\overset{i)}{\Rightarrow} uv \in \{0\}^{*} &\Rightarrow & u,v \in \{0\}^{*} \\
&\overset{ii)}{\Rightarrow} & v \in \{0\}^{+} \text{ d.h. } v=a^{m} \text{ für } 1 \leq m \leq n
\end{alignat*}\\
Mit iii) folgt für $i = 0$, dass $uv^{i}w = 0^{n-m}1^{2n} \not\in L \lightning$.\\
Also $L \not\in REG$.
\section{} %3.5
\subsection{} %1.
Die Menge $\overline{\{a\}^{*}}$ lässt sich über dem Alphabet $\Sigma$ auch so schreiben: $\Sigma^{*} \setminus \{a\}^{*}$. Die regulären Ausdrücke $A = (a|b)^{*}$ und $B = a^{*}$ beschreiben jeweils $\Sigma^{*}$ und $\{a\}^{*}$.
Der Ausdruck $C = B \cdot b^{+} \cdot A$ beschreibt die gesuchte Menge $\overline{\{a\}^{*}}$.\\
Dieser Ausdruck akzeptiert beliebig viele $a$ am Anfang (auch keine), erwartet dann mindestens ein $b$ aber auch nicht mehr und gibt einem danach die freie Wahl zwischen beliebig vielen $a$ und $b$ ohne Beschränkung. Dadurch ist gewährleistet, dass in jedem Ausdruck mindestens ein $b$ vorkommt, womit weder das leere Wort noch eine Zeichenkette, die nur aus dem Buchstaben $a$ besteht, akzeptiert werden. Somit wird kein Element von $\{a\}^{*}$ akzeptiert, welches dem Komplement eben dieser Menge entspricht.
\subsection{} %2.
Berechnung eines regulären Ausdruckes für den gegebenen DFA mithilfe des Kleene-Verfahrens.
\begin{alignat*}{2}
R_{1,1}^{0} &=& \{\epsilon , b\} \\
R_{1,2}^{0} &=& \{a\} \\
R_{1,3}^{0} &=& R_{2,1}^{0} = R_{3,1}^{0} = \emptyset \\
R_{2,2}^{0} &=& \{\epsilon \} \\
R_{2,3}^{0} &=& \{a\} \\
R_{3,2}^{0} &=& \{b\} \\
R_{3,3}^{0} &=& \{\epsilon \} \end{alignat*}
\begin{alignat*}{2}
R_{1,2}^{1} &=& R_{1,2}^{0} \cup R_{1,1}^{0} \cdot (R_{1,1}^{0})^{*} \cdot R_{1,2}^{0} \\
&=& R_{1,2}^{0} \cup (R_{1,1}^{0})^{+} \cdot R_{1,2}^{0} \\
&=& (R_{1,1}^{0})^{+} \cdot R_{1,2}^{0} \cup R_{1,2}^{0} \\
&=& (R_{1,1}^{0})^{+} \cdot R_{1,2}^{0} \\
&=& \{\epsilon , b \}^{+} \cdot \{a\} \\
&=& \{b\}^{*} \cdot \{a\} \\
&\overset{\sim}{=}& ((b)^{*} \cdot a) \\
%
R_{2,3}^{1} &=& R_{2,3}^{0} \cup R_{2,1}^{0} \cdot (R_{1,1}^{0})^{*} \cdot R_{1,3}^{0} \\
&=& R_{2,3}^{0} \cup R_{2,1}^{0} \cdot (R_{1,1}^{0})^{*} \cdot R_{1,3}^{0} \\
&=& \{a\} \cup \emptyset \cdot \{\epsilon , b\}^{*} \cdot \emptyset \\
&=& \{a\} \cup \emptyset \\
&=& \{a\} \\
&\overset{\sim}{=}& a \\
%
R_{3,2}^{1} &=& R_{3,2}^{0} \cup R_{3,1}^{0} \cdot (R_{1,1}^{0})^{*} \cdot R_{1,3}^{0} \\
&=& \{b\} \cup \emptyset \cdot \{\epsilon , b\}^{*} \cdot \emptyset \\
&=& \{b\} \cup \emptyset \cdot \{b\}^{*} \cdot \emptyset \\
&=& \{b\} \cup \emptyset \\
&=& \{b\} \\
&\overset{\sim}{=}& b \\
%
R_{2,2}^{1} &=& R_{2,2}^{0} \cup R_{2,1}^{0} \cdot (R_{1,1}^{0})^{*} \cdot R_{1,2}^{0} \\
&=& \{\epsilon \} \cup (\emptyset \cdot \{\epsilon , b\}^{*} \cdot \{a\}) \\
&=& \{\epsilon \} \cup (\emptyset \cdot \{b\}^{*} \cdot \{a\}) \\
&=& \{\epsilon \} \cup \emptyset \\
&=& \{\epsilon \} \\
&\overset{\sim}{=}& \emptyset^{*} \\
%
R_{3,3}^{1} &=& R_{3,3}^{0} \cup R_{3,1}^{0} \cdot (R_{1,1}^{0})^{*} \cdot R_{1,3}^{0} \\
&=& \{\epsilon \} \cup \emptyset \cdot \{\epsilon , b\}^{*} \cdot \emptyset \\
&=& \{\epsilon \} \cup \emptyset \cdot \{b\}^{*} \cdot \emptyset \\
&=& \{\epsilon \} \cup \emptyset \\
&=& \{\epsilon \} \\
&\overset{\sim}{=}& \emptyset^{*}
\end{alignat*}\\
\begin{alignat*}{2}
R_{1,3}^{1} &=& R_{1,3}^{0} \cup R_{1,1}^{0} \cdot (R_{1,1}^{0})^{*} \cdot R_{1,3}^{0} \\
&=& R_{1,3}^{0} \cup (R_{1,1}^{0})^{+} \cdot R_{1,3}^{0} \\
&=& (R_{1,1}^{0})^{+} \cdot R_{1,3}^{0} \cup R_{1,3}^{0}\\
&=& (R_{1,1}^{0})^{+} \cdot R_{1,3}^{0}\\
&=& \{\epsilon , b \}^{+} \cdot \emptyset \\
&=& \{b \}^{*} \cdot \emptyset \\
&=& \emptyset \\
%
R_{1,2}^{2} &=& R_{1,2}^{1} \cup R_{1,2}^{1} \cdot (R_{2,2}^{1})^{*} \cdot R_{2,2}^{1}) \\
&=& R_{1,2}^{1} \cdot (\{\epsilon \} \cup (R_{2,2}^{1})^{*} \cdot R_{2,2}^{1}) \\
&=& R_{1,2}^{1} \cdot (\{\epsilon \} \cup (R_{2,2}^{1})^{+}) \\
&=& R_{1,2}^{1} \cdot (R_{2,2}^{1})^{*} \\
&=& (\{b\}^{*} \cdot \{a\}) \cdot \{\epsilon \}^{*} \\
&=& \{b\}^{*} \cdot \{a\}\\
&\overset{\sim}{=}& ((b)^{*} \cdot a) \\
%
R_{1,3}^{2} &=& R_{1,3}^{1} \cup R_{1,2}^{1} \cdot (R_{2,2}^{1})^{*} \cdot R_{2,3}^{1} \\
&=& \emptyset \cup (\{a\} \cdot \{\epsilon \}^{*} \cdot \{a\}) \\
&=& \emptyset \cup (\{a\} \cdot \{a\}) \\
&=& \emptyset \cup \{aa\} \\
&=& \{aa\} \\
&\overset{\sim}{=}& aa \\
%
R_{3,3}^{2} &=& R_{3,3}^{1} \cup R_{3,2}^{1} \cdot (R_{2,2}^{1})^{*} \cdot R_{2,3}^{1} \\
&=& \{\epsilon \} \cup (\{b\} \cdot \{\epsilon \}^{*} \cdot \{a\}) \\
&=& \{\epsilon \} \cup (\{b\} \cdot \{a\}) \\
&=& \{\epsilon \} \cup \{ba\} \\
&=& \{\epsilon , ba\} \\
&\overset{\sim}{=}& \emptyset^{*} + b \cdot a
\end{alignat*}\\
\begin{alignat*}{2}
R_{3,2}^{2} &=& R_{3,2}^{1} \cup R_{3,2}^{1} \cdot (R_{2,2}^{1})^{*} \cdot R_{2,2}^{1} \\
&=& R_{3,2}^{1} \cdot (\{\epsilon \} \cup (R_{2,2}^{1})^{*} \cdot R_{2,2}^{1}) \\
&=& R_{3,2}^{1} \cdot (\{\epsilon \} \cup (R_{2,2}^{1})^{+}) \\
&=& R_{3,2}^{1} \cdot (R_{2,2}^{1})^{*} \\
&=& \{b\} \cdot \{\epsilon \}^{*} \\
&=& \{b\} \\
&\overset{\sim}{=}& b \\
%
R_{1,3}^{3} &=& R_{1,3}^{2} \cup R_{1,3}^{2} \cdot (R_{3,3}^{2})^{*} \cdot R_{3,3}^{2} \\
&=& R_{1,3}^{2} \cdot (\{\epsilon \} \cup (R_{3,3}^{2})^{*} \cdot R_{3,3}^{2}) \\
&=& R_{1,3}^{2} \cdot (\{\epsilon \} \cup (R_{3,3}^{2})^{+}) \\
&=& R_{1,3}^{2} \cdot (R_{3,3}^{2})^{*} \\
&=& \{aa\} \cdot \{\epsilon , ba\}^{*} \\
&=& \{aa\} \cdot \{ba\}^{*} \\
&\overset{\sim}{=}& a \cdot a \cdot (b \cdot a)^{*} \\
%
R_{1,2}^{3} &=& R_{1,2}^{2} \cup R_{1,3}^{2} \cdot (R_{3,3}^{2})^{*} \cdot R_{3,2}^{2} \\
&=& (\{b\}^{*} \cdot \{a\}) \cup (\{aa\} \cdot \{\epsilon , ba\}^{*} \cdot \{b\}) \\
&=& (\{b\}^{*} \cdot \{a\}) \cup (\{aa\} \cdot \{ba\}^{*} \cdot \{b\}) \\
&\overset{\sim}{=}& (b^{*} \cdot a) + (aa \cdot (b \cdot a)^{*} \cdot b) \\
\end{alignat*}
Der reguläre Ausdruck ergibt sich aus der Vereinigung von $R_{1,3}^{3}$ und $R_{1,2}^{3}$. Es ergibt sich daher folgendes:\\
\begin{alignat*}{2}
R_{1,3}^{3} \cup R_{1,2}^{3} &=& (\{aa\} \cdot \{ba\}^{*}) \cup ((\{b\}^{*} \cdot \{a\}) \cup (\{aa\} \cdot \{ba\}^{*} \cdot \{b\})) \\
&\overset{\sim}{=}& (aa \cdot (ba)^{*}) + (b^{*} \cdot a) + (aa \cdot (ba)^{*} \cdot b)
\end{alignat*}
Der reguläre Ausdruck ist $(aa \cdot (ba)^{*}) + (b^{*} \cdot a) + (aa \cdot (ba)^{*} \cdot b)$.
\section{} %3.6
\subsection{} %1.
\subsection{} %2.
\end{document}
