\documentclass[10pt,a4paper,oneside,ngerman,numbers=noenddot]{scrartcl}
\usepackage[T1]{fontenc}
\usepackage[utf8]{inputenc}
\usepackage[ngerman]{babel}
\usepackage{amsmath}
\usepackage{amsfonts}
\usepackage{amssymb}
\usepackage{paralist}
\usepackage{gauss}
\usepackage{stmaryrd}
\usepackage[locale=DE,exponent-product=\cdot,detect-all]{siunitx}
\usepackage{tikz}
\usetikzlibrary{automata,matrix,fadings,calc,positioning,decorations.pathreplacing,arrows,decorations.markings}
\usepackage{polynom}
\polyset{style=C, div=:,vars=x}
\pagenumbering{arabic}
\def\thesection{11.\arabic{section})}
\def\thesubsection{\arabic{subsection}.}
\def\thesubsubsection{(\alph{subsubsection})}
\setcounter{section}{2}
\makeatletter
\renewcommand*\env@matrix[1][*\c@MaxMatrixCols c]{%
  \hskip -\arraycolsep
  \let\@ifnextchar\new@ifnextchar
  \array{#1}}
\makeatother
\addtolength{\parskip}{\baselineskip}

\begin{document}
\author{Jim Martens}
\title{Hausaufgaben zum 25. Juni}
\maketitle

\section{} %11.3
\subsection{} %1.
\subsubsection{} %a
$M = \{\{A\}\}$\\
\begin{tabular}{c|cc}
 & A \\
 \hline
$\mathcal{A}_{0}$ & 0 \\
$\mathcal{A}_{1}$ & 1
\end{tabular}
\subsubsection{} %b
$M = \{\{A,B,C\}\}$\\
\begin{tabular}{c|ccc|ccc}
& A & B & C & $\{A,B,C\}$ & $\{\{A,B,C\}\}$ & $A \vee (B \vee C)$\\
\hline
$\mathcal{A}_{0}$ & 0 & 0 & 0 & 0 & 0 & 0 \\
$\mathcal{A}_{1}$ & 0 & 0 & 1 & 1 & 1 & 1 \\
$\mathcal{A}_{2}$ & 0 & 1 & 0 & 1 & 1 & 1\\
$\mathcal{A}_{3}$ & 0 & 1 & 1 & 1 & 1 & 1\\
$\mathcal{A}_{4}$ & 1 & 0 & 0 & 1 & 1 & 1\\
$\mathcal{A}_{5}$ & 1 & 0 & 1 & 1 & 1 & 1\\
$\mathcal{A}_{6}$ & 1 & 1 & 0 & 1 & 1 & 1\\
$\mathcal{A}_{7}$ & 1 & 1 & 1 & 1 & 1 & 1\\
\end{tabular}
\subsubsection{} %c
$M = \{\{A\}, \{B\}, \{C\}\}$\\
\begin{tabular}{c|ccc|ccc|cc}
& A & B & C & $\{A\}$ & $\{B\}$ & $\{C\}$ & $\{\{A\}, \{B\}, \{C\}\}$ & $A \wedge (B \wedge C)$\\
\hline
$\mathcal{A}_{0}$ & 0 & 0 & 0 & 0 & 0 & 0 & 0 & 0 \\
$\mathcal{A}_{1}$ & 0 & 0 & 1 & 0 & 0 & 1 & 0 & 0 \\
$\mathcal{A}_{2}$ & 0 & 1 & 0 & 0 & 1 & 0 & 0 & 0\\
$\mathcal{A}_{3}$ & 0 & 1 & 1 & 0 & 1 & 1 & 0 & 0\\
$\mathcal{A}_{4}$ & 1 & 0 & 0 & 1 & 0 & 0 & 0 & 0\\
$\mathcal{A}_{5}$ & 1 & 0 & 1 & 1 & 0 & 1 & 0 & 0\\
$\mathcal{A}_{6}$ & 1 & 1 & 0 & 1 & 1 & 0 & 0 & 0\\
$\mathcal{A}_{7}$ & 1 & 1 & 1 & 1 & 1 & 1 & 1 & 1\\
\end{tabular}
\subsection{} %2.
\begin{alignat*}{2}
M_{1} &= \{\{A\}\} \\
M_{2} &= \{\{\neg A\}\} \\
M_{3} &= \{\{A, \neg A\}\} \\
M_{4} &= \{\{A\}, \{\neg A\}\} \\
M_{5} &= \{\{A\}, \{A, \neg A\}\} \\
M_{6} &= \{\{\neg A\}, \{A, \neg A\}\} \\
M_{7} &= \{\{A\}, \{\neg A\}, \{A, \neg A\}\}
\end{alignat*}\\
$M_{3}$ ist allgemeingültig. $M_{4}$ und $M_{7}$ sind unerfüllbar. $M_{1}$ und $M_{5}$, sowie $M_{2}$ und $M_{6}$ sind äquivalent zueinander.

%\begin{tabular}{c|c|ccccccc}
%& A & $\{\{A\}\}$ & $\{\{\neg A\}\}$ & $\{\{A, \neg A\}\}$ & $\{\{A\}, \{\neg A\}\}$ & $\{\{A\}, \{A, \neg A\}\}$ & $\{\{\neg A\}, \{A, \neg A\}\}$ & $\{\{A\}, \{\neg A\}, \{A, \neg A\}\}$\\
%\hline
%$\mathcal{A}_{0}$ & 0 & 0 & 1 & 1 & 0 & 0 & 1 & 0 \\
%$\mathcal{A}_{1}$ & 1 & 1 & 0 & 1 & 0 & 1 & 0 & 0
%\end{tabular}
\subsection{} %3.
Behauptung: $M \cup \{K\}$ ist genau dann erfüllbar, wenn M erfüllbar ist.\\
Beweis:\\
Es ist zu zeigen, dass (1) $M \cup \{K\}$ erfüllbar ist, wenn M erfüllbar ist und dass (2) M erfüllbar ist, wenn $M \cup \{K\}$ erfüllbar ist.\\
\\
Teilbeweis von (1):\\
M sei erfüllbar.\\
$M \cup \{K\}$ ist erfüllbar, wenn es mindestens eine Belegung gibt, für die sowohl M als auch K erfüllt sind.\\
K ist erfüllt, wenn mindestens eines der Literale in K erfüllt ist.\\
\\
Es gilt $A, \neg A \in K$. Unabhängig von der Belegung von A ist K daher immer erfüllt, denn für $\mathcal{A}(A) = 0$ gilt $\mathcal{A}(\neg A) = 1$ und für $\mathcal{A}(A) = 1$ gilt $\mathcal{A}(A) = 1$.\\
Da K immer erfüllt ist, stellt es bei der Vereinigung mit M keine einschränkende Bedingung dar. Ob die vereinigte Menge erfüllbar ist, hängt damit von M ab. Da M aufgrund der Annahme erfüllbar ist, ist somit auch $M \cup \{K\}$ erfüllbar.\\
\\
Teilbeweis für (2):\\
$M \cup \{K\}$ sei erfüllbar.\\
$M \cup \{K\}$ ist erfüllbar, wenn es mindestens eine Belegung gibt, für die sowohl M als auch K erfüllt sind. Wie bereits gezeigt, ist K für jede Belegung erfüllt. Damit stellt K keine zusätzliche Bedingung und $M \cup \{K\}$ ist erfüllbar, wenn M erfüllbar ist.\\
Aufgrund der Annahme ist $M \cup \{K\}$ erfüllbar, womit M ebenfalls erfüllbar ist.\\
\\
Es konnten beide Richtungen gezeigt werden, womit die Behauptung bewiesen ist.
\section{} %11.4
\subsection{} %1.
\setcounter{subsubsection}{0}
\subsubsection{} %a
\begin{alignat*}{2}
F_{3} &=& C \wedge (\neg A \vee \neg E \vee D) \wedge E \wedge (\neg C \vee B) \wedge (\neg E  \vee \neg B \vee A) \wedge (\neg D \vee \neg C \vee \neg A) \\
\intertext{Umformen in Implikationsschreibweise}
&=& (T \Rightarrow C) \wedge ((A \wedge E) \Rightarrow D) \wedge (T \Rightarrow E) \wedge (C \Rightarrow B) \wedge ((E \wedge B) \Rightarrow A) \wedge ((D \wedge C \wedge A) \Rightarrow \perp) \\
\intertext{Anwenden des Markierungsalgorithmus}
&=& (T \Rightarrow C^{1}) \wedge ((A^{3} \wedge E^{1}) \Rightarrow D^{4}) \wedge (T \Rightarrow E^{1}) \wedge (C^{1} \Rightarrow B^{2}) \wedge ((E^{1} \wedge B^{2}) \Rightarrow A^{3}) \wedge ((D^{4} \wedge C^{1} \wedge A^{3}) \Rightarrow \perp) \\
\intertext{Die Formel ist unerfüllbar.}
\end{alignat*}\\
\\\\\\
Anwenden der P-Resolution:\\
\begin{tabular}{cccccc}
($\neg D \vee \neg C \vee \neg A$) & C & ($\neg C \vee B$) & ($\neg E \vee \neg B \vee A$) & E & ($\neg A \vee \neg E \vee D$)  \\
& ($\neg D \vee \neg A$) & B & ($\neg B \vee A$) & & ($\neg A \vee D$) \\
& & & A & & \\
& & $\neg D$ & & D & \\
& & & $\Box$ & &
\end{tabular} \\
Die Formel ist auch nach der P-Resolution unerfüllbar.
\subsubsection{} %b
\begin{alignat*}{2}
F_{4} &=& \neg A \wedge (\neg A \vee C \vee \neg E) \wedge (A \vee \neg C \vee \neg E) \wedge B \wedge (\neg B \vee D) \wedge (\neg B \vee \neg D \vee E) \\
\intertext{Umformen in Implikationsschreibweise}
&=& (A \Rightarrow \perp) \wedge ((A \wedge E) \Rightarrow C) \wedge ((C \wedge E) \Rightarrow A) \wedge (T \Rightarrow B) \wedge (B \Rightarrow D) \wedge ((B \wedge D) \Rightarrow E) \\
\intertext{Anwenden des Markierungsalgorithmus}
&=& (A \Rightarrow \perp) \wedge ((A \wedge E^{3}) \Rightarrow C) \wedge ((C \wedge E^{3}) \Rightarrow A) \wedge (T \Rightarrow B^{1}) \wedge (B^{1} \Rightarrow D^{2}) \wedge ((B^{1} \wedge D^{2}) \Rightarrow E^{3}) \\
\intertext{Die Formel ist erfüllbar.}
\end{alignat*}\\
Anwenden der P-Resolution:\\
\begin{tabular}{cccccc}
($\neg A \vee C \vee \neg E$) & ($\neg B \vee D$) & B & ($\neg B \vee \neg D \vee E$) & ($A \vee \neg C \vee \neg E$) \\
& D & & ($\neg D \vee E$) & & \\
& & E & & & \\
& ($\neg A \vee C$) & & ($A \vee \neg C$) & &
\end{tabular}\\
Es können keine weiteren Resolventen mit der P-Resolution gebildet werden. Die Formel ist daher nicht unerfüllbar und damit erfüllbar.
\subsection{} %2.
\begin{alignat*}{2}
F_{5} &=& (A \vee B) \wedge (A \vee D) \wedge (C \vee D) \wedge (\neg A \vee \neg C) \wedge (\neg B \vee \neg D) \wedge (B \vee \neg A \vee \neg D)
\end{alignat*}\\
Anwenden der P-Resolution:\\
\begin{tabular}{cccccc}
\{A,B\} & \{$\neg B, \neg D$\} & \{A,D\} & \{$\neg A, \neg C$\} & \{C,D\} & \{$B, \neg A, \neg D$\} \\
& & \{$A, \neg B$\} & & & \\
& \{A\} & & & & \\
& & \{$\neg C$\} & & \{$B, \neg D$\} & \\
& & & \{D\} & & \\
& \{$\neg B$\} & & & \{B\} & \\
& & & $\Box$ & &
\end{tabular}\\
Die Formel ist nach der P-Resolution unerfüllbar.
\subsection{} %3.
\begin{alignat*}{2}
F_{6} &=& (A \Rightarrow C) \Rightarrow ((B \Rightarrow C) \Rightarrow ((A \vee B) \Rightarrow C)) \\
\intertext{Eliminieren von Implikationen}
&=& \neg(\neg A \vee C) \vee (\neg(\neg B \vee C) \vee (\neg(A \vee B) \vee C)) \\
\intertext{Negationen nach innen ziehen}
&=& (A \wedge \neg C) \vee ((B \wedge \neg C) \vee ((\neg A \wedge \neg B) \vee C))
\end{alignat*}\\
\end{document}
