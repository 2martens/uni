\documentclass[10pt,a4paper,oneside,ngerman,numbers=noenddot]{scrartcl}
\usepackage[T1]{fontenc}
\usepackage[utf8]{inputenc}
\usepackage[ngerman]{babel}
\usepackage{amsmath}
\usepackage{amsfonts}
\usepackage{amssymb}
\usepackage{paralist}
\usepackage{gauss}
\usepackage{stmaryrd}
\usepackage[locale=DE,exponent-product=\cdot,detect-all]{siunitx}
\usepackage{tikz}
\usetikzlibrary{automata,matrix,fadings,calc,positioning,decorations.pathreplacing,arrows,decorations.markings}
\usepackage{polynom}
\polyset{style=C, div=:,vars=x}
\pagenumbering{arabic}
\def\thesection{4.\arabic{section})}
\def\thesubsection{\arabic{subsection}.}
\def\thesubsubsection{(\roman{subsubsection})}
\setcounter{section}{3}
\makeatletter
\renewcommand*\env@matrix[1][*\c@MaxMatrixCols c]{%
  \hskip -\arraycolsep
  \let\@ifnextchar\new@ifnextchar
  \array{#1}}
\makeatother

\begin{document}
\author{Jim Martens}
\title{Hausaufgaben zum 30. April}
\maketitle
\section{} %4.4
\subsection{} %1.
$G_{1}$:\\
\begin{alignat*}{2}
S &\rightarrow & aS | AB  \\
A &\rightarrow & aAb | \lambda \\
B &\rightarrow & bBc | \lambda
\end{alignat*}
Ich behaupte $L(G_{1}) = \{a^{i + j}b^{i + k}c^{k}|i,j,k \geq 0\} =: M$.
Zunächst zeige ich $L(G_{1}) \subseteq M$. Sei dazu $w \in L(G_{1})$, d.h. es gilt $S \overset{*}{\Longrightarrow} w$. Wie kann $w \in V_{T}^{*}$ entstanden sein? Durch $j$-malige Anwendung der Produktion $S \rightarrow aS$ erhält man zunächst $S \overset{*}{\Longrightarrow} a^{j}S$. Anschließend muss die Produktion $S \rightarrow AB$ benutzt werden. Dies führt zu $S \overset{*}{\Longrightarrow} a^{j}S \overset{*}{\Longrightarrow} a^{j}AB$. Durch $i$-malige Anwendung der Produktion $A \rightarrow aAb$ erhält man $S \overset{*}{\Longrightarrow} a^{j}S \overset{*}{\Longrightarrow} a^{j}AB \overset{*}{\Longrightarrow} a^{j}a^{i}Ab^{i}B$. Dann muss die Produktion $A \rightarrow \lambda$ gewählt werden. Es ergibt sich daher 
$S \overset{*}{\Longrightarrow} a^{j}S \overset{*}{\Longrightarrow} a^{j}AB \overset{*}{\Longrightarrow} a^{j}a^{i}Ab^{i}B \Longrightarrow a^{j}a^{i}b^{i}B$. Durch $k$-malige Anwendung der Produktion $B \rightarrow  bBc$ erhält man $S \overset{*}{\Longrightarrow} a^{j}S \overset{*}{\Longrightarrow} a^{j}AB \overset{*}{\Longrightarrow} a^{j}a^{i}Ab^{i}B \Longrightarrow a^{j}a^{i}b^{i}B \overset{*}{\Longrightarrow} a^{j}a^{i}b^{i}b^{k}Bc^{k}$. Abschließend muss die Produktion $B \rightarrow \lambda$ gewählt werden.\\
\\
Insgesamt muss $w$ damit die Form $a^{j}a^{i}b^{i}b^{k}c^{k} = a^{i + j}b^{i + k}c^{k}$ haben. Es gilt also $w \in M$.\\
\\
Sei umgekehrt $w \in M$. Hat $w$ die Form $a^{i + j}b^{i + k}c^{k} = a^{j}a^{i}b^{i}b^{k}c^{k}$ für $i,j,k \geq 0$, so kann $w$ durch $j$-malige Anwendung der Regel $S \rightarrow aS$, anschließende Anwendung der Regel $S \rightarrow AB$, $i$-malige Anwendung der Regel $A \rightarrow aAb$ mit anschließender Anwendung von $A \rightarrow \lambda$, $k$-maliger Anwendung von $B \rightarrow bBc$ und anschließender Anwendung von $B \rightarrow \lambda$ abgeleitet werden. Damit gilt für jedes $w \in M$ also auch $w \in L(G_{1})$ und die Behauptung ist gezeigt.
\subsection{} %2.
$G_{2}$:\\
\begin{alignat*}{2}
S &\rightarrow & 0ABD | B1CD | A1CE | AE2D \\
A &\rightarrow & 0A | \lambda \\
B &\rightarrow & 0B1 | \lambda \\
C &\rightarrow & 1C | \lambda \\
D &\rightarrow & 2D | \lambda \\
E &\rightarrow & 1E2 | \lambda
\end{alignat*}
Ich behaupte $L(G_{2}) = \{0^{i + j}1^{i}2^{n}, 0^{i}1^{i + k}2^{n}, 0^{n}1^{i + l}2^{i}, 0^{n}1^{i}2^{i + m}|i,n \geq 0 \wedge j,k,l,m \geq 1\} =: M$.\\
Zunächst zeige ich $L(G_{2}) \subseteq M$. Sei dazu $w \in L(G_{2})$, d.h. es gilt $S \overset{*}{\Longrightarrow} w$. Wie kann $w \in V_{T}^{*}$ entstanden sein? Es gibt vier Fälle für Wörter von $L_{2}$:
\begin{enumerate}
	\item $|w|_{0} > |w|_{1}$ \\
	\item $|w|_{0} < |w|_{1}$ \\
	\item $|w|_{1} > |w|_{2}$ \\
	\item $|w|_{1} < |w|_{2}$
\end{enumerate}
Es ist natürlich möglich, dass mehrere von diesen Fällen zur gleichen Zeit zutreffen, allerdings muss immer mindestens einer zutreffen. Die jeweils nicht genannten Anzahlen sind beliebig.

Für den ersten Fall ergibt sich folgende Ableitung:\\
Zunächst wird die Produktion $S \rightarrow 0ABD$ gewählt. Durch $j-1$-malige Anwendung der Produktion $A \rightarrow 0A$ erhält man zunächst $S \overset{*}{\Longrightarrow} 00^{j-1}ABD$. Als nächstes wird die Produktion $A \rightarrow \lambda$ gewählt. Durch $i$-malige Anwendung der Produktion $B \rightarrow 0B1$ erhält man $S \overset{*}{\Longrightarrow} 00^{j-1}ABD \overset{*}{\Longrightarrow} 00^{j-1}0^{i}B1^{i}D$. Es wird die Produktion $B \rightarrow \lambda$ gewählt. Durch $n$-malige Anwendung von $D \rightarrow 2D$ erhält man $S \overset{*}{\Longrightarrow} 00^{j-1}ABD \overset{*}{\Longrightarrow} 00^{j-1}0^{i}B1^{i}D \overset{*}{\Longrightarrow} 00^{j-1}0^{i}1^{i}2^{n}D$. Abschließend wird $D \rightarrow \lambda$ gewählt. Das Wort $w$ hat demnach die Form $00^{j-1}0^{i}1^{i}2^{n} = 0^{i + j}1^{i}2^{n}$ mit $i, n \geq 0 \wedge j \geq 1$.\\
\\
Für den zweiten Fall ergibt sich folgende Ableitung:\\
Zunächst wird die Produktion $S \rightarrow B1CD$ gewählt. Durch $i$-malige Anwendung der Produktion $B \rightarrow 0B1$ erhält man $S \overset{*}{\Longrightarrow}0^{i}B1^{i}1CD$. Es wird die Produktion $B \rightarrow \lambda$ gewählt. Durch $k-1$-malige Anwendung von $C \rightarrow 1C$ erhält man $S \overset{*}{\Longrightarrow}0^{i}B1^{i}1CD \overset{*}{\Longrightarrow} 0^{i}1^{i}11^{k-1}CD$. Durch $n$-malige Anwendung von $D \rightarrow 2D$ erhält man $S \overset{*}{\Longrightarrow}0^{i}B1^{i}1CD \overset{*}{\Longrightarrow} 0^{i}1^{i}11^{k-1}CD \overset{*}{\Longrightarrow} 0^{i}1^{i}11^{k-1}2^{n}D$. Abschließend wird $D \rightarrow \lambda$ gewählt. Das Wort $w$ hat demnach die Form $0^{i}1^{i}11^{k-1}2^{n} = 0^{i}1^{i + k}2^{n}$ mit $i, n \geq 0 \wedge k \geq 1$.\\
\\
Für den dritten Fall ergibt sich diese Ableitung:\\
Zunächst wird die Produktion $S \rightarrow A1CE$ gewählt. Durch $n$-malige Anwendung der Produktion $A \rightarrow 0A$ erhält man zunächst $S \overset{*}{\Longrightarrow} 0^{n}A1CE$. Direkt danach folgt die Produktion $A \rightarrow \lambda$. Durch $l-1$-malige Anwendung der Produktion $C \rightarrow 1C$ erhält man $S \overset{*}{\Longrightarrow} 0^{n}A1CE \overset{*}{\Longrightarrow} 0^{n}11^{l-1}CE$. Anschließend folgt die Produktion $C \rightarrow \lambda$. Mit $i$-maliger Anwendung der Produktion  $E \rightarrow 1E2$ erhält man $S \overset{*}{\Longrightarrow} 0^{n}A1CE \overset{*}{\Longrightarrow} 0^{n}11^{l-1}CE \overset{*}{\Longrightarrow} 0^{n}11^{l-1}1^{i}E2^{i}$. Abschließend wird die Produktion $E \rightarrow \lambda$ angewendet. Das Wort $w$ hat demnach die Form $0^{n}11^{l-1}1^{i}2^{i} = 0^{n}1^{i + l}2^{i}$ mit $i,n \geq 0 \wedge l \geq 1$. \\
\\
Für den vierten Fall ergibt sich diese Ableitung:\\
Zunächst wird die Produktion $S \rightarrow AE2D$ gewählt. Durch $n$-malige Anwendung der Produktion $A \rightarrow 0A$ erhält man zunächst $S \overset{*}{\Longrightarrow} 0^{n}AE2D$. Direkt danach folgt die Produktion $A \rightarrow \lambda$. Mit $i$-maliger Anwendung der Produktion  $E \rightarrow 1E2$ erhält man $S \overset{*}{\Longrightarrow} 0^{n}AE2D \overset{*}{\Longrightarrow} 0^{n}1^{i}E2^{i}2D$. Abschließend wird die Produktion $E \rightarrow \lambda$ angewendet. Durch $m-1$-malige Anwendung der Produktion $D \rightarrow 2D$ erhält man $S \overset{*}{\Longrightarrow} 0^{n}AE2D \overset{*}{\Longrightarrow} 0^{n}1^{i}E2^{i}2D \overset{*}{\Longrightarrow} 0^{n}1^{i}2^{i}22^{m-1}D$. Abschließend wird die Produktion $D \rightarrow \lambda$ gewählt. Das Wort $w$ hat demnach die Form $0^{n}1^{i}2^{i}22^{m-1} = 0^{n}1^{i}2^{i + m}$ mit $i,n \geq 0 \wedge m \geq 1$. \\
\\
Insgesamt muss $w$ damit die Form $0^{i + j}1^{i}2^{n}, 0^{i}1^{i + k}2^{n}, 0^{n}1^{i + l}2^{i}$ oder $0^{n}1^{i}2^{i + m}$ haben. Es gilt also $w \in M$.\\
\\
Sei umgekehrt $w \in M$. Hat $w$ die Form $0^{i + j}1^{i}2^{n}$, so kann $w$ durch die im ersten Fall beschriebene Ableitungsfolge abgeleitet werden. Hat $w$ die Form $0^{i}1^{i + k}2^{n}$, so kann $w$ durch die im zweiten Fall beschriebene Ableitungsfolge abgeleitet werden. Hat $w$ die Form $0^{n}1^{i + l}2^{i}$, so kann $w$ durch die im dritten Fall beschriebene Ableitungsfolge abgeleitet werden. Hat $w$ die Form $0^{n}1^{i}2^{i + m}$, so kann $w$ durch die im vierten Fall beschriebene Ableitungsfolge abgeleitet werden. Damit gilt für jedes $w \in M$ also auch $w \in L(G_{2})$ und die Behauptung ist gezeigt.
\section{} %4.5
Zunächst werden alle $\lambda$-Produktionen nach dem Verfahren von Theorem 9.12 entfernt.
\begin{alignat*}{2}
M_{0} &=& \{C\} \\
M_{1} &=& \{C,F,S\} \\
M_{2} &=& \{C,F,S\} = M_{1} \\
V_{\lambda} &=& M_{2}
\end{alignat*}
Daraus ergibt sich folgende Grammatik:\\
\begin{alignat*}{2}
S &\rightarrow & AB | AB1B | AD | DE | C \\
A &\rightarrow & EBEE \\
B &\rightarrow & A | E | SE \\
D &\rightarrow & CBG | BG | 0G | 1D \\
E &\rightarrow & 0 | 1 \\
F &\rightarrow & 0E | EF | CC | C | E
\end{alignat*}\\
Anschließend wird nach dem Verfahren von Theorem 9.11 reduziert. Zunächst werden dabei alle unproduktiven Nonterminale entfernt.\\
\begin{alignat*}{2}
M_{0} &=& \{0,1\} \\
M_{1} &=& \{0,1,E\} \\
M_{2} &=& \{0,1,E,F,B\} \\
M_{3} &=& \{0,1,E,F,B,A\} \\
M_{4} &=& \{0,1,E,F,B,A,S\} \\
M_{5} &=& \{0,1,E,F,B,A,S\} = M_{4}
\end{alignat*}
Es ergibt sich folgende Grammatik:\\
\begin{alignat*}{2}
S &\rightarrow & AB | AB1B \\
A &\rightarrow & EBEE \\
B &\rightarrow & A | E | SE \\
E &\rightarrow & 0 | 1 \\
F &\rightarrow & 0E | EF | E
\end{alignat*}
Daraufhin werden alle Produktionen entfernt, die auf der linken oder rechten Seite unerreichbare Nonterminale haben.
\begin{alignat*}{2}
M_{0} &=& \{S\} \\
M_{1} &=& \{S,A,B\} \\
M_{2} &=& \{S,A,B,E\} \\
M_{3} &=& \{S,A,B,E\} = M_{2}
\end{alignat*}
Es ergibt sich die Grammatik:\\
\begin{alignat*}{2}
S &\rightarrow & AB | AB1B \\
A &\rightarrow & EBEE \\
B &\rightarrow & A | E | SE \\
E &\rightarrow & 0 | 1
\end{alignat*}
Im dritten Schritt werden die Kettenregeln entfernt. Daraus ergibt sich diese Grammatik:\\
\begin{alignat*}{2}
S &\rightarrow & AB | AB1B \\
A &\rightarrow & EBEE \\
B &\rightarrow & EBEE | 0 | 1 | SE \\
E &\rightarrow & 0 | 1
\end{alignat*}
Im vierten Schritt werden lange Terminalregeln ersetzt. Daraus ergibt sich diese Grammatik:\\
\begin{alignat*}{2}
S &\rightarrow & AB | AB\left\langle 1 \right\rangle B \\
A &\rightarrow & EBEE \\
B &\rightarrow & EBEE | 0 | 1 | SE \\
E &\rightarrow & 0 | 1 \\
\left\langle 1 \right\rangle &\rightarrow & 1
\end{alignat*}
Im fünften Schritt werden zu lange Regeln verkürzt. Daraus ergibt sich diese Grammatik:\\
\begin{alignat*}{2}
S &\rightarrow & AB | \left\langle AB \right\rangle \left\langle \left\langle 1 \right\rangle B \right\rangle \\
\left\langle AB \right\rangle &\rightarrow & AB \\
\left\langle \left\langle 1 \right\rangle B \right\rangle &\rightarrow &\left\langle 1 \right\rangle B \\
A &\rightarrow & \left\langle EB \right\rangle \left\langle EE \right\rangle \\
\left\langle EB \right\rangle &\rightarrow & EB \\
\left\langle EE \right\rangle &\rightarrow & EE \\
B &\rightarrow & \left\langle EB \right\rangle \left\langle EE \right\rangle | 0 | 1 | SE \\
E &\rightarrow & 0 | 1 \\
\left\langle 1 \right\rangle &\rightarrow & 1
\end{alignat*}
Im sechsten Schritt wird die ursprüngliche Sprache wiederhergestellt, indem eine $\lambda$-Regel hinzugenommen wird. Daraus ergibt sich diese Grammatik:\\
\begin{alignat*}{2}
S_{neu} &\rightarrow & \lambda | AB | \left\langle AB \right\rangle \left\langle \left\langle 1 \right\rangle B \right\rangle \\
\left\langle AB \right\rangle &\rightarrow & AB \\
\left\langle \left\langle 1 \right\rangle B \right\rangle &\rightarrow &\left\langle 1 \right\rangle B \\
A &\rightarrow & \left\langle EB \right\rangle \left\langle EE \right\rangle \\
\left\langle EB \right\rangle &\rightarrow & EB \\
\left\langle EE \right\rangle &\rightarrow & EE \\
B &\rightarrow & \left\langle EB \right\rangle \left\langle EE \right\rangle | 0 | 1 | SE \\
E &\rightarrow & 0 | 1 \\
\left\langle 1 \right\rangle &\rightarrow & 1
\end{alignat*}
\section{} %4.6
\end{document}
