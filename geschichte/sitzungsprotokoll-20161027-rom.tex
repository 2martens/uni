\documentclass[10pt,a4paper,oneside,ngerman,numbers=noenddot]{scrartcl}
\usepackage[T1]{fontenc}
\usepackage[utf8]{inputenc}
\usepackage[ngerman]{babel}
\usepackage{amsmath}
\usepackage{amsfonts}
\usepackage{amssymb}
\usepackage{bytefield}
\usepackage{paralist}
\usepackage{gauss}
\usepackage{pgfplots}
\usepackage{textcomp}
\usepackage[locale=DE,exponent-product=\cdot,detect-all]{siunitx}
\usepackage{tikz}
\usepackage{algpseudocode}
\usepackage{algorithm}
\usepackage{mathtools}
\usepackage{hyperref}
%\usepackage{algorithmic}
%\usepackage{minted}
\usetikzlibrary{automata,matrix,fadings,calc,positioning,decorations.pathreplacing,arrows,decorations.markings}
\usepackage{polynom}
\polyset{style=C, div=:,vars=x}
\pgfplotsset{compat=1.8}
\pagenumbering{arabic}
%\def\thesection{\arabic{section})}
%\def\thesubsection{(\alph{subsection})}
%\def\thesubsubsection{(\roman{subsubsection})}
\makeatletter
\renewcommand*\env@matrix[1][*\c@MaxMatrixCols c]{%
  \hskip -\arraycolsep
  \let\@ifnextchar\new@ifnextchar
  \array{#1}}
\makeatother
\parskip 12pt plus 1pt minus 1pt
\parindent 0pt

\DeclarePairedDelimiter\abs{\lvert}{\rvert}%
\DeclarePairedDelimiter{\ceil}{\lceil}{\rceil}

\newenvironment{myitemize}{\begin{itemize}\itemsep -9pt}{\end{itemize}} % Zeilenabstand in Aufzählungen geringer

%switch starred and non-starred (auto-size)
\makeatletter
\let\oldabs\abs
\def\abs{\@ifstar{\oldabs}{\oldabs*}}
\makeatother

\hypersetup{
    colorlinks,
    citecolor=black,
    filecolor=black,
    linkcolor=black,
    urlcolor=black
}

\begin{document}
\author{Jim Martens (6420323)}
\title{Vorlesung Römische Geschichte 1: Römische Republik}
\subtitle{Vorlesungsprotokoll vom 10. Oktober 2016}
\date{10. Oktober 2016}
\maketitle

\section*{Übersicht}
\begin{myitemize}
\item Einführung in Römische Republik
\item Quellenkunde, das alte Italien, die Etrusker, das frühe Rom
\item Verfassung wird nicht niedergeschrieben
\item Bundesgenossensystem macht Rom extrem stark
\item Expansionspolitik stellt inneres System vor Zerreißprobe
\item Probleme im Inneren werden nicht gelöst, es kommt zum Bundesgenossenkrieg
\item nach Bundesgenossenkrieg kommt es zu Bürgerkriegen
\end{myitemize}
- seit Sulla bewegt sich Rom in monarchischem Handlungsrahmen, auch wenn dies von den Leuten nicht erkannt wird
- im SoSe wird es um die Kaiserzeit gehen

\section*{Quellen}
\subsection*{Nicht-literarische Quellen}
\begin{myitemize}
    \item archäologische Zeugnisse
    \item "schwarzer Stein"
    \item die 12 Tafeln
    \item Sprachgeschichte
\end{myitemize}
\subsection*{Literarische Quellen}

\begin{myitemize}
    \item griechische Fragmente
    \item Fasten
    \item Werke von Cicero
    \item Mythengeschichte
    \item Werke von Livius
    \begin{myitemize}
        \item Livius lebt in augustinischer Zeit
        \item er schreibt sehr schön
        \item alle Vorgänger wurden durch ihn in den Schatten gestellt
        \item entwirft eine "Mastergeschichte" des frühen Roms
    \end{myitemize}
    \item Plutarch ist ein Großautor
    \begin{myitemize}
        \item Parallelbiografien (bedeutender Grieche vs bedeutender Römer)
        \item Demostenes vs Cicero ist berühmtes Beispiel davon
        \item Viten
        \item eine der wichtigsten Quellen für alte Geschichte Roms
    \end{myitemize}
    \item Werke von Plautus und Terenz (beide sind römische Komödienautoren)
    \item Komödien sind wichtige Quellen für die Sozialstruktur in Rom
    \item Werke von Naevius und Ennius (Epiker, "Nationaldichter")
    \begin{myitemize}
        \item Ennius schreibt Epos über gesamte römische Zeit bis zu Ennius Zeit
        \item es treten Konsuln auf, die belegt sind
        \item ähnlich zu Homer
        \item Ennius dichtet als Erster im Hexameter
    \end{myitemize}
\end{myitemize}

\subsubsection*{Geschichte der literarischen Quellen}
\begin{myitemize}
    \item Annalistik ist Form der Geschichtsschreibung (es wird nach den Konsuln datiert: Liste mit Jahr und den beiden Konsuln)
    \item bekannte Autoren der Annalistik: Livius, Polybios, Plutarch
    \item Latein war anfangs nicht als Literatursprache geeignet
    \item römische Geschichtsschreibung fand daher zunächst in Griechisch statt
    \item Römer beginnen später aus dem Griechischen zu übersetzen
    \item Livius Antonius übersetzt die Odyssee ins Lateinische 
\end{myitemize}

\section*{Besiedlung Italiens}

\begin{myitemize}
    \item Etrusker sind ab dem 7 Jhd. v.Chr. in Italien greifbar und eine vorindoeuropäische Gruppe
    \item Kernbereich von Rom im Süden bis Ano
    \item von Küste im Westen bis Appinin(?) im Osten
    \item ab 1000 vChr. sind indogermanische/indoeuropäische Gruppen in Italien belegt
    \item aus latino-phaliskischer Sprachgruppe entsteht Rom
    \item Griechen siedelten im Süden
    \item Kelten siedelten im Norden
    \item illyrische Sprachgruppe (Veneter und Illyrer)
    \item vor den Römern gab es ein buntgefleckte Kulturlandschaft
    \item Sizilien wichtig für die Griechen
    \item Kolonisation der Griechen ungebremst bis zur Seeschlacht bei Alelia
    \item Griechen bringen Stadtkultur nach Italien (Polis)
    \item Städte betreiben Ackerbau und Handel
    \item es gab kein griechisches Reich
    \item Magna Graecia ist kein Reich
    \item Reichsgründung gab es erst unter den Hellenen (Alexander der Große)
    \item Polis (plural?) untereinander in Konkurrenz, Krieg eher gegeneinander als zusammen
    \item Katharger sitzen im Westen Siziliens und Griechen im Osten
    \item Katharger gehören zur semitischen Sprachgruppe
\end{myitemize}

\subsection*{Etrusker}

\begin{myitemize}
    \item Etrusker betreiben Erzbergbau, Handel
    \item unterschiedliche Theorien zur Herkunft von Etruskern
    \begin{myitemize}
        \item von Villa Nova bis Etrusker ähnliche Bestattungsrituale
        \item Einfluss/Einwanderung von Osten
        \item Villa Nova + Einwanderung mit Assimilation von Osten (Ethnogenese der Etrusker)
    \end{myitemize}
    \item wichtige Städte sind Capua, Palestrina, Veji
    \item Frauen genossen hohes Sozialprestige
    \item im 6. Jhd. expandieren Etrusker nach Norden und Süden (Latium)
    \item es wird ein Bündnis mit Kathargern gegen die Griechen eingegangen
    \begin{myitemize}
        \item erste Konflikte werden in Seeschlacht bei Aleria ausgetragen
        \item große Niederlage gegen Griechen in Cumae
        \item es kommt zu Machtverlust des etruskischen Adels in Roms
        \item Nutznießer sind Römer, die sich von den Etruskern lösen
    \end{myitemize}
    \item Galliereinfall zerstört etruskische Kulturgegenstände
    \item Rom attackiert Veji
    \item später zerstören Römer religiöses Zentrum der Etrusker
    \item Zerfall der Etrusker zieht sich vom 5. Jhd. bis zum 3. Jhd. v.Chr.
    \item etruskische Sprache nicht verständlich (vorindoeuropäisch)
    \item etliche Fresken und Bilder der Etrusker (Formensprache griechisch)
    \item bei Besuch in Rom: Besuch beim etruskischen Museum Pflicht
    \item Leber von Piacenza (Benutzung der Leber für Zukunftsschau)
\end{myitemize}

\section*{Mythengeschichte}
\begin{myitemize}
    \item Römer schließen sich an die Trojaner an (Fiktion)
    \item mythologische Zurückführung Roms auf Mars
    \item später wird das Jahr 753 als Himmelfahrt von Romulus gesetzt
    \item Caesar und Augustus leiten den Stammbaum auf Venus zurück
\end{myitemize}

\section*{Gründung Roms}
\begin{myitemize}
    \item ist Rom allmählich gegründet worden oder gibt es eine feste Gründung (Synoikismos)?
    \item Synoikismos: mehrere Häuser/Dörfer schließen sich bewusst zusammen
    \item wenn feste Gründung überhaupt stattfand, geschah das erst gegen 600 v. Chr.
    \item erste Funde am Forum Buarium, weitere am Südhang des Kapitols
    \item weitere Funde am Forum Romanium
    \item Rom liegt an Salzstraße bei Furt von Tiber
    \item deutliche Siedlungsspuren auf dem Palatin (9. Jhd. v. Chr.)
\end{myitemize}
\section*{Etruskischer Einfluss}
\begin{myitemize}
    \item im 6. Jhd. gibt es wohlgeordnete Stadt etruskischen Zuschnitts
    \item schon zu dieser Zeit ist das Kapitol ein zentrales religiöses Zentrums
    \item Stadtwerdung ohne etruskischen Einfluss undenkbar
    \item Geschlecht der Romulier ist rein etruskisch
    \item seit dem 8. Jhd. Städtekultur in Etrurien
    \item Etrusker sind Transmissionsrahmen für griechische Schrift
    \item Römer bekamen Schrift von den Etruskern
    \item dreigliedriges Namenssystem (Vorname . Beiname)
    \item Schnabelschuhe (von Etruskern zu Römern zu Päpsten)
    \item Gladiatorenkämpfe kommen aus dem etruskischen Gräberritual?
    \item Idee eines Senats (Ältestenrats) kommen von den Etruskern
    \item Konzept der Zukunftschau kommt von den Etruskern
    \item Könige in Rom haben etruskische Namen
    \item einer der Könige hat begonnen die Kurie zu bauen
    \item Latiner werden von den Etruskern bevormundet aber nicht unterdrückt
    \item Triumph könnte von den Etruskern kommen
    \item archaische Sozialstruktur wird von den Etruskern übernommen
    \item Senat ist Ältestenrat des Adels
    \item etruskisches Erbe wird von den Römern gesehen und geachtet
    \item Identität der Etrusker ging nicht verloren
\end{myitemize}

\end{document}