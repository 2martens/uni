\documentclass[10pt,a4paper,oneside,ngerman,numbers=noenddot]{scrartcl}
\usepackage[T1]{fontenc}
\usepackage[utf8]{inputenc}
\usepackage[ngerman]{babel}
\usepackage{amsmath}
\usepackage{amsfonts}
\usepackage{amssymb}
\usepackage{bytefield}
\usepackage{paralist}
\usepackage{gauss}
\usepackage{pgfplots}
\usepackage{textcomp}
\usepackage[locale=DE,exponent-product=\cdot,detect-all]{siunitx}
\usepackage{tikz}
\usepackage{algpseudocode}
\usepackage{algorithm}
\usepackage{mathtools}
\usepackage{hyperref}
%\usepackage{algorithmic}
%\usepackage{minted}
\usetikzlibrary{automata,matrix,fadings,calc,positioning,decorations.pathreplacing,arrows,decorations.markings}
\usepackage{polynom}
\polyset{style=C, div=:,vars=x}
\pgfplotsset{compat=1.8}
\pagenumbering{arabic}
%\def\thesection{\arabic{section})}
%\def\thesubsection{(\alph{subsection})}
%\def\thesubsubsection{(\roman{subsubsection})}
\makeatletter
\renewcommand*\env@matrix[1][*\c@MaxMatrixCols c]{%
  \hskip -\arraycolsep
  \let\@ifnextchar\new@ifnextchar
  \array{#1}}
\makeatother
\parskip 12pt plus 1pt minus 1pt
\parindent 0pt

\DeclarePairedDelimiter\abs{\lvert}{\rvert}%
\DeclarePairedDelimiter{\ceil}{\lceil}{\rceil}

\newenvironment{myitemize}{\begin{itemize}\itemsep -2pt}{\end{itemize}} % Zeilenabstand in Aufzählungen geringer

%switch starred and non-starred (auto-size)
\makeatletter
\let\oldabs\abs
\def\abs{\@ifstar{\oldabs}{\oldabs*}}
\makeatother

\hypersetup{
    colorlinks,
    citecolor=black,
    filecolor=black,
    linkcolor=black,
    urlcolor=black
}

\begin{document}
\author{Jim Martens (6420323)}
\title{Die Mittelalterliche Geschichte der Hanse}
\subtitle{Vorlesungsprotokoll vom 12. April 2016}
\date{12. April 2016}
\maketitle

\section*{Überblick über Geschichte Lübecks}

\begin{myitemize}
    \item Lübeck war zentral für die Hanse
    \item spätestens im 14. Jhd. war Lübeck Hauptort der Hanse
    \item Rolle u.a. durch geografische Position begründet
    \item während einer Verfassungskrise in Lübeck ging die Führung an Hamburg
    \item nachdem in Hamburg kurz darauf auch eine Krise ausbrach, ging Führung an Lüneburg
    \item dies war auch keine dauerhafte Lösung, nach Ende der Krise in Lübeck ging Führung zurück an Lübeck
    \item auf einem Hansetag wurde das Zusammenspiel zwischen Städten geregelt und Lübeck übernahm Geschäftsführung
    \item Lübeck teilte diese Verantwortung mit englischen Städten
    \item viele Hansetage fanden in Lübeck statt
    \item Lübeck war Nachfolge für Schleswig und slawische Städte (u.a. für Alt-Lübeck)
    \item von 1203 bis 1225 herrschten die Dänen über Lübeck
\end{myitemize}

\section*{Alt-Lübeck}

\subsection*{Überblick}
\begin{myitemize}
    \item 1819 wurde Überreste einer slawischen Burg gefunden
    \item befand sich näher am Meer, als das heutige Lübeck
    \item unklar, welcher slawische Herrscher die Burg baute
    \item unklar, warum die Burg erbaut wurde
    \item es kann nicht von dauerhafter Nutzung der Burg ausgegangen werden
    \item im späten 8. Jhd. und im 9. Jhd. war die Burganlage teilweise unbenutzt
    \item im 11. Jhd. wurde die Burg erneuert und auch mit einem Kloster ausgestattet
\end{myitemize}

\subsection*{Herrschergeschichte}

\begin{myitemize}
    \item Gottschalk war der herrschende Fürst
    \item wurde in einem Kloster christlich erzogen
    \item nach Ermordung seines Vaters wandte er sich gegen Christen
    \item wollte Rache für Ermordung seines Vaters
    \item wurde nach England verbannt
    \item heiratete später eine Dänin
    \item kehrte nach einigen Jahren zurück in die Heimat
    \item konnte Herrschaft festigen
    \item 1066 gab es einen Aufstand gegen die Christianisierung durch Gottschalk
    \item sein Sohn Heinrich floh während des Aufstandes nach Dänemark
    \item Gottschalk wurde im weiteren Verlauf der Rebellion gegen ihn getötet
    \item nach einigen Jahren konnte Heinrich zurückkehren und einige der väterlichen Ländereien bekommen
    \item er überlistete seine Gegner
    \item Heinrich residierte in Alt-Lübeck und nahm einige Burgen des früheren slawischen Fürsten ein
    \item Alt-Lübeck wurde ausgebaut
    \item Kaufleute aus dem HRE und aus Skandinavien wurden angezogen
    \item Heinrich ging Bündnis mit Sachsen ein und kämpfte gegen Slawen
    \item er dehnte sein Gebiet bis an untere Oder und Havel aus
    \item zusammen mit christlicher Hilfe konnte er Belagerung von Alt-Lübeck zerschlagen
    \item nach dem Tode Heinrichs stritten dessen Sööhne um die Macht
    \item nach einiger Zeit starben beide Söhne und damit starb die Familie aus
    \item Versuch einer Neubesetzung der Herrschaft scheiterte mit dem Tod des dänischen Königs 1131
    \item im weiteren Verlauf wurde Alt-Lübeck zerstört und nicht wieder besiedelt
\end{myitemize}

\section*{Gründung Lübecks}

\subsection*{Erster Anlauf}
\begin{myitemize}
    \item Sachsen fiel an die Welfen
    \item es wird für Siedler geworben, die nach Sachsen ziehen sollen
    \item Adolph war zuständig für Werbung
    \item er suchte nach einem Siedlungsplatz für eine neue Stadt
    \item einige Siedler siedelten bei slawischen Burgen (Lüneburg, Oldenburg)
    \item an etwas südlicherer Stelle im Verhältnis zum früheren Alt-Lübeck wurde die neue Stadt Lübeck gegründet
    \item wahrscheinlich wurde die Siedlung von Beginn an mit Rechten ausgestattet, ist aber nichts überliefert
    \item Siedlung wuchs schnell
    \item das Wachstum führte zum Streit zwischen dem Herzog und Adolph
    \item 1127 wurde die Stadt zerstört
\end{myitemize}

\subsection*{Zweiter Anlauf}
\begin{myitemize}
    \item etwas weiter nördlich wurde neue Stadt durch den Herzog gegründet
    \item diese war aber schlecht platziert
    \item Adolph und Heinrich (Herzog) einigten sich
    \item Kaufleute gingen zurück nach Lübeck und bauten Stadt wieder auf
    \item Heinrich baut Lübeck als Handelsstadt aus und stattete sich mit Rechten aus
    \item es wurden diplomatische Beziehungen u.a. mit Russland aufgenommen
    \item 1161 Vertrag u.a. mit Russland, um freien Handel zu ermöglichen
    \item die Übernahme der Stadtherrschaft durch Heinrich war nicht der Startschuss des Handels
    \item entsprechende Beziehungen gab es bereits vorher
\end{myitemize}

\section*{Wachstum Lübecks}

Das Wachstum Lübecks gründet sich auf drei Aspekte:

\begin{myitemize}
    \item Erstens: Feste Siedlung mit Fernhandelskaufleuten, anders als z.B. Schleswig
    \item dadurch höhere Rechtssicherheit
    \item Zweitens: Weg zur Ostsee verkürzte sich über Lübeck
    \item Drittens: Lübeck lag an Handelsstraße, die Bardowick, slawische Dörfer und dänische Gebiete verband
    \item dadurch konnte das Salz besser abgesetzt werden
\end{myitemize}

\section*{Wechselnde Machtverhältnisse}

\subsection*{Freiheit der Stadt unter Kontrolle des Kaisers}

\begin{myitemize}
    \item Kaiser sah sich bei Feldzügen in Italien zu wenig von Heinrich unterstützt
    \item hinzu kam Klage des Herrschers von Köln gegen Heinrich
    \item Heinrich erschien nicht bei Gericht und wurde geächtet
    \item 1180 verlor er all seine Lehen
    \item Herzogtum Sachsen wurde aufgeteilt zwischen kölnischem Herrscher und neuem Herzog Sachsens
    \item Heinrich hat Kontrolle aber nicht komplett abgegeben
    \item Kaiser begann Reichskrieg gegen Heinrich
    \item daraufhin baute Heinrich etliche Städte mit Mauern aus
    \item 1160 übersiedelte der Bischof in Lübeck auf südliche Seite der Halbinsel Lübecks
    \item 1172/75 Ausbau der Stadt Lübeck
    \item Notwendigkeit der Befestigung wurde durch Belagerung der Stadt durch Kaiser offenbar
    \item 1181 verlangte der Kaiser die Übergabe der Stadt
    \item Bürger übergaben die Stadt nicht direkt, sondern verhandelten
    \item zunächst wurden sie von Heinrich von der Pflicht befreit die Stadt weiter zu verteidigen
    \item anschließend baten sie den Kaiser um Bestätigung der Freiheit der Stadt
    \item Kaiser bestätigte die Rechte und betrat anschließend die Stadt
    \item 1188 ließ Friedrich Barbarossa angeblich neue Privilegien folgen (Fälschung)
\end{myitemize}

\subsection*{Übergabe an holsteinischen Herzog}
\begin{myitemize}
    \item Stadt wurde durch mehrere Herrscher umklammert
    \item Lübecker wandten sich an die Dänen, die durch den Heringsmarkt reich waren
    \item 1191 gab es Überlegungen sich unter dänische Herrschaft zu stellen
    \item zu der Zeit wurde Stadt durch holsteinischen Herzog belagert
    \item Hälfte der Bürger wollte Stadt übergeben, andere Hälfte nicht
    \item schließlich wurde Stadt durch den Kaiser an den holsteinischen Herzog übergeben
    \item Siedlungsgebiet wurde ausgedehnt
    \item Lübecker schufen sich immer neue eigene Strukturen
\end{myitemize}

\subsection*{Dänische Herrschaft}

\begin{myitemize}
    \item nach einem Krieg fiel Lübeck an die Dänen
    \item Waldemar ging 1203 nach Lübeck und wurde begeistert empfangen
    \item dänisches Reich war weit ausgedehnt
    \item Waldemar bestätigte die Rechte der Stadt
    \item zwischen 1203 und 1209 folgte die Privilegierung der Lübecker
    \item 1216 ist Holstenbrücke erstmals erwähnt
    \item 1217 wurden Mauern erneuert
    \item 1220 wurde Lübeck vom Strandrecht befreit und Waldemar sperrte den Hafen
\end{myitemize}

\subsection*{Aufstieg zur Reichsstadt}
\begin{myitemize}
    \item 1226 wechselten die Lübecker während einem Krieg die Seiten und bekamen Reichsfreiheitsbrief
    \item Lübeck wurde zur Reichsstadt
    \item dem Privileg folgte der weitere Aufstieg der Stadt
\end{myitemize}

\section*{Anfänge des Gotlandhandels und des Kontors in Novgorod}

- Anfänge des Gotlandhandels und des Kontors in Novgorod
- Lübeck war Anfang von weiteren Stadtgründungen im Ostseeraum, die dann auch Lübecker Recht verwendeten
- niederdeutsche Kaufleute kamen früh über Lübeck nach Gotland
- es bildeten sich genossenschaftliche Strukturen

# 60-69 min
- deutsche Händler durften Niederlassung der gotländischen Kaufleute nutzen, bevor sie ihre eigenen Niederlassungen hatten
- Novgoroder Kontor war nur zeitweilig besetzt
- Anfänge des Gotlandhandels sind auf Heinrich dem Löwen zurückzuführen
- Ansprechpartner waren nicht nur Dänemark, sondern auch Gotland
- wann die Handelsbeziehungen mit Gotland begannen, ist nicht ganz klar
- 1161 Artlenburger Vertrag
- Handelsbeziehungen mit Gotland bestanden bereits vor Gründung von Lübeck
- Vertrag verweist auf Konflikte zwischen deutschen und gotländischen Kaufleuten
- unklar, ob es um Konflikte auf Gotland geht
- umfassender Rechtsschutz für Gotländer, wenn deutsche Kaufleute die gleichen Rechte dort bekommen
- Überlegung, dass dies als Vorbereitung für Handel mit Gotland gemacht wurde
- 1170/90 wurde Kirche in Gotland vermutlich begonnen
- 1225 gab es Weihbrief der Kirche, Kaufleute waren zu dem Zeitpunkt schon da
- Zusammenhang zwischen Gotland und Novgorodfahrt
- Wahrung der Rechte der dt. Kaufleute wird deren Gesandtem übergeben
- dies beruht wieder auf Gegenseitigkeit

# 70-79 min
- deutsche Kaufleute, die nach Gotland gingen, waren wohl schon genossenschaftlich organisiert
- man findet im Nord- und Ostseeraum Hinweise auf gotländische Kaufleute
- in Urkunde der Herrscherin von Flandern finden sich entsprechende Hinweise
- Kaufleute aus Gotland jedweder Nationalität waren nach außen mit einem Siegel vertreten
- 1280 ein Siegel zeigt Gemeinschaft der in Gotland bleibenden dt. Kaufleute
- man kann den Artlenburger Vertrag als Gründungsdokument der Gemeinschaft der Kaufleute ansehen
- 1252 gab es angeblich Vertrag zwischen Schweden und Lübeck, der die Privilegierung der Kaufleute
  in Visby(Gotland) bestätigt
- baltischer Raum gewann im späten 12. Jhd. zunehmend an Bedeutung
- Lübeck und Gotland stellten wichtige Ausgangspunkte fast jährlicher Kreuzzüge in Region Livland, Lettland, Estland
- 1201 entstand mit Riga wichtiges städtisches Zentrum
- 1211 war Riga gesichert
- dt. Kaufleute bekamen Rechte und Privilegien und wurden vom Gottesurteil befreit

# 80-89 min
- ohne Zustimmung des Bischofs sollten die Kaufleute in Riga keine Gilde bauen
- Stadt sollte eigenes Recht bekommen
- Stadtmauern von Riga sind bis heute erhalten
- Gilde in Gotland war Mittel alle Kaufleute zusammenzufassen
- schwierig, ob man dies als Start der Hanse betrachten kann
- 1188 gab es Verhandlungen mit Russland
- man kann von einem gemeinsamen Agieren der Kaufleute in Gotland ausgehen
- auch nach Ende der dänischen Vorherrschaft im Ostseeraum bestand der Gemeinschaft weiter
- ein Vertrag berechtigt Kaufleute zur Errichtung eines Kontors in Smolensk
- dieser Vertrag beinhaltet wieder Siegel der Kaufleute, welches auf genossenschaftliche Struktur
  hinweist
- gotländische Händler bereits Ende des 12. Jhd. in Bergen
- in englischer Kämmerei finden sich nach 1226 immer wieder Verweise auf gotländische Kaufleute
- dänische Krone hat genossenschaftliche Gruppe mehrfach privilegiert
- 1226 verfügt Kaiser durch Reichsfreiheitsurkunde die Gleichheit der Lübecker Kaufleute mit anderen
  Handelspartner
- englischer König nahm Händler von Gotland unter seinen Schutz
- engl. König hat Urkunden sowohl für Händler aus Gotland als auch für Händler aus HRE ausgestellt

- Anfänge des Novgoroder Kontor angelehnt an Geschichte des Handels mit Gotland
- Luxuswaren für russische Oberschicht wurden exportiert
- Großgrundbesitzer hatten mehr Macht als Fürsten oder Versammlung in Novgorod
- Novgorod war wichtigste Stadt für Hansisch-Russischen Handel
- als dt. Kaufleute nach Novgorod kamen, waren skandinavische Händler bereits anwesend
- 1165 gibt es Privileg für Bürger Kölns, die wegen Handel in Novgorod sind
- aus 1191 stammt der älteste erhaltenste Vertrag mit russischem Landesherrn
- Vertrag bezieht sich auf frühere Abmachungen

# 90-99 min
- gegenseitige Privilegien und Strafen zwischen novgorodischen und gotländischen/deutschen Kaufleuten
- gleiche Strafen für gleiche Vergehen
- Unterscheidung zwischen Sommer- und Winterfahrer nach Novgorod
- 1231 heißt es, die deutschen Kaufleute hätten Stadt mittels Lieferung vor Hungersnot gerettet
- ältere Statuten regeln Saisonalität des Handels
- genossenschaftliche Strukturen erkennbar
- 4 Schlüssel für Truhe
    - Gotländer, Sooster, Lübecker und Dortmunder hatten Zugriff auf Truhe

\end{document}