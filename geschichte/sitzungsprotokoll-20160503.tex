\documentclass[10pt,a4paper,oneside,ngerman,numbers=noenddot]{scrartcl}
\usepackage[T1]{fontenc}
\usepackage[utf8]{inputenc}
\usepackage[ngerman]{babel}
\usepackage{amsmath}
\usepackage{amsfonts}
\usepackage{amssymb}
\usepackage{bytefield}
\usepackage{paralist}
\usepackage{gauss}
\usepackage{pgfplots}
\usepackage{textcomp}
\usepackage[locale=DE,exponent-product=\cdot,detect-all]{siunitx}
\usepackage{tikz}
\usepackage{algpseudocode}
\usepackage{algorithm}
\usepackage{mathtools}
\usepackage{hyperref}
%\usepackage{algorithmic}
%\usepackage{minted}
\usetikzlibrary{automata,matrix,fadings,calc,positioning,decorations.pathreplacing,arrows,decorations.markings}
\usepackage{polynom}
\polyset{style=C, div=:,vars=x}
\pgfplotsset{compat=1.8}
\pagenumbering{arabic}
%\def\thesection{\arabic{section})}
%\def\thesubsection{(\alph{subsection})}
%\def\thesubsubsection{(\roman{subsubsection})}
\makeatletter
\renewcommand*\env@matrix[1][*\c@MaxMatrixCols c]{%
  \hskip -\arraycolsep
  \let\@ifnextchar\new@ifnextchar
  \array{#1}}
\makeatother
\parskip 12pt plus 1pt minus 1pt
\parindent 0pt

\DeclarePairedDelimiter\abs{\lvert}{\rvert}%
\DeclarePairedDelimiter{\ceil}{\lceil}{\rceil}

\newenvironment{myitemize}{\begin{itemize}\itemsep -8pt}{\end{itemize}} % Zeilenabstand in Aufzählungen geringer

%switch starred and non-starred (auto-size)
\makeatletter
\let\oldabs\abs
\def\abs{\@ifstar{\oldabs}{\oldabs*}}
\makeatother

\hypersetup{
    colorlinks,
    citecolor=black,
    filecolor=black,
    linkcolor=black,
    urlcolor=black
}

\begin{document}
\author{Jim Martens (6420323)}
\title{Geschichte der USA von 1776-1945}
\subtitle{Vorlesungsprotokoll vom 3. Mai 2016}
\date{3. Mai 2016}
\maketitle

\section*{Hauptstadt}

\subsection*{Gründung}
\begin{myitemize}
    \item durch Schaffung der Stadt hat sich Volk selber gefeiert
    \item Washington D.C. entstand als Idee 1790, Bautätigkeit 1791/92,
          Hauptstadt seit 1800
    \item Washington ist dem Kongress direkt unterstellt
    \item Entscheidung der Bennenung ist noch zu Lebzeiten Washingtons passiert (1791)
    \item Washington D.C. liegt unmittelbar an Ostküste, 35km westlich
          der Chessapea Bay und an der Mündung des Anacostia-Flusses
    \item entstand an ihrem Platz wegen der Nähe zum Atlantik
    \item wurde auf Land gebaut, welches durch Marylnd und Virginia abgetreten wurde
    \item Washington (Person) besaß selber Land in dem Gebiet, welches er an den Bund verkauft hat
    \item ursprünglicher Distrikt hatte Fläche von 100 Quadratmeilen
    \item entsprach der Maximalgröße, die in der Verfassung vorgeschrieben war
    \item war ein Quadrat, deren Ecken in alle vier Himmelsrichtungen zeigten
    \item ungenutzte Fläche wurde an Virginia zurückgegeben
    \item besteht heutzutage nur noch aus ehemaligen Gebieten von Maryland
\end{myitemize}
\subsection*{Stadtplanung}
\begin{myitemize}
    \item sollte städtebaulich durchgeplant und für damalige Verhältnisse
          modern sein
    \item Linien zu den Ecken gehen vom Kapitol aus
    \item diagonal verlaufene Straßen werden Avenues genannt
    \item ost-west alphabetisch und nord-süd numerisch
    
    \item Stadt wurde durch Architekt geplant
    \item Jefferson legte dem Architekten mehrere Pläne vor, welche von Jeffersons Europareise stammten 
          (Frankfurt am Main, Karlsruhe, Amsterdam, Paris, Mailand) 
    \item Architekt entwickelte basierend auf Plänen einen ersten Entwurf
    \item es gab Überwürfnis mit Auftraggebern
    \item Landvermesser hat Pläne des Architekten übernommen, aber auch einige
          Änderungen vorgenommen
    \item endgültige Planung weist Parallelen zu Karlsruhe auf
    \item Bau des Weißen Hauses (13. Oktober 1792) begann zuerst
    \item Kapitol seit 1793 bezugsfähig
    \item John Adams siedelte im Juni 1800 mit Regierung nach Washington um
    \item D.C. kam unter direkte Verwaltung des Kongresses
\end{myitemize}
\subsection*{Heutige Stadt}
\begin{myitemize}
    \item Bürger der Stadt sind in Wahlrechten eingeschränkt
    \item erst seit 23. Verfassungszusatz 1961 dürfen Bewohner der Stadt
          den Präsidenten mitwählen (3 Wahlmänner)
    \item im Repräsentantenhaus ist District mit einem nicht stimmberechtigen
          Mitglied vertreten, im Senat gar nicht
    \item mit Ausnahme der Wahl zum Präsidenten haben Bewohner keine garantierten
          Wahlrechte
    \item auf Autos finden sich häufig Slogans "taxations without representation"
    \item Verbesserung der Wahlrechte gingen mit besserem Wahlergebnis für Demokraten einher
    \item Republikaner wollen dies nicht
    \item bei National Mall sind viele Museen angesiedelt (Smithonian, National Gallery of Art, Museum for history of Afro-Americans, Museum about Holocaust)
    \item Ostseite durch Kapitol bestimmt und Westseite durch Lincoln Memorial
    \item Washington Memorial ist von überall zu sehen
    \item Staat wird gefeiert im Kapitol
    \item Personenkult im Washington Memorial gefeiert
    \item Reunion im Lincoln Memorial gefeiert 
    \item wichtige staatliche Akte werden auf der National Mall gefeiert
    \item Gedanken der Planer sind in den Köpfen der Menschen angekommen
\end{myitemize}

    %\item vieles der Verfassung ist aus dem HRE übernommen wurden
\section*{Die Flagge}
\subsection*{Allgemein}
\begin{myitemize}
    \item keine Flagge ist so mit patriotischen Werten durchtränkt, wie die US-Flagge
    \item in Smithonian-Umfrage gaben 2007 Amerikaner mehrheitlich an, tägliche die Flagge zur Schau zu stellen
    \item nur knapp ein Fünftel der Befragten hat keine Flagge
    \item 14. Juni gilt bis heute als National Flag Day
    \item bei Gründungsvätern fand Flagge wenig Beachtung
    \item Sterne vom Himmel, rot vom Mutterland, weiß für Freiheit geht nicht von Washington aus
    \item Ursprünge des Flaggenkultes geht auf den zweiten Krieg der USA gegen UK zurück 1812-1814
    \item es entstand ein Lied und die Flagge avancierte zum Verteidigungsobjekt
    \item Lied wurde erst 1931 zur offiziellen Nationalhymne
    \item Flagge wurde als spar-sprangled banner bezeichnet
    \item Meta: Vorspielung des Liedes
    \item 1777 Flagge wird entwickelt, lokales Phänomen
    \item 1812 Flagge wird besungen, bleibt auf Maryland und Süden begrenzt
    \item großer Schwung kommt erst 1861 im amerikanischen Bürgerkrieg
    \item in dem Krieg offerten sich tausende Soldaten
    \item je mehr starben, desto mehr wurden zerstörte Flaggen zum Relikt
    \item Blut und Flagge wurde zum Symbol
    \item Flagge wurde in damals bekannten Arten gefeiert (Lied und Gedicht)
\end{myitemize}
\subsection*{Flaggengesetz}
\begin{myitemize}
    \item Flagge besteht aus 7 roten und 6 weißen Streifen für die 13 Gründungsstaaten
    \item weißer Sternenring trägt so viele Sterne wie es Staaten gibt
    \item rot, weiß, blau wird auf Union Jack zurückgeführt
    \item weiß: für Reinheit und Unschuld, rot: Tapferkeit und Widerstandsfähigkeit, blau: Beharrlichkeit und Gerechtigkeit
    \item Kongress beschloss, dass Streifen nicht mehr werden sollten
    \item anfangs gab es keine Richtlinien, wie Sterne erscheinen sollten
    \item es gab daher etliche Flaggenvarianten
    \item 1942 wurde vom Kongress der US Flag Codex beschlossen
    \item in diesem Codex ist alles in Bezug zur Flagge sehr penibel geregelt
    \item ausgefranzte Flagge muss verbrannt werden
    \item Pledge of the Legion/Union muss von jedem Schulkind abgegeben werden
\end{myitemize}

\section*{Die Konstante Land}
\subsection*{Allgemein}
\begin{myitemize}
    \item John Jey? empfand starke Bindung zwischen Bewohner der USA und dem Land in den Federalist Papers
    \item er formulierte später, dass das Land für die US-Amerikaner geschaffen wurde
    \item 1848 wurde es von Journalisten als "manifest destiny" gefasst
    \item Henry Sourough? empfand die "manifest destiny" nicht als die seinige
    \item seitdem die USA von dem UK anerkannt wurde, beschäftigte man sich auch mit Land
    \item es gab Disput um einige Gebiete
    \item Ostküste wird als Bezugspunkt genommen
    \item Northwest Territory wird als westlichster Endpunkt der Expansion gesehen
    \item UK hat Northwest Territory als Pufferzone zwischen Louisiana (Spanien) und Kanada
          angesehen und die Gebiete den indigenen Völkern zugesprochen
\end{myitemize}
\subsection*{Expansion}
\begin{myitemize}
    \item es gab etliche Konflikte zwischen den Kolonien und mit den indigenen Völkern
    \item Kolonisten waren sehr landhungrig
    \item ärgerten sich über Zurückhaltung der engl. Regierung
    \item nach Unabhängigkeit tritt USA auch als binnenländische Kolonialmacht auf
    \item Staaten übertrugen Anspruchsrechte an den Bund

    \item mehrere Landnahmestrategien:
    \begin{myitemize}
        \item  Besetzung
        \item  Besiedlung
        \item  Vertreibung, Tötung der Einwohner
        \item  Erwerb in Verhandlungen
        \item  Kauf
    \end{myitemize}
    \item Land bildete wesentlichen Bestandteil der US-amerik. Politik
    \item trotz Bedenken der Regierungsfähigkeit ging der Staat von Beginn an auf Expansionskurs
    \item 1803 wurde das damalige Lousiana gekauft und war nach Westen offen
    \item 1819 Florida
    \item 1836 Oregon
    \item 1848 xx Gebiete
    \item 1850 Texas
    \item 1850 Kalifornien
    \item 1867 Alaska
    \item 1890 schien "manifest destiny" erfüllt
    \item 1890 fortführend auf Philippinen und Samoa engagiert
    \item 1890 Krieg mit Kuba
    \item in knapp 50 Jahren wandelte sich USA von Kolonie zur Hegemoniemacht
    \item Verschiebung der Orientierung

    \item hat Auswirkungen für Landbesitz der Menschen, die in den USA leben
    \item 2,3 Mio. km² zu Lebzeiten Washingtons
    \item 1803: 4,4 Mio. km²
    \item 1845: 5,6 Mio. km²
    \item 1853: 7,8 Mio. km²
    \item 1895: 9,5 Mio. km²
    \item 75 Mio. Menschen damals in den USA
    \item Land als konkreter Raum, als idelogische Vereinnahmung, als ästhetische Fläche eine
    \item große Bedeutung für die Bevölkerung
    \item es fehlt Beziehung zwischen konkreten Territorien und ideologischer Bedeutung
\end{myitemize}

\end{document}