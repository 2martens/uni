\documentclass[10pt,a4paper,oneside,ngerman,numbers=noenddot]{scrartcl}
\usepackage[T1]{fontenc}
\usepackage[utf8]{inputenc}
\usepackage[ngerman]{babel}
\usepackage{amsmath}
\usepackage{amsfonts}
\usepackage{amssymb}
\usepackage{bytefield}
\usepackage{paralist}
\usepackage{gauss}
\usepackage{pgfplots}
\usepackage{textcomp}
\usepackage[locale=DE,exponent-product=\cdot,detect-all]{siunitx}
\usepackage{tikz}
\usepackage{algpseudocode}
\usepackage{algorithm}
\usepackage{mathtools}
\usepackage{hyperref}
%\usepackage{algorithmic}
%\usepackage{minted}
\usetikzlibrary{automata,matrix,fadings,calc,positioning,decorations.pathreplacing,arrows,decorations.markings}
\usepackage{polynom}
\polyset{style=C, div=:,vars=x}
\pgfplotsset{compat=1.8}
\pagenumbering{arabic}
%\def\thesection{\arabic{section})}
%\def\thesubsection{(\alph{subsection})}
%\def\thesubsubsection{(\roman{subsubsection})}
\makeatletter
\renewcommand*\env@matrix[1][*\c@MaxMatrixCols c]{%
  \hskip -\arraycolsep
  \let\@ifnextchar\new@ifnextchar
  \array{#1}}
\makeatother
\parskip 12pt plus 1pt minus 1pt
\parindent 0pt

\DeclarePairedDelimiter\abs{\lvert}{\rvert}%
\DeclarePairedDelimiter{\ceil}{\lceil}{\rceil}

\newenvironment{myitemize}{\begin{itemize}\itemsep -8pt}{\end{itemize}} % Zeilenabstand in Aufzählungen geringer

%switch starred and non-starred (auto-size)
\makeatletter
\let\oldabs\abs
\def\abs{\@ifstar{\oldabs}{\oldabs*}}
\makeatother

\hypersetup{
    colorlinks,
    citecolor=black,
    filecolor=black,
    linkcolor=black,
    urlcolor=black
}

\begin{document}
\author{Jim Martens (6420323)}
\title{Geschichte der USA von 1776-1945}
\subtitle{Vorlesungsprotokoll vom 12. April 2016}
\date{12. April 2016}
\maketitle

\section*{Hauptstadt}

- letzte Woche mit Amt des Präsidenten beschäftigt
- Ausgestaltung des Amtes durch George Washington
- wie ist Personenkult erstanden
- wie ist Hauptstadt namens Washington entstanden
- durch Schaffung der Stadt hat sich Volk selber gefeiert
- Washington D.C. entstand als Idee 1790, Bautätigkeit 1791/92,
  Hauptstadt seit 1800
- Washington ist Kongress direkt unterstellt
- Entscheidung der Bennenung ist noch zu Lebzeiten Washingtons passiert (1791)
- Columbia war Ableitung des Namens von Kolumbus
- Columbia war poetische Bezeichnung der USA in dem 18. Jhd.
- "Frau" Columbia wurde als Gegenstück zur Frau Britannia gefeiert
- Washington D.C. liegt unmittelbar an Ostküste, 35km westlich
  der Chessapea Bay, Mündung des Anacostia-Rivers?
- gebildet durch Land abgetreten durch Maryland und Virginia
- sollte städtebaulich durchgeplant und für damalige Verhältnisse
  modern sein
- entstand an ihrem Platz wegen der Nähe zum Atlantik
- Washington hat Land an den Bund verkauft
- Washington lebte nah an der Stadt

# 10-19 min
- Washington legte Wert auf gute Kutsche und gute Pferde
- ursprünglicher Distrikt hatte Fläche von 100 Quadratmeilen
- entsprach der Maximalgröße, die in der Verfassung vorgeschrieben waren
- war ein Quadrat, deren Ecken in alle vier Himmelsrichtungen zeigten
- ungenutzte Fläche wurde an Virginia zurückgegeben
- besteht heutzutage nur noch aus ehemaligen Gebieten von Maryland
- Linien zu den Ecken gehen vom Kapitol aus
- diagonal verlaufene Straßen werden Avenues genannt
- ost-west alphabetisch und nord-süd numerisch
- Stadt wurde durch Architekt geplant
    - Lonfont?
- Jefferson legte dem Architekten mehrere Pläne vor
    - diese stammten von dessen Europareise
    - Städte: Frankfurt am Main, Karlsruhe, Amsterdam, Paris, Mailand
- vieles der Verfassung ist aus dem HRE übernommen wurden
- Architekt entwickelte basierend auf Plänen einen ersten Entwurf
- es gab Überwürfnis mit Auftraggebern
- Landvermesser hat Pläne des Architekten übernommen, aber auch einige
  Änderungen vorgenommen
- endgültige Planung weist Parallelen zu Karlsruhe auf
- Bau des Weißen Hauses (13. Oktober 1792) begann zuerst
    - anwesend: Washington, Jeffers, Daniel Carrol und Landvermesser
- Kapitol seit 1793 bezugsfähig
- John Adams siedelte im Juni 1800 mit Regierung nach Washington um
- D.C. kam unter direkte Verwaltung des Kongresses
- Bürger der Stadt sind in Wahlrechten eingeschränkt
- erst seit 23. Verfassungszusatz 1961 dürfen Bewohner der Stadt
  den Präsidenten mitwählen (3 Wahlmänner)
- im Repräsentantenhaus ist District mit einem nicht stimmberechtigen
  Mitglied vertreten, im Senat gar nicht

# 20-29 min
- mit Ausnahme der Wahl zum Präsidenten haben Bewohner keine garantierten
  Wahlrechte
- auf Autos finden sich häufig Slogans "taxations without representation"
- Verbesserung der Wahlrechte gingen mit besserem Wahlergebnis für Demokraten einher
- Republikaner wollen dies nicht
- bei National Mall sind viele Museen angesiedelt
    - Smithonian
    - national gallery of art
    - museum für geschichte der afroamericans
    - museum für holocaust
- Ostseite durch Kapitol bestimmt und Westseite durch Lincoln Memorial
- Washington Memorial ist von überall zu sehen
    - sollte in Sichtachse des Weißen Hauses sein
    - ist nicht ganz der fall
- Staat wird gefeiert im Kapitol
- Personenkult im Washington Memorial gefeiert
- Reunion im Lincoln Memorial gefeiert 
- wichtige staatliche Akte werden auf der National Mall gefeiert
- Gedanken der Planer sind in den Köpfen der Menschen angekommen

# 30-39 min
- Kapitel 2 die Flagge
- keine Flagge ist so mit patriotischen Werten durchtränkt, wie die US-Flagge
- in Smithonian-Umfrage gaben 2007 Amerikaner mehrheitlich an, tägliche die Flagge zur Schau zu stellen
- nur knapp ein Fünftel der Befragten hat keine Flagge
- 14. Juni gilt bis heute als National Flag Day
- bei Gründungsvätern fand Flagge wenig Beachtung
- Sterne vom Himmel, rot vom Mutterland, weiß für Freiheit geht nicht von Washington aus
- Ursprünge des Flaggenkultes geht auf den zweiten Krieg der USA gegen UK zurück 1812-1814
- es entstand ein Lied und die Flagge avancierte zum Verteidigungsobjekt
- Lied wurde erst 1931 zur offiziellen Nationalhymne
- Flagge wurde als spar-sprangled banner bezeichnet
- Meta: Vorspielung des Liedes
- 1777 Flagge wird entwickelt, lokales Phänomen
- 1812 Flagge wird besungen, bleibt auf Maryland und Süden begrenzt
- großer Schwung kommt erst 1861 im amerikanischen Bürgerkrieg
- in dem Krieg offerten sich tausende Soldaten
- je mehr starben, desto mehr wurden zerstörte Flaggen zum Relikt
- Blut und Flagge wurde zum Symbol
- Flagge wurde in damals bekannten Arten gefeiert (Lied und Gedicht)

# 40-49 min
- von 1800 bis 1880 schrieb jeder Gedichte
- Franzis Lieber war deutscher Emigrant und wurde amerikanischer Patriot (seit 1859 in NYC)
- schrieb ein Gedicht, welches er ein Lied nannte, und sollte zu einer europäischen Melodie gesungen werden
  ("Ein freies Leben führen wir")
  - rot: Blut, weiß: Reinheit, Wahrheit
  - blau: See, die wir gerne durchpflügen, zwischen alter und noch älterer Welt
  - feiert die Union und spricht alles Positive dem Süden ab
- Meta: Vortragen des Gedichts
- dieses Gedicht gehört zu einem der besten Gedichte von Lieber
Flaggengesetz
- Flagge besteht aus 7 roten und 6 weißen Streifen für die 13 Gründungsstaaten
- weißer Sternenring trägt so viele Sterne wie es Staaten gibt
- rot, weiß, blau wird auf Union Jack zurückgeführt
- weiß: für Reinheit und Unschuld, rot: Tapferkeit und Widerstandsfähigkeit, blau: Beharrlichkeit und Gerechtigkeit
- Kongress beschloss, dass Streifen nicht mehr werden sollten
- anfangs gab es keine Richtlinien, wie Sterne erscheinen sollten
- es gab daher etliche Flaggenvarianten
- 1942 wurde vom Kongress der US Flag Codex beschlossen

# 50-59 min
- in diesem Codex ist sehr penibel geregelt
- 1953 wurde entschieden, dass Streifenbanner über xy gehisst werden darf?
- ausgefranzte Flagge muss verbrannt werden
- Pledge of the Legion/Union muss von jedem Schulkind abgegeben werden
- Kapitel 3 die Konstante Land
- Was wird dargestellt auf Flagge?
    - 13 Streifen für Gründungsstaaten
    - Sterne für Bundesstaaten
    - Landausdehnung
- John Jey? empfand starke Bindung zwischen Bewohner der USA und dem Land in den Federalist Papers
- er formulierte später, dass das Land für die US-Amerikaner geschaffen wurde
- 1848 wurde es von Journalisten als "manifest destiny" gefasst
- Henry Sourough? fand die "manifest destiny" nicht als die seinige
- seitdem die USA von der UK anerkannt wurde, beschäftigte man sich auch mit Land

# 60-69 min
- es gab Disput um einige Gebiete
- Ostküste wird als Bezugspunkt genommen
- Northwest Territory wird als westlichster Endpunkt der Expansion gesehen
- UK hat Northwest Territory als Pufferzone zwischen Louisiana (Spanien) und Kanada
  angesehen und die Gebiete den indigenen Völkern zugesprochen
--------
- es gab etliche Konflikte zwischen den Kolonien und mit den indigenen Völkern
- Kolonisten waren sehr landhungrig
- ärgerten sich über Zurückhaltung der engl. Regierung
- nach Unabhängigkeit tritt USA auch als binnenländische Kolonialmacht auf
- Staaten übertrugen Anspruchsrechte an den Bund

# 70-79 min
- mehrere Landnahmestrategien:
    - Besetzung
    - Besiedlung
    - Vertreibung, Tötung der Einwohner
    - Erwerb in Verhandlungen
    - Kauf
- Land bildete wesentlichen Bestandteil der US-amerik. Politik
- trotz Bedenken der Regierungsfähigkeit ging der Staat von Beginn an auf Expansionskurs
- 1803 wurde das damalige Lousiana gekauft und war nach Westen offen
- 1819 Florida
- 1836 Oregon
- 1848 xx Gebiete
- 1850 Texas
- 1850 Kalifornien
- 1867 Alaska
- 1890 schien "manifest destiny" erfüllt
- 1890 fortführend auf Philippinen und Samoa engagiert
- 1890 Krieg mit Kuba
- in knapp 50 Jahren wandelte sich USA von Kolonie zur Hegemoniemacht
- Verschiebung der Orientierung

# 80-89 min
- hat Auswirkungen für Landbesitz der Menschen, die in den USA leben
- 2,3 Mio. km² zu Lebzeiten Washingtons
- 1803: 4,4 Mio. km²
- 1845: 5,6 Mio. km²
- 1853: 7,8 Mio. km²
- 1895: 9,5 Mio. km²
- 75 Mio. Menschen damals in den USA
- Land als konkreter Raum, als idelogische Vereinnahmung, als ästhetische Fläche eine
  große Bedeutung für die Bevölkerung
- es fehlt Beziehung zwischen konkreten Territorien und ideologischer Bedeutung


\end{document}