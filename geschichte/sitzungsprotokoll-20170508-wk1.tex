\documentclass[10pt,a4paper,oneside,ngerman,numbers=noenddot]{scrartcl}
\usepackage[T1]{fontenc}
\usepackage[utf8]{inputenc}
\usepackage[ngerman]{babel}
\usepackage{amsmath}
\usepackage{amsfonts}
\usepackage{amssymb}
\usepackage{bytefield}
\usepackage{paralist}
\usepackage{gauss}
\usepackage{pgfplots}
\usepackage{textcomp}
\usepackage[locale=DE,exponent-product=\cdot,detect-all]{siunitx}
\usepackage{tikz}
\usepackage{algpseudocode}
\usepackage{algorithm}
\usepackage{mathtools}
\usepackage{hyperref}
\usepackage[german=quotes]{csquotes}
%\usepackage{algorithmic}
%\usepackage{minted}
\usetikzlibrary{automata,matrix,fadings,calc,positioning,decorations.pathreplacing,arrows,decorations.markings}
\usepackage{polynom}
\polyset{style=C, div=:,vars=x}
\pgfplotsset{compat=1.8}
\pagenumbering{arabic}
%\def\thesection{\arabic{section})}
%\def\thesubsection{(\alph{subsection})}
%\def\thesubsubsection{(\roman{subsubsection})}
\makeatletter
\renewcommand*\env@matrix[1][*\c@MaxMatrixCols c]{%
  \hskip -\arraycolsep
  \let\@ifnextchar\new@ifnextchar
  \array{#1}}
\makeatother
\parskip 12pt plus 1pt minus 1pt
\parindent 0pt

\DeclarePairedDelimiter\abs{\lvert}{\rvert}%
\DeclarePairedDelimiter{\ceil}{\lceil}{\rceil}

\newenvironment{myitemize}{\begin{itemize}\itemsep -9pt}{\end{itemize}} % Zeilenabstand in Aufzählungen geringer

%switch starred and non-starred (auto-size)
\makeatletter
\let\oldabs\abs
\def\abs{\@ifstar{\oldabs}{\oldabs*}}
\makeatother

\hypersetup{
    colorlinks,
    citecolor=black,
    filecolor=black,
    linkcolor=black,
    urlcolor=black
}

\MakeOuterQuote{"}

\begin{document}
\author{Jim Martens (6420323)}
\title{Vorlesung Osteuropa im 1. Weltkrieg}
\subtitle{Vorlesungsprotokoll vom 8. Mai 2017}
\date{8. Mai 2017}
\maketitle

\section*{Übersicht}

Im Vorfeld des Ersten Weltkriegs lässt sich beobachten, dass bis auf
Österreich-Ungarn alle Großmächte (Deutschland, Frankreich, Vereinigtes
Königreich, Russland, Osmanisches Reich, Italien) Vielvölkerstaaten waren.
Es lässt sich jedoch ebenfalls beobachten, dass in direktem Zusammenhang mit
dem Ersten Weltkrieg eine nationale Identität heraufbeschworen wurde, die es
so zu jener Zeit gar nicht gab. Vielmehr befinden sich alle Staaten in einem
Nationalisierungsprozess, der vielerorts auch auf Widerstände traf und alles
andere als abgeschlossen war.

Für die Mobilisierung zu einem Krieg gegen andere Nationen war es allerdings
notwendig zu begründen, warum dieser Krieg so wichtig sei. Dabei lässt sich eine
charakteristische Komponente erkennen, die sich in allen Großmächten wiederfindet.
Die Minderheiten in den anzugreifenden Staaten wurden für die eigenen Zwecke
instrumentalisiert. Man trat an diese Gruppen zu "befreien" von dem Joch der
Unterdrückung. Dies war auch die Legitimation für den Kriegseintritt. Um die
Interessen dieser Minderheiten ging es dabei jedoch nicht. Interessanterweise waren
Minderheiten immer zudem nur in den anderen Staaten unterdrückt - nie im eigenen.
Diese Instrumentalisierung von Minderheiten wurde diesen nicht selten in ihrem
eigenen Staat zum Verhängnis, da sie fortan als Verbündete der Gegner galten.

\section*{Nationalismusvarianten}

Zur damaligen Zeit gab es vier Nationalismusansätze. Der "demokratische"
Nationalismus tolerierte andere Nationen neben der eigenen. Er beinhaltete auch
einige Bedingungen:

\begin{enumerate}
    \item Die Nation umfasst alle Bewohner des nationalen Territoriums; denn
    alle haben den gleichen Anspruch auf Menschen- und Bürgerrechte.
    \item Alle Mitglieder der Nation sind berechtigt und sollen befähigt sein, an
    deren politischer Kultur teilzuhaben und die Solidarität der Nation zu erfahren;
    dazu beizutragen, sind alle gleichermaßen verpflichtet.
    \item Eine Nation hat das Recht auf politische Selbstverwaltung innerhalb ihres
    Territoriums; das Prinzip der Volkssouveränität soll die Norm der Staaten sein.
    \item Alle Völker haben ein gleiches Recht auf Existenz, auf Nationsbildung und
    auf Selbstbestimmung innerhalb ihres Siedlungsgebiets.
\end{enumerate}

Diese Form des Nationalismus hat nur eine sehr beschränkte Rolle gespielt. So
wurde er in den Balkankriegen und im Krieg von Italien gegen Österreich
verwendet.

Der "integrale" Nationalismus spielte vor dem Ersten Weltkrieg kaum eine Rolle
gespielt. Er beschreibt "eine kämpferische [...] Sammlungsbewegung von rechts,
die eine qualitativ andere Nation im Visier hatte". In dieser Nation ging es
nicht mehr um Menschenrechte und Gleichberechtigung, sondern um ein elitär
verstandenes Volkstum. Dieser Nationalismus fand seinen Auftrieb in den 1930er
Jahren u.a. im Deutschen Reich in Form des Nationalsozialismus.

Schließlich gab es noch den inklusiven bzw. exklusiven Nationalismus. Der Erstere
schließt alle politischen und kulturellen Gruppen ein und entwickelt damit eine
integrierende Wirkung. Der exklusive Nationalismus grenzt sich zu anderen Staaten
und Nationen ab und überhöht die eigenen nationalen Elemente. Er verkleinert damit
im Gegensatz zum inklusiven Nationalismus die eigene Gruppe.

\section*{Unabhängigkeit Polens}

Nach den Befreiungskriegen war Polen dreigeteilt und nicht mehr als unabhängiger
Staat vorhanden. Russland, Preußen und Österreich-Ungarn kontrollierten je einen
Teil des ehemals polnischen Gebiets. Entsprechend interessant ist es, dass
ausgerechnet die polnische Unabhängigkeit in Verbindung mit dem Ersten Weltkrieg
so eine Relevanz bekam. Es gab Bestrebungen von der österreichischen Seite und
von der russischen.
Die österreichische Bewegung verkauft den russischen Teil
des polnischen Teilungsgebiets als unfrei. Es wird daher zum Befreiungskampf
gegen die Russen aufgerufen. Dabei wird behauptet, dass Österreich für die
polnische Sache kämpfe. Die österreichische Seite wird entsprechend als Land der
Freiheit verkauft. Allerdings hat Österreich 1914 gar kein Interesse am Krieg
mit Russland und eine Unabhängigkeit Polens will Österreich auch nicht.
Die Kampagne hat wenig Erfolg.

Die russische Kampagne erreicht eine höhere Verbreitung auch außerhalb des von
Russland kontrollierten Gebiets. Sie wird auf Russisch und Polnisch veröffentlicht,
obwohl polnische Veröffentlichungen eigentlich verboten waren. In dem Aufruf
wird von einer "frohen Botschaft" gesprochen. Dieser Begriff wird eigentlich
für das Evangelium verwendet. Es wird gefordert, dass sich der russisch-polnische
Teil mit den anderen Teilen verbindet. Dieses zusammenhängende Polen soll allerdings
nicht etwa wirklich unabhängig sein, sondern unter der Herrschaft des Zaren stehen.
Damit ist die frohe Botschaft also die Kontrolle ganz Polens durch Russland.
Nach der Eroberung soll es Freiheit und Selbstverwaltung geben. Russland hätte
jedoch bereits seit Jahrzehnten die Möglichkeit gehabt dies in seinem Bereich
umzusetzen. Tatsächlich erfolgte im russischen Bereich mehr Unterdrückung als
im österreichischen Teil. Dies schien jedoch egal zu sein, denn bis zur Einnahme
Warschaus durch die Mittelmächte herrschte dort eine pro-russische Stimmung.
Dies kann auch als Beginn des integralen Nationalismus gesehen werden. So war
der Zusammenhang der slawischen Völker wichtiger als die tatsächliche Unterdrückung.
Es kommt sogar zur Gründung eines polnischen Nationalkommittees im russischen
Bereich, welches sich für das Zarenreich ausspricht.

Bei der wenig erfolgreichen österreichischen Bewegung kann jedoch davon ausgegangen
werden, dass eine Trennung von Österreich nach der Einnahme des russischen Teils
geplant war.

\section*{Befreiung der Juden in Russland}

Von der deutschen Seite gab es Bestrebungen zur "Befreiung" der Juden in Russland.
Allerdings kommt diese Idee von den Zionisten in Deutschland und nicht von der
Generalität oder dem Kaiser. Die Idee beinhaltet die Agitation der Juden in
Russland gegen Russland. In den zu besetzenden Gebieten sei nur auf die Juden
Verlass. Auch hier geht es um eine "frohe Botschaft". Für diese Kampagne wurde
eine Zeitschrift auf Hebräisch und Jiddisch verfasst, welche insgesamt zwei
Ausgaben erhielt. Es ist unklar, ob die Zeitschrift tatsächlich über russischem
Gebiet abgeworfen wurde. Nach Kriegsbeginn kamen die Deutschen in Polen aber nicht
sonderlich gut voran, sodass eine eventuelle Wirkung im russischen Teil verpuffte.
Im deutschen Heer machte sich die Einstellung breit, die Juden würden sich nicht
genügend für die Kriegszwecke der Deutschen einsetzen. Die antijüdischen Kräfte
gewannen die Oberhand und es fand die sogenannte Judenzählung statt.

Ein weiterer Effekt der Kampagne sollte ein besseres Verhältnis der Deutschen
gegenüber den Juden sein. Dieser Effekt stellte sich jedoch nicht ein. Die ganze
Aktion blieb dennoch nicht ohne Wirkung. Die Juden in Russland wurden von der
Grenzregion ins Landesinnere deportiert.
\end{document}
