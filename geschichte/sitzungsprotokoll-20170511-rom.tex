\documentclass[10pt,a4paper,oneside,ngerman,numbers=noenddot]{scrartcl}
\usepackage[T1]{fontenc}
\usepackage[utf8]{inputenc}
\usepackage[ngerman]{babel}
\usepackage{amsmath}
\usepackage{amsfonts}
\usepackage{amssymb}
\usepackage{bytefield}
\usepackage{paralist}
\usepackage{gauss}
\usepackage{pgfplots}
\usepackage{textcomp}
\usepackage[locale=DE,exponent-product=\cdot,detect-all]{siunitx}
\usepackage{tikz}
\usepackage{algpseudocode}
\usepackage{algorithm}
\usepackage{mathtools}
\usepackage{hyperref}
\usepackage[german=quotes]{csquotes}
%\usepackage{algorithmic}
%\usepackage{minted}
\usetikzlibrary{automata,matrix,fadings,calc,positioning,decorations.pathreplacing,arrows,decorations.markings}
\usepackage{polynom}
\polyset{style=C, div=:,vars=x}
\pgfplotsset{compat=1.8}
\pagenumbering{arabic}
%\def\thesection{\arabic{section})}
%\def\thesubsection{(\alph{subsection})}
%\def\thesubsubsection{(\roman{subsubsection})}
\makeatletter
\renewcommand*\env@matrix[1][*\c@MaxMatrixCols c]{%
  \hskip -\arraycolsep
  \let\@ifnextchar\new@ifnextchar
  \array{#1}}
\makeatother
\parskip 12pt plus 1pt minus 1pt
\parindent 0pt

\DeclarePairedDelimiter\abs{\lvert}{\rvert}%
\DeclarePairedDelimiter{\ceil}{\lceil}{\rceil}

\newenvironment{myitemize}{\begin{itemize}\itemsep -9pt}{\end{itemize}} % Zeilenabstand in Aufzählungen geringer

%switch starred and non-starred (auto-size)
\makeatletter
\let\oldabs\abs
\def\abs{\@ifstar{\oldabs}{\oldabs*}}
\makeatother

\hypersetup{
    colorlinks,
    citecolor=black,
    filecolor=black,
    linkcolor=black,
    urlcolor=black
}

\MakeOuterQuote{"}

\begin{document}
\author{Jim Martens (6420323)}
\title{Vorlesung Römische Kaiserzeit}
\subtitle{Vorlesungsprotokoll vom 11. Mai 2017}
\date{11. Mai 2017}
\maketitle

\section*{Übersicht}

Die römische Sozialstruktur gliedert sich in verschiedene Ebenen. Auf der
untersten Ebene finden sich die Sklaven. Darüber stehen die Freigelassenen,
gefolgt von den freien Provinzialen. Die Kinder von Freigelassenen sind jedoch
römische Bürger und daher freien Provinzialen gegenüber bevorteilt.
Auf der nächsten Ebene stehen die römischen Bürger. Über ihnen stehen die
Ritter und schließlich die Senatoren. In der Kaiserzeit folgt ganz oben noch
der Kaiser.

Die soziale Struktur ändert sich im Laufe der Zeit erstaunlich wenig. Trotz
zahlreicher politischer Veränderungen bleibt die römische Sozialstruktur
von der frühen Republik bis weit in die Kaiserzeit relativ stabil. Diese
Stabilität bedeutet aber auch, dass die mit der Struktur verbundenen
systematischen Probleme bestehen bleiben. Trotz erheblicher innenpolitischer
Konflikte wird das System selber nie in Frage gestellt. Stattdessen werden
die Probleme personalisiert auf die aktuell Herrschenden bezogen.

\section*{Sklaven}

Die Position der Sklaven in der römischen Gesellschaft war nie gut. Allerdings
hat sich die Situation in der Kaiserzeit im Vergleich zur späten Republik
etwas gebessert. So kommt es im Prinzipat zu weniger Sklavenaufständen als
in der späten Republik. Häufiger als ein Aufstand ist die Flucht einzelner
Sklaven, was eine neue Berufsgruppe der Sklavenfänger entstehen lässt.
Dies ist auch vor dem Hintergrund zu sehen, dass Magistrate und Statthalter
entlaufene Sklaven nur halbherzig verfolgen.
Die Flucht von Sklaven scheint jedoch ein derartiges Problem zu sein, dass
es unter hohe Strafe gestellt wird einem Sklaven zu helfen.

Im Jahre 61 wurde der Stadtpräfekt Roms durch einen Sklaven aus seinem Haushalt
ermordet. Alle Sklaven des Hauses (rund 400) sollen hingerichtet werden. Dies
stößt auch im Senat eine große Diskussion an, da es zum Teil als überzogen
angesehen wird. Die Hardliner können sich aber durchsetzen und die 400 Sklaven,
darunter Frauen und Kinder, werden hingerichtet. Es kommt nicht zur
Ursachenanalyse.

Ab Tiberius wird der Stadtpräfekt zur Appellationsinstanz für Sklaven und diese
können bei Kaiserstatuen um Asyl bitten und den eigenen Verkauf an einen anderen
Herrn erreichen, um sich vor den größten Schikanierungen zu schützen. Hadrian
verbannte sogar eine reiche Dame für 5 Jahre aus Rom, weil sie Sklaven
misshandelte.

Wenn der Herr es zulässt, können Sklaven ein kleines Vermlgen anhäufen (perculium)
und sich nach Ersparen der eigenen Kaufsumme freikaufen. Bei einem solchen
Freikauf oder der Freilassung durch den Herrn werden die Sklaven zu Freigelassenen
(libertus) und sind noch keine römischen Bürger.

\section*{Freigelassene}

Bei der Freilassung ändert sich das Verhältnis von dominus \(\rightarrow\) servus
zu patronus \(\rightarrow\) cliens. Der Patron ist nicht mehr für die
Altersvorsorge der Freigelassenen verantwortlich. Daher rechnet es sich durchaus
Sklaven freizulassen, da diese einem auf Lebenszeit zu Dank verpflichtet sind,
häufig weiter für einen arbeiten und sich meistens längst amortisiert haben.

In der Republik gibt es drei Formen von Freilassungen: (1) per Testament,
(2) durch Zensus (alle 5 Jahre kann vor dem Zensor freigelassen werden) und
(3) fingierter Freilassungsprozess vor einem Magistrat. Im Kaiserreich gibt
es zwei weitere Formen: (4) Freilassung unter Freunden (Freunde als Zeugen)
und (5) schriftliche Freilassung. Hauptsächlich finden (1), (4) und (5) Anwendung.

Die Freigelassenen sind in vielen Bereichen tätig und verewigen ihren Aufstieg
in Inschriften. Der Leistungsgedanke ist für sie zentral. Denn nach der Freilassung
stehen sie unter einem gewissen Legitimationsdruck zu zeigen, dass sie zurecht
freigelassen wurden. Die Freilassung wird aber auch als Machtinstrument verwendet,
um Sklaven zu Wohlverhalten zu zwingen: Nur gutes Benehmen als Sklave ermöglicht
Freilassung und die Freigelassenen sind ihrem vormaligen Herrn zu Dank verpflichtet.

Freigelassene sind in den collegia - den Berufsgemeinschaften - aktiv und
konkurrieren bei Reichtum mit den Stadträten um die meisten Stiftungen. Ehrungen
bleiben den Freigelassenen jedoch meistens verwehrt. Im Rahmen der collegia
bekommen sie jedoch Ehrenstatuen.

Unter Claudius regierten sogar Freigelassene. Für den Kaiser waren sie zudem
nützlich, da es am Hof des Kaisers keine Lobby für sie gab. Im Verlauf des
Kaiserreichs stammen viele Ritter und Senatoren von Freigelassenen ab. In Rom
stellen sie die Polizei und die Feuerwehr.

\section*{Leben in den Provinzen}

Um 14 n.Chr. beträgt die Gesamtbevölkerung des Reiches knapp 60 Millionen.
Der Großteil der Bevölkerung lebt in Asien, Syrien, Palästina und Ägypten.
Im westlichen Teil des Reichs lebt nur ein kleiner Teil der Bevölkerung.
Dort bedeutet Romanisierung also im wesentlichen Urbanisierung. Insgesamt
sind nur knapp 5 Millionen freie römische Bürger. Der überwiegende Teil
der Bevölkerung sind demnach freie Provinziale (peregrini). Diese Gruppe
ist sehr heterogen und die Rechte und Pflichten der Provinzialen ergeben sich
aus den Rechtsverfassungen ihrer Heimatstädte.

Es gibt Kolonien und Municipia mit jeweils römischen oder latinischem Recht.
Dann gibt es einheimische Städte, die sich in freie Städte ohne Steuerpflicht,
föderale Städte und steuerpflichtige Städte unterteilen. In latinischen
Siedlungen gibt es das römische Bürgerrecht nur für Menschen in politischen Ämtern.

Oft bestimmen die einheimischen Kulturen das Alltagsleben und die Statthalter sind
nur für Streitschlichtung zuständig. Die Eingriffe der römischen Provinzialverwaltung
beschränken sich auf ein absolutes Minimum, weil ein größeres Engagement mangels
effizienter Verwaltungsstrukturen schlicht nicht möglich war. Alte Stammes- und
Stadtverfassungen bleiben in den einheimischen Städten in Kraft. In religiösen
Fragen mischt sich der römische Staat nicht ein.

Die Aufgaben der Statthalter beschränken sich auf die Aufrechterhaltung von Ruhe
und Ordnung, dem Schutz von Leben und Eigentum (Schutz der Eliten) und die Garantie
der Funktion der lokalen Selbstverwaltung und Rechtsprechung.

Einheimische Sprachen bestehen ebenso weiter. In der Spätantike wurde am Kaiserhof
in Trier noch Keltisch gesprochen. In Syrien und Palästina wurde von der
Oberschicht Griechisch und von dem allgemeinen Volk Aramäisch gesprochen. In
Afrika ist Punisch weiterhin in aktiver Benutzung. Mehrsprachigkeit war demnach
ein häufigeres Phänomen als in der Neuzeit.

Von 600 Senatoren war nur ein kleiner Teil in den Provinzen. Dazu kommen einige
Tausend Ritter und die Stadträte, welche zusammen für die gesamte Verwaltung
zuständig waren. Erst Diocletian bläht die Verwaltung auf, was jedoch auch
wieder Nachteile mit sich bringt. Im gesamten Reichsgebiet gibt es also nur eine
dünne Suprastruktur, die alles zusammenhält. Und diese Struktur löst sich in der
sog. Völkerwanderung auf.


\end{document}
