\documentclass[10pt,a4paper,oneside,ngerman,numbers=noenddot]{scrartcl}
\usepackage[T1]{fontenc}
\usepackage[utf8]{inputenc}
\usepackage[ngerman]{babel}
\usepackage{amsmath}
\usepackage{amsfonts}
\usepackage{amssymb}
\usepackage{bytefield}
\usepackage{paralist}
\usepackage{gauss}
\usepackage{pgfplots}
\usepackage{textcomp}
\usepackage[locale=DE,exponent-product=\cdot,detect-all]{siunitx}
\usepackage{tikz}
\usepackage{algpseudocode}
\usepackage{algorithm}
\usepackage{mathtools}
\usepackage{hyperref}
\usepackage[german=quotes]{csquotes}
%\usepackage{algorithmic}
%\usepackage{minted}
\usetikzlibrary{automata,matrix,fadings,calc,positioning,decorations.pathreplacing,arrows,decorations.markings}
\usepackage{polynom}
\polyset{style=C, div=:,vars=x}
\pgfplotsset{compat=1.8}
\pagenumbering{arabic}
%\def\thesection{\arabic{section})}
%\def\thesubsection{(\alph{subsection})}
%\def\thesubsubsection{(\roman{subsubsection})}
\makeatletter
\renewcommand*\env@matrix[1][*\c@MaxMatrixCols c]{%
  \hskip -\arraycolsep
  \let\@ifnextchar\new@ifnextchar
  \array{#1}}
\makeatother
\parskip 12pt plus 1pt minus 1pt
\parindent 0pt

\DeclarePairedDelimiter\abs{\lvert}{\rvert}%
\DeclarePairedDelimiter{\ceil}{\lceil}{\rceil}

\newenvironment{myitemize}{\begin{itemize}\itemsep -9pt}{\end{itemize}} % Zeilenabstand in Aufzählungen geringer

%switch starred and non-starred (auto-size)
\makeatletter
\let\oldabs\abs
\def\abs{\@ifstar{\oldabs}{\oldabs*}}
\makeatother

\hypersetup{
    colorlinks,
    citecolor=black,
    filecolor=black,
    linkcolor=black,
    urlcolor=black
}

\MakeOuterQuote{"}

\begin{document}
\author{Jim Martens (6420323)}
\title{Vorlesung Geschichte der Lateinamerikanerinnen und Lateinamerikaner}
\subtitle{Vorlesungsprotokoll vom 17. Mai 2017}
\date{17. Mai 2017}
\maketitle

\section*{Geschichte der Sklaverei}

Kolumbus hat den Aufbau von Handel zum Ziel. Entsprechend werden die gefundenen
Gebiete als reich beschrieben. Die Karibik ist für Spanien im 15. Jahrhundert
jedoch nahezu uninteressant. Die einheimische Bevölkerung stirbt aufgrund der
aus Europa mitgebrachten Krankheiten fast vollständig aus. Auf dem Festland
ist die Situation jedoch eine andere. Dort gibt es Gold und Silber, welche
dann auch der Grund für die Kolonialisierung Südamerikas waren. Nach einer Weile
wird Zucker entdeckt. Dieser Fund ändert die Geschichte Amerikas grundlegend.
Denn Zucker ist quasi weißes Gold, da es zur damaligen Zeit nur sehr wenige
aus Europa bekannte Zuckeranbaugebiete gab. Zucker war also wertvoll.

Es gab jedoch das Problem, dass in den Gebieten, wo einheimische Bevölkerung
überlebte, kein Zucker angebaut werden konnte und in den Zuckeranbaugebieten
keine Bevölkerung mehr lebte. Für den Anbau wurden also Menschen benötigt. In
Europa gab es damals kaum "verfügbare" Menschen. Die Lösung aus Sicht der
Kolonialmächte waren Sklaven aus Afrika. Die Sklaven wurden nur zur Produktion
von lukrativen Produkten verwendet. Das ging sogar so weit, dass auf einigen
Karibikinseln ausschließlich Zucker angebaut wurde und die Nahrung für u.a. die
Sklaven importiert werden musste. Der Fund von Zucker war demnach der Startpunkt
des Sklavenhandels.

Über die Dauer des Sklavenhandels kommen insgesamt rund 9,5 Millionen Sklaven
vom 16. bis 19. Jahrhundert nach Amerika. Dabei kommen die meisten Sklaven im
18. und 19. Jahrhundert. Zur Zeit der Aufklärung in Europa gibt es also die
Hochphase des Sklavenhandels. Lediglich während der französischen Revolution
wurde unter der Herrschaft Robespierres der Sklavenhandel Frankreichs kurzfristig
ausgesetzt. Die erste Abschaffung des Sklavenhandels geht nicht von Aufklärern aus,
sondern von einer Sklavenerhebung in Haiti. Ebenso sind die religiösen
Fundamentalisten gegen den Sklavenhandel und setzen sich im frühen 19. Jahrhundert
gegen die vernünftigen und rationalen Menschen in Europa durch.

Im britischen Nordamerika kommen "nur" 361.000 Sklaven an, was ungefähr 3\% der
gesamten Menge aus dem Sklavenhandel sind. In der Karibik landen knapp 34-40\%.
Die Karibik ist jedoch ein schlechter Ort zum Leben - für Sklaven und Europäer
gleichermaßen. Die Mortalitätsrate der Sklaven dort ist sehr hoch. Ein ähnlich
großer Teil der Sklaven kommt nach Brasilien. Die Mortalitätsrate von Sklaven
und Europäern ist in Brasilien nicht ganz so hoch wie in der Karibik, aber
deutlich höher als in den USA.

Das Ende der Sklaverei erfolgte in einem dreistufigen Verfahren. Zunächst wurde
der atlantische Sklavenhandel beendet. Darauf folgte das Ende der Vererbung von
Sklaverei und schließlich die gänzliche Abschaffung derselben. Der atlantische
Sklavenhandel bricht weitestgehend zusammen als das britische Parlament beschließt,
dass die Royal Navy in brasilianische Häfen einfahren darf, um den Handel zu
stoppen. Diese Drohung reicht aus, damit Brasilien den Handel einstellt. In Folge
dessen steigen zunehmend Länder aus dem Handel aus.

Mit dem Ende des Sklavenhandels verringert sich die Anzahl der Sklaven in Kuba
und Brasilien, da aufgrund der hohen Sterblichkeitsrate und ohne externen Nachschub
kein nachhaltiges Wachstum möglich ist. Ganz anders in den USA: Obwohl der
atlantische Sklavenhandel relativ früh beendet wird, kann sich die Sklaverei
noch sehr lange halten. Dies liegt an den guten materiellen Bedingungen und damit
einem Wachstum der Anzahl der Sklaven durch Geburt, sowie einer Staatlichkeit,
welche die Sklaverei durchsetzen kann.

Die letztendliche Abschaffung erfolgt erst nach dem Bürgerkrieg, der sich um
die Frage drehte, ob die USA ein Bundesstaat oder ein Staatenbund sind und ob
demnach die Sklaverei auf Bundesebene abgeschafft werden kann. Vor Beginn des
Bürgerkriegs gibt es jahrzehntelang Propaganda gegen die Sklaverei. Die Frage
der Sklaverei war denn auch eine große Diskussion in der US-amerikanischen
Öffentlichkeit. Ein Teil der Bewegung gegen die Sklaverei waren die sog.
slave narratives.

\section*{Slave narratives}

Die slave narratives sind buchlange Erzählungen des eigenen Lebens. Ihre Ursprünge
liegen in den captivity-narratives und der Erlösungsliteratur. Sie treten
hauptsächlich im 19. Jahrhundert auf, hatten aber auch schon Vorläufer im 18.
Jahrhundert. Berühmte slave narratives sind z.B. "Twelve Years a Slave" von
Solomon Northup aus dem Jahr 1853, "A Narrative of the life of Frederick Douglass,
an American Slave" von Frederick Douglass aus dem Jahr 1845 sowie "Incidents in
the Life of a Slave Girl" von Harriet Jacobs aus dem Jahr 1861. Kein eigentlicher
slave narrative war "Uncle Tom's Cabin" von Harriet Beecher Stove aus dem Jahr
1852. Dieser Roman war jedoch DER Anti-Sklaverei-Roman des 19. Jahrhunderts.

Bei einem slave narrative ist ein (Ex-)Sklave die erzählende Person. Die vermittelte
Botschaft ist, dass ein Sklave ein Mensch und Autor seines Lebens ist. Ebenfalls
zeigen diese narratives, dass sich Sklaven bilden können und Sklaverei daher
moralisch schlecht und unnatürlich ist. Häufige Vorkommnisse in diesen Erzählungen
sind die Trennung von Mutter und Kindern, die Trennung von Eheleuten,
ungerechtfertigte Bestrafungen und der moralische Verfall der Herren. Die
narratives zeigen in der Essenz die Menschlichkeit der Sklaven.

Diese slave narratives gibt es in der Form nur aus den USA. Die feste
Institutionalisierung der Sklaverei in den USA ist der Hauptgrund für ihre Existenz.
Denn mit ihnen soll eine Debatte über die Sklaverei und mittels der Debatte eine
politische Mehrheit zur Abschaffung der Sklaverei organisiert werden.

In Lateinamerika entsteht eine solche Debatte nicht, da sich die Eliten dort
einig sind, dass die Sklaverei abgeschafft gehört. Auch das Genre der slave
narratives ensteht in Lateinamerika nicht. Lediglich drei Werke kommen in die
Nähe dieser slave narratives. "El Cimarron" soll über das Leben des Esteban Montejo
handeln. Die Geschichte passt jedoch sehr in die Erzählung der kubanischen
Revolution. Das Werk "Biography" soll von dem Sklaven Mahommah Gardo Baquaqua
handeln. Es wurde jedoch von US-Amerikanern für ein US-amerikanisches Publikum
erstellt und basiert lediglich auf Unterhaltungen mit Baquaqua.

Das dritte
Werk wurde von Juan Francisco Manzano verfasst und kommt den slave narratives
ein wenig nahe. Es wurde von einem kubanischen Sklaven geschrieben, welcher
seit seiner Jugend Gedichte schrieb. Das Verfassen der Autobiografie war die
Voraussetzung dafür, dass del Monte ihn freikaufte. Das Werk ist sehr nah an
der Sprechfassung und wurde 1840 zunächst auf Englisch in London veröffentlicht.
Die Erstveröffentlichung war also kein Text für Kuba. Es wird ebenfalls vermutet,
dass nur ein Teil der Autobiografie erhalten blieb, da diese nur 45-50 Seiten
ausmacht. Thematisch wird eine gute Herrin beschrieben, die eines Tages stirbt.
Der neue Herr ist die Ursache für viele grausame Strafen. Ebenfalls wird die
besondere Rolle der Mutter beschrieben.

Aktuell gibt es drei Fassungen des Werkes: "Autiobiografie de un esclavo", welche
1975 in Madrid publiziert wurde, "The autobiography of a slave" aus Detroit im Jahr
1996 und die jüngste Fassung aus dem Jahr 2007. Man kann sagen, dass Manzano neu
entdeckt wurde.

\end{document}
