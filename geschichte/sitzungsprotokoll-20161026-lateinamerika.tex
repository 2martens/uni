\documentclass[10pt,a4paper,oneside,ngerman,numbers=noenddot]{scrartcl}
\usepackage[T1]{fontenc}
\usepackage[utf8]{inputenc}
\usepackage[ngerman]{babel}
\usepackage{amsmath}
\usepackage{amsfonts}
\usepackage{amssymb}
\usepackage{bytefield}
\usepackage{paralist}
\usepackage{gauss}
\usepackage{pgfplots}
\usepackage{textcomp}
\usepackage[locale=DE,exponent-product=\cdot,detect-all]{siunitx}
\usepackage{tikz}
\usepackage{algpseudocode}
\usepackage{algorithm}
\usepackage{mathtools}
\usepackage{hyperref}
\usepackage[german=quotes]{csquotes}
%\usepackage{algorithmic}
%\usepackage{minted}
\usetikzlibrary{automata,matrix,fadings,calc,positioning,decorations.pathreplacing,arrows,decorations.markings}
\usepackage{polynom}
\polyset{style=C, div=:,vars=x}
\pgfplotsset{compat=1.8}
\pagenumbering{arabic}
%\def\thesection{\arabic{section})}
%\def\thesubsection{(\alph{subsection})}
%\def\thesubsubsection{(\roman{subsubsection})}
\makeatletter
\renewcommand*\env@matrix[1][*\c@MaxMatrixCols c]{%
  \hskip -\arraycolsep
  \let\@ifnextchar\new@ifnextchar
  \array{#1}}
\makeatother
\parskip 12pt plus 1pt minus 1pt
\parindent 0pt
\MakeOuterQuote{"}

\DeclarePairedDelimiter\abs{\lvert}{\rvert}%
\DeclarePairedDelimiter{\ceil}{\lceil}{\rceil}

\newenvironment{myitemize}{\begin{itemize}\itemsep -9pt}{\end{itemize}} % Zeilenabstand in Aufzählungen geringer

%switch starred and non-starred (auto-size)
\makeatletter
\let\oldabs\abs
\def\abs{\@ifstar{\oldabs}{\oldabs*}}
\makeatother

\hypersetup{
    colorlinks,
    citecolor=black,
    filecolor=black,
    linkcolor=black,
    urlcolor=black
}

\begin{document}
\author{Jim Martens (6420323)}
\title{Vorlesung Geschichte Lateinamerikas}
\subtitle{Vorlesungsprotokoll vom 26. Oktober 2016}
\date{26. Oktober 2016}
\maketitle


\section*{Was ist das 20. Jahrhundert in Lateinamerika?}

Die historische Einteilung in Jahrhunderte ist oftmals nicht übereinstimmend
mit der kalendarischen. So wird in Bezug auf Europa häufig vom langen 19. Jahrhundert
und kurzen 20. Jahrhundert geredet, obwohl beide kalendarisch 100 Jahre lang sind.
Auf Lateinamerika bezogen ist die Abgrenzung zwischen dem 19. und 20. Jahrhundert
allerdings nicht so einfach, wie bei Europa, da sich die einzelnen Länder unterschiedlich
entwickelten.

Im Fall Brasilien endete die Sklaverei 1888 und die Republik besteht seit 1891.
Ein Schnittpunkt wäre somit 1891. In Argentinien gibt es erst seit 1912 das
allgemeine Männerwahlrecht und 1916 hat die Union Civica Radical gewonnen.
Demnach wäre 1912 bzw. 1916 ein Startpunkt für das 20. Jhd. In Mexiko begann
1910/11 die Revolution und mündete 1917 in einer Verfassung. Daher wäre 1917
ein möglicher Epochenwechsel für Mexiko. Daran wird deutlich, dass es keinen
einheitlichen Start gegeben hat. Das Ende kann dagegen klar bestimmt werden:
1989. Mit Beginn des Neoliberalismus endet eindeutig das 20. Jahrhundert.

Auch die Rolle der USA dient nicht als Indikator, denn bis 1945 hatte die USA
keinen nennenswerten Einfluss auf Lateinamerika und bis 1920 war noch Großbritannien
der wichtigste Handelspartner.

Daher gibt es eine alternative Gliederung für Lateinamerika bzgl. des 20. Jahrhunderts:

\begin{myitemize}
    \item ca. 1810/20 - ca. 1870/80 Konsolidierung der Nationalstaaten (Grenzen, politische
          Systeme, etc.) bei geringer Einwanderung
    \item ca. 1870/80 bis 1930 Blüte und Zerfall der alten Ordnung bei starker Einwanderung,
          Beginn der Industrialisierung durch Eisenbahnbau ab 1870
    \item ca. 1930-1989 Instabilität durch Modernisierung mit Beginn der Auswanderung,
          zweiter Schub der Industrialisierung von 1940 bis 1970
\end{myitemize}

Neben der zeitlichen Einordnung stellt sich auch die Frage, wie sich das Jahrhundert
charakterisiert. Das Zeitalter der Extreme passt auf Europa, aber nicht auf
Lateinamerika. Es war in Bezug auf Lateinamerika auch nicht das Jahrhundert der
USA. Vielleicht war es das Jahrhundert der Kinder und Frauen.

\end{document}