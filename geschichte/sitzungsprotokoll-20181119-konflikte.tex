\documentclass[10pt,a4paper,oneside,ngerman,numbers=noenddot]{scrartcl}
\usepackage[T1]{fontenc}
\usepackage[utf8]{inputenc}
\usepackage[ngerman]{babel}
\usepackage{amsmath}
\usepackage{amsfonts}
\usepackage{amssymb}
\usepackage{bytefield}
\usepackage{paralist}
\usepackage{gauss}
\usepackage{pgfplots}
\usepackage{textcomp}
\usepackage[locale=DE,exponent-product=\cdot,detect-all]{siunitx}
\usepackage{tikz}
\usepackage{algpseudocode}
\usepackage{algorithm}
\usepackage{mathtools}
\usepackage{hyperref}
\usepackage[german=quotes]{csquotes}
%\usepackage{algorithmic}
%\usepackage{minted}
\usetikzlibrary{automata,matrix,fadings,calc,positioning,decorations.pathreplacing,arrows,decorations.markings}
\usepackage{polynom}
\polyset{style=C, div=:,vars=x}
\pgfplotsset{compat=1.8}
\pagenumbering{arabic}
%\def\thesection{\arabic{section})}
%\def\thesubsection{(\alph{subsection})}
%\def\thesubsubsection{(\roman{subsubsection})}
\makeatletter
\renewcommand*\env@matrix[1][*\c@MaxMatrixCols c]{%
  \hskip -\arraycolsep
  \let\@ifnextchar\new@ifnextchar
  \array{#1}}
\makeatother
\parskip 12pt plus 1pt minus 1pt
\parindent 0pt

\DeclarePairedDelimiter\abs{\lvert}{\rvert}%
\DeclarePairedDelimiter{\ceil}{\lceil}{\rceil}

\newenvironment{myitemize}{\begin{itemize}\itemsep -9pt}{\end{itemize}} % Zeilenabstand in Aufzählungen geringer

%switch starred and non-starred (auto-size)
\makeatletter
\let\oldabs\abs
\def\abs{\@ifstar{\oldabs}{\oldabs*}}
\makeatother

\hypersetup{
    colorlinks,
    citecolor=black,
    filecolor=black,
    linkcolor=black,
    urlcolor=black
}

\MakeOuterQuote{"}

\begin{document}
\author{Jim Martens (6420323)}
\title{Vorlesung Religiöse Konflikte im Mittelalter}
\subtitle{Essay zu Mönchswesen und Klöstern}
\date{1. Dezember 2018}
\maketitle

\section{Konfliktlösung der Gegenwart}

Heutzutage werden unterschiedliche Meinungen in Demokratien gesetzlich geschützt.
Auch die öffentliche Verneinung der Existenz einer wie auch immer gestalteten
höheren Instanz ist möglich ohne Repression durch den Staat zu erfahren. Die
institutionelle Repression von Religionen beschränkt sich auf rein religiöse
Repressionen für Menschen, welche dieser Religion angehören. Der Papst kann
weiterhin Katholiken exkommunizieren und sie von religiösen Handlungen ausschließen.
Allerdings hat diese Exkommunikation keinerlei weltliche Bewandnis mehr und
exkommunizierte Menschen könnten in der Theorie problemlos staatliche Ämter
bekleiden.
Der Prozess für Streitigkeiten ist ebenfalls klar geregelt in Form einer
Zivil- und Strafprozessordnung im Rahmen einer unabhängigen Justiz. Ebenso gelten
Menschenrechte auch in Klöstern und Misshandlungen auch im Rahmen religiöser Kontexte
werden vor weltlichen Gerichten verhandelt.
Der blutige Konflikt verschiedener Konfessionen innerhalb des Christentums ist
ebenso beigelegt und es gibt regelmäßig ökumenische Gottesdienste.

Selbstredend beziehe ich mich hier auf eine rechtsstaatliche Demokratie wie
Deutschland. Auch ist dies der Idealfall. Die Missbrauchsskandale zeigen,
dass wir auch hier von einem Idealzustand noch entfernt sind und die Positionen
der katholischen Kirche zu Abtreibung und Homosexualität stellen einen modernen
Konflikt dar zwischen weltlichem Verfassungsrecht wie den Menschenrechten
auf der einen Seite und religiösen Überzeugungen auf der anderen Seite.

Eine Auflösung dieses Konfliktes ist bislang nicht absehbar. Aufgrund von
weiterhin bestehender Einwirkung gerade der katholischen Positionen auf die
Politik hat diese Position auch einen Einfluss auf Menschen, welche nicht
der katholischen Kirche angehören. Selbst der "Reform-Papst" Franziskus vermag
dies nicht zu lösen. Denn hier geht es um die grundlegenden Lehren der Kirche,
welche wenn überhaupt auf den Synoden/Konzilen überarbeitet werden können. Diese
sehen aber (meines Wissens nach) bis heute keinen geregelten Prozess zur
Überwindung von Konflikten vor. Im Zweifel wird es immer einige machtvolle
Personen in der Kirche gegen Abtreibung und Homosexualität geben. In diesen
Fällen ist ein Konsens also schwer möglich.

\section{Mönchskonflikte}

Asketische Lebensformen gab es bereits in der Antike. Ein Mönchtum entwickelte
sich erst im 4. Jahrhundert. Dabei zogen sich Menschen bspw. in die Wüste zurück,
um ein gottgefälliges Leben zu führen. Diese Art des Mönchtums war sehr
egoistisch geprägt und es ging weniger darum für das Wohl anderer Menschen
zu sorgen. Die monastische Lebensform begeisterte viele Leutes, sodass
sich Einsiedlerkolonien bildeten. Aus diesen Kolonien entstand die Notwendigkeit
einer Führungsperson. Mangels einer wirklichen beschlossenen Ordnung des
Zusammenlebens wird dies mit Sicherheit zu dem einen oder anderen Konflikt
geführt haben.

Im weiteren Verlauf entstanden dann Klöster, bei denen es vielmehr um ein
kollektives Leben geht. Dort gab es dann eine klare Hierarchie mit dem Abt bzw.
der Äbtissin, welche nahezu absolute Kontrolle über die Mönche in ihrer Aufsicht
hatten. Die Mönche dort befanden sich fast immer in der Gemeinschaft, hatten
einen strikten Zeitplan und es muss eine nahezu militärische Disziplin vorgeherrscht
haben.

Beide Formen des Mönchtums begannen in Ägypten noch zu Zeit der Herrschaft des
(west)römischen Reiches. Das christliche Mönchtum breitete sich fortan
im Herrschaftsbereich des römischen Reiches aus. Es sind ebenso römische Berichte
zu finden, welche abfällig über Einsiedler berichten.

Im Widerspruch zu diesen Mönchsformen bildete sich auch eine Art Salonaskese,
wo zwar auf Wiederheirat und Unterhaltung verzichtet wird, ansonsten aber ein
annehmbarer Lebensstil geführt wird.

Die Klöster bestanden weitgehend autark und das inkludiert die kirchlichen Hierarchien.
Diese Unabhängigkeit hat nicht selten zu Konflikten mit den religiösen Autoritäten
geführt. Zu dieser Zeit gibt es noch keine Ordnen, nur einzelne Klöster und Klöstergruppen.
Regeltexte gab es ebenfalls noch nicht und Anweisungen wurden einzeln gegeben.
Idealerweise richteten sich die Anweisungen dabei an den individuellen Fähigkeiten
aus.

Mit der Zeit wurde dann die Benediktregel entworfen. Diese besagt, dass bei
Fehlverhalten der Abt entscheiden müsse, wie interveniert wird. Dies könne
bspw. das Überzeugen der fehlerhaft handelnden Person sein, aber auch die Bestrafung
dieser. Der Regeltext ist allerdings so allgemein gehalten, dass die tatsächlichen
Abläufe im Kloster sich daraus nicht ergeben.

Es ist jedoch sehr klar, dass der Abt die absolute Schlüsselstellung inne hat.
Ihm ist die Fürsorge der Untergebenen zugeteilt, ebenso eine hohe Straf-
und Normierungsgewalt. Er ist zugleich Stellvertreter Christi bzw. Gottes,
übt ein Regime von Überwachung und Strafen aus und soll regelmäßig überprüfen,
dass Mönche kein Privateigentum haben, außer Dinge, welche ihnen vom Abt
gegeben wurden. Aus dieser starken Machtposition ergeben sich einige mögliche
Konflikte.

Zwar solle ein Abt vor einer Entscheidung alle anhören, die Entscheidung
trifft aber allein der Abt und alle müssen sich unbedingt daran halten. Es ist
also gar nicht vorgesehen, dass in einer Art Prozess Meinungsverschiedenheiten
beigelegt werden.
Vielmehr gibt es nach Auffassung der Benediktregel gar keine Konflikte,
sondern nur Fehlverhalten, welches dann zu sanktionieren sei. Mögliche Sanktionen
sind bspw. Ausschluss vom Gebet oder vom Essen. Die sanktionierten Personen
sollen sich regelmäßig dem Abt zu Fuße werfen und um Gnade bitten.

Weitere Konflikte entstehen dadurch, dass häufig die nicht-erstgeborenen
Kinder von Adligen in Klöster gegeben werden, da sie nichts erben und sonst
nur ein Machtrisiko für die erbenden Geschwister wären. Diese Übergabe
an Klöster kann aber gegen den Willen der Kinder erfolgen. Kann jemand gezwungen
werden Mönch zu bleiben, auch wenn keine aktive eigene Entscheidung für das
Mönchtum vorlag? Ein gewisser Gottschalk ist aus einem Kloster abgehauen mit
der Argumentation, dass er sich nie selbst für das Kloster entschieden habe.
Der Abt war anderer Ansicht und ließ Gottschalk durch halb Europa verfolgen.

Außerdem stellt sich natürlich die Frage, wie ein Abt zu dieser Position kommt.
Dafür ist kein klares Verfahren geregelt. Er solle einmütig von allen gewählt werden,
von einer klügeren Minderheit oder von außerhalb ausgewählt werden. Daraus
ergeben sich zwangsläufig Konflikte, gerade vor dem Hintergrund, dass Äbte auf
Lebenszeit gewählt werden.

\section{Konflikte im Kontext}

Im größeren Kontext der Entwicklungen des Mittelalters spielen Klöster
und somit ihre Konflikte eine zentrale Rolle. Klöster haben Literatur
gehütet und verbreitet, waren oftmals die einzigen Orte um das Lesen- und Schreiben
zu lernen und konnten als Kollektiv große Reichtümer anhäufen. Mit der Zeit
wurden viele Klöster zu Dienstleistern für professionelles Gebet und somit Ziel
von großzügigen Spenden.

Aufgrund von Unzufriedenheit mit dem Reichtum der Klöster gründeten sich dann
auch Orden, die auch für die Gemeinschaft Armut vorschrieben, die sog. Bettelorden.
Aufgrund ihrer Autarkie konnten diese Orden auch für die Kirche gefährlich werden,
denn qua Definition lebten Mönche das gottgefälligste Leben. Starke Bettelorden
widersprachen somit klar und deutlich der gelebten Praxis auch in der katholischen
Kirche, welche unzählige Reichtümer anhäufte, jedoch für sich in Anspruch nahm
für Gott zu sprechen. Wer war glaubwürdiger? Ein reicher Kardinal mit Ländereien
und mächtiger Familie im Hintergrund (z.B. de Medici) oder ein armer Mönch?

Diese Situation führte dann auch zu teilweisen Verboten von Bettelorden. Ebenso
wurden diese aber auch zeitweise als Machtmittel des Papstes gegen Ortsbischöfe
eingesetzt. Anders als die früheren Ordensgemeinschaften ließen sich Bettelorden
bevorzugt in Städten nieder und ihre Mitglieder übten dort Priester- und Lehrtätigkeiten
aus.

\end{document}
