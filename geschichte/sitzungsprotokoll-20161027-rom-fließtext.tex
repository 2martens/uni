\documentclass[10pt,a4paper,oneside,ngerman,numbers=noenddot]{scrartcl}
\usepackage[T1]{fontenc}
\usepackage[utf8]{inputenc}
\usepackage[ngerman]{babel}
\usepackage{amsmath}
\usepackage{amsfonts}
\usepackage{amssymb}
\usepackage{bytefield}
\usepackage{paralist}
\usepackage{gauss}
\usepackage{pgfplots}
\usepackage{textcomp}
\usepackage[locale=DE,exponent-product=\cdot,detect-all]{siunitx}
\usepackage{tikz}
\usepackage{algpseudocode}
\usepackage{algorithm}
\usepackage{mathtools}
\usepackage{hyperref}
\usepackage[german=quotes]{csquotes}
%\usepackage{algorithmic}
%\usepackage{minted}
\usetikzlibrary{automata,matrix,fadings,calc,positioning,decorations.pathreplacing,arrows,decorations.markings}
\usepackage{polynom}
\polyset{style=C, div=:,vars=x}
\pgfplotsset{compat=1.8}
\pagenumbering{arabic}
%\def\thesection{\arabic{section})}
%\def\thesubsection{(\alph{subsection})}
%\def\thesubsubsection{(\roman{subsubsection})}
\makeatletter
\renewcommand*\env@matrix[1][*\c@MaxMatrixCols c]{%
  \hskip -\arraycolsep
  \let\@ifnextchar\new@ifnextchar
  \array{#1}}
\makeatother
\parskip 12pt plus 1pt minus 1pt
\parindent 0pt
\MakeOuterQuote{"}

\DeclarePairedDelimiter\abs{\lvert}{\rvert}%
\DeclarePairedDelimiter{\ceil}{\lceil}{\rceil}

\newenvironment{myitemize}{\begin{itemize}\itemsep -9pt}{\end{itemize}} % Zeilenabstand in Aufzählungen geringer

%switch starred and non-starred (auto-size)
\makeatletter
\let\oldabs\abs
\def\abs{\@ifstar{\oldabs}{\oldabs*}}
\makeatother

\hypersetup{
    colorlinks,
    citecolor=black,
    filecolor=black,
    linkcolor=black,
    urlcolor=black
}

\begin{document}
\author{Jim Martens (6420323)}
\title{Vorlesung Römische Geschichte 1: Römische Republik}
\subtitle{Vorlesungsprotokoll vom 27. Oktober 2016}
\date{27. Oktober 2016}
\maketitle

\section*{Quellen}

Ein wichtiger Bereich für die Geschichtsforschung sind immer die Quellen. Für den Zeitraum der Römischen Republik gibt
es etliche Quellen. Allgemein können sie in literarische und nicht-literarische Quellen aufgeteilt werden. Zu den nicht-literarischen
Quellen gehören beispielsweise archäologische Funde. Im Folgenden wird der Fokus jedoch auf den literarischen Quellen liegen.

Einige der bekanntesten Schriftstücke für den Zeitraum stammen von Cicero. Es gibt jedoch viel mehr Autoren und Schriftstücke.
Zu Beginn der Republik war Latein noch nicht als Literatursprache geeignet, weswegen auch die römische Geschichtsschreibung in
Griechisch erfolgte. Aus der Zeit stammen griechische Fragmente, die Aufschluss über die frühe Geschichte Roms geben können.
Erst später fangen die Römer an aus dem Griechischen ins Lateinische zu übersetzen. Livius Antonius beispielsweise übersetzt
die Griechische Odyssee ins Lateinische.

In der römischen Geschichtsschreibung ist besonders die Annalistik interessiert. Dort wird nach den Konsuln datiert. Dies geschieht,
indem das Jahr und die beiden gewählten Konsuln genannt werden und anschließend nur noch auf die Konsuln referenziert wird.
Dies kann bspw. so lauten (frei erfunden): "Unter Konsul Antonius haben wir eine Reise nach Athen gemacht. Später wurde Athen unter
Konsul Julius Teil der römischen Einflusssphäre". Mithilfe der sog. Fasten konnten auch normale Bürger Roms eine Form von
Geschichtsschreibung tun. Auf den Plätzen standen diese Fasten mit den aktuellen Konsuln und jeder konnte Ereignisse dort notieren.

Weitere literarische Werke sind jene von Livius (nicht Livius Antonius), Plutarch, Plautus und Terenz. Livius lebte erst in augustinischer
Zeit und berichtet somit rückblickend über die Republik. Aufgrund seines schönen Schreibstils stellte er alle vorigen Quellen bzgl.
der Frühgeschichte Roms in den Schatten. Er entwarf im Prinzip eine "Mastergeschichte" des frühen Roms. Plautus und Terenz sind
römische Komödienautoren. Dies macht sie interessant, da Komödien eine wichtige Quelle für die Sozialstruktur in Rom sind. Die Werke
von Naevius und Ennius sind ebenfalls zu erwähnen. Sie sind Epiker bzw. "Nationaldichter".


\section*{Besiedlung Italiens}

Italien wurde vor der Eroberung durch Rom von verschiedenen Gruppen bevölkert. Die Etrusker sind bspw. ab dem 7. Jahrhundert v. Chr.
in Italien greifbar und sie sind eine vorindoeuropäische Gruppe. Ihr Kerngebiet erstreckte sich von Rom im Süden bis zum Fluss Arno im Norden,
sowie von der westlichen Mittelmeerküste bis zum Appennin-Gebirge. Bereits ab 1000 v. Chr. sind indogermanische bzw. indoeuropäische
Gruppen in Italien belegt.
Die Kelten siedelten im Norden Italiens und die Griechen im Süden. Im Nordosten (beim heutigen Venedig) waren die Veneter und Illyrer
ansässig. Letztere siedelten auch an der Küste des heutigen Balkans.

Die Griechen siedelten in Kolonien bzw. Städten, die jeweils Ackerbau und Handel betrieben. Mit den Griechen und der Idee der Polis
kam auch die Stadtkultur nach Italien. Es gab jedoch kein griechisches Reich. Der Siedlungsbereich der Griechen wurde zwar Magna Graecia
(Großgriechenland) genannt, meinte jedoch ebenfalls kein Reich. Eine solche Reichsgründung gab es erst unter den Hellenen (Alexander der
Große). Die einzelnen Städte befanden sich eher untereinander in Konkurrenz und hatten häufiger Krieg miteinander als zusammen gegen
äußere Bedrohungen. Neben dem südlichen Italien war auch Sizilien für die Griechen von Wichtigkeit. Dort stoßen sie mit den Kathargern
zusammen, die im Westen der Insel siedelten, was später zu Konflikten führte.

\subsection*{Etrusker}

Von diesen Gruppen sind die Etrusker besonders interessant, da sie den größten Kontakt mit den Römern hatten und vieles der etruskischen
Kultur in Rom eine Fortführung erfuhr. Die Etrusker betrieben Erzbergbau und Handel. Ihre wichtigen Städte waren Capua, Palestrina und Veji.
Im 6. Jhd. expandierten die Etrusker nach Norden und Süden (Latium). Sie gehen auch ein Bündnis mit den Kathargern gegen Griechenland ein.
Dieser Konflikt mit den Griechen schwächt die Etrusker und führt zum Machtverlust des etruskischen Adels in Rom, sodass Rom schließlich 
die Chance nutzt und sich von den Etruskern löst. Dabei zieht sich der Zerfall der Etrusker vom 5. Jhd. bis zum 3. Jhd. v.Chr. 
Insbesondere Angriffe Roms auf Veji, dem religiösen Zentrum der Etrusker, und ein Galliereinfall, der etruskische Kulturgegenstände zerstört, 
setzen den Etruskern schwer zu.

Frauen besaßen bei den Etruskern ein hohes Sozialprestige. Es stellt sich aber die Frage, inwieweit dieses Prestige mit der heutigen Stellung
von Frauen vergleichbar ist und woran die Höhe des Prestiges gemessen wird. Die vorindoeuropäische Sprache der Etrusker ist nicht verständlich.
Interessant ist vor diesem Hintergrund aber, dass etliche Fresken und Bilder der Etrusker griechische Formensprache benutzten.

\subsection*{Gründung Roms}

Zur Gründung Roms gibt es unterschiedliche Theorien. Die Hauptfrage ist, ob Rom allmählich gegründet wurde oder es eine feste Gründung 
(Synoikismos) gab. Mit Sicherheit kann gesagt werden, dass 753 v. Chr. nicht das Gründungsjahr von Rom war. Falls eine feste Gründung
überhaupt stattfand, so geschah das erst gegen 600 v.Chr. Die Position von Rom ist strategisch sehr günstig gelegen. So führt die Salzstraße
direkt an Rom vorbei. Trotz der späten Stadtgründung von Rom gibt es bereits im 9. Jhd. v. Chr. deutliche Siedlungsspuren auf dem Hügel Palatin.
Archäologische Funde gab es am Forum Buarium, am Südhang des Kapitols und am Forum Romanium.

\section*{Mythengeschichte}

Neben der tatsächlichen Geschichte Roms gibt es auch etliche Mythen, die bereits zu damaliger Zeit als solche erkennbar waren. Die Römer sollen demnach
auf die Trojaner zurückgehen. Die Stadt Rom wird auf den Kriegsgott Mars zurückgeführt. Caesar und Augustus leiten ihren Stammbaum auf die 
Göttin Venus zurück.

Das Jahr 753 v.Chr. wird später als das Jahr der Himmelfahrt von Romulus gesetzt und damit der mythologische Startschuss für die Gründung Roms.

\end{document}