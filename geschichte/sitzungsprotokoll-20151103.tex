\documentclass[10pt,a4paper,oneside,ngerman,numbers=noenddot]{scrartcl}
\usepackage[T1]{fontenc}
\usepackage[utf8]{inputenc}
\usepackage[ngerman]{babel}
\usepackage{amsmath}
\usepackage{amsfonts}
\usepackage{amssymb}
\usepackage{bytefield}
\usepackage{paralist}
\usepackage{gauss}
\usepackage{pgfplots}
\usepackage{textcomp}
\usepackage[locale=DE,exponent-product=\cdot,detect-all]{siunitx}
\usepackage{tikz}
\usepackage{algpseudocode}
\usepackage{algorithm}
\usepackage{mathtools}
\usepackage{hyperref}
%\usepackage{algorithmic}
%\usepackage{minted}
\usetikzlibrary{automata,matrix,fadings,calc,positioning,decorations.pathreplacing,arrows,decorations.markings}
\usepackage{polynom}
\polyset{style=C, div=:,vars=x}
\pgfplotsset{compat=1.8}
\pagenumbering{arabic}
%\def\thesection{\arabic{section})}
%\def\thesubsection{(\alph{subsection})}
%\def\thesubsubsection{(\roman{subsubsection})}
\makeatletter
\renewcommand*\env@matrix[1][*\c@MaxMatrixCols c]{%
  \hskip -\arraycolsep
  \let\@ifnextchar\new@ifnextchar
  \array{#1}}
\makeatother
\parskip 12pt plus 1pt minus 1pt
\parindent 0pt

\DeclarePairedDelimiter\abs{\lvert}{\rvert}%
\DeclarePairedDelimiter{\ceil}{\lceil}{\rceil}

\newenvironment{myitemize}{\begin{itemize}\itemsep -2pt}{\end{itemize}} % Zeilenabstand in Aufzählungen geringer

%switch starred and non-starred (auto-size)
\makeatletter
\let\oldabs\abs
\def\abs{\@ifstar{\oldabs}{\oldabs*}}
\makeatother

\hypersetup{
    colorlinks,
    citecolor=black,
    filecolor=black,
    linkcolor=black,
    urlcolor=black
}

\begin{document}
\author{Jim Martens (6420323)}
\title{England im Spätmittelalter}
\subtitle{Vorlesungsprotokoll vom 3. November 2015}
\date{3. November 2015}
\maketitle

\section*{Regentschaft von Isabella und Mortimer}

\begin{myitemize}
	\item Eduard III. wurde 1327 König (war noch minderjährig) \(\rightarrow\) Regentschaft durch Mutter Isabella und Mortimer
	\item nutzten Stellung, um sich zu bereichern
	\item Kritik an Mortimer aufgrund von Außenpolitik zu Schottland
	\item schlechter Kriegsverlauf zwang Mortimer dazu viele Zugeständnisse an Schottland zu machen
	\item 1328 Friedensvertrag von Northampton: garantierte Schottlands Unabhängigkeit von England
	\item Friedensvertrag wurde in England als Schwäche angesehen
	\item anwachsen von Widerstand, welcher im Oktober 1330 unter Beteiligung Eduard III. die Regentschaft stürzt
	\item im einberufenen Parlament wurde Mortimer wegen notorischem Hofverrates zum Tode verurteilt
	\item Isabella durfte sich nicht mehr politisch beteiligen, ansonsten jedoch keine wirkliche Strafe
	\item Ratgeber der Regentschaft fielen nicht in Ungnade
	\item Übergang zwischen Regentschaft und Herrschaft Eduard III. dadurch relativ problemlos
\end{myitemize}

\section*{Anfänge der Regierung Eduard III. und Entwicklung innerer Verhältnisse}

\begin{myitemize}
	\item Sturz der Regentschaft landesweit begrüßt
	\item Eduard III. wollte Verhältnis zu Baronen verbessern
	\item 1333 erfolgreicher Feldzug gegen Schottland, dabei erstmals Einsatz der berühmten Langbögen
	\item Konflikte mit Frankreich nehmen zu und führen zum Hundertjährigen Krieg
	\item zunehmend schwierigere finanzielle Lage
	\item mehrmals Auseinandersetzungen mit Parlament über Steuern und Abgaben
	\item 1376 "`gutes Parlament"' mit vielen Zugeständnissen an Commons
	\item Vorgang des Impeachment eingeführt
	\item nach Abschluss des Parlaments wird vorheriger Status weitestgehend wiederhergestellt
	\item Eduard III. stirbt kurz darauf
\end{myitemize}

\section*{1. Phase des Hundertjährigen Krieges bis Friedensverhandlungen unter Richard II.}
\begin{myitemize}
	\item 1337 öffentliche Erklärung des Anspruchs auf französische Krone
	\item 1337-1341 Eduard III. baut auf Allianzen mit Flandern und deutschen Prinzen
	\item 1342-1360 erfolgreich durch kleinere Feldzüge
	\item 1361-1372 Ausweitung des Krieges, Stellvertreterkriege
	\item 1340 Erfolg von Sluys, französische Flotte geschlagen
	\item 1342 Kämpfe um Bretagne
	\item 1343 Waffenstillstand
	\item 1345/46 erneuter Angriff auf Frankreich geplant 
	\item einige Erfolge im Westen (Bretagne, Gascogne), Landung in Normandie statt Flandern
	\item unerwartet für Frankreich, sodass Eduard bis kurz vor Paris ziehen kann
	\item deutlicher Sieg gegen Frankreich in einer Schlacht
	\item Eduard III. marschiert nach Calais, welches 1347 kapituliert
	\item schwarzer Prinz gewinnt Schlacht zahlenmäßig unterlegen \(\rightarrow\) Johann II. wird gefangen genommen
	\item Oktober 1360 Frieden, Eduard III. verzichtet auf Anspruch und bekommt dafür Unabhängigkeit für Aquitanien
	\item zwischenzeitlich Stellvertreterkriege in Kastilien, sowie Kreuzzug gegen Frankreich
	\item 1390 28-jähriger Waffenstillstand
\end{myitemize}

\section*{Innere Geschichte Englands unter Richard II.}

\begin{myitemize}
	\item 1381 Bauernaufstand
	\item Resultat: höheres Selbstbewusstsein Richard II., Kopfsteuer rotes Tuch in englischer Politik bis ins 20. Jhd.
	\item Geschichtsbild verzerrt durch Lancaster-Geschichtsschreibung
	\item 1388 "`gnadenloses Parlament"'
	\item neues Verfahren: Appeal
	\item 1397/98 stärkere Position Richard II. als vor Problemen mit Parlament
	\item Heinrich v. Lancaster rebelliert und Richard II. wird abgesetzt
	\item Heinrichs Einsetzung nicht unumstritten
	\item einige sehen ihn als Ursupator
	\item weitere Person mit Thronanspruch aus Haus York
	\item wird später zu Rosenkriegen führen
\end{myitemize}

\end{document}