\documentclass[10pt,a4paper,oneside,ngerman,numbers=noenddot]{scrartcl}
\usepackage[T1]{fontenc}
\usepackage[utf8]{inputenc}
\usepackage[ngerman]{babel}
\usepackage{amsmath}
\usepackage{amsfonts}
\usepackage{amssymb}
\usepackage{bytefield}
\usepackage{paralist}
\usepackage{gauss}
\usepackage{pgfplots}
\usepackage{textcomp}
\usepackage[locale=DE,exponent-product=\cdot,detect-all]{siunitx}
\usepackage{tikz}
\usepackage{algpseudocode}
\usepackage{algorithm}
\usepackage{mathtools}
\usepackage{hyperref}
\usepackage[german=quotes]{csquotes}
%\usepackage{algorithmic}
%\usepackage{minted}
\usetikzlibrary{automata,matrix,fadings,calc,positioning,decorations.pathreplacing,arrows,decorations.markings}
\usepackage{polynom}
\polyset{style=C, div=:,vars=x}
\pgfplotsset{compat=1.8}
\pagenumbering{arabic}
%\def\thesection{\arabic{section})}
%\def\thesubsection{(\alph{subsection})}
%\def\thesubsubsection{(\roman{subsubsection})}
\makeatletter
\renewcommand*\env@matrix[1][*\c@MaxMatrixCols c]{%
  \hskip -\arraycolsep
  \let\@ifnextchar\new@ifnextchar
  \array{#1}}
\makeatother
\parskip 12pt plus 1pt minus 1pt
\parindent 0pt

\DeclarePairedDelimiter\abs{\lvert}{\rvert}%
\DeclarePairedDelimiter{\ceil}{\lceil}{\rceil}

\newenvironment{myitemize}{\begin{itemize}\itemsep -9pt}{\end{itemize}} % Zeilenabstand in Aufzählungen geringer

%switch starred and non-starred (auto-size)
\makeatletter
\let\oldabs\abs
\def\abs{\@ifstar{\oldabs}{\oldabs*}}
\makeatother

\hypersetup{
    colorlinks,
    citecolor=black,
    filecolor=black,
    linkcolor=black,
    urlcolor=black
}

\MakeOuterQuote{"}

\begin{document}
\author{Jim Martens (6420323)}
\title{Vorlesung Spätantike}
\subtitle{Vorlesungsprotokoll vom 09. November 2017}
\date{09. November 2017}
\maketitle

\section*{Julian}

Julian wird 331 in Konstantinopel als Flavius Claudius Julianus geboren.
Als 6-Jähriger wurde er auf ein Landgut verbracht und dort unter Hausarrest
gestellt. Während des Hausarrests hatte er griechische Lehrer und ist somit
nicht nur mit Griechisch aufgewachsen, sondern wurde auch ein großer Fan der
alten Griechen. Julian betrieb intensive Studien über Nauplatoniker, Bibel und
das Judentum. Im Jahr 351 wurde der Bruder von Julian zum Caesar im Osten.
Dies bedeutete mehr Bewegungsfreiheit für Julian. Diese wurde von Julian für
ein vierjähriges Rhetorikstudium genutzt. Er wurde auch in Athen in die Mythologie
eingeweiht und hatte Kontakt mit Basilius dem Großen.

Im Jahre 355 wurde Julian Caesar für Gallien und bekam einen Heermeister zur
Seite gestellt. Die Franken hatten Köln eingenommen. Insgesamt waren 40 Städte
gefallen. Julian erobert Köln zurück und bezwingt die Allemannen und Saalfranken.
Die Franken werden nach Toxandrien zurückgedrängt, das heutige Belgien. Ebenso
wird die Schiffsverbindung zwischen Britannien und dem Rhein wieder gesichert
und die zerstörten Städte werden wieder aufgebaut. Julian geht auch gegen die
Korruption vor. Durch diese Maßnahmen gelingt ihm eine musterhafte Verwaltung von
Gallien.

Constantius fordert Truppen für den Osten an. Doch Julian schickt nur einen Teil
der Truppen. Denn ein gewichtiger Teil der gallischen Truppen besteht aus Föderaten,
so genannten Wehrbauern, die lokal gebunden sind und nicht nach Persien möchten.
Von seinen Truppen wird Julian im Jahr 360 zum Augustus ernannt. Um einem Bürgerkrieg
aus dem Weg zu gehen, bittet er Constantius diplomatisch um den Verzicht auf
Truppen. Dieser Verzicht wird verweigert. Es droht ein Bürgerkrieg. Kurz darauf
stirbt jedoch Constantius und benennt kurz vor seinem Tod Julian als seinen
Nachfolger. Die Machtübergabe erfolgt problemlos und Julian wird von allen
Stellen anerkannt.

Als Zeichen der Versöhnung wird der Vetter von Constantius konsekriert. Ansonsten
beginnt Julian umgehend damit seine Reformvorhaben umzusetzen. Der Hof wird
verkleinert, eine traditionsreiche Polizeitruppe wird aufgelöst und Julian gibt
sich als Bürgerkaiser - als Primus inter pares. Er möchte nach Leistung befördern,
senkt Steuern, füllt die Stadträte wieder auf und gibt Tempelland an die Städte
zurück. Den Städten werden die Schulden bzw. ein Teil dieser erlassen. Konstantinopel
wird ausgebaut und eine große Bibliothek angelegt.

Religiös wendet sich Julian dem alten Glauben zu. Er lässt sich einen Philosophen-Bart
wachsen, erlaubt 361 wieder die Verehrung der alten Götter und besetzt alte Priesterstellen
erneut. Er gibt die Güter den Tempeln zurück, die vorher von den Christen entwendet
wurden. Die Privilegien des Klerus werden abgeschafft. Insbesondere haben Bischöfe
keine Rechte im staatlichen Gerichtswesen mehr. Das Ziel ist die Schwächung
der Amtskirche und eine Zersplitterung der Kirche. Nichtsdestotrotz ist Julian
tolerant gegenüber Christen und verfolgt diese nicht. Er schützt sogar die bisher
unterdrückten christlichen Sonderkirchen. Diese Toleranz wird aber nicht von allen
Heiden geteilt. Es werden teils grausame Racheakte begangen.
Im Jahr 362 erlässt Julian das umstrittene Rhetorenedikt. Es verlangt, dass Lehrer
nicht nur eine gute Lehrqualität, sondern auch charakterliche Integrität haben
müssen.

Julian erklärt sich die Missionserfolge der Christen mit dem sozialen
Versagen der Heiden. Für ihn sind die Götter die Naturkräfte. Die Naturwissenschaften
lehnt er indes ab, da diese ihm zu viel erklären. Er verehrt auch den Sonnengott.
Die übrigen Götter sind für ihn Ausflüsse des Sonnengottes. Julian sieht wenig
Unterschied zwischen Neuplatonikern und den Christen. Er liebt die neuplatonische
Theologie und unterscheidet nicht zwischen Philosophie und Theologie. Beides führt
für ihn zu einem moralisch integrem Leben. Das höchste Fest ist für ihn das
Weihnachtsfest.

Im Jahr 362 bricht Julian nach Persien auf, um das zu beenden, was Constantius
nicht mehr schaffte. In Folge der Kämpfe erleidet er jedoch Verletzungen, die
letztlich 363 tödlich enden. Er wird in Tarsus beigesetzt. Die Heiden waren tief
bestürzt und Julian wird vom Senat umgehend konsekriert. Seine Schriftwerke werden
gesammelt. Aufgrund seiner starken literarischen Tätigkeit ist er eine der
interessantesten Gestalten der Spätantike. Er wurde auch zum Beginn der Faustgestalt
und war später Liebling der Aufklärer.

Die neuere Forschung bewertet ihn als Mensch sehr gut. Seine Politik als
Kaiser im Hinblick auf die Religiösität wird aber als aussichtslos beurteilt.

\section*{Valens und Valentinian}

Nach dem Tode Julians wird für kurze Zeit Jovian Kaiser. Er vereinbart einen
30-jährigen Frieden mit den Persern und tritt dafür Nordmesopotamien mit 15
Festungen und Armenien an die Perser ab. Er erneuerte sofort die Rechte der
Christen und kehrte Julians Politik weitgehend um. Jovian stirbt 364.

Valentinian und Valens werden zu Augusti ernannt. Valentinian verwaltet dabei
den Westen und Valens den Osten. Die Stadt Naissus repräsentierte dabei die
Grenze der Reichshälften. Alle offiziellen Gesetze wurden im Namen beider
Kaiser ausgestellt. Valentinian regierte von Mailand aus, Valens in Konstantinopel.

Im Jahr 366 gab es einen verlustreichen Sieg der Römer gegen die Allemannen.
Valentinian unternimmt einen Rachefeldzug gegen die Allemannen und schließt
diesen mit einem Frieden ab. Ebenso wird ein Pakt mit Burgund geschlossen.
Die Franken und Sachsen bleiben hingegen ein Problem. Die Pikten, Sachsen und
weitere greifen in Britannien von Norden an. In Folge wird der Hadrianswall
noch einmal ausgebaut und eine neue Provinz eingerichtet.

Es gibt Zaubereiprozesse gegen Senatoren in Rom. Valentinian hat die Stadt
Rom jedoch nie betreten. Die Heermeister werden befördert und den Prätorianerpräfekten
gleichgestellt. Valentinian geht gegen die Korruption vor. Den Tempeln werden
wieder die Tempelgüter weggenommen. Schließlich stirbt Valentinian an einem
Wutanfall.

Valens befestigt die Donaugrenze und schickt Truppen zur Donau. Diese rufen
einen Gegenkaiser aus, der aber schnell besiegt werden kann. Die Goten wurden
vom Gegenkaiser zu Hilfe gerufen, können aber zurückgeschlagen werden. Valens
unternimmt einen Gegenschlag und überschreitet die Donau. Die Goten spalten sich.
Wulfius wird als Bischof zu den Goten geschickt.

Die Situation der Heiden ist unter Valens relativ gut. Er schließt einen Frieden
mit den Katholiken. Es gibt jedoch eine Menge an Majestätsprozessen. Askese wird
erstmals als Lifestyle gefeiert und das Mönchsleben erstarkt.

Die Goten waren seit dem 3. Jahrhundert am Schwarzen Meer. Die Hunnen greifen
die Ostgoten an. Diese werden Untertanen der Hunnen. Westgoten erleiden eine große
Niederlage. Ein großer Teil geht an die Donau und bittet um Aufnahme ins Reich.
Gotische Boten kommen zu Valens. Dieser ordnet eine Aufnahme der Westgoten an.
Die Ostgoten schließen sich jedoch an. Es kommt zu Versorgungsproblemen an der
Donau. Den Goten wird daraufhin der Zugang zum Markt verweigert. Die Folge
ist ein schicksalsträchtiger Krieg an dessem Ende die gesamte Ostarmee aufgerieben
sein wird.


\end{document}
