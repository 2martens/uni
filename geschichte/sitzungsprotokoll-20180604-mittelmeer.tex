\documentclass[10pt,a4paper,oneside,ngerman,numbers=noenddot]{scrartcl}
\usepackage[T1]{fontenc}
\usepackage[utf8]{inputenc}
\usepackage[ngerman]{babel}
\usepackage{amsmath}
\usepackage{amsfonts}
\usepackage{amssymb}
\usepackage{bytefield}
\usepackage{paralist}
\usepackage{gauss}
\usepackage{pgfplots}
\usepackage{textcomp}
\usepackage[locale=DE,exponent-product=\cdot,detect-all]{siunitx}
\usepackage{tikz}
\usepackage{algpseudocode}
\usepackage{algorithm}
\usepackage{mathtools}
\usepackage{hyperref}
\usepackage[german=quotes]{csquotes}
%\usepackage{algorithmic}
%\usepackage{minted}
\usetikzlibrary{automata,matrix,fadings,calc,positioning,decorations.pathreplacing,arrows,decorations.markings}
\usepackage{polynom}
\polyset{style=C, div=:,vars=x}
\pgfplotsset{compat=1.8}
\pagenumbering{arabic}
%\def\thesection{\arabic{section})}
%\def\thesubsection{(\alph{subsection})}
%\def\thesubsubsection{(\roman{subsubsection})}
\makeatletter
\renewcommand*\env@matrix[1][*\c@MaxMatrixCols c]{%
  \hskip -\arraycolsep
  \let\@ifnextchar\new@ifnextchar
  \array{#1}}
\makeatother
\parskip 12pt plus 1pt minus 1pt
\parindent 0pt

\DeclarePairedDelimiter\abs{\lvert}{\rvert}%
\DeclarePairedDelimiter{\ceil}{\lceil}{\rceil}

\newenvironment{myitemize}{\begin{itemize}\itemsep -9pt}{\end{itemize}} % Zeilenabstand in Aufzählungen geringer

%switch starred and non-starred (auto-size)
\makeatletter
\let\oldabs\abs
\def\abs{\@ifstar{\oldabs}{\oldabs*}}
\makeatother

\hypersetup{
    colorlinks,
    citecolor=black,
    filecolor=black,
    linkcolor=black,
    urlcolor=black
}

\MakeOuterQuote{"}

\begin{document}
\author{Jim Martens (6420323)}
\title{Vorlesung Das Mittelmeer: Ein Binnenmer als globale Drehscheibe im Mittelalter}
\subtitle{Vorlesungsprotokoll vom 4. Juni 2018}
\date{4. Juni 2018}
\maketitle

\section{Rahmenbedingungen des Ersten Kreuzzugs}

In der Zeit vor dem Ersten Kreuzzug gab es keine großen stehenden Heere und daher
auch keine Logistik für die Versorgung entsprechender Truppen. Fernab der Heimat
wurde häufig geplündert, um sich zu ernähren. Der Erfolg von militärischen Kampagnen
hing daher stark von örtlichen Begebenheiten ab.

In diesem Zusammenhang sind im Grunde drei machtpolitische Faktoren besonders
relevant. Die Expansion der Seldschuken veränderte die nahöstlichen Machtverhältnisse,
die mediterranen Verhältnisse wurden durch das Auftreten der Normannen neu
geordnet und die Entwicklung des byzantinischen Reiches spielte eine wesentliche
Rolle.

Das byzantinische Reich sah sich als legitime Nachfolge des antiken römischen
Reichs. Die lateinischen Kaiser dagegen nahmen diese Nachfolge für sich in
Anspruch. Byzanz hielt die lateinischen Kaiser für unterlegen. Faktisch
sah die Situation im 11. Jahrhundert jedoch anders aus. Nach dem Tode des
Kaisers Basileus II im Jahre 1025 befand sich das Reich in einer Krise. Seit
der Mitte des 11. Jahrhunderts gab es zudem kleinere Grenzkonflikte im Balkan
und mit den Seldschuken im nordöstlichen Kleinasien.
In der gleichen Zeit erhebt Byzanz Anspruch auf Süditalien und Sizilien. General
Georgios Maniakes landete in Sizilien und heuerte Normannen aus der Familie
Hauteville als Söldner an. Es gab militärische Erfolge, die Kampagne scheiterte
letztlich jedoch an innerbyzantinischen Streitigkeiten. Der Kaiser zweifelte
an der Loyalität des Generals, berief ihn ab, wodurch die militärische Präsenz
von Byzanz zusammenbrach. Die Normannen hingegen blieben auf Dauer.
Der Konflikt mit den Seldschuken breitete sich aus und in der Schlacht von Manzikert
im Jahre 1071 verlor Byzanz das anatolische Hochland. Byzanz war zur Regionalmacht
verkommen und das Reich der Rum-Seldschuken wurde etabliert. Der Regierungssitz
der Seldschuken befand sich in der früheren Kaiserresidenz in Nizäa und damit
in der Nähe Konstantinopels.

Die Seldschuken waren auch an anderer Front erfolgreich. Sie besiegten das
Abbasidenreich und konfrontierten die Fatimiden in Syrien und Ägypten. Letztlich
stabilisierte sich die Front in einer Weise, dass sie im Bereich der Levante
verlief, letztlich also im Bereich der zukünftigen Kreuzfahrerstaaaten.

Die Normannen etablierten sich als starke Macht in Süditalien. Es fanden
Eroberungen durch Robert Guiscard statt. Der Enkel von Robert Guiscard begründete
später das Königreich Süditalien. In den 1080er Jahren griff Robert Guiscard
zusammen mit seinem Sohn das byzantinische Reich auf dem Balkan an. Das byzantinische
Reich schloss ein Bündnis mit Venedig zum Schutz vor den Normannen.

Unter Alexios Komnenos wurde das byzantinische Reich reorganisiert und konnte
sich selbst behaupten. Die fatimidische und seldschukische Herrschaft war
in den 1090ern instabil. Im Jahr 1095 richtete Alexios ein Hilfeersuchen
an den Papst Urban II mit der Bitte Söldner für den Krieg gegen die Seldschuken
zu schicken.

\section{Erster Kreuzzug}

Der Papst rief in Clermont zum Kreuzzug auf. Allerdings handelte es sich um mehr
als lediglich einen Aufruf Byzanz zu unterstützen. Auch die heiligen Stätten
sollten erobert werden. Dies kann wahrscheinlich aber als PR abgetan werden
und vermutlich dachte niemand daran, dass dies tatsächlich gelingen könnte.

In Folge des Aufrufs zogen mehrere Kontingente los. Zunächst ein "Volkskreuzzug"
unter Peter von Amiens im Jahr 1096, der Progrome im Rheinland an Juden durchführte.
Das war der Beginn der Gewalttradition gegen Juden für den Rest des Mittelalters
und darüber hinaus. Später zogen adlige Kontingente aus Nordfrankreich/Flandern,
Südfrankreich/Provence und von den Normannen in Süditalien los. Der Treffpunkt
sollte Konstantinopel zwischen Herbst 1096 und Frühjahr 1097 sein.

Byzanz war überrascht über die zehntausenden Kämpfer statt der geforderten
Söldner. Denn all diese Truppen wollten versorgt werden und mussten auf die
asiatische Seite übergesetzt werden. Der Kaiser setzte einen Loyalitätseid durch,
nach dem alle neu eroberten Gebiete unter die Oberhoheit des Kaisers fallen
sollten, sofern diese vorher schon einmal zu Byzanz gehörten. Der Kaiser
achtete ferner auf eine getrennte Überfahrt und Unterbringung, sodass ein Treffen
erst auf der asiatischen Seite erfolgte.

Als erste Aktion wurde zusammen mit den Byzantinern Nizäa erobert, wobei die
Byzantiner eine friedliche Übergabe unter Wahrung von Besitz und Leben vereinbarten.
Die Kreuzfahrer wollten hingegen plündern. Es folgte ein mehrjähriger
Eroberungszug von 1097 bis 1099 mit der Eroberung von Antiochia im Jahr 1098
und Jerusalem im Jahr 1099. Es wurden vier Kreuzfahrerherrschaften etabliert:

\begin{itemize}
    \item Königreich Jerusalem (Gottfried von Bouillon)
    \item Fürstentum Antiochia (Bohemund von Tarent, Sohn von Robert Guiscard)
    \item Grafschaft Edessa (Balduin von Boulogne)
    \item Grafschaft Tripolis (Raimund von Toulouse)
\end{itemize}

\section{Folgen des Ersten Kreuzzugs}

Aus der Sicht von Byzanz war der Kreuzzug ein Schalg ins Wasser ohne den
erhofften Erfolg. Außerdem bekamen sie dadurch einen neuen problematischen, unruhigen
und militärisch potenten Nachbarn im Südosten. Die lateinischen Kreuzfahrer
erreichten hingegen ihr Ziel. Das Ziel war die politische Kontrolle der Pilgerwege
nach Jerusalem. Die Gegend war multireligiös und blieb es auch, da der Kreuzzug
kein Missionskrieg war. Letztlich wurde die Herrschaftsebene ausgetauscht, die
Gesellschaft blieb weitgehend bestehen.

Die Kreuzfahrerstaaaten stellten einen neuen Machtfaktor in der Levante dar.
Sie standen in enger Kommunikation mit den europäischen Heimatregionen der
"Franchi". Aus der Perspektive der Seldschuken, Fatimiden und später der
Ayyubiden bedeuteten diese Staaten eine neue Herrschaftsbildung an der Peripherie
der Einflussbereiche dieser Großreiche, welche bei militärischem Druck nicht
dauerhaft widerstandsfähig sein würden.

\end{document}
