\documentclass[10pt,a4paper,oneside,ngerman,numbers=noenddot]{scrartcl}
\usepackage[T1]{fontenc}
\usepackage[utf8]{inputenc}
\usepackage[ngerman]{babel}
\usepackage{amsmath}
\usepackage{amsfonts}
\usepackage{amssymb}
\usepackage{bytefield}
\usepackage{paralist}
\usepackage{gauss}
\usepackage{pgfplots}
\usepackage{textcomp}
\usepackage[locale=DE,exponent-product=\cdot,detect-all]{siunitx}
\usepackage{tikz}
\usepackage{algpseudocode}
\usepackage{algorithm}
\usepackage{mathtools}
\usepackage{hyperref}
\usepackage[german=quotes]{csquotes}
%\usepackage{algorithmic}
%\usepackage{minted}
\usetikzlibrary{automata,matrix,fadings,calc,positioning,decorations.pathreplacing,arrows,decorations.markings}
\usepackage{polynom}
\polyset{style=C, div=:,vars=x}
\pgfplotsset{compat=1.8}
\pagenumbering{arabic}
%\def\thesection{\arabic{section})}
%\def\thesubsection{(\alph{subsection})}
%\def\thesubsubsection{(\roman{subsubsection})}
\makeatletter
\renewcommand*\env@matrix[1][*\c@MaxMatrixCols c]{%
  \hskip -\arraycolsep
  \let\@ifnextchar\new@ifnextchar
  \array{#1}}
\makeatother
\parskip 12pt plus 1pt minus 1pt
\parindent 0pt

\DeclarePairedDelimiter\abs{\lvert}{\rvert}%
\DeclarePairedDelimiter{\ceil}{\lceil}{\rceil}

\newenvironment{myitemize}{\begin{itemize}\itemsep -9pt}{\end{itemize}} % Zeilenabstand in Aufzählungen geringer

%switch starred and non-starred (auto-size)
\makeatletter
\let\oldabs\abs
\def\abs{\@ifstar{\oldabs}{\oldabs*}}
\makeatother

\hypersetup{
    colorlinks,
    citecolor=black,
    filecolor=black,
    linkcolor=black,
    urlcolor=black
}

\MakeOuterQuote{"}

\begin{document}
\author{Jim Martens (6420323)}
\title{Vorlesung Kalter Krieg}
\subtitle{Konflikt entsteht}
\date{24. Juni 2019}
\maketitle

\section{Erholung versus Demontage}

Direkt nach dem Krieg waren alle Alliierten für eine Demontage Deutschlands. In den USA hat sich diese Meinung
aber schnell geändert: Europa sollte sich fortan erholen; zu diesem Zweck wurde der Marshall-Plan entworfen.
Offiziell hieß das Programm "European Recovery Program" und umfasste \$13,4 Milliarden; das entspricht circa \$134 
Milliarden in der heutigen Zeit. Davon hat die Bundesrepublik Deutschland nur \$1,4 Milliarden bekommen, Frankreich 
\$2,8 Mrd und das Vereinigte Königreich \$3,4 Mrd; das deutsche Wirtschaftswunder kann daher schlecht alleine
auf den Marshall-Plan zurückgeführt werden. Zudem besteht die Hilfe hauptsächlich aus Wirtschaftsgütern, die
in den USA produziert wurden; das Programm war somit auch ein Konjunkturprogramm für die US-Wirtschaft.
 
Die Sowjetunion bekam Ende Juni 1947 auch das Angebot sich am Marshall-Plan zu beteiligen, entschied sich kurz
darauf aber dagegen; sie untersagte auch allen Staaten im sowjetischen Einflussbereich daran teilzunehmen. Die
Vertreter\*innen der Tschechoslowakei mussten zurückfahren. Die Sowjetunion vertrat weiterhin eine Politik der
Demontage und protestierte gegen den Politikwechsel bei den USA; Frankreich war ebenfalls weiter an Demontage 
interessiert, mussten unter Druck der USA dann aber der Erholung Deutschlands zustimmen.

Die Position der Sowjetunion war gedeckt durch die Beschlüsse der Konferenzen von Jalta und Potsdam. Der alliierte
Kontrollrat wurde Ende Juli 1945 eingesetzt und sollte Deutschland einstimmig regieren; zu den Aufgaben zählten
die Aufhebung von Nazi-Gesetzen, eine gemeinsame Organisation der Verwaltung, die Entnazifizierung und
Demontage und Reparation; eine Erholung war nicht Teil der Aufgaben. Insofern änderte sich die Position der
USA zu dem Kontrollrat und die Sowjetunion bestand ohne Erfolg darauf die Absprachen einzuhalten. 

Das Vereinigte Königreich hatte enorme wirtschaftliche Probleme in Folge des Krieges, Kolonien wurden unabhängig
und Lebensmittel waren bis 1954 rationiert. In diesem Umfeld wurde die britische Zone und die us-amerikanische 
Zone zur Bi-Zone vereinigt; Frankreich schloss sich erst 1947 an und bekam dafür das Saargebiet als Entschädigung.
In der Folge wurde der Sowjetunion verboten Reperationen aus der Westzone zu bekommen.

\section{Risse bilden sich}

Die USA ist ein föderaler Staat; formal war dies die Sowjetunion auch, aber faktisch wurde zentral in Moskau
entschieden. Für Deutschland wollte die Sowjetunion eine zentrale Verwaltung. Frankreich ist selber zentral 
organisiert, wollte für Deutschland aber genauso wie die USA keine zentrale Führung.

In ihrem Einflussbereich annektierte die Sowjetunion die Gebiete, welche ihr im Hitler-Stalin-Pakt zugeschrieben
worden waren; außerdem sorgte sie für homogenisierte Staaten in ihrem Einflussbereich. Sie versuchte Europa
sicher zu machen, dazu gehörte ein schwaches Deutschland; dies stand im Widerspruch zur US-Politik der 
Erholung. Auch wollte sie eine provisorische deutsche Zentralregierung schaffen und Deutschland demokratisieren.
In der Ostzone wurde im April 1946 die KPD und die SPD zwangsvereinigt unter der Führung der KPD.

Im März 1948 trafen sich die Westalliierten und die Benelux-Staaten zur Sechs-Mächte-Konferenz: Am 6. März
wurde das Kommuniqué veröffentlicht, welches eine enge Verbindung der Wirtschaftsräume in Westdeutschland mit
dem Rest von Europa vorsah; am 17. März gründete das Vereinigte Königreich ein Militärbündnis mit Frankreich und 
den Benelux-Staaten; am 20. März verließ die Sowjetunion den alliierten Kontrollrat. 

\section{Währungsreform und Grundgesetz}

Im Juni 1948 wurde ein Deutschland-Kommuniqué veröffentlicht: Die deutschen Ministerpräsidenten bekommen die
Vollmacht eine verfassungsgebende Versammlung einzuberufen, die Verfassung soll eine föderative Regierungsform
festschreiben und die Rechte und Freiheiten der Individuen garantieren. Diese Forderungen richteten
sich klar gegen die Sowjetunion ohne deren Namen zu nennen, denn diese wollte eine zentrale Verwaltung.

Wenige Tage nach dem Kommuniqué wurde in Westdeutschland eine Währungsreform durchgeführt. Zwar hatte der
alliierte Kontrollrat sich für eine Währungsreform ausgesprochen, dies sollte aber zu einem unbestimmten Zeitpunkt
weit in der Zukunft stattfinden. Die Sowjetunion wurde durch diese Reform überrascht und hatte sich nicht
darauf eingestellt; die Reichsmark im Westen wurde nicht eingesammelt und somit befürchtete die Sowjetunion einen
Run von Westdeutschen auf die Ostzone. Sie sah sich gezwungen ebenfalls eine Währungsreform durchzuführen:
alte Reichsmarkscheine wurden lediglich überdruckt. Aufgrund der Politik der Demontage konnte mit der Ostmark fast 
nichts gekauft werden, die Deutsche Mark genoss im Vergleich deutlich mehr Vertrauen.
Die Währungsreform in Westen war allerdings auch nicht perfekt: Alle bekamen die gleiche Menge Bargeld, 
immaterielle Vermögensgegenstände wie Wohnungen blieben aber bei ihren Eigentümern; somit war die Reform
insbesondere zugunsten der Reichen und Wohlhabenden.

Die Sowjetunion blockierte Berlin und wollte wohl die Westalliierten zwingen sich an Absprachen zu halten,
diese protestierten gegen die Blockade; im Sommer 1948 trafen sich die Militärgouverneure und im Oktober 
beschäftigte sich der UN-Sicherheitsrat mit der Thematik. Schließlich wurde im November/Dezember 1948 die
Trennung von West- und Ostberlin zementiert: die Verwaltung wurde gedoppelt, die Freie Universität Berlin entsteht.
Am Ende bekommt die Sowjetunion lediglich dieses Zugeständnis: Berlin bleibt unter Kontrolle der vier Mächte;
dies wurde am 14. Mai 1949 im kleinen Besatzungsstatut festgelegt, am 23. Mai 1949 tritt das Grundgesetz in Kraft.

\section{Fazit}

Die USA haben sich von den Absprachen der Roosevelt-Regierung entfernt, die Sowjetunion bleibt bei diesen.
Naiv gesehen läge die "Schuld" des Kalten Krieges somit bei den USA, allerdings hat die Sowjetunion mit dem 
Rücktritt aus dem Kontrollrat und der Blockade Berlins ebenso taktische Fehler begangen und ihren Teil zum Konflikt
beigetragen. 

\end{document}
