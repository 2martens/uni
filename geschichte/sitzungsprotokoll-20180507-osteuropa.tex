\documentclass[10pt,a4paper,oneside,ngerman,numbers=noenddot]{scrartcl}
\usepackage[T1]{fontenc}
\usepackage[utf8]{inputenc}
\usepackage[ngerman]{babel}
\usepackage{amsmath}
\usepackage{amsfonts}
\usepackage{amssymb}
\usepackage{bytefield}
\usepackage{paralist}
\usepackage{gauss}
\usepackage{pgfplots}
\usepackage{textcomp}
\usepackage[locale=DE,exponent-product=\cdot,detect-all]{siunitx}
\usepackage{tikz}
\usepackage{algpseudocode}
\usepackage{algorithm}
\usepackage{mathtools}
\usepackage{hyperref}
\usepackage[german=quotes]{csquotes}
%\usepackage{algorithmic}
%\usepackage{minted}
\usetikzlibrary{automata,matrix,fadings,calc,positioning,decorations.pathreplacing,arrows,decorations.markings}
\usepackage{polynom}
\polyset{style=C, div=:,vars=x}
\pgfplotsset{compat=1.8}
\pagenumbering{arabic}
%\def\thesection{\arabic{section})}
%\def\thesubsection{(\alph{subsection})}
%\def\thesubsubsection{(\roman{subsubsection})}
\makeatletter
\renewcommand*\env@matrix[1][*\c@MaxMatrixCols c]{%
  \hskip -\arraycolsep
  \let\@ifnextchar\new@ifnextchar
  \array{#1}}
\makeatother
\parskip 12pt plus 1pt minus 1pt
\parindent 0pt

\DeclarePairedDelimiter\abs{\lvert}{\rvert}%
\DeclarePairedDelimiter{\ceil}{\lceil}{\rceil}

\newenvironment{myitemize}{\begin{itemize}\itemsep -9pt}{\end{itemize}} % Zeilenabstand in Aufzählungen geringer

%switch starred and non-starred (auto-size)
\makeatletter
\let\oldabs\abs
\def\abs{\@ifstar{\oldabs}{\oldabs*}}
\makeatother

\hypersetup{
    colorlinks,
    citecolor=black,
    filecolor=black,
    linkcolor=black,
    urlcolor=black
}

\MakeOuterQuote{"}

\begin{document}
\author{Jim Martens (6420323)}
\title{Vorlesung Osteuropa im 1. Weltkrieg}
\subtitle{Vorlesungsprotokoll vom 7. Mai 2018}
\date{7. Mai 2018}
\maketitle

\section{Entwicklung im Kaukasus}

Nach dem Ende des Ersten Weltkriegs entstanden die drei kaukasischen Staaten
Azerbaidschan, Armenien und Georgien. Im kaukasischen Teil Russlands formierte
sich 1919 die Freiwilligenarmee der "weißen" Gegenbewegung zu den Bolschewiken.
Azerbaidschan und Georgien schlossen ein Verteidigungsbündnis gegen diese
Armee, wohingegen Armenien sich mit ihr verbündete. Grund hierfür ist eine
Auseinandersetzung von Armenien mit Azerbaidschan. Dieser Konflikt besteht
im Übrigen auch heute noch. Nach dem endgültigen Ende der weißen Kräfte fehlte
der Partner für Armenien. Dieses wechselte schnell die Seiten und verbündete sich
mit den Bolschewiken, was weniger ideologisch und mehr geopolitisch begründet
war.

Im Osmanischen Reich kam 1920 die nationalistische Bewegung an die Macht, verlegte
die Hauptstadt nach Ankara und nannte den Staat fortan Türkei. Aus Sicht der
Bolschewiken stellte die kemalistische Türkei das Gegenteil zum reaktionären
Osmanischen Reich dar, was zu einem Bündnis zwischen beiden Kräften führte.
In der Folge einigte sich die Türkei mit Russland auf die Eroberung von Azerbaidschan.
Die Rote Armee marschierte in Azerbaidschan ein. Daraufhin trat am 1. April 1920
die Regierung von Azerbaidschan zurück und am 28. April wurde die sozialistische
Sowjetrepublik Azerbaidschan ausgerufen. Die Bolschewiken hatten ein großes
Interesse an der Kontrolle von Azerbaidschan und damit Baku, weil sich dort
große Erdölvorkommen befanden. Schon im Zarenreich wurde 1844 eine erste
Ölquelle dort durch Bohrung erschlossen. Die Nutzung von Öl in Baku
reicht bis ins 13. Jahrhundert zurück.

Mit der Quasi-Annexion von Azerbaidschan hatte Armenien erneut ein Problem.
Im Herbst 1920 begann ein Krieg mit der Türkei, in dem die türkischen Truppen
in Armenien einmarschierten, um ehemals osmanische Gebiete zurückzuerobern.
Armenien versuchte eine Kriegslist und besetzte zu Georgien gehörendes Gebiet
im Versuch die türkischen Truppen zu umgehen. Bevor die türkischen Truppen
die Städte erreichten, wurden Progrome an der türkischen Bevölkerung verübt.
Knapp einen Monat nach Beginn des Kriegs ersuchte Armenien im Oktober 1920 um die
Hilfe von Großbritannien, Frankreich und Italien. Sie bekamen allerdings keine
Unterstützung. Lediglich Griechenland schickte einige hundert Truppen nach
Armenien. Nachdem vorher georgische Gebiete besetzt wurden, ersuchte Armenien
nun um die Hilfe Georgiens.

Es folgte auf Vermittlung der Bolschewiken ein türkisch-armenischer Friedensvertrag,
der allerdings unmittelbar gebrochen wurde. Die türkischen Truppen marschierten
einfach weiter. Am 8. November übermittelte die Türkei ihre Bedingungen. Der
Waffenstillstand wurde am 18. November geschlossen. Am 29. November putschten
sich armenische Bolschewiki an die Macht. Drei Tage später wurde die
heutige Grenze eingerichtet, die Armee entwaffnet und auf Gebiete verzichtet,
welche Armenien im Friedensvertrag mit dem Osmanischen Reich zugesichert bekam.
Im Dezember marschierte dann auch die Rote Armee von Azerbaidschan aus in Armenien
ein. Kurz darauf wurde die armenische sozialistische Sowjetrepublik ausgerufen.
Mit Georgien wurde ähnlich verfahren.

Im Jahr 1921 wurde ein Bündnisvertrag zwischen Armenien und Russland geschlossen,
wodurch sämtliche Kompetenzen an Moskau abgegeben wurden. 1922 wurden Armenien,
Azerbaidschan und Georgien zur Transkaukasischen sozialistischen föderativen
Sowjetrepublik. Kurz darauf wurde allerdings der föderative Teil entfernt und
ein Einheitsstaat gebildet. Im November 1922 trat diese Republik der Union
der sozialistischen Sowjetrepubliken bei.

\section{Abtrennungsfreiheit}

Nach der sogenannten Oktoberrevolution, die vielmehr eine Machtergreifung
einer Minderheit darstellte, erließ die neue Regierung Dekrete zur Umsetzung
ihrer Versprechen. Eine der Deklarationen beinhaltete das Verhältnis zu den Völkern
Russlands:

\begin{enumerate}
    \item Gleichheit und Souveranität der Völker Russlands
    \item Recht der Völker Russlands auf freie Selbstbestimmung, bis hin zur
    Loslösung und Bildung eines selbstständigen Staates.
    \item Aufhebung aller und jeglicher nationaler und nationalreligiöser
    Privilegien und Einschränkungen
    \item Freie Entfaltung nationaler Minderheiten und ethnographischer Gruppen,
    die das Gebiet Russlands bewohnen.
\end{enumerate}

Diese Deklaration klang zwar auf den ersten Blick ganz sympathisch, hatte aber
einen Haken. Die Abtrennungsfreiheit wurde nur als Übergangsperiode auf dem
Weg hin zur Verschmelzung der Nationen angesehen. Finnland nahm die Abspaltung
in Anspruch, was genehmigt wurde. Im Anschluss wurde die sowjetische Revolution
in Finnland unterstützt. Es gab also zwei Phasen. Die Loslösung in bürgerlicher
Phase und die Verschmelzung in sowjetischer Phase.

\section{Entwicklung in Ukraine}

In der Ukraine gab es eine linke politische Bewegung, die formale Gleichberechtigung
mit dem Rest Russlands wollte. Am 17. März 1917 wurde eine Konferenz einberufen,
um eine politische Vertretung zu schaffen. Erste Beschlüsse des Zentralrats
der Ukraine gab es am 7. April 1917. Die linke Bewegung möchte für die Autonomie
der Ukraine mit legalen Mitteln kämpfen. Auf Einladung schlossen sich weitere
linke Parteien dem Zentralrat an. Das Ziel war Autonomie im Rahmen eines
demokratischen und föderalen russischen Staates. Die Autonomie wurde am
23. Juni 1917 proklamiert. Es wurde ein allukrainisches Parlament gefordert,
welches Gesetze erlassen könnte, die dann später von der allrussischen
Vertretung bestätigt werden müssten.

Die provisorische Regierung Russlands wusste nicht, wie sie mit der Situation
umgehen sollte. Entsprechend gingen die Gespräche zwischen dem Zentralrat
und der Regierung immer hin und her. Schließlich wurde die Zustimmung erteilt
und dies in einem Universal des Zentralrats verkündet. Die Regionalregierung der
Ukraine nennt sich "Generalsekretariat". Im dritten Universal wird die
Selbstständigkeit und im vierten Universal die Unabhängigkeit im Januar 1918
erklärt. Dies passierte nachdem im Oktober 1917 die Bolschewiken die Macht
übernahmen. Die Bolschewiken verlangten die gleiche Anerkennung, wie sie
der provisorischen Regierung entgegengebracht wurde. Dies lehnte die Ukraine
jedoch ab, welche die sowjetische Regierung einzig als russische Regionalregierung
ansah.

Aufgrund der linken Politik des Zentralrats, die als revolutionshemmend angesehen wurde,
konnte sich dieser aus Sicht der Bolschewiken auch nicht auf die Abtrennungsfreiheit
berufen. Die Erklärung der Unabhängigkeit war daher aus bolschewistischer Sicht
nicht erlaubt. Es kam zum Krieg Russlands gegen die Ukraine.

\end{document}
