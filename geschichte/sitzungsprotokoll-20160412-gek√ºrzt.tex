\documentclass[10pt,a4paper,oneside,ngerman,numbers=noenddot]{scrartcl}
\usepackage[T1]{fontenc}
\usepackage[utf8]{inputenc}
\usepackage[ngerman]{babel}
\usepackage{amsmath}
\usepackage{amsfonts}
\usepackage{amssymb}
\usepackage{bytefield}
\usepackage{paralist}
\usepackage{gauss}
\usepackage{pgfplots}
\usepackage{textcomp}
\usepackage[locale=DE,exponent-product=\cdot,detect-all]{siunitx}
\usepackage{tikz}
\usepackage{algpseudocode}
\usepackage{algorithm}
\usepackage{mathtools}
\usepackage{hyperref}
%\usepackage{algorithmic}
%\usepackage{minted}
\usetikzlibrary{automata,matrix,fadings,calc,positioning,decorations.pathreplacing,arrows,decorations.markings}
\usepackage{polynom}
\polyset{style=C, div=:,vars=x}
\pgfplotsset{compat=1.8}
\pagenumbering{arabic}
%\def\thesection{\arabic{section})}
%\def\thesubsection{(\alph{subsection})}
%\def\thesubsubsection{(\roman{subsubsection})}
\makeatletter
\renewcommand*\env@matrix[1][*\c@MaxMatrixCols c]{%
  \hskip -\arraycolsep
  \let\@ifnextchar\new@ifnextchar
  \array{#1}}
\makeatother
\parskip 12pt plus 1pt minus 1pt
\parindent 0pt

\DeclarePairedDelimiter\abs{\lvert}{\rvert}%
\DeclarePairedDelimiter{\ceil}{\lceil}{\rceil}

\newenvironment{myitemize}{\begin{itemize}\itemsep -8pt}{\end{itemize}} % Zeilenabstand in Aufzählungen geringer

%switch starred and non-starred (auto-size)
\makeatletter
\let\oldabs\abs
\def\abs{\@ifstar{\oldabs}{\oldabs*}}
\makeatother

\hypersetup{
    colorlinks,
    citecolor=black,
    filecolor=black,
    linkcolor=black,
    urlcolor=black
}

\begin{document}
\author{Jim Martens (6420323)}
\title{Die Mittelalterliche Geschichte der Hanse}
\subtitle{Vorlesungsprotokoll vom 12. April 2016}
\date{12. April 2016}
\maketitle

\section*{Überblick über Geschichte Lübecks}

\begin{myitemize}
    \item Lübeck war zentral für die Hanse
    \item spätestens im 14. Jhd. war Lübeck Hauptort der Hanse
    \item Rolle u.a. durch geografische Position begründet
    \item während einer Verfassungskrise in Lübeck ging die Führung an Hamburg
    \item nachdem in Hamburg kurz darauf auch eine Krise ausbrach, ging Führung an Lüneburg
    \item dies war auch keine dauerhafte Lösung, nach Ende der Krise in Lübeck ging Führung zurück an Lübeck
    \item auf einem Hansetag wurde das Zusammenspiel zwischen Städten geregelt und Lübeck übernahm Geschäftsführung
    \item Lübeck teilte diese Verantwortung mit englischen Städten
    \item viele Hansetage fanden in Lübeck statt
    \item Lübeck war Nachfolge für Schleswig und slawische Städte (u.a. für Alt-Lübeck)
    \item von 1203 bis 1225 herrschten die Dänen über Lübeck
\end{myitemize}

\section*{Gründung Lübecks}

\subsection*{Erster Anlauf}
\begin{myitemize}
    \item Sachsen fiel an die Welfen
    \item es wird für Siedler geworben, die nach Sachsen ziehen sollen
    \item Adolph war zuständig für Werbung
    \item er suchte nach einem Siedlungsplatz für eine neue Stadt
    \item einige Siedler siedelten bei slawischen Burgen (Lüneburg, Oldenburg)
    \item an etwas südlicherer Stelle im Verhältnis zum früheren Alt-Lübeck wurde die neue Stadt Lübeck gegründet
    \item wahrscheinlich wurde die Siedlung von Beginn an mit Rechten ausgestattet, ist aber nichts überliefert
    \item Siedlung wuchs schnell
    \item das Wachstum führte zum Streit zwischen dem Herzog und Adolph
    \item 1127 wurde die Stadt zerstört
\end{myitemize}

\subsection*{Zweiter Anlauf}
\begin{myitemize}
    \item etwas weiter nördlich wurde neue Stadt durch den Herzog gegründet
    \item diese war aber schlecht platziert
    \item Adolph und Heinrich (Herzog) einigten sich
    \item Kaufleute gingen zurück nach Lübeck und bauten Stadt wieder auf
    \item Heinrich baut Lübeck als Handelsstadt aus und stattete sich mit Rechten aus
    \item es wurden diplomatische Beziehungen u.a. mit Russland aufgenommen
    \item 1161 Vertrag u.a. mit Russland, um freien Handel zu ermöglichen
    \item die Übernahme der Stadtherrschaft durch Heinrich war nicht der Startschuss des Handels
    \item entsprechende Beziehungen gab es bereits vorher
\end{myitemize}

\section*{Wachstum Lübecks}

Das Wachstum Lübecks gründet sich auf drei Aspekte:

\begin{myitemize}
    \item Erstens: Feste Siedlung mit Fernhandelskaufleuten, anders als z.B. Schleswig
    \item dadurch höhere Rechtssicherheit
    \item Zweitens: Weg zur Ostsee verkürzte sich über Lübeck
    \item Drittens: Lübeck lag an Handelsstraße, die Bardowick, slawische Dörfer und dänische Gebiete verband
    \item dadurch konnte das Salz besser abgesetzt werden
\end{myitemize}

\section*{Wechselnde Machtverhältnisse}

Nach Differenzen zwischen Heinrich und dem Kaiser und einem folgenden Reichskrieg gegen Heinrich, kam
die Stadt unter die Kontrolle Kaisers, wobei die Rechte der Bürger der Stadt bestätigt wurden.

Mehrere Herrscher hatten Interesse an Lübeck. Nach Überlegungen seitens der Bürger die Stadt an die Dänen zu 
übergeben (Hälfte dafür, Hälfte dagegen), wurde die Stadt vom Kaiser an den holsteinischen Herzog
übergeben.

Nach einem Krieg fiel Lübeck schließlich doch an die Dänen, welche 1203 in der Stadt begeistert empfangen
wurden. Die Rechte der Stadt wurden erneut bestätigt.

1226 wechselten die Lübecker in einem Krieg die Seiten und bekamen vom Kaiser den Reichsfreiheitsbrief.
Damit wurde Lübeck zur Reichsstadt. Dieser Privilegierung folgte der weitere Aufstieg der Stadt.

\section*{Anfänge des Gotlandhandels und des Kontors in Novgorod}

\subsection*{Gotlandhandel}
\begin{myitemize}
    \item Anfänge des Gotlandhandels sind auf Heinrich dem Löwen zurückzuführen
    \item Ansprechpartner waren nicht nur Dänemark, sondern auch Gotland
    \item wann die Handelsbeziehungen mit Gotland begannen, ist nicht ganz klar
    \item 1161 Artlenburger Vertrag
    \item Handelsbeziehungen mit Gotland bestanden bereits vor Gründung von Lübeck
    \item Vertrag verweist auf Konflikte zwischen deutschen und gotländischen Kaufleuten
    \item unklar, ob es um Konflikte auf Gotland geht
    \item umfassender Rechtsschutz für Gotländer, wenn deutsche Kaufleute die gleichen Rechte dort bekommen
    \item Überlegung, dass dies als Vorbereitung für Handel mit Gotland gemacht wurde
    \item 1170/90 wurde Kirche in Gotland vermutlich begonnen
    \item 1225 gab es Weihbrief der Kirche, Kaufleute waren zu dem Zeitpunkt schon da
    \item Wahrung der Rechte der dt. Kaufleute wird deren Gesandtem übergeben
\end{myitemize}

\subsection*{Bildung genossenschaftlicher Strukturen}
\begin{myitemize}
    \item niederdeutsche Kaufleute kamen früh über Lübeck nach Gotland
    \item es bildeten sich genossenschaftliche Strukturen
    \item deutsche Händler durften Niederlassung der gotländischen Kaufleute nutzen, bevor sie ihre eigenen Niederlassungen hatten
    \item deutsche Kaufleute, die nach Gotland gingen, waren wohl schon genossenschaftlich organisiert
    \item man findet im Nord- und Ostseeraum Hinweise auf gotländische Kaufleute
    \item in Urkunde der Herrscherin von Flandern finden sich entsprechende Hinweise
    \item Kaufleute aus Gotland jedweder Nationalität waren nach außen mit einem Siegel vertreten
    \item 1280 ein Siegel zeigt Gemeinschaft der in Gotland bleibenden dt. Kaufleute
    \item man kann den Artlenburger Vertrag als Gründungsdokument der Gemeinschaft der Kaufleute ansehen
    \item 1252 gab es angeblich Vertrag zwischen Schweden und Lübeck, der die Privilegierung der Kaufleute
          in Visby(Gotland) bestätigt
\end{myitemize}

\subsection*{Gemeinsame Außenvertretung gotländischer Kaufleute}
\begin{myitemize}
    \item Gilde in Gotland war Mittel alle Kaufleute zusammenzufassen
    \item schwierig, ob man dies als Start der Hanse betrachten kann
    \item 1188 gab es Verhandlungen mit Russland
    \item man kann von einem gemeinsamen Agieren der Kaufleute in Gotland ausgehen
    \item auch nach Ende der dänischen Vorherrschaft im Ostseeraum bestand die Gemeinschaft weiter
    \item ein Vertrag berechtigt Kaufleute zur Errichtung eines Kontors in Smolensk
    \item dieser Vertrag beinhaltet wieder Siegel der Kaufleute, welches auf genossenschaftliche Struktur
          hinweist
    \item gotländische Händler bereits Ende des 12. Jhd. in Bergen
    \item in englischer Kämmerei finden sich nach 1226 immer wieder Verweise auf gotländische Kaufleute
    \item dänische Krone hat genossenschaftliche Gruppe mehrfach privilegiert
    \item 1226 verfügt Kaiser durch Reichsfreiheitsurkunde die Gleichheit der Lübecker Kaufleute mit anderen
          Handelspartnern
    \item englischer König nahm Händler von Gotland unter seinen Schutz
    \item engl. König hat Urkunden sowohl für Händler aus Gotland als auch für Händler aus HRE ausgestellt
\end{myitemize}

\subsection*{Novgoroder Kontor}
\begin{myitemize}
    \item Zusammenhang zwischen Gotland und Novgorodfahrt
    \item Anfänge des Novgoroder Kontor angelehnt an Geschichte des Handels mit Gotland
    \item Novgorod war wichtigste Stadt für Hansisch-Russischen Handel
    \item als dt. Kaufleute nach Novgorod kamen, waren skandinavische Händler bereits anwesend
    \item 1165 gibt es Privileg für Bürger Kölns, die wegen Handel in Novgorod sind
    \item aus 1191 stammt der älteste erhaltenste Vertrag mit russischem Landesherrn
    \item Vertrag bezieht sich auf frühere Abmachungen
    \item gegenseitige Privilegien und Strafen zwischen novgorodischen und gotländischen/deutschen Kaufleuten
    \item gleiche Strafen für gleiche Vergehen
    \item Novgoroder Kontor war nur zeitweilig besetzt
    \item Unterscheidung zwischen Sommer- und Winterfahrer nach Novgorod
    \item genossenschaftliche Strukturen erkennbar
    \item 4 Schlüssel für Truhe: Gotländer, Sooster, Lübecker und Dortmunder hatten Zugriff auf Truhe
\end{myitemize}
\end{document}