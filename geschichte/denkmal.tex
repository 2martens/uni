\documentclass[10pt,a4paper,oneside,ngerman,numbers=noenddot]{scrartcl}
\usepackage[T1]{fontenc}
\usepackage[utf8]{inputenc}
\usepackage[ngerman]{babel}
\usepackage{amsmath}
\usepackage{amsfonts}
\usepackage{amssymb}
\usepackage{bytefield}
\usepackage{paralist}
\usepackage{gauss}
\usepackage{pgfplots}
\usepackage{textcomp}
\usepackage[locale=DE,exponent-product=\cdot,detect-all]{siunitx}
\usepackage{tikz}
\usepackage{algpseudocode}
\usepackage{algorithm}
\usepackage{mathtools}
\usepackage{hyperref}
\usepackage[german=quotes]{csquotes}
%\usepackage{algorithmic}
%\usepackage{minted}
\usetikzlibrary{automata,matrix,fadings,calc,positioning,decorations.pathreplacing,arrows,decorations.markings}
\usepackage{polynom}
\polyset{style=C, div=:,vars=x}
\pgfplotsset{compat=1.8}
\pagenumbering{arabic}
%\def\thesection{\arabic{section})}
%\def\thesubsection{(\alph{subsection})}
%\def\thesubsubsection{(\roman{subsubsection})}
\makeatletter
\renewcommand*\env@matrix[1][*\c@MaxMatrixCols c]{%
  \hskip -\arraycolsep
  \let\@ifnextchar\new@ifnextchar
  \array{#1}}
\makeatother
\parskip 12pt plus 1pt minus 1pt
\parindent 0pt

\MakeOuterQuote{"}

\DeclarePairedDelimiter\abs{\lvert}{\rvert}%
\DeclarePairedDelimiter{\ceil}{\lceil}{\rceil}

\newenvironment{myitemize}{\begin{itemize}\itemsep -2pt}{\end{itemize}} % Zeilenabstand in Aufzählungen geringer

%switch starred and non-starred (auto-size)
\makeatletter
\let\oldabs\abs
\def\abs{\@ifstar{\oldabs}{\oldabs*}}
\makeatother

\hypersetup{
    colorlinks,
    citecolor=black,
    filecolor=black,
    linkcolor=black,
    urlcolor=black
}

\begin{document}
\author{Jim Martens (6420323)}
\title{Geschichte eines Hamburger Denkmals}
\subtitle{"Pferdestall" am Allendeplatz 1 und Umgebung}
\date{3. Dezember 2016}
\maketitle

Der heutige "Pferdestall" am Allendeplatz 1 beinhaltet das soziologische Institut,
die Pony-Bar und auch die T-Stube. Das Gebäude, welches zwischenzeitlich dem
Von-Melle-Park mit der Hausnummer 15 zugeordnet war, ist in einer Weise die
ausgestreckte Hand der Universität hinein in das Grindelviertel. Auf der anderen
Seite des nach dem früheren chilenischen Präsidenten Salvador Allende benannten
Allendeplatzes steht ein ehemaliger Luftschutzbunker aus dem Zweiten Weltkrieg.
Früher hieß der Platz Bornplatz und verband den "Pferdestall" mit der ehemaligen
Synagoge am Bornplatz. Die Synagoge wurde am 9. November 1938 in der Reichspogromnacht
niedergebrannt. Heute befindet sich an ihrer Stelle ein Platz, auf dem die Grundrisse
der Synagoge zu sehen sind.

Über den Grindelhof ist der "Pferdestall" innerhalb weniger Gehminuten mit der
großen Grindelallee und damit einer der Hauptverkehrsachsen Hamburgs verbunden.
Auf ihr verkehrt jeutzutage die Metrobuslinie 5, die das Zentrum Hamburgs mit
Niendorf verbindet.
In die andere Richtung ist in ähnlich kurzer Entfernung die Talmud-Tora-Gesamtschule
erreicht.

Nach den Entwürfen von Wilhelm Göhre wurde 1908 der heutige "Pferdestall" erbaut.
Allerdings war das Gebäude nicht als Universitätsgebäude geplant, denn damals
existierte die Universität Hamburg noch nicht einmal. Erst 1919 wurde sie gegründet
und erst nach dem Zweiten Weltkrieg wurde der Campus am Von-Melle-Park erbaut.

Stattdessen wurde das Gebäude nach dem Vorbild des königlich-preußischen Marstalls
in Berlin errichtet. Es diente einem Fuhrunternehmen zur Unterbringung des
Wagenparks und knapp 200 Pferden. Zwei Jahrzehnte später (ungefähr 1928-1930)
ging das Gebäude in Staatsbesitz über und wurde seitdem von der Universität genutzt.
Entlang dem Gebäude verlief die Beneckestraße und war wohl über die Fläche des
heutigen Von-Melle-Parks mit der Schlüterstraße verbunden.

Auf der linken Seite des Gebäudes befanden seit 1895 sich bis zur Zerstörung im
Zweiten Weltkrieg die Gebäude Beneckestraße 2, 4 und 6. Sie beheimateten im Laufe
der Jahre die Beratungsstelle für Jüdische Wirtschaftshilfe, die Verwaltung des
Jüdischen Religionsverbandes und des Bezirks Nordwestdeutschland der Reichsvereinigung
der Juden in Deutschland, die Bibliothek und Lesehalle der Gemeinde, verschiedene
jüdische Jugendverbände und das Jüdische Alters- und Pflegeheim. Ab 1942 mussten
diese Häuser als sogenannte "Judenhäuser" dienen und ein Jahr später auch als
Deportationsstätten. Die Gestapo deportierte in 7 Transporten mehr als 400 Menschen
nach Auschwitz und Theresienstadt.

Im Hof des Gebäudes Beneckestraße 4 befand sich die 1895 geweihte
Neue-Dammtor-Synagoge. In ihr wirkten die Rabbiner Dr. Max Grunwald (1895-1903),
Dr. Abraham Löwenthal (1903-1917) und Dr. Paul Holzer (1923-1938). Paul Holzer
wurde 1938 von den Nazis gezwungen Hamburg zu verlassen.
Auch die Neue-Dammtor-Synagoge wurde während der Reichspogromnacht geschändet,
konnte aber Anfang 1939 wiederhergerichtet und genutzt werden. Im Juni 1943 wurde
sie von der Gestapo beschlagnahmt und am 27. Juli 1943 durch Bomben zerstört.

Erst seit der Wiedervereinigung erinnern Gedenktafeln an das Schicksal dieser
Gebäude und die Geschichte hinter ihnen und den dort lebenden Menschen. Anhand
der Bebauung des Von-Melle-Parks, an dessen Stelle vor dem Zweiten Weltkrieg
mit Sicherheit ein Teil des Grindelviertels stand, wird die Geschichtsträchtigkeit
des Ortes ansonsten wenig deutlich. Ein Großteil eines Stadtteils wurde im Krieg
zerstört und nur die Stolpersteine auf dem Universitätsgelände erinnern daran, dass
auf der Fläche des heutigen Hauptcampus der Universität Hamburg einst das jüdische
Leben in Hamburg pulsierte.

Das Gebäude der Talmud-Tora-Gesamtschule blieb erhalten, wurde aber erst 2004
wieder der jüdischen Gemeinde in Hamburg übergeben. Der östliche Teil des ehemaligen
Bornplatzes wurde 1989 nach Joseph-Carlebach benannt. Er war seit 1936 der
Oberrabbiner der Deutsch-Israelitischen Gemeinde und wurde am 26. März 1942
von den Nationalsozialisten im Wald von Bikernieki bei Riga ermordet.




\end{document}
