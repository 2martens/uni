\documentclass[10pt,a4paper,oneside,ngerman,numbers=noenddot]{scrartcl}
\usepackage[T1]{fontenc}
\usepackage[utf8]{inputenc}
\usepackage[ngerman]{babel}
\usepackage{amsmath}
\usepackage{amsfonts}
\usepackage{amssymb}
\usepackage{bytefield}
\usepackage{paralist}
\usepackage{gauss}
\usepackage{pgfplots}
\usepackage{textcomp}
\usepackage[locale=DE,exponent-product=\cdot,detect-all]{siunitx}
\usepackage{tikz}
\usepackage{algpseudocode}
\usepackage{algorithm}
\usepackage{mathtools}
\usepackage{hyperref}
\usepackage[german=quotes]{csquotes}
%\usepackage{algorithmic}
%\usepackage{minted}
\usetikzlibrary{automata,matrix,fadings,calc,positioning,decorations.pathreplacing,arrows,decorations.markings}
\usepackage{polynom}
\polyset{style=C, div=:,vars=x}
\pgfplotsset{compat=1.8}
\pagenumbering{arabic}
%\def\thesection{\arabic{section})}
%\def\thesubsection{(\alph{subsection})}
%\def\thesubsubsection{(\roman{subsubsection})}
\makeatletter
\renewcommand*\env@matrix[1][*\c@MaxMatrixCols c]{%
  \hskip -\arraycolsep
  \let\@ifnextchar\new@ifnextchar
  \array{#1}}
\makeatother
\parskip 12pt plus 1pt minus 1pt
\parindent 0pt

\MakeOuterQuote{"}

\DeclarePairedDelimiter\abs{\lvert}{\rvert}%
\DeclarePairedDelimiter{\ceil}{\lceil}{\rceil}

\newenvironment{myitemize}{\begin{itemize}\itemsep -2pt}{\end{itemize}} % Zeilenabstand in Aufzählungen geringer

%switch starred and non-starred (auto-size)
\makeatletter
\let\oldabs\abs
\def\abs{\@ifstar{\oldabs}{\oldabs*}}
\makeatother

\hypersetup{
    colorlinks,
    citecolor=black,
    filecolor=black,
    linkcolor=black,
    urlcolor=black
}

\begin{document}
\author{Jim Martens (6420323)}
\title{Geschichte eines Hamburger Denkmals}
\subtitle{"Pferdestall" am Allendeplatz 1}
\date{3. Dezember 2016}
\maketitle

Der heutige "Pferdestall" am Allendeplatz 1 beinhaltet das soziologische Institut,
die Pony-Bar und auch die T-Stube. Das Gebäude, welches früher dem Von-Melle-Park
mit der Hausnummer 15 zugeordnet war, ist in einer Weise die ausgestreckte Hand
der Universität hinein in das Grindelviertel. Auf der anderen Seite des nach dem
früheren chilenischen Präsidenten Salvador Allende benannten Allendeplatzes steht
ein ehemaliger Luftschutzbunker aus dem Zweiten Weltkrieg. Früher hieß der Platz
Bornplatz und verband den "Pferdestall" mit der ehemaligen Synagoge am Bornplatz.
Die Synagoge wurde am 9. November 1938 in der Reichspogromnacht niedergebrannt.
Heute befindet sich an ihrer Stelle ein Platz, auf dem die Grundrisse der Synagoge
zu sehen sind.

Über den Grindelhof ist der "Pferdestall" innerhalb weniger Gehminuten mit der
großen Grindelallee und damit einer der Hauptverkehrsachsen Hamburgs verbunden.
Auf ihr verkehrt jeutzutage die Metrobuslinie 5, die das Zentrum Hamburgs mit
Niendorf verbindet.
In die andere Richtung ist in ähnlich kurzer Entfernung die Talmud-Tora-Gesamtschule
erreicht.

Nach den Entwürfen von Wilhelm Göhre wurde 1908 der heutige "Pferdestall" erbaut.
Allerdings war das Gebäude nicht als Universitätsgebäude geplant, denn damals
existierte die Universität Hamburg noch nicht einmal. Erst 1919 wurde sie gegründet
und erst nach dem Zweiten Weltkrieg wurde der Campus am Von-Melle-Park erbaut.
TODO:
Stattdessen war das Gebäude damals tatsächlich ein Pferdestall.




\end{document}
