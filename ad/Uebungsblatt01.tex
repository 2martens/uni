\documentclass[10pt,a4paper,oneside,ngerman,numbers=noenddot]{scrartcl}
\usepackage[T1]{fontenc}
\usepackage[utf8]{inputenc}
\usepackage[ngerman]{babel}
\usepackage{amsmath}
\usepackage{amsfonts}
\usepackage{amssymb}
\usepackage{paralist}
\usepackage{gauss}
\usepackage{pgfplots}
\usepackage[locale=DE,exponent-product=\cdot,detect-all]{siunitx}
\usepackage{tikz}
\usetikzlibrary{matrix,fadings,calc,positioning,decorations.pathreplacing,arrows,decorations.markings}
\usepackage{polynom}
\polyset{style=C, div=:,vars=x}
\pgfplotsset{compat=1.8}
\pagenumbering{arabic}
\def\thesection{\arabic{section})}
\def\thesubsection{(\alph{subsection})}
\def\thesubsubsection{(\roman{subsubsection})}
\makeatletter
\renewcommand*\env@matrix[1][*\c@MaxMatrixCols c]{%
  \hskip -\arraycolsep
  \let\@ifnextchar\new@ifnextchar
  \array{#1}}
\makeatother

\begin{document}
\author{Tronje Krabbe, Jim Martens, Julian Tobergte}
\title{Hausaufgaben zum 22. Oktober}
\maketitle
\section{} %1
	\subsection{} %a
	\subsection{} %b
		\subsubsection{} %i
		Zu zeigen: für beliebige $b > 1$ gilt: $\log_{b}(n) \in \mathcal{\theta} (\log_{2}n)$\\
		Dies ist durch die Logarithmusgesetze einfach zu zeigen. $\log_{b}(n)$ kann demnach auch als $\frac{\log_{2}(n)}{\log_{2}(b)}$ dargestellt werden. Dabei geht dieser Term asymptotisch gegen $\log_{2}(n)$, da $\log_{2}(b)$ eine Konstante ist und daher nicht beachtet werden muss. Damit ist die Aussage gezeigt, dass $\log_{b}(n)$ für $b > 1$ asymptotisch genau so schnell wächst wie $\log_{2}(n)$.
		\subsubsection{} %ii
		Behauptung: $f \in \mathcal{O}(g) \Rightarrow g \in \omega (f)$\\
		Diese Behauptung gilt nicht. Dies kann mithilfe der Definition der Landau-Symbole erklärt werden. $f \in \mathcal{O}(g)$ besagt, dass $f$ maximal so schnell wie $g$ wächst. Dabei ist auch der Fall enthalten, dass $f$ und $g$ gleich schnell wachsen.
		
		Der zweite Teil der Behauptung erfordert jedoch, dass $g$ in jedem Fall schneller als $f$ wächst. Dies steht aber im Widerspruch zu dem ersten Teil der Behauptung. Damit ist die Behauptung widerlegt. 
		\subsubsection{} %iii
\section{} %2
	\subsection{} %a
	\subsection{} %b
\section{} %3
	\subsection{} %a
	\subsection{} %b
	$X^{64}$ kann geschickter berechnet werden, wenn man die Ergebnisse von vorigen Multiplikationen speichert. Damit lässt sich $X^{64}$ auf diese Weise berechnen:\\
	\begin{alignat*}{3}
	& X &\cdot & X &=& X^{2} \\
	& X^{2} &\cdot & X^{2} &=& X^{4} \\
	& X^{4} &\cdot & X^{4} &=& X^{8} \\
	& X^{8} &\cdot & X^{8} &=& X^{16} \\
	& X^{16} &\cdot & X^{16} &=& X^{32} \\
	& X^{32} &\cdot & X^{32} &=& X^{64}
	\end{alignat*}
	Auf eben gezeigte Weise kann man $X^{64}$ mit nur $6$ Multiplikationen ausrechnen.
	\subsection{} %c
\end{document}
