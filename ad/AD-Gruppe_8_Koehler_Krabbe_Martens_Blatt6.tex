\documentclass[10pt,a4paper,oneside,ngerman,numbers=noenddot]{scrartcl}
\usepackage[T1]{fontenc}
\usepackage[utf8]{inputenc}
\usepackage[ngerman]{babel}
\usepackage{amsmath}
\usepackage{amsfonts}
\usepackage{amssymb}
\usepackage{bytefield}
\usepackage{paralist}
\usepackage{gauss}
\usepackage{pgfplots}
\usepackage{textcomp}
\usepackage[locale=DE,exponent-product=\cdot,detect-all]{siunitx}
\usepackage{tikz}
\usepackage{algorithm}
\usepackage{algorithmic}
%\usepackage{minted}
\usetikzlibrary{automata,matrix,fadings,calc,positioning,decorations.pathreplacing,arrows,decorations.markings}
\usepackage{polynom}
\polyset{style=C, div=:,vars=x}
\pgfplotsset{compat=1.8}
\pagenumbering{arabic}
\def\thesection{\arabic{section})}
\def\thesubsection{(\alph{subsection})}
\def\thesubsubsection{(\roman{subsubsection})}
\makeatletter
\renewcommand*\env@matrix[1][*\c@MaxMatrixCols c]{%
  \hskip -\arraycolsep
  \let\@ifnextchar\new@ifnextchar
  \array{#1}}
\makeatother
\parskip 12pt plus 1pt minus 1pt
\parindent 0pt

\begin{document}
\author{Reinhard Köhler (6425686), Tronje Krabbe (6435002), \\
Jim Martens (6420323)}
\title{Hausaufgaben zum 15. Januar}
\subtitle{Gruppe 8}
\maketitle

\section{} %1
	\subsection{} %a
		\begin{tabular}{c|c|c|c|c|c|c|c|c}
			EXTRACT & 1 & 2 & 3 & 4 & 5 & 6 & 7 &\\
			\hline
			- & 0 & $\infty$ & $\infty$ & $\infty$ & $\infty$ & $\infty$ & $\infty$ & (v.dist) \\
			& - & - & - & - & - & - & - & (v.$\pi$) \\
			\hline
			1 & 0 & 4 & $\infty$ & $\infty$ & $\infty$ & 5 & $\infty$ & \\
			& - & 1 & - & - & - & 1 & - & \\
			\hline
			2 & 0 & 4 & 14 & $\infty$ & $\infty$ & 5 & 7 & \\
			& - & 1 & 2 & - & - & 1 & 2 & \\
			\hline
			6 & 0 & 4 & 14 & $\infty$ & 14 & 5 & 7 & \\
			& - & 1 & 2 & - & 6 & 1 & 2 & \\
			\hline
			7 & 0 & 4 & 13 & $\infty$ & 10 & 5 & 7 & \\
			& - & 1 & 7 & - & 7 & 1 & 2 & \\
			\hline
			5 & 0 & 4 & 12 & 15 & 10 & 5 & 7 & \\
			& - & 1 & 5 & 5 & 7 & 1 & 2 & \\
			\hline
			3 & 0 & 4 & 12 & 14 & 10 & 5 & 7 & \\
			& - & 1 & 5 & 3 & 7 & 1 & 2 & \\
			\hline
			4 & 0 & 4 & 12 & 14 & 10 & 5 & 7 & \\
			& - & 1 & 5 & 3 & 7 & 1 & 2 & 
		\end{tabular}
		
		Der kürzeste Pfad von 1 nach 4 verläuft über 2, 7, 5 und 3 nach 4. Insgesamt kostet der Pfad 14.
	\subsection{} %b
		In $G_{2}$ ist 3 die Quelle. 
		
		\begin{tabular}{c|c|c|c|c|c|c|c|c}
			EXTRACT & 3 & 1 & 2 & 4 & 5 &\\
			\hline
			- & 0 & $\infty$ & $\infty$ & $\infty$ & $\infty$ & (v.dist) \\
			& - & - & - & - & - & (v.$\pi$) \\
			\hline
			3 & 0 & $\infty$ & 9 & 4 & $\infty$ & \\
			& - & - & 3 & 3 & - & \\
			\hline
			4 & 0 & 5 & 8 & 4 & 6 & \\
			& - & 4 & 4 & 3 & 4 & \\
			\hline
			1 & 0 & 5 & 8 & 4 & 6 & \\
			& - & 4 & 4 & 3 & 4 & \\
			\hline
			5 & 0 & 1 & 8 & 4 & 6 & \\
			& - & 5 & 4 & 3 & 4 & \\
		\end{tabular}
		
		Durch die negative Kante von 5 nach 1, würde sich der kürzeste Pfad von 1 von 5 auf 1 ändern, was jedoch nicht geht, da 1 bereits besucht wurde. Daher liefert Dijkstra für das Single-Source-Shortest-Path Problem in $G_{2}$ ein falsches Ergebnis.
\section{} %2
\section{} %3
	\subsection{} %a
	\subsection{} %b
	\subsection{} %c
\section{} %4
	\subsection{} %a
	\subsection{} %b

\end{document}
