\documentclass[10pt,a4paper,oneside,ngerman,numbers=noenddot]{scrartcl}
\usepackage[T1]{fontenc}
\usepackage[utf8]{inputenc}
\usepackage[ngerman]{babel}
\usepackage{amsmath}
\usepackage{amsfonts}
\usepackage{amssymb}
\usepackage{bytefield}
\usepackage{paralist}
\usepackage{gauss}
\usepackage{pgfplots}
\usepackage{textcomp}
\usepackage[locale=DE,exponent-product=\cdot,detect-all]{siunitx}
\usepackage{tikz}
\usepackage{algpseudocode}
\usepackage{algorithm}
%\usepackage{algorithmic}
%\usepackage{minted}
\usetikzlibrary{automata,matrix,fadings,calc,positioning,decorations.pathreplacing,arrows,decorations.markings}
\usepackage{polynom}
\polyset{style=C, div=:,vars=x}
\pgfplotsset{compat=1.8}
\pagenumbering{arabic}
\def\thesection{\arabic{section})}
\def\thesubsection{(\alph{subsection})}
\def\thesubsubsection{(\roman{subsubsection})}
\makeatletter
\renewcommand*\env@matrix[1][*\c@MaxMatrixCols c]{%
  \hskip -\arraycolsep
  \let\@ifnextchar\new@ifnextchar
  \array{#1}}
\makeatother
\parskip 12pt plus 1pt minus 1pt
\parindent 0pt

\begin{document}
\author{Reinhard Köhler (6425686), Tronje Krabbe (6435002), \\
Jim Martens (6420323)}
\title{Hausaufgaben zum 22. Januar}
\subtitle{Gruppe 8}
\maketitle

\section{} %1
	\subsection{} %a
	\subsection{} %b
	\subsection{} %c
	\subsection{} %d
\section{} %2
\setcounter{section}{6}
\section{} %7
	\setcounter{subsection}{4}
	\subsection{} %e
	\begin{tikzpicture}
		\node (592) {6};
		\node (301) [right=of 592] {3};
		\node (297) [right=of 301] {4};
		\node (170) [right=of 297] {2};
		\node (92) [right=of 170] {1};
		\node (87) [right=of 92] {5};
		\node (86) [right=of 87] {0};
		\node (79) [right=of 86] {8};
		\node (69) [right=of 79] {9};
		\node (68) [right=of 69] {7};
		
		\node[state] (137) [above right=1 and 0.3 of 69] {137};
		\node[state] (165) [above right=1 and 0.3 of 86] {165};
		\node[state] (179) [above right=1 and 0.3 of 92] {179};
		\node[state] (302) [above right=of 165] {302};
		\node[state] (349) [above left=of 179] {349};
		\node[state] (598) [above right=1 and 0.3 of 301] {598};
		\node[state] (651) [above right=2 and 3 of 349] {651};
		\node[state] (1190) [above left=of 598] {1190};
		\node[state] (1841) [above left=of 651] {1841};
		
		\path (69) edge (137)
		      (68) edge (137)
		      (86) edge (165)
		      (79) edge (165)
		      (92) edge (179)
		      (87) edge (179)
		      (165) edge (302)
		      (137) edge (302)
		      (170) edge (349)
		      (179) edge (349)
		      (301) edge (598)
		      (297) edge (598)
		      (592) edge (1190)
		      (598) edge (1190)
		      (349) edge (651)
		      (302) edge (651)
		      (1190) edge (1841)
		      (651) edge (1841);
	\end{tikzpicture}
\end{document}
