\documentclass[10pt,a4paper,oneside,ngerman,numbers=noenddot]{scrartcl}
\usepackage[T1]{fontenc}
\usepackage[utf8]{inputenc}
\usepackage[ngerman]{babel}
\usepackage{amsmath}
\usepackage{amsfonts}
\usepackage{amssymb}
\usepackage{bytefield}
\usepackage{paralist}
\usepackage{gauss}
\usepackage{pgfplots}
\usepackage{textcomp}
\usepackage[locale=DE,exponent-product=\cdot,detect-all]{siunitx}
\usepackage{tikz}
\usepackage{algpseudocode}
\usepackage{algorithm}
%\usepackage{algorithmic}
%\usepackage{minted}
\usetikzlibrary{automata,matrix,fadings,calc,positioning,decorations.pathreplacing,arrows,decorations.markings}
\usepackage{polynom}
\polyset{style=C, div=:,vars=x}
\pgfplotsset{compat=1.8}
\pagenumbering{arabic}
\def\thesection{\arabic{section})}
\def\thesubsection{(\alph{subsection})}
\def\thesubsubsection{(\roman{subsubsection})}
\makeatletter
\renewcommand*\env@matrix[1][*\c@MaxMatrixCols c]{%
  \hskip -\arraycolsep
  \let\@ifnextchar\new@ifnextchar
  \array{#1}}
\makeatother
\parskip 12pt plus 1pt minus 1pt
\parindent 0pt

\begin{document}
\author{Reinhard Köhler (6425686), Tronje Krabbe (6435002), \\
Jim Martens (6420323)}
\title{Hausaufgaben zum 22. Januar}
\subtitle{Gruppe 8}
\maketitle

\section{} %1
	\subsection{} %a
		\[
			(F, H), (E, G), (D, F), (C, E), (E, F), (I, J), (C, B), (B, A), (G, I)
		\]
	\subsection{} %b
		\begin{tikzpicture}
			\node (A) {$A^{1}$};
			\node (B) [below=of A] {$B^{0}$};
			\node (C) [above right=of A] {$C^{3}$};
			\node (D) [below=of C] {$D^{0}$};
			\node (E) [right=of D] {$E^{1}$};
			\node (F) [below=of E] {$F^{0}$};
			\node (G) [right=of E] {$G^{2}$};
			\node (H) [below=of G] {$H^{0}$};
			\node (I) [right=of H] {$I^{1}$};
			\node (J) [below=of I] {$J^{0}$};
			
			\path[->,line width=1pt] 
			    (B) edge (A)
			    (D) edge (C)
			    (F) edge (E)
			    (H) edge (G)
			    (J) edge (I)
			    (E) edge (C)
			    (I) edge (G)
			    (A) edge (C)
			    (G) edge (C);
		\end{tikzpicture}
	\subsection{} %c
		\[
			(A, B), (B, G), (E, G), (C, E), (E, F), (F, H), (D, F), (A, I), (I, J)
		\]
	\subsection{} %d
		Beim Algorithmus von Prim wird zunächst eine for-Schleife über alle Knoten durchlaufen, welche damit die Laufzeit $\mathcal{O}(|V|)$ hat. Hinzu kommt eine while-Schleife über eine Queue mit initial $|V|$ Einträgen. Damit wird die while-Schleife genau $|V|$ mal durchlaufen, da in jedem Durchgang ein Eintrag entfernt wird. Die Operation deletemin hat eine Laufzeit von $\mathcal{O}(\log V)$, womit die Aufrufe von deletemin insgesamt $\mathcal{O}(V \cdot \log V)$ Zeit erfordern.
		
		Die for each Schleife wird \underline{insgesamt} $2|E|$ mal ausgeführt, was $\mathcal{O}(E)$ entspricht. Die decreasekey-Operation hat eine Laufzeit von $\mathcal{O}(\log V)$, womit sich insgesamt eine Laufzeit von $\mathcal{O}(V + V \cdot \log V + E \cdot \log V) = \mathcal{O}(E \cdot \log V)$ ergibt.		
		
\section{} %2
	\begin{algorithmic}[1]
		\Function{CanBeDivided}{s}
			\State $l \gets \Call{Length}{s}$
			\If{\Call{Dict}{s} = 1}
				\State \Return 1
			\ElsIf{$l$ = 0}
				\State \Return 0
			\Else
				\State $Q \gets \Call{CreateQueue}$
				\For{$i = 1 ... l-1$}
					\If{\Call{Dict}{$s_{1}...s_{i}$} = 1}
						\State $Q.enqueue(i+1)$
					\EndIf
				\EndFor
				\State $result \gets 0$
				\While{Q not empty}
					\State $pos \gets Q.dequeue()$
					\If{$!result$}
						\State $result \gets \Call{CanBeDivided}{s_{pos}...s_{l}}$
					\EndIf
				\EndWhile
				\State \Return $result$	
			\EndIf
		\EndFunction
	\end{algorithmic}
\setcounter{section}{6}
\section{} %7
	\setcounter{subsection}{4}
	\subsection{} %e
	\begin{tikzpicture}
		\node (592) {6};
		\node (301) [right=of 592] {3};
		\node (297) [right=of 301] {4};
		\node (170) [right=of 297] {2};
		\node (92) [right=of 170] {1};
		\node (87) [right=of 92] {5};
		\node (86) [right=of 87] {0};
		\node (79) [right=of 86] {8};
		\node (69) [right=of 79] {9};
		\node (68) [right=of 69] {7};
		
		\node[state] (137) [above right=1 and 0.3 of 69] {137};
		\node[state] (165) [above right=1 and 0.3 of 86] {165};
		\node[state] (179) [above right=1 and 0.3 of 92] {179};
		\node[state] (302) [above right=of 165] {302};
		\node[state] (349) [above left=of 179] {349};
		\node[state] (598) [above right=1 and 0.3 of 301] {598};
		\node[state] (651) [above right=2 and 3 of 349] {651};
		\node[state] (1190) [above left=of 598] {1190};
		\node[state] (1841) [above left=of 651] {1841};
		
		\path (69) edge (137)
		      (68) edge (137)
		      (86) edge (165)
		      (79) edge (165)
		      (92) edge (179)
		      (87) edge (179)
		      (165) edge (302)
		      (137) edge (302)
		      (170) edge (349)
		      (179) edge (349)
		      (301) edge (598)
		      (297) edge (598)
		      (592) edge (1190)
		      (598) edge (1190)
		      (349) edge (651)
		      (302) edge (651)
		      (1190) edge (1841)
		      (651) edge (1841);
	\end{tikzpicture}
\end{document}
