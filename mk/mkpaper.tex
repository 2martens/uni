\documentclass[12pt,twoside,ngerman]{scrartcl}
%%%%%%%%%%%%%%%%%%%%%%%%%%%%%%%%%%%%%%%%%%%%%%%%%%%%%%%%%%%%%
% Meta informations:
\newcommand{\trauthor}{Jim Martens}
\newcommand{\trtype}{Paper} %{Seminararbeit} %{Proseminararbeit}
\newcommand{\trcourse}{Einf\"uhrung in das wissenschaftliche Arbeiten}
\newcommand{\trtitle}{Elektronische Demokratie}
\newcommand{\trmatrikelnummer}{6420323}
\newcommand{\tremail}{2martens@informatik.uni-hamburg.de}
\newcommand{\trarbeitsbereich}{}
\newcommand{\trdate}{06.11.2013}

%%%%%%%%%%%%%%%%%%%%%%%%%%%%%%%%%%%%%%%%%%%%%%%%%%%%%%%%%%%%%
% Languages:

% Falls die Ausarbeitung in Deutsch erfolgt:
\usepackage[ngerman]{babel}
\usepackage[T1]{fontenc}
%\usepackage[latin1]{inputenc}
\usepackage[utf8]{inputenc}	 				
\selectlanguage{ngerman}

% If the thesis is written in English:
%\usepackage[english]{babel} 						
%\selectlanguage{english}
%\addto{\captionsenglish}{\renewcommand{\refname}{Bibliography}}
%%%%%%%%%%%%%%%%%%%%%%%%%%%%%%%%%%%%%%%%%%%%%%%%%%%%%%%%%%%%%
% Bind packages:
\usepackage{acronym}                    % Acronyms
\usepackage{algorithmic}								% Algorithms and Pseudocode
\usepackage{algorithm}									% Algorithms and Pseudocode
\usepackage{amsfonts}                   % AMS Math Packet (Fonts)
\usepackage{amsmath}                    % AMS Math Packet
\usepackage{amssymb}                    % Additional mathematical symbols
\usepackage{amsthm}
\usepackage{booktabs}                   % Nicer tables
%\usepackage[font=small,labelfont=bf]{caption} % Numbered captions for figures
\usepackage{color}                      % Enables defining of colors via \definecolor
\definecolor{uhhRed}{RGB}{254,0,0}		  % Official Uni Hamburg Red
\definecolor{uhhGrey}{RGB}{122,122,120} % Official Uni Hamburg Grey
\usepackage{fancybox}                   % Gleichungen einrahmen
\usepackage{fancyhdr}										% Packet for nicer headers
%\usepackage{fancyheadings}             % Nicer numbering of headlines

%\usepackage[outer=3.35cm]{geometry} 	  % Type area (size, margins...) !!!Release version
%\usepackage[outer=2.5cm]{geometry} 		% Type area (size, margins...) !!!Print version
%\usepackage{geometry} 									% Type area (size, margins...) !!!Proofread version
\usepackage[outer=3.15cm]{geometry} 	  % Type area (size, margins...) !!!Draft version
\geometry{a4paper,body={5.8in,9in}}

\usepackage{graphicx}                   % Inclusion of graphics
%\usepackage{latexsym}                  % Special symbols
\usepackage{longtable}									% Allow tables over several parges
\usepackage{listings}                   % Nicer source code listings
\usepackage{multicol}										% Content of a table over several columns
\usepackage{multirow}										% Content of a table over several rows
\usepackage{rotating}										% Alows to rotate text and objects
\usepackage[hang]{subfigure}            % Allows to use multiple (partial) figures in a fig
%\usepackage[font=footnotesize,labelfont=rm]{subfig}	% Pictures in a floating environment
\usepackage{tabularx}										% Tables with fixed width but variable rows
\usepackage{url,xspace,boxedminipage}   % Accurate display of URLs

%%%%%%%%%%%%%%%%%%%%%%%%%%%%%%%%%%%%%%%%%%%%%%%%%%%%%%%%%%%%%
% Configurationen:

\parskip 12pt plus 1pt minus 1pt
\parindent 0pt

\hyphenation{whe-ther} 									% Manually use: "\-" in a word: Staats\-ver\-trag

%\lstloadlanguages{C}                   % Set the default language for listings
\DeclareGraphicsExtensions{.pdf,.svg,.jpg,.png,.eps} % first try pdf, then eps, png and jpg
\graphicspath{{./src/}} 								% Path to a folder where all pictures are located
\pagestyle{fancy} 											% Use nicer header and footer

% Redefine the environments for floating objects:
\setcounter{topnumber}{3}
\setcounter{bottomnumber}{2}
\setcounter{totalnumber}{4}
\renewcommand{\topfraction}{0.9} 			  %Standard: 0.7
\renewcommand{\bottomfraction}{0.5}		  %Standard: 0.3
\renewcommand{\textfraction}{0.1}		  	%Standard: 0.2
\renewcommand{\floatpagefraction}{0.8} 	%Standard: 0.5

% Tables with a nicer padding:
\renewcommand{\arraystretch}{1.2}

%%%%%%%%%%%%%%%%%%%%%%%%%%%%
% Additional 'theorem' and 'definition' blocks:
\theoremstyle{plain}
\newtheorem{theorem}{Theorem}[section]
%\newtheorem{theorem}{Satz}[section]		% Wenn in Deutsch geschrieben wird.
\newtheorem{axiom}{Axiom}[section] 	
%\newtheorem{axiom}{Fakt}[chapter]			% Wenn in Deutsch geschrieben wird.
%Usage:%\begin{axiom}[optional description]%Main part%\end{fakt}

\theoremstyle{definition}
\newtheorem{definition}{Definition}[section]

%Additional types of axioms:
\newtheorem{lemma}[axiom]{Lemma}
\newtheorem{observation}[axiom]{Observation}

%Additional types of definitions:
\theoremstyle{remark}
%\newtheorem{remark}[definition]{Bemerkung} % Wenn in Deutsch geschrieben wird.
\newtheorem{remark}[definition]{Remark} 

%%%%%%%%%%%%%%%%%%%%%%%%%%%%
% Provides TODOs within the margin:
\newcommand{\TODO}[1]{\marginpar{\emph{\small{{\bf TODO: } #1}}}}

%%%%%%%%%%%%%%%%%%%%%%%%%%%%
% Abbreviations and mathematical symbols
\newcommand{\modd}{\text{ mod }}
\newcommand{\RS}{\mathbb{R}}
\newcommand{\NS}{\mathbb{N}}
\newcommand{\ZS}{\mathbb{Z}}
\newcommand{\dnormal}{\mathit{N}}
\newcommand{\duniform}{\mathit{U}}

\newcommand{\erdos}{Erd\H{o}s}
\newcommand{\renyi}{-R\'{e}nyi}
%%%%%%%%%%%%%%%%%%%%%%%%%%%%%%%%%%%%%%%%%%%%%%%%%%%%%%%%%%%%%
% Document:
\begin{document}
\renewcommand{\headheight}{14.5pt}

\fancyhead{}
\fancyhead[LE]{ \slshape \trauthor}
\fancyhead[LO]{}
\fancyhead[RE]{}
\fancyhead[RO]{ \slshape \trtitle}

%%%%%%%%%%%%%%%%%%%%%%%%%%%%
% Cover Header:
\begin{titlepage}
	\begin{flushleft}
		Universit\"at Hamburg\\
		Fachbereich Informatik\\
%		\trarbeitsbereich\\
	\end{flushleft}
	\vspace{3.5cm}
	\begin{center}
		\huge \trtitle\\
	\end{center}
	\vspace{3.5cm}
	\begin{center}
		\normalsize\trtype\\
		[0.2cm]
		\Large\trcourse\\
		[1.5cm]
		\Large \trauthor\\
%		[0.2cm]
%		\normalsize Matr.Nr. \trmatrikelnummer\\
		[0.2cm]
		\normalsize\tremail\\
		[1.5cm]
		\Large \trdate
	\end{center}
	\vfill
\end{titlepage}

	%backsite of cover sheet is empty!
\thispagestyle{empty}
\hspace{1cm}
\newpage

%%%%%%%%%%%%%%%%%%%%%%%%%%%%
% Abstract:

% Abstract gives a brief summary of the main points of a paper:
\section*{Abstract}
	Elektronische Demokratie bietet viele Chancen zur Verbesserung der vorhandenen Interaktion mit Regierungen in Demokratien. Allerdings birgt sie auch einige Risiken, die jedoch nicht unüberwindbar sind.

% Lists:
\setcounter{tocdepth}{2} 					% depth of the table of contents (for Seminars 2 is recommended)
\tableofcontents
\pagenumbering{arabic}
\clearpage

%%%%%%%%%%%%%%%%%%%%%%%%%%%%
% Content:

% the actual content, usually separated over a number of sections
% each section is assigned a label, in order to be able to put a
% crossreference to it

\section{Einleitung}
\label{sec:introduction}

	Elektronische Demokratie hat viele Bedeutungen. Die wohl eingängigste ist das Verlagern des Wahlvorganges selber auf Computer oder sogar das Internet.\cite{Mohen2001} Solch ein Vorgehen ist mit vielen Hürden verbunden, da sichergestellt werden muss, dass die Wahl nicht gefälscht werden kann, keine versehentliche doppelte Stimmabgabe möglich ist und nur Wahlberechtigte abstimmen können. Ein solches Vorgehen wird von Mohen\cite{Mohen2001} anhand einer Fallstudie aus Arizona beschrieben. Diese Art des elektronischen Wählens wird auch E-Voting genannt.\cite{Spirakis2010}
	
	Allerdings ist dies nur ein Aspekt von elektronischer Demokratie (E-Democracy). E-Democracy hat weitaus mehr Bedeutungen. Watson\cite{Watson2001} beschreibt eine strategische Perspektive zur Umsetzung von elektronischer Demokratie. Für ihn gehört zu E-Democracy ebenso auch E-Government und E-Politics.
	
	Eine weitere Sicht auf E-Democracy wird durch Spirakis\cite{Spirakis2010} vorgenommen. Dabei wird der Einfluss von E-Government auf die Demokratie in den Fokus genommen und Literatur in drei Teilen analysiert: (a) der Definition und Bedeutung von E-Government, (b) der Informationspolitik und (c) E-Democracy durch Information und E-Participation.
	
	In diesem Paper werden Erkenntnisse von Spirakis vorgestellt und schließlich die Chancen und Risiken von E-Democracy anhand von Spirakis beleuchtet. Abschließend wird ein Fazit gezogen.
	
\section{Erkenntnisse Spirakis}
\label{sec:erkenntnisseSpirakis}

\subsection*{Definition von E-Government}
	E-Government beschreibt die Nutzung von Informations- und Kommunikationstechnologien (IKT) während der Transformation von Regierungen zur Verbesserung der Barrierefreiheit, Effektivität und Verantwortung. Laut des Europäischen Kommittees des norwegischen Parlaments kann diese Transformation mit Veränderungen der Organisation und neuen Fähigkeiten einhergehen, sodass die öffentlichen Dienste verbessert werden, die demokratische Teilhabe vergrößert und die Implementation von öffentlichen Richtlinien gestärkt wird.\cite{Spirakis2010} 
	
\subsection*{Bedeutung von E-Government}

	E-Government bietet wichtige Vorteile auf nationaler Ebene, wie z.B. effektiven Zugang zu den öffentlichen Diensten, Kostenreduzierung öffentlicher Dienste und erhöhte Zugangsmöglichkeiten von Bürgern für Informationen bezüglich dem nationalen Budget und anderen Regierungstätigkeiten.\cite{Spirakis2010}
	
\subsection*{Informationspolitik}
	
	Es gibt keine eindeutige Definition von "`Informationspolitik"' und im Paper von Spirakis\cite{Spirakis2010} zeichnet sich auch nicht einmal eine Richtung ab. Da der Begriff der Informationspolitik bei Watson so in der Form nicht auftaucht, wird hier nicht weiter darauf eingegangen.
	
\subsection*{Elektronische Demokratie}

	Elektronische Demokratie beinhaltet technische Innovationen, die eine Verbesserung und Stärkung demokratischer Institutionen mit oder ohne Benutzung des Internets erlauben. E-Democracy ist ein Mechanismus basierend auf IKT, der es Bürgern erlaubt aktiv am Entscheidungsprozess für öffentliche Angelegenheiten mitzuwirken.\cite{Spirakis2010}
	
	Das Modell von E-Democracy beinhaltet vier Schritte. Im ersten Schritt machen die meisten Regierungsorganisationen Informationen im Internet verfügbar. Durch den zweiten Schritt entwickelt sich eine Kommunikation in beide Richtungen: von der Regierung zu den Bürgern und umgekehrt. Der dritte Schritt meint den Wandel von Kommunikation zu Zusammenarbeit. Schließlich zeigt der vierte Schritt den Einfluss der Bürger auf das Resultat des Entscheidungsprozesses.\cite{Spirakis2010}

\section{Chancen und Risiken von E-Democracy}
\label{sec:chanceRisk}

	E-Democracy hat wie jede Sache Chancen und Risiken. Inwieweit diese umgesetzt werden können bzw. eintreffen hängt von dem jeweiligen staatlichen Gebilde und dessen Rahmenbedingungen ab.\cite{Spirakis2010}
	
	Dabei ist keine Sache perfekt, auch nicht Demokratie. Es gibt einen andauernden Prozess neuer Herausforderungen für Bürger und Regierungen. Die womöglich wichtigste Herausforderung ist die aktive Teilnahme der Bürger am politischen Prozess, denn traditionelle repräsentative Demokratien beschränken die Aktivität der Bürger meist auf das Wählen.\cite{Spirakis2010}
	
	E-Democracy ermöglicht die aktivere Teilnahme von Bürgern am politischen Geschehen und verbessert die Kommunikation und schlussendlich die Zusammenarbeit zwischen Bürgern und Regierung, sowie Bürgern und Bürgern. Dies geschieht sowohl auf nationaler als auch auf subnationaler Ebene. Allerdings ist eine wichtige Voraussetzung für E-Democracy, dass die demokratischen Prinzipien, wie Meinungsfreiheit, Menschenrechte und Gesetz eingehalten werden. Denn E-Democracy kann nur ein Instrument oder Werkzeug für den politischen Entscheidungsprozess sein. Diese Einschränkungen sind nötig, damit das volle Potenzial ausgeschöpft werden kann.\cite{Spirakis2010}
	
	Allerdings gibt es auch einige Risiken oder Nachteile von E-Democracy. So basiert es auf Informatik, der damit verbundenen Software und der Möglichkeit von Bürgern auf diese elektronischen Dienste zuzugreifen. Wird in Betracht gezogen, dass Teile der Bevölkerung arm und schlecht gebildet sind, dann ergibt sich ein großer Nachteil, denn diese Bevölkerungsgruppen sind von E-Democracy ausgeschlossen. Die Kosten von Computern und anderen notwendigen technischen Geräten, sowie die Verfügbarkeit einer Internetverbindung sind Hauptfaktoren bei der Implementation von E-Democracy. Desweiteren ist die Technologieabdeckung auf die Welt bezogen längst nicht homogen. Viele isolierte Gebiete haben keinen Anschluss an das Internet oder neue Technologien im Allgemeinen. Dies birgt ein großes Risiko bei der Umsetzung von E-Democracy.\cite{Spirakis2010}
	
\section{Fazit}
\label{sec:concl}

	E-Democracy bietet riesige Chancen für die positive Entwicklung einer Gesellschaft, indem sich Regierungen und Bürger auf Augenhöhe begegnen und als Partner ansehen und nicht als Gegner.
	Dafür ist es aber nötig, dass alle Bürger Zugang zu E-Democracy haben, was durch deren Abhängigkeit von Computertechnologie zur Zeit noch sehr schwierig ist. Mit der Entwicklung eines kostengünstigen Systems für E-Democracy kann dies jedoch behoben werden.
	
	Von zentraler Wichtigkeit ist, dass die demokratischen Prinzipien auch für E-Democracy gelten.

%%%%%%%%%%%%%%%%%%%%%%%%%%%%%%%%%%%%%%
% hier werden - zum Ende des Textes - die bibliographischen Referenzen
% eingebunden
%
% Insbesondere stehen die eigentlichen Informationen in der Datei
% ``bib.bib''
%
\clearpage
\bibliography{bib}
\bibliographystyle{ieeetr}
\addcontentsline{toc}{section}{Literatur}% Add to the TOC

\end{document}


