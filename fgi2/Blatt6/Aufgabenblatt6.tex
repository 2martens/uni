\documentclass[10pt,a4paper,oneside,ngerman,numbers=noenddot]{scrartcl}
\usepackage[T1]{fontenc}
\usepackage[utf8x]{inputenc}
\usepackage[ngerman]{babel}
\usepackage{amsmath}
\usepackage{amsfonts}
\usepackage{amssymb}
\usepackage{paralist}
\usepackage{gauss}
\usepackage{pgfplots}
\usepackage[locale=DE,exponent-product=\cdot,detect-all]{siunitx}
\usepackage{tikz}
\usetikzlibrary{automata,matrix,fadings,calc,positioning,decorations.pathreplacing,arrows,decorations.markings}
\usepackage{polynom}
\usepackage{multirow}
\usepackage[german]{fancyref}
\polyset{style=C, div=:,vars=x}
\pgfplotsset{compat=1.8}
\pagenumbering{arabic}
% ensures that paragraphs are separated by empty lines
\parskip 12pt plus 1pt minus 1pt
\parindent 0pt
% define how the sections are rendered
\def\thesection{6.\arabic{section})}
\def\thesubsection{\arabic{subsection}.}
\def\thesubsubsection{(\alph{subsubsection})}
% some matrix magic
\makeatletter
\renewcommand*\env@matrix[1][*\c@MaxMatrixCols c]{%
  \hskip -\arraycolsep
  \let\@ifnextchar\new@ifnextchar
  \array{#1}}
\makeatother

\begin{document}
\author{Benjamin Kuffel, Jim Martens\\Gruppe 6}
\title{Hausaufgaben zum 24. November}
\maketitle

\setcounter{section}{2}
\section{} %6.3
	\subsection{}
	\begin{alignat*}{2}
		VC(\phi_{01}) &=& (1,0,0,0) \\
		VC(\phi_{11}) &=& (1,1,0,0) \\
		VC(\phi_{21}) &=& (0,0,1,0) \\
		VC(\phi_{31}) &=& (0,0,0,1) \\
		VC(\phi_{02}) &=& (2,0,0,0) \\
		VC(\phi_{12}) &=& (1,2,0,1) \\
		VC(\phi_{22}) &=& (2,0,2,0) \\
		VC(\phi_{32}) &=& (0,0,1,2) \\
		VC(\phi_{33}) &=& (0,0,1,3) \\
		VC(\phi_{03}) &=& (3,0,1,3) \\
		VC(\phi_{13}) &=& (1,3,0,1) \\
		VC(\phi_{34}) &=& (0,0,1,4) \\
		VC(\phi_{23}) &=& (2,0,3,3) \\
		VC(\phi_{04}) &=& (4,0,1,3) \\
		VC(\phi_{24}) &=& (2,3,4,3) \\
		VC(\phi_{35}) &=& (4,0,1,4) \\
		VC(\phi_{25}) &=& (2,3,5,3) \\
		VC(\phi_{05}) &=& (5,3,5,3) \\
		VC(\phi_{06}) &=& (6,3,5,3) \\
		VC(\phi_{14}) &=& (6,4,5,3) \\
		VC(\phi_{36}) &=& (4,0,	1,5) \\
		VC(\phi_{26}) &=& (4,3,6,5)
	\end{alignat*}
	
	\subsection{}
	\begin{alignat*}{1}
		p_{0}&: \phi_{03}\\
		p_{1}&: \phi_{14}\\
		p_{2}&: \phi_{21}\\
		p_{3}&: \phi_{33}
	\end{alignat*}
	
	\subsection{}
	\begin{alignat*}{1}
		p_{0}&: \phi_{02}\\
		p_{1}&: \phi_{11}\\
		p_{2}&: \phi_{21}\\
		p_{3}&: \phi_{31}
	\end{alignat*}
\section{} %6.4
\end{document}
