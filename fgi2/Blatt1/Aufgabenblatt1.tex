\documentclass[10pt,a4paper,oneside,ngerman,numbers=noenddot]{scrartcl}
\usepackage[T1]{fontenc}
\usepackage[utf8]{inputenc}
\usepackage[ngerman]{babel}
\usepackage{amsmath}
\usepackage{amsfonts}
\usepackage{amssymb}
\usepackage{paralist}
\usepackage{gauss}
\usepackage{pgfplots}
\usepackage[locale=DE,exponent-product=\cdot,detect-all]{siunitx}
\usepackage{tikz}
\usetikzlibrary{automata,matrix,fadings,calc,positioning,decorations.pathreplacing,arrows,decorations.markings}
\usepackage{polynom}
\usepackage{multirow}
\usepackage[german]{fancyref}
\polyset{style=C, div=:,vars=x}
\pgfplotsset{compat=1.8}
\pagenumbering{arabic}
% ensures that paragraphs are separated by empty lines
\parskip 12pt plus 1pt minus 1pt
\parindent 0pt
% define how the sections are rendered
\def\thesection{1.\arabic{section})}
\def\thesubsection{\arabic{subsection}.}
\def\thesubsubsection{(\roman{subsubsection})}
% some matrix magic
\makeatletter
\renewcommand*\env@matrix[1][*\c@MaxMatrixCols c]{%
  \hskip -\arraycolsep
  \let\@ifnextchar\new@ifnextchar
  \array{#1}}
\makeatother

\begin{document}
\author{Benjamin Kuffel, Jim Martens, Sabrina Mehrens\\Gruppe 6}
\title{Hausaufgaben zum 20. Oktober}
\maketitle

\setcounter{section}{2}
\section{} %1.3
	\subsection{} %1.
	\[\underbrace{aaa \cdots aaa}_{\text{n times}}d + d + \underbrace{\underbrace{aaa \cdots aaa}_{\text{2k - 1 times}} \cdot c \cdot \underbrace{bbb \cdots bbb}_{\text{2k - 1 times}}}_{\forall k \in \mathbb{N}| 0 < k \leq \frac{1}{2}n} + \underbrace{\underbrace{aaa \cdots aaa}_{\text{2k times}} \cdot d \cdot \underbrace{bbb \cdots bbb}_{\text{2k times}}}_{\forall k \in \mathbb{N}| 0 < k \leq \frac{1}{2}n}\]
	\subsection{}
	\[L(A_{n}) = \bigcup\limits_{k = 1}^{\frac{1}{2}n} \left\lbrace \underbrace{aaa \cdots aaa}_{\text{2k -1 times}} c \underbrace{bbb \cdots bbb}_{\text{2k -1 times}}\right\rbrace \cup \bigcup\limits_{k = 1}^{\frac{1}{2}n} \left\lbrace \underbrace{aaa \cdots aaa}_{\text{2k times}} d \underbrace{bbb \cdots bbb}_{\text{2k times}}\right\rbrace \cup \left\lbrace \underbrace{aaa \cdots aaa}_{\text{n times}}d\right\rbrace \cup \{d\}\]
	\subsection{}
	\subsection{}
	\(L(A_{n})\) ist für ein beliebiges fest gewähltes \(n\) regulär, da die Sprache durch einen deterministischen endlichen Automaten akzeptiert wird (siehe Aufgabe 1.3).
\section{} %1.4
\label{sec:14}
	\subsection{}
	Von einem gegebenen Automaten werden Start- und Endzustände vertauscht, sowie alle Kantenbeziehungen umgekehrt. Für den resultierenden NFA (in den meisten Fällen nicht mehr deterministisch) wird nun ein Potenzautomaten gebildet, welcher vollständig gemacht wird.
	\subsection{}
	\subsection{}
	\[f^{*}(fe)^{*}ee^{*}f(e + f)^{*}\]
	\subsection{}
	\label{subsec:144}
	Nach Umkehrung der Kantenbeziehungen und Vertauschen des Start- und Endzustands, ergibt sich der Automat auf \fref{fig:1}.
	% TODO: Automat
	\begin{figure}
	\begin{tikzpicture}[node distance=2cm]
		\node[state,initial] (p3) {\(p_{3}\)};
		\node[state] (p2) [right=of p3] {\(p_{2}\)};
		\node[state] (p1) [right=of p2] {\(p_{1}\)};
		\node[state,accepting] (p0) [right=of p1] {\(p_{0}\)};
		\path[->] (p3) edge[loop above] node [above] {e,f} (p3)
			(p3) edge node [above] {f} (p2)
			(p2) edge[loop above] node [above] {e} (p2)
			(p2) edge node [above] {e} (p1)
			(p1) edge[bend left] node [above] {e} (p0)
			(p0) edge[bend left] node [below] {f} (p1)
			(p0) edge[loop above] node [above] {f} (p0);
	\end{tikzpicture}
	\caption{Automat \(A'\) nach Umkehrung aller Kantenbeziehungen  und Vertauschen des Start- und Endzustands des Automaten \(A\)}
	\label{fig:1}
	\end{figure}
	
	Anschließend wird der Potenzautomat gebildet, der auf \fref{fig:2} zu sehen ist.
	
	\begin{figure}
	\begin{tikzpicture}[node distance=2cm]
		\node[state,initial] (p3) {\(\{p_{3}\}\)};
		\node[state] (p3p2) [right=of p3] {\(\{p_{2},p_{3}\}\)};
		\node[state] (p3p2p1) [right=of p3p2] {\(\{p_{1},p_{2},p_{3}\}\)};
		\node[state,accepting] (p3p2p1p0) [below=of p3p2p1] {\(\{p_{0},p_{1},p_{2},p_{3}\}\)};
		\path[->] (p3) edge[loop above] node [above] {e} (p3)
			(p3) edge node [above] {f} (p3p2)
			(p3p2) edge[bend left] node [above] {e} (p3p2p1)
			(p3p2) edge[loop above] node [above] {f} (p3p2)
			(p3p2p1) edge node [right] {e} (p3p2p1p0)
			(p3p2p1) edge[bend left] node [below] {f} (p3p2)
			(p3p2p1p0) edge[loop right] node [right] {e,f} (p3p2p1p0);
	\end{tikzpicture}
	\caption{Potenzautomat \(A''\) des Automaten \(A'\)}
	\label{fig:2}
	\end{figure}
	
	Da der somit gebildete Automat bereits vollständig ist, erübrigt sich ein extra Schritt zum Vervollständigen, womit das Verfahren vollständig durchlaufen wurde.
	\subsection{}
	Der auf \fref{fig:2} zu sehende Automat lässt sich durch den folgenden regulären Ausdruck beschreiben:
	
	\[e^{*}ff^{*}e(f \cdot e)^{*}e(e + f)^{*}\]
	
	Anhand dieses Ausdrucks wird deutlich, dass die Bedingung an den Automaten erfüllt wird (alle durch \(A\) akzeptierten Worte müssen in gespiegelter Form durch den Automaten \(A''\) akzeptiert werden): Es werden alle Worte mit einem \(fee\) an einer beliebigen Stelle akzeptiert. Da durch \(A\) alle Worte mit einem \(eef\) an beliebiger Stelle akzeptiert werden, wird die Gesamtheit der akzeptierten Worte von \(A\) in gespiegelter Form durch \(A''\) akzeptiert.
\end{document}
