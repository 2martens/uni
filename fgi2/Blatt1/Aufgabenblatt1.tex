\documentclass[10pt,a4paper,oneside,ngerman,numbers=noenddot]{scrartcl}
\usepackage[T1]{fontenc}
\usepackage[utf8]{inputenc}
\usepackage[ngerman]{babel}
\usepackage{amsmath}
\usepackage{amsfonts}
\usepackage{amssymb}
\usepackage{paralist}
\usepackage{gauss}
\usepackage{pgfplots}
\usepackage[locale=DE,exponent-product=\cdot,detect-all]{siunitx}
\usepackage{tikz}
\usetikzlibrary{matrix,fadings,calc,positioning,decorations.pathreplacing,arrows,decorations.markings}
\usepackage{polynom}
\usepackage{multirow}
\polyset{style=C, div=:,vars=x}
\pgfplotsset{compat=1.8}
\pagenumbering{arabic}
% ensures that paragraphs are separated by empty lines
\parskip 12pt plus 1pt minus 1pt
\parindent 0pt
% define how the sections are rendered
\def\thesection{1.\arabic{section})}
\def\thesubsection{\arabic{subsection}.}
\def\thesubsubsection{(\roman{subsubsection})}
% some matrix magic
\makeatletter
\renewcommand*\env@matrix[1][*\c@MaxMatrixCols c]{%
  \hskip -\arraycolsep
  \let\@ifnextchar\new@ifnextchar
  \array{#1}}
\makeatother

\begin{document}
\author{Jim Martens (6420323)}
\title{Hausaufgaben zum 20. Oktober}
\maketitle

\setcounter{section}{2}
\section{} %1.3
	\subsection{} %1.
	\[\underbrace{aaa \cdots aaa}_{\text{n times}}d + d + \underbrace{\underbrace{aaa \cdots aaa}_{\text{2k - 1 times}} \cdot c \cdot \underbrace{bbb \cdots bbb}_{\text{2k - 1 times}}}_{\forall k \in \mathbb{N}| 0 < k \leq \frac{1}{2}n} + \underbrace{\underbrace{aaa \cdots aaa}_{\text{2k times}} \cdot d \cdot \underbrace{bbb \cdots bbb}_{\text{2k times}}}_{\forall k \in \mathbb{N}| 0 < k \leq \frac{1}{2}n}\]
	\subsection{}
	\[\bigcup\limits_{k = 1}^{\frac{1}{2}n} \left\lbrace \underbrace{aaa \cdots aaa}_{\text{2k -1 times}} c \underbrace{bbb \cdots bbb}_{\text{2k -1 times}}\right\rbrace \cup \bigcup\limits_{k = 1}^{\frac{1}{2}n} \left\lbrace \underbrace{aaa \cdots aaa}_{\text{2k times}} d \underbrace{bbb \cdots bbb}_{\text{2k times}}\right\rbrace \cup \left\lbrace \underbrace{aaa \cdots aaa}_{\text{n times}}d\right\rbrace \cup \{d\}\]
	\subsection{}
	\subsection{}
	\(L(A_{n})\) ist für ein beliebigesb fest gewähltes \(n\) regulär, da die Sprache durch einen deterministischen endlichen Automaten akzeptiert wird (siehe Aufgabe 1.3).
\section{} %1.4
	\subsection{}
	Von einem gegebenen Automaten werden Start- und Endzustände vertauscht, sowie alle Kantenbeziehungen umgekehrt. Für den resultierenden NFA (in den meisten Fällen nicht mehr deterministisch) wird nun ein Potenzautomaten gebildet, welcher vollständig gemacht wird.
	\subsection{}
	\subsection{}
	\[f^{*}(fe)^{*}ee^{*}f(e + f)^{*}\]
	\subsection{}
	Nach Umkehrung der Kantenbeziehungen und Vertauschen des Start- und Endzustands, ergibt sich dieser Automat.
	% TODO: Automat
	
	Anschließend wird der Potenzautomat gebildet.
	
	Abschließend wird dieser vollständig gemacht.
	\subsection{}
\end{document}
