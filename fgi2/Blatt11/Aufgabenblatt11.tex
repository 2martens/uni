\documentclass[10pt,a4paper,oneside,ngerman,numbers=noenddot]{scrartcl}
\usepackage[T1]{fontenc}
\usepackage[utf8x]{inputenc}
\usepackage[ngerman]{babel}
\usepackage{amsmath}
\usepackage{amsfonts}
\usepackage{amssymb}
\usepackage{paralist}
\usepackage{gauss}
\usepackage{pgfplots}
\usepackage[locale=DE,exponent-product=\cdot,detect-all]{siunitx}
\usepackage{tikz}
\usetikzlibrary{automata,matrix,fadings,calc,positioning,decorations.pathreplacing,arrows,decorations.markings,petri,shapes}
\usepackage{polynom}
\usepackage{multirow}
\usepackage[german]{fancyref}
\usepackage{morefloats}
\polyset{style=C, div=:,vars=x}
\pgfplotsset{compat=1.8}
\pagenumbering{arabic}
% ensures that paragraphs are separated by empty lines
\parskip 12pt plus 1pt minus 1pt
\parindent 0pt
% define how the sections are rendered
\def\thesection{11.\arabic{section})}
\def\thesubsection{\arabic{subsection}.}
\def\thesubsubsection{(\alph{subsubsection})}
% some matrix magic
\makeatletter
\renewcommand*\env@matrix[1][*\c@MaxMatrixCols c]{%
  \hskip -\arraycolsep
  \let\@ifnextchar\new@ifnextchar
  \array{#1}}
\makeatother

\tikzset{
    place/.style={
        circle,
        thick,
        draw=black,
        fill=white,
        minimum size=6mm,
        font=\bfseries
    },
    transitionH/.style={
        rectangle,
        thick,
        draw=black,
        fill=white,
        minimum width=8mm,
        inner ysep=4pt,
        font=\bfseries
    },
    transitionV/.style={
        rectangle,
        thick,
        fill=black,
        minimum height=8mm,
        inner xsep=2pt
    }
}

\renewcommand{\vec}[1]{\underline{#1}}

\begin{document}
\author{Benjamin Kuffel, Jim Martens\\Gruppe 6}
\title{Hausaufgaben zum 12. Januar}
\maketitle

\setcounter{section}{2}
\section{} %11.3
\subsection{}
Mithilfe der Regeln können Workflownetze soweit vergröbert werden, bis nur noch der Start- und Endplatz und eine Transition übrig bleiben. Lässt sich ein Netz nicht in einer solchen Weise mit den gegebenen Regeln vergröbern, dann ist es kein Workflownetz.

\subsection{}
Das kleinste korrekte Workflownetz besteht aus zwei Plätzen und einer Transition, wobei die beiden Plätze den Start- bzw. Endplatz darstellen.

\setcounter{subsection}{3}
\subsection{}
\begin{tabular}{l|l|l}
    Nr. & Transformationsregel & Hinzugekommene Elemente \\
    \hline
    0 & - & p17 \\
    1 & P-Seq & p16, t9 \\
    2 & P-Seq & p14, t8 \\
    3 & Par & p15 \\
    4 & P-Seq & p6, t7 \\
    5 & Par & p13 \\
    6 & P-Seq & a, t1 \\
    7 & Par & p12 \\
    8 & P-Seq & p8, t6 \\
    9 & Par & p11 \\
    10 & P-Seq & p3, t4 \\
    11 & Par & p10 \\
    12 & P-Seq & p7, t5 \\
    13 & Par & p9 \\
    14 & P-Seq & p2, t2 \\
    15 & Par & p5 \\
    16 & P-Seq & p4, t3
\end{tabular}

\setcounter{subsection}{5}
\subsection{}
\begin{tabular}{l|l|l|l}
    Nr. & Transformationsregel & Verbliebene Knoten & Entfernte Knoten \\
    \hline
   1 & Zusammenfassung & p17,t10,t11,p18,p19,t12,p20,t15,p23 & t14,p22,t21,p31,t22,p32 \\
   & & p24,t16,t17,p25,p26,t13,p21,t18,p27 & p33,t23,p34,t25\\
   & & t19,p28,p29,t20,p30,t24,p35,t26,p36,t28,e & \\
   \hline
   2 & P-Seq & p17,t10,t11,p18,p19,t12,p20,t15,p23 & t19,p27 \\
   & & p24,t16,t17,p25,p26,t13,p21,t18,p28 &\\
   & & p29,t20,p30,t24,p35,t26,p36,t28,e & \\
   \hline
   3 & Par & p17,t10,t11,p18,p19,t12,p20,t15,p23 & p28 \\
   & & p24,t16,t17,p25,p26,t13,p21,t18,p29 &\\
   & & t20,p30,t24,p35,t26,p36,t28,e & \\
   \hline
   4 & P-Seq & p17,t10,t11,p18,p19,t12,p20,t15,p23,p24 & p29,t20 \\
   & & t16,t17,p25,p26,t13,p21,t18 &\\
   & & p30,t24,p35,t26,p36,t28,e & \\
   \hline
   5 & P-Seq & p17,t10,t11,p18,p19,t12,p20,t15,p23,p24 & p30,t24 \\
   & & t16,t17,p25,p26,t13,p21,t18 &\\
   & & p35,t26,p36,t28,e &\\
   \hline
   6 & T-Seq & p17,t10,t11,p18,p19,t12,p20,t15,p23,p24 & p35,t26 \\
   & & t16,t17,p25,p26,t13,p21,t18 &\\
   & & p36,t28,e &\\
   \hline
   7 & P-Seq & p17,t10,t11,p18,p19,t12,p20,t15,p24 & p23,t16 \\
   & & t17,p25,p26,t13,p21,t18 &\\
   & & p36,t28,e &\\
   \hline
   8 & Par & p17,t10,t11,p18,p19,t12,p20,t15,p24 & p25 \\
   & & t17,p26,t13,p21,t18 &\\
   & & p36,t28,e &\\
   \hline
   9 & Zusammenfassen & p17,t10,p18,t12,p20,t13,p21 & t11,p19,t15,p24 \\
   & & t18,p36,t28,e &t17,p26,t27\\
   \hline
   10 & T-Seq & p17,t10,p18,t12,p20,t13,p21 & p36,t28 \\
   & & t18,e &\\
   \hline
   11 & T-Seq & p17,t10,p18,t12,p21,t18,e & p20,t13 \\
   \hline
   12 & T-Seq & p17,t10,p21,t18,e & p18,t12 \\
   \hline
   13 & T-Seq & p17,t10,e & p21,t18 \\
   \hline
   14 & P-Seq & e & p17,t10
\end{tabular}

\subsection{}
Es ist nicht möglich die Korrektheit mit den bekannten Regeln nachzuweisen, da bei keiner Regel mehr als zwei Transitionen beteiligt sind. Im Fall von p37 sind aber genau drei Transitionen beteiligt, wodurch keine der Regeln angewendet werden kann.
\section{} %11.4
\end{document}

