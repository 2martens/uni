\documentclass[10pt,a4paper,oneside,ngerman,numbers=noenddot]{scrartcl}
\usepackage[T1]{fontenc}
\usepackage[utf8x]{inputenc}
\usepackage[ngerman]{babel}
\usepackage{amsmath}
\usepackage{amsfonts}
\usepackage{amssymb}
\usepackage{paralist}
\usepackage{gauss}
\usepackage{pgfplots}
\usepackage[locale=DE,exponent-product=\cdot,detect-all]{siunitx}
\usepackage{tikz}
\usetikzlibrary{automata,matrix,fadings,calc,positioning,decorations.pathreplacing,arrows,decorations.markings}
\usepackage{polynom}
\usepackage{multirow}
\usepackage[german]{fancyref}
\polyset{style=C, div=:,vars=x}
\pgfplotsset{compat=1.8}
\pagenumbering{arabic}
% ensures that paragraphs are separated by empty lines
\parskip 12pt plus 1pt minus 1pt
\parindent 0pt
% define how the sections are rendered
\def\thesection{7.\arabic{section})}
\def\thesubsection{\arabic{subsection}.}
\def\thesubsubsection{(\alph{subsubsection})}
% some matrix magic
\makeatletter
\renewcommand*\env@matrix[1][*\c@MaxMatrixCols c]{%
  \hskip -\arraycolsep
  \let\@ifnextchar\new@ifnextchar
  \array{#1}}
\makeatother

\begin{document}
\author{Benjamin Kuffel, Jim Martens\\Gruppe 6}
\title{Hausaufgaben zum 1. Dezember}
\maketitle

\setcounter{section}{2}
\section{} %7.3
	\setcounter{subsection}{1}
	\subsection{}
	Das Netz \(N_{1}\) ist nicht lebendig, beschränkt oder reversibel. Die Transitionen gta, rz, s und f sind lebendig. Alle Plätze außer "`Bauteile"' und "`fertige Bauteile"' sind beschränkt.
\section{} %7.4
	\subsection{}
	Der Erreichbarkeitsgraph für das Netz \(N_{7.4a}\) befindet sich auf \fref{fig:1}. Der Erreichbarkeitsgraph für das Netz \(N_{7.4b}\) befindet sich auf \fref{fig:2} und der Algorithmus hätte in der Markierung \((3,0,0,3)^T\) abgebrochen.
	\begin{figure}
	\begin{tikzpicture}[node distance=2cm]
		\node (m0) {\((0,0,2,4)^T\)};
		\node (m1) [right=of m0] {\((1,0,1,3)^T\)};
		\node (m2) [right=of m1] {\((2,0,0,2)^T\)};
		\node (m3) [below=of m2] {\((0,1,1,2)^T\)};
		\node (m4) [right=of m3] {\((1,1,0,1)^T\)};
		\path[->] (m0) edge node[above] {v} (m1)
			(m1) edge node[above] {v} (m2)
			(m2) edge node[right] {u} (m3)
			(m3) edge node[above right] {t} (m1)
			(m3) edge node[above] {v} (m4)
			(m4) edge node[above right] {t} (m2);
	\end{tikzpicture}
	\caption{Erreichbarkeitsgraph für Netz \(N_{7.4a}\)}
	\label{fig:1}
	\end{figure}
	
	\begin{figure}
	\begin{tikzpicture}[node distance=2cm]
		\node (m0) {\((0,0,1,4)^T\)};
		\node (m1) [right=of m0] {\((3,0,0,3)^T\)};
		\node (m2) [right=of m1] {\((1,2,1,3)^T\)};
		\node (m3) [below=of m2] {\((3,0,1,4)^T\)};
		\node (m4) [left=of m3] {\((3,2,0,2)^T\)};
		\node (m5) [below=of m3] {\((6,0,0,3)^T\)};
		\node (m6) [below=of m5] {\((4,2,1,3)^T\)};
		\node (m7) [left=of m5] {\((1,2,2,4)^T\)};
		\node (m8) [left=of m7] {\((1,4,1,2)^T\)};
		\node (m9) [left=of m4] {\((5,0,0,3)^T\)};
		\path[->] (m0) edge node[above] {v} (m1)
			(m1) edge node[above] {u} (m2)
			(m2) edge node[right] {t} (m3)
			(m2) edge node[below right] {v} (m4)
			(m3) edge node[right] {v} (m5)
			(m4) edge node[below right] {u} (m8)
			(m4) edge node[above] {t} (m9)
			(m5) edge node[right] {u} (m6)
			(m3) edge node[below right] {u} (m7);
	\end{tikzpicture}
	\caption{Erreichbarkeitsgraph für Netz \(N_{7.4b}\)}
	\label{fig:2}
	\end{figure}
\end{document}
