\documentclass[10pt,a4paper,oneside,ngerman,numbers=noenddot]{scrartcl}
\usepackage[T1]{fontenc}
\usepackage[utf8]{inputenc}
\usepackage[ngerman]{babel}
\usepackage{amsmath}
\usepackage{amsfonts}
\usepackage{amssymb}
\usepackage{paralist}
\usepackage{gauss}
\usepackage{pgfplots}
\usepackage[locale=DE,exponent-product=\cdot,detect-all]{siunitx}
\usepackage{tikz}
\usetikzlibrary{automata,matrix,fadings,calc,positioning,decorations.pathreplacing,arrows,decorations.markings}
\usepackage{polynom}
\usepackage{multirow}
\usepackage[german]{fancyref}
\polyset{style=C, div=:,vars=x}
\pgfplotsset{compat=1.8}
\pagenumbering{arabic}
% ensures that paragraphs are separated by empty lines
\parskip 12pt plus 1pt minus 1pt
\parindent 0pt
% define how the sections are rendered
\def\thesection{3.\arabic{section})}
\def\thesubsection{\arabic{subsection}.}
\def\thesubsubsection{(\roman{subsubsection})}
% some matrix magic
\makeatletter
\renewcommand*\env@matrix[1][*\c@MaxMatrixCols c]{%
  \hskip -\arraycolsep
  \let\@ifnextchar\new@ifnextchar
  \array{#1}}
\makeatother

\begin{document}
\author{Benjamin Kuffel, Jim Martens\\Gruppe 6}
\title{Hausaufgaben zum 3. November}
\maketitle

\setcounter{section}{2}
\section{} %3.3
	\subsection{}
		\begin{alignat*}{2}
			L(A_{1}) &=& (a^{*}ba^{*}b)^{*}(a^{*} + a^{*}ba^{*}) \\
			L(A_{2}) &=& a^{*}b(a^{*}ba^{*}b)^{*} \\
			L^{\omega}(A_{1}) &=& (a^{*}ba^{*}b)^{\omega} + ((a^{*}ba^{*}b)^{*}a^{*}ba^{\omega}) \\
			L^{\omega}(A_{2}) &=& (a^{*}ba^{*}b)^{\omega}
		\end{alignat*}
	\subsection{}
		\begin{figure}
			\begin{tikzpicture}[node distance=2cm]
				\node[state,initial] (qs) {\((q, s)\)};
				\node[state] (rt) [right=of qs] {\((r, t)\)};
				\node[state,accepting] (pt) [above=of qs] {\((p, t)\)};
				\path[->] (qs) edge[bend left] node [above] {\(b\)} (rt)
					(qs) edge[loop below] node [below] {\(a\)} (qs)
					(rt) edge[loop right] node [right] {\(a\)} (rt)
					(rt) edge[bend left] node [below] {\(b\)} (qs)
					(qs) edge node [left] {\(b\)} (pt)
					(pt) edge[loop right] node [right] {\(a\)} (pt);
			\end{tikzpicture}
			\caption{initiale Zusammenhangskomponente des Produktautomaten \(A_{3}\)} 
			\label{fig:1}
		\end{figure}
		Siehe \fref{fig:1} für die Konstruktion.
	\subsection{}
		\begin{alignat*}{2}
			L(A_{3}) &=& (a^{*}ba^{*}b)^{*}ba^{*} \\
			L^{\omega}(A_{3}) &=& (a^{*}ba^{*}b)^{*}a^{*}ba^{\omega} \\
			L(A_{1}) \cap L(A_{2}) &=& a^{*}b(a^{*}ba^{*}b)^{*} = (a^{*}ba^{*}b)^{*}a^{*}b \\
			L^\omega(A_{1}) \cap L^{\omega}(A_{2}) &=& (a^{*}ba^{*}b)^{\omega}
		\end{alignat*}
		
		Es gilt \(L(A_{3}) \not\subseteq L(A_{1}) \cap L(A_{2})\), weil \(ba \in L(A_{3})\), aber \(ba \not\in L(A_{1}) \cap L(A_{2})\). Es gilt \(L(A_{1}) \cap L(A_{2}) \not\subseteq L(A_{3})\), weil \(ab \in L(A_{1}) \cap L(A_{2})\), aber \(ab \not\in L(A_{3})\). Daher gilt \(L(A_{3}) \neq L(A_{1}) \cap L(A_{2})\).
		
		Es gilt \(L^{\omega}(A_{3}) \not\subseteq L^{\omega}(A_{1}) \cap L^{\omega}(A_{2})\), weil \(ba^{\omega} \in L^{\omega}(A_{3})\), aber \(ba^{\omega} \not\in L^{\omega}(A_{1}) \cap L^{\omega}(A_{2})\). Es gilt \(L^{\omega}(A_{1}) \cap L^{\omega}(A_{2}) \not\subseteq L^{\omega}(A_{3})\), weil \((a^{*}ba^{*}b)^{\omega} \in L^{\omega}(A_{1}) \cap L^{\omega}(A_{2})\), aber \((a^{*}ba^{*}b)^{\omega} \not\in L^{\omega}(A_{3})\). Daher gilt \(L^{\omega}(A_{3}) \neq L^{\omega}(A_{1}) \cap L^{\omega}(A_{2})\).
	\subsection{}
		\begin{figure}
			\begin{tikzpicture}[node distance=2cm]
				\node[state,initial,accepting] (qs1) {\((q, s, 1)\)};
				\node[state] (qs2) [above=of qs1] {\((q, s, 2)\)};
				\node[state] (rt1) [right=of qs1] {\((r, t, 1)\)};
				\node[state,accepting] (pt1) [below=of qs1] {\((p, t, 1)\)};
				\node[state] (rt2) [above=of rt1] {\((r, t, 2)\)};
				\node[state] (pt2) [right=of pt1] {\((p, t, 2)\)};
				\path[->] (qs1) edge[bend right] node [above left] {\(b\)} (rt2)
					(qs1) edge node [right] {\(a\)} (qs2)
					(qs2) edge node [above] {\(b\)} (rt2)
					(qs2) edge[loop left] node [left] {\(a\)} (qs2)
					(rt2) edge node [below right] {\(a\)} (rt1)
					(rt2) edge[bend right] node [below right] {\(b\)} (qs1)
					(rt1) edge[loop right] node [right] {\(a\)} (rt1)
					(rt1) edge node [below] {\(b\)} (qs1)
					(qs1) edge[bend left] node [above right] {\(b\)} (pt2)
					(pt2) edge[bend left] node [below] {\(a\)} (pt1)
					(pt1) edge[bend left] node [below] {\(a\)} (pt2);
			\end{tikzpicture}
			\caption{initiale Zusammenhangskomponente des Produktautomaten \(A_{4}\)} 
			\label{fig:2}
		\end{figure}
		Siehe \fref{fig:2} für die Konstruktion.
	\subsection{}
		\begin{alignat*}{2}
			L(A_{4}) &=& ((aa^{*}b + b)(b + aa^{*}b))^{*} + (((aa^{*}b + b)(b + aa^{*}b))^{*} + ba(aa)^{*}) \\
			L^{\omega}(A_{4}) &=& ((aa^{*}b + b)(b + aa^{*}b))^{\omega} + (((aa^{*}b + b)(b + aa^{*}b))^{*} + ba(aa)^{\omega})	
		\end{alignat*}
		
		Es gilt \(L(A_{4}) \not\subseteq L(A_{1}) \cap L(A_{2})\), weil \(\epsilon \in L(A_{4})\), aber \(\epsilon \not\in L(A_{1}) \cap L(A_{2})\).
		Es gilt \(L(A_{1}) \cap L(A_{2}) \not\subseteq L(A_{4})\), weil \(b \in L(A_{1}) \cap L(A_{2})\), aber \(b \not\in L(A_{4})\). Damit gilt \(L(A_{4}) \neq L(A_{1}) \cap L(A_{2})\).
		
		Es gilt \(L^{\omega}(A_{4}) \not\subseteq L^{\omega}(A_{1}) \cap L^{\omega}(A_{2})\), weil \(ba(aa)^{\omega} \in L^{\omega}(A_{4})\), aber \(ba(aa)^{\omega} \not\in L^{\omega}(A_{1}) \cap L^{\omega}(A_{2})\).
		Es gilt \(L^{\omega}(A_{1}) \cap L^{\omega}(A_{2}) \subseteq L^{\omega}(A_{4})\), weil \(((aa^{*}b + b)(b + aa^{*}b))^{\omega} = (a^{+}ba^{+}b + a^{+}bb + ba^{+}b + bb)^{\omega} = (a^{*}ba^{*}b)^{\omega}\). Damit gilt \(L^{\omega}(A_{1}) \cap L^{\omega}(A_{2}) \subset L^{\omega}(A_{4})\).
\section{} %3.4	
	\(TS_{1}\) und \(TS_{2}\) sind bisimilar. Die dazugehörige Relation lautet:
	\[
		\mathcal{B}_{1} = \{(P_{0}, Q_{0}), (P_{1}, Q_{1}), (P_{2}, Q_{1}), (P_{3}, Q_{2}), (P_{0}, Q_{3})\}
	\]
	
	\(TS_{3}\) und \(TS_{4}\) sind nicht bisimilar. Dies lässt sich an der theoretisch benötigten Relation sehen.
	\[
		\mathcal{B}_{2} = \{(R_{0}, S_{0}), (R_{0}, S_{1}), (R_{1}, S_{2}), (R_{1}, S_{3}), (R_{2}, S_{2}), (R_{2}, S_{3}), (R_{3}, S_{4}), (R_{4}, S_{4})\}
	\]
	
	Es ist deutlich zu sehen, dass \(R_{1}\) und \(S_{2}\) ein Paar sein müssten. Von \(R_{1}\) gehen zwei Kanten ab, einmal \(b\) und einmal \(c\). Von \(S_{2}\) geht jedoch nur eine Kante ab und zwar eine mit \(b\). Da es keine Kante mit einem \(c\) gibt, sind die beiden Transitionsnetze nicht bisimilar.
\end{document}
