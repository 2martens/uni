\documentclass[10pt,a4paper,oneside,ngerman,numbers=noenddot]{scrartcl}
\usepackage[T1]{fontenc}
\usepackage[utf8]{inputenc}
\usepackage[ngerman]{babel}
\usepackage{amsmath}
\usepackage{amsfonts}
\usepackage{amssymb}
\usepackage{paralist}
\usepackage{gauss}
\usepackage{pgfplots}
\usepackage[locale=DE,exponent-product=\cdot,detect-all]{siunitx}
\usepackage{tikz}
\usetikzlibrary{automata,matrix,fadings,calc,positioning,decorations.pathreplacing,arrows,decorations.markings}
\usepackage{polynom}
\usepackage{multirow}
\usepackage[german]{fancyref}
\polyset{style=C, div=:,vars=x}
\pgfplotsset{compat=1.8}
\pagenumbering{arabic}
% ensures that paragraphs are separated by empty lines
\parskip 12pt plus 1pt minus 1pt
\parindent 0pt
% define how the sections are rendered
\def\thesection{2.\arabic{section})}
\def\thesubsection{\arabic{subsection}.}
\def\thesubsubsection{(\roman{subsubsection})}
% some matrix magic
\makeatletter
\renewcommand*\env@matrix[1][*\c@MaxMatrixCols c]{%
  \hskip -\arraycolsep
  \let\@ifnextchar\new@ifnextchar
  \array{#1}}
\makeatother

\begin{document}
\author{Benjamin Kuffel, Jim Martens, Sabrina Mehrens\\Gruppe 6}
\title{Hausaufgaben zum 27. Oktober}
\maketitle

\setcounter{section}{2}
\section{} %2.3
	\subsection{} % 1.
		\[L(A_{2.3}) = a \cdot (ba^{*}c)^{*} + bc(abc)^{*} \cdot (a + e)\]
		\[L^{\omega}(A_{2.3}) = a(ba^{*}c)^{\omega} + bc(abc)^{\omega}\]
		\[(L(A_{2.3}))^{\omega} = (a \cdot (ba^{*}c)^{*} + bc(abc)^{*} \cdot (a + e))^ {\omega}\]
	\subsection{} % 2.
		Es sei \(w_{1}\) ein Wort aus der Sprache \(L^{\omega}(A_{2.3})\) und \(w_{2}\) ein Wort aus der Sprache \((L(A_{2.3}))^{\omega}\). Es gelte \(w_{1} = bc(abc)^{\omega}\) und \(w_{2} = (bcabce)^{\omega}\).
		
		Bei der Sprache \(L^{\omega}(A_{2.3})\) hat man Wörter, die zum Teil aus unendlich oft auftretenden Zeichenfolgen bestehen. Die Sprache \((L(A_{2.3}))^{\omega}\) enthält alle Wörter der Sprache \(L(A_{2.3})\), welche unendlich oft hintereinander kommen können. Die erste Sprache enthält zum Beispiel nicht das Wort \(w_{2}\), da es auf \(e\) endet und dieser Endzustand nicht unendlich oft durchlaufen werden kann.
	\subsection{} % 3.
	\begin{figure}
	\begin{tikzpicture}[node distance=2cm]
		\node[state,initial] (q0) {\(q_{0}\)};
		\node[state,accepting] (q1) [below=of q0] {\(q_{1}\)};
		\node[state] (q2) [right=of q1] {\(q_{2}\)};
		\node[state] (q3) [right=of q0] {\(q_{3}\)};
		\node[state] (q4) [right=of q3] {\(q_{4}\)};
		\node[state,accepting] (q6) [right=of q4] {\(q_{6}\)};
		\node[state,accepting] (q5) [below=of q4] {\(q_{5}\)};
		\node[state] (q7) [below=of q1] {\(q_{7}\)};
		\path[->] (q0) edge node [above] {\(b\)} (q3)
			(q0) edge node [left] {\(a\)} (q1)
			(q1) edge[bend left] node [above] {\(b\)} (q2)
			(q2) edge[loop below] node [below] {\(a\)} (q2)
			(q2) edge[bend left] node [below] {\(c\)} (q1)
			(q3) edge node [above] {\(c\)} (q4)
			(q4) edge node [right] {\(a\)} (q5)
			(q5) edge node [below left] {\(b\)} (q3)
			(q4) edge node [above] {\(e\)} (q6)
			(q6) edge[bend right] node [above] {\(b\)} (q3)
			(q1) edge node [left] {\(a\)} (q7)
			(q7) edge node [below right] {\(b\)} (q2);
	\end{tikzpicture}
	\caption{Büchi-Automat \(A'_{2.3}\), der \((L(A_{2.3}))^{\omega}\) akzeptiert.} 
	\label{fig:1}
	\end{figure}
	Es sind zwei Teilmengenbeziehungen zu zeigen. Die erste Richtung besagt \(L^{\omega}(A'_{2.3}) \subseteq (L(A_{2.3}))^{\omega}\) und die zweite \((L(A_{2.3}))^{\omega} \subseteq L^{\omega}(A'_{2.3})\). Es gelte \(A'_{2.3} = (Q', \Sigma ', \delta ', Q'^{0}, F')\). Desweiteren gelte \(A_{2.3} = (Q, \Sigma, \delta, Q^{0}, F)\). Für die Konstruktion von \(A'_{2.3}\) siehe \fref{fig:1}.
	
	Zunächst wird \(L^{\omega}(A'_{2.3}) \subseteq (L(A_{2.3}))^{\omega}\) gezeigt.
	
	\begin{alignat*}{2}
	w \in L^{\omega}(A'_{2.3}) &\Rightarrow & A'_{2.3} \text{ akzeptiert }w \\
								&\Rightarrow & \exists \text{ Erfolgsrechnung von \(A'_{2.3}\) auf }w \\
								&\Rightarrow & \exists p = z_{0}z_{1}z_{2}... | z_{0} \in Q'^{0} \wedge F' \cap inf(p) \neq \emptyset \\
								&\Rightarrow & \exists z_{i} \text{ in } p | i \in \mathbb{N} : z_{i} \in F'
	\end{alignat*}
	
\section{} %2.4
	\subsection{} % 1.
		\begin{enumerate}
			\item Voraussetzen, dass \(W\) regulär und \(U\) \(\omega\)-regulär ist
			\item Zeigen, dass jedes Wort aus \(W \cdot U\) mit einem Wort aus \(U\) endet
			\item Zeigen, dass jedes Wort aus \(W \cdot U\) mindestens einen Endzustand unendlich durchläuft
		\end{enumerate}
	\subsection{} % 2.
		\begin{enumerate}
			\item endlichen Automaten zu \(W\) konstruieren
			\item Büchi-Automaten zu \(U\) konstruieren
			\item Kantenbeziehungen von jedem Endzustand des endlichen Automaten (folgend: \(A_{W}\)) zu den Folgezuständen des Startzustands des Büchi-Automaten (folgend: \(A_{U}\) ergänzen
			\item Startzustand von \(A_{U}\) zu normalem Status degradieren
			\item Endzustände von \(A_{W}\) zu normalen Stati degradieren
		\end{enumerate}
	\subsection{} % 3.
		
	\subsection{} % 4.
\end{document}
