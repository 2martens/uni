\documentclass[10pt,a4paper,oneside,ngerman,numbers=noenddot]{scrartcl}
\usepackage[T1]{fontenc}
\usepackage[utf8]{inputenc}
\usepackage[ngerman]{babel}
\usepackage{amsmath}
\usepackage{amsfonts}
\usepackage{amssymb}
\usepackage{paralist}
\usepackage{gauss}
\usepackage{pgfplots}
\usepackage[locale=DE,exponent-product=\cdot,detect-all]{siunitx}
\usepackage{tikz}
\usetikzlibrary{automata,matrix,fadings,calc,positioning,decorations.pathreplacing,arrows,decorations.markings}
\usepackage{polynom}
\usepackage{multirow}
\usepackage[german]{fancyref}
\polyset{style=C, div=:,vars=x}
\pgfplotsset{compat=1.8}
\pagenumbering{arabic}
% ensures that paragraphs are separated by empty lines
\parskip 12pt plus 1pt minus 1pt
\parindent 0pt
% define how the sections are rendered
\def\thesection{2.\arabic{section})}
\def\thesubsection{\arabic{subsection}.}
\def\thesubsubsection{(\roman{subsubsection})}
% some matrix magic
\makeatletter
\renewcommand*\env@matrix[1][*\c@MaxMatrixCols c]{%
  \hskip -\arraycolsep
  \let\@ifnextchar\new@ifnextchar
  \array{#1}}
\makeatother

\begin{document}
\author{Benjamin Kuffel, Jim Martens, Sabrina Mehrens\\Gruppe 6}
\title{Hausaufgaben zum 27. Oktober}
\maketitle

\setcounter{section}{2}
\section{} %2.3
	\subsection{} % 1.
		\[L(A_{2.3}) = a \cdot (ba^{*}c)^{*} + bc(abc)^{*} \cdot (a + e)\]
		\[L^{\omega}(A_{2.3}) = a(ba^{*}c)^{\omega} + bc(abc)^{\omega}\]
		\[\]
	\subsection{} % 2.
	\subsection{} % 3.
\section{} %2.4
	\subsection{} % 1.
	\subsection{} % 2.
	\subsection{} % 3.
	\subsection{} % 4.
\end{document}
