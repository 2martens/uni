\documentclass[10pt,a4paper,oneside,ngerman,numbers=noenddot]{scrartcl}
\usepackage[T1]{fontenc}
\usepackage[utf8x]{inputenc}
\usepackage[ngerman]{babel}
\usepackage{amsmath}
\usepackage{amsfonts}
\usepackage{amssymb}
\usepackage{paralist}
\usepackage{gauss}
\usepackage{pgfplots}
\usepackage[locale=DE,exponent-product=\cdot,detect-all]{siunitx}
\usepackage{tikz}
\usetikzlibrary{automata,matrix,fadings,calc,positioning,decorations.pathreplacing,arrows,decorations.markings}
\usepackage{polynom}
\usepackage{multirow}
\usepackage[german]{fancyref}
\polyset{style=C, div=:,vars=x}
\pgfplotsset{compat=1.8}
\pagenumbering{arabic}
% ensures that paragraphs are separated by empty lines
\parskip 12pt plus 1pt minus 1pt
\parindent 0pt
% define how the sections are rendered
\def\thesection{4.\arabic{section})}
\def\thesubsection{\arabic{subsection}.}
\def\thesubsubsection{(\roman{subsubsection})}
% some matrix magic
\makeatletter
\renewcommand*\env@matrix[1][*\c@MaxMatrixCols c]{%
  \hskip -\arraycolsep
  \let\@ifnextchar\new@ifnextchar
  \array{#1}}
\makeatother

\begin{document}
\author{Benjamin Kuffel, Jim Martens\\Gruppe 6}
\title{Hausaufgaben zum 10. November}
\maketitle

\setcounter{section}{2}
\section{} %4.3
	\subsection{}
	\begin{alignat*}{2}
		L(TS_{kuchen\_teil}) &=& (v(htwb^{*}k)^{*}o + r)^{*} \\
		L^{\omega}(TS_{kuchen\_teil}) &=& (v(htwb^{*}k)^{*}o + r)^{\omega}
	\end{alignat*}
	\subsection{}
	\[
		SS(M) = 1(3(5763)^{*}1 + 1)^{\omega}
	\]
	\subsection{}
	\[
		ES(M) = 1(3(5763)^{*}1 + 1)^{\omega}
	\]
	\subsection{}
	\begin{alignat*}{2}
		Sat(Teig \vee \lnot Hitze) &=& \{1, 4, 5\} \\
		Sat(\lnot Teig) &=& \{1, 2, 3, 6, 7\}
	\end{alignat*}	
	
\section{} %4.4
	\begin{tabular}{l|l|l}
		\(f\) & \(M_{kuchen} \models f \) & \(M_{kuchen}, \pi \models f\)\\
		\hline
		\(\lozenge \square(\lnot Teig \vee Hitze) \)& nein & nein \\
		\(\square	\lozenge(\lnot Teig \vee Hitze)\) & ja & ja \\
		\(\square	(Hitze \;\mathcal{U}\; Backen)\) & nein & nein \\
		\(\square	\lozenge (Hitze \Rightarrow XX\lnot Backen)\) & ja & ja
	\end{tabular}
\end{document}
