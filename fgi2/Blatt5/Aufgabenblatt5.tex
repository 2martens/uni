\documentclass[10pt,a4paper,oneside,ngerman,numbers=noenddot]{scrartcl}
\usepackage[T1]{fontenc}
\usepackage[utf8x]{inputenc}
\usepackage[ngerman]{babel}
\usepackage{amsmath}
\usepackage{amsfonts}
\usepackage{amssymb}
\usepackage{paralist}
\usepackage{gauss}
\usepackage{pgfplots}
\usepackage[locale=DE,exponent-product=\cdot,detect-all]{siunitx}
\usepackage{tikz}
\usetikzlibrary{automata,matrix,fadings,calc,positioning,decorations.pathreplacing,arrows,decorations.markings}
\usepackage{polynom}
\usepackage{multirow}
\usepackage[german]{fancyref}
\polyset{style=C, div=:,vars=x}
\pgfplotsset{compat=1.8}
\pagenumbering{arabic}
% ensures that paragraphs are separated by empty lines
\parskip 12pt plus 1pt minus 1pt
\parindent 0pt
% define how the sections are rendered
\def\thesection{5.\arabic{section})}
\def\thesubsection{\arabic{subsection}.}
\def\thesubsubsection{(\alph{subsubsection})}
% some matrix magic
\makeatletter
\renewcommand*\env@matrix[1][*\c@MaxMatrixCols c]{%
  \hskip -\arraycolsep
  \let\@ifnextchar\new@ifnextchar
  \array{#1}}
\makeatother

\begin{document}
\author{Benjamin Kuffel, Jim Martens\\Gruppe 6}
\title{Hausaufgaben zum 17. November}
\maketitle

\setcounter{section}{2}
\section{} %4.3
	\subsection{}
	Der Abwicklungsbaum befindet sich auf \fref{fig:531}.
	\begin{figure}
		\begin{tikzpicture}[node distance=1cm]
			\node[state,initial] (s0start) {\(\{b\}\)};
			\node (t0start) [above=0.0001 of s0start] {\(s_{0}\)};
			\node[state] (s01) [below left=of s0start] {\(\{b\}\)};
			\node (t01) [above=0.0001 of s01] {\(s_{0}\)};
			\node[state] (s11) [below right=of s0start] {\(\{s\}\)};
			\node (t11) [above=0.0001 of s11] {\(s_{1}\)};
			\node[state] (s02) [below left=of s01] {\(\{b\}\)};
			\node (t02) [above=0.0001 of s02] {\(s_{0}\)};
			\node[state] (s12) [below right=of s01] {\(\{s\}\)};
			\node (t12) [above=0.0001 of s12] {\(s_{1}\)};
			\node[state] (s22) [below right=of s11] {\(\{g\}\)};
			\node (t22) [above=0.0001 of s22] {\(s_{2}\)};
			\node[state] (s03) [below left=of s02] {\(\{b\}\)};
			\node (t03) [above=0.0001 of s03] {\(s_{0}\)};
			\node[state] (s13) [below right=of s02] {\(\{s\}\)};
			\node (t13) [above=0.0001 of s13] {\(s_{1}\)};	
			\node[state] (s23) [below right=of s12] {\(\{g\}\)};
			\node (t23) [above=0.0001 of s23] {\(s_{2}\)};
			\node[state] (s23-1) [below right=of s22] {\(\{g\}\)};
			\node (t23-1) [above=0.0001 of s23-1] {\(s_{2}\)};
			\node[state] (s04) [below left=of s03] {\(\{b\}\)};
			\node (t04) [above=0.0001 of s04] {\(s_{0}\)};
			\node[state] (s14) [below right=of s03] {\(\{s\}\)};
			\node (t14) [above=0.0001 of s14] {\(s_{1}\)};
			\node[state] (s24) [below right=of s13] {\(\{g\}\)};
			\node (t24) [above=0.0001 of s24] {\(s_{2}\)};
			\node[state] (s24-1) [below right=of s23] {\(\{g\}\)};
			\node (t24-1) [above=0.0001 of s24-1] {\(s_{2}\)};
			\node[state] (s24-2) [below right=of s23-1] {\(\{g\}\)};
			\node (t24-2) [above=0.0001 of s24-2] {\(s_{2}\)};
			\path[->] (s0start) edge (s01)
				(s0start) edge (s11)
				(s01) edge (s02)
				(s01) edge (s12)
				(s11) edge (s22)
				(s02) edge (s03)
				(s02) edge (s13)
				(s12) edge (s23)
				(s22) edge (s23-1)
				(s03) edge (s04)
				(s03) edge (s14)
				(s13) edge (s24)
				(s23) edge (s24-1)
				(s23-1) edge (s24-2);
		\end{tikzpicture}
		\caption{Abwicklungsbaum \(Abwicklung_{AKW}\) der Tiefe 4}
		\label{fig:531}
	\end{figure}
	
	\subsection{}
		\subsubsection{}
		\[Sat(\alpha _{1}) = \{s_{0},s_{1},s_{2}\}\]
		\subsubsection{}
		\[Sat(\textbf{AG}\alpha _{1}) = \{s_{0},s_{1},s_{2}\}\]
		\subsubsection{}
		\[Sat(\textbf{EG}\lnot b) = \{s_{1},s_{2}\} \]
		\subsubsection{}
		\[Sat(\textbf{AX}\lnot g) = \{s_{0}\} \]
	\subsection{}
	\(\beta _{1} = \textbf{AGEX}(s \vee g)\) bedeutet, dass für alle Pfade immer gilt, dass es einen Pfad gibt, für den im nächsten Schritt \(s\) oder \(g\) gilt.
	
	\(\beta _{2} = \textbf{EG}\lnot b\) bedeutet, dass es einen Pfad gibt, für den immer \(\lnot b\) gilt.
		
		\subsubsection{}
		\(\beta _{1}\) gilt nach 5.3.2.(b) für alle Zustände und somit auch für den Startzustand. Damit gilt \(M_{AKW} \models \textbf{AGEX}(s \vee g)\).
		\subsubsection{}
		\(\beta _{2}\) gilt nach 5.3.2.(c) nur für \(s_{1}\) und \(s_{2}\) und somit nicht für den Startzustand. Damit gilt \(M_{AKW} \models \textbf{EG}\lnot b\) nicht.
\section{} %4.4
	\subsection{}
	Umwandeln von \(\beta _{1}\) in \(\beta '_{1}\):
	\[\beta '_{1} = \lnot EF(\lnot EX \lnot s)\]
	Umwandeln von \(\beta _{2}\) in \(\beta '_{2}\):
	\[\beta '_{2} = EX(\lnot EF s)\]
	
	\subsection{}
	\begin{tabular}{c|c|c|c}
		Teilformel & Zustand \(s_{0}\) & Zustand \(s_{1}\) & Zustand \(s_{2}\) \\
		\hline
		\hline
		\(s\) & -- & + & -- \\
		\(\lnot s\) & + & -- & + \\
		\(\textbf{EX}\lnot s\) & + & + & + \\
		\(\lnot \textbf{EX}\lnot s\) & -- & -- & -- \\
		\(\textbf{EF}(\lnot \textbf{EX}\lnot s) \) & -- & -- & -- \\
		\(\lnot \textbf{EF}(\lnot \textbf{EX}\lnot s) \) & + & + & + \\
		\(\textbf{EF}s\) & + & + & -- \\
		\(\lnot \textbf{EF}s\) & -- & -- & + \\
		\(\textbf{EX}(\lnot \textbf{EF}s) \) & -- & + & +		
	\end{tabular}
	
	\subsection{}
	\(M_{AKW} \models \beta _{1}\) gilt, aber \(M_{AKW} \models \beta _{2}\) gilt nicht.
\end{document}
