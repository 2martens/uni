\documentclass[10pt,a4paper,oneside,ngerman,numbers=noenddot]{scrartcl}
\usepackage[T1]{fontenc}
\usepackage[utf8x]{inputenc}
\usepackage[ngerman]{babel}
\usepackage{amsmath}
\usepackage{amsfonts}
\usepackage{amssymb}
\usepackage{paralist}
\usepackage{gauss}
\usepackage{pgfplots}
\usepackage[locale=DE,exponent-product=\cdot,detect-all]{siunitx}
\usepackage{tikz}
\usetikzlibrary{automata,matrix,fadings,calc,positioning,decorations.pathreplacing,arrows,decorations.markings,petri,shapes}
\usepackage{polynom}
\usepackage{multirow}
\usepackage[german]{fancyref}
\usepackage{morefloats}
\polyset{style=C, div=:,vars=x}
\pgfplotsset{compat=1.8}
\pagenumbering{arabic}
% ensures that paragraphs are separated by empty lines
\parskip 12pt plus 1pt minus 1pt
\parindent 0pt
% define how the sections are rendered
\def\thesection{12.\arabic{section})}
\def\thesubsection{\arabic{subsection}.}
\def\thesubsubsection{(\alph{subsubsection})}
% some matrix magic
\makeatletter
\renewcommand*\env@matrix[1][*\c@MaxMatrixCols c]{%
  \hskip -\arraycolsep
  \let\@ifnextchar\new@ifnextchar
  \array{#1}}
\makeatother

\tikzset{
    place/.style={
        circle,
        thick,
        draw=black,
        fill=white,
        minimum size=6mm,
        font=\bfseries
    },
    transitionH/.style={
        rectangle,
        thick,
        draw=black,
        fill=white,
        minimum width=8mm,
        inner ysep=4pt,
        font=\bfseries
    },
    transitionV/.style={
        rectangle,
        thick,
        fill=black,
        minimum height=8mm,
        inner xsep=2pt
    }
}

\renewcommand{\vec}[1]{\underline{#1}}

\begin{document}
\author{Benjamin Kuffel, Jim Martens\\Gruppe 6}
\title{Hausaufgaben zum 19. Januar}
\maketitle

\setcounter{section}{2}
\section{} %12.3
    \subsection{}
    Der Prozessgraph für \(t_3\) ist auf \fref{fig:t3-graph} sichtbar. Der Prozessgraph für \(t_4\) ist auf \fref{fig:t4-graph} sichtbar.
    \begin{figure}
        \begin{tikzpicture}[node distance=1cm]
            \node (t3-start) {\(a \cdot (bb + a) \cdot b + a \cdot (b + (b + bb))\)};
            % left side
            \node (t3-a-left) [below left=of t3-start] {\((bb + a) \cdot b\)};
            \node (t3-b-left) [below=of t3-a-left] {\(b\)};
            \node (t3-finish-left) [below=of t3-b-left] {\(\sqrt{}\)};
            % right side
            \node (t3-a-right) [below right=of t3-start] {\(b + (b + bb)\)};
            \node (t3-finish-right) [below=2.0 of t3-a-right] {\(\sqrt{}\)};

            \path[->] (t3-start) edge node[above left] {a} (t3-a-left)
                (t3-a-left) edge[bend right] node[left] {bb} (t3-b-left)
                (t3-a-left) edge[bend left] node[right] {a} (t3-b-left)
                (t3-b-left) edge node[left] {b} (t3-finish-left)
                (t3-start) edge node[above right] {a} (t3-a-right)
                (t3-a-right) edge[bend right] node[left] {b} (t3-finish-right)
                (t3-a-right) edge node[left] {b} (t3-finish-right)
                (t3-a-right) edge[bend left] node[right] {bb} (t3-finish-right);
        \end{tikzpicture}
        \caption{Prozessgraph für \(t_3\)}
        \label{fig:t3-graph}
    \end{figure}


    \begin{figure}
        \begin{tikzpicture}[node distance=1cm]
            \node (t4-start) {\(a \cdot (bbb + (ab + b) \cdot (b + b))\)};
            \node (t4-a) [below=of t4-start] {\(bbb + (ab + b) \cdot (b + b)\)};
            % left side
            \node (t4-finish-left) [below left=of t4-a] {\(\sqrt{}\)};
            % right side
            \node (t4-b-plus-b-right) [below=of t4-a] {\(b + b\)};
            \node (t4-finish-right) [below=of t4-b-plus-b-right] {\(\sqrt{}\)};

            \path[->] (t4-start) edge node[left] {a} (t4-a)
                (t4-a) edge node[above left] {bbb} (t4-finish-left)
                (t4-a) edge[bend right] node[left] {ab} (t4-b-plus-b-right)
                (t4-a) edge[bend left] node[right] {b} (t4-b-plus-b-right)
                (t4-b-plus-b-right) edge[bend right] node[left] {b} (t4-finish-right)
                (t4-b-plus-b-right) edge[bend left] node[right] {b} (t4-finish-right);
        \end{tikzpicture}
        \caption{Prozessgraph für \(t_4\)}
        \label{fig:t4-graph}
    \end{figure}

    \subsection{}
    Die Knoten \(b + b\) im Graph von \(t_4\) und \(b\) im Graph von \(t_3\) sind bisimilar. Alle anderen Knoten sind nicht bisimilar.

    \subsection{}
    \(t_3\) und \(t_4\) sind nicht bisimilar.

    \subsection{}
    Erste Ableitung:
    \begin{alignat*}{2}
        a \overset{a}{\rightarrow} \sqrt{} \;&&\; (A_0), x = a, v = a \\
        a \cdot (b + (b + bb)) \overset{a}{\rightarrow} b + (b + bb) \;&&\; (T.^{\sqrt{}}), x = a, v = a, y = b + (b + bb) \\
        a \cdot (bb + a) \cdot b + a \cdot (b + (b + bb)) \overset{a}{\rightarrow} b + (b + bb) \;&&\; (T_{+L}), v = a, y = a \cdot (b + (b + bb))\\
        &&\; x =  a \cdot (bb + a) \cdot b, y' = b + (b + bb)
    \end{alignat*}

    Zweite Ableitung:
    \begin{alignat*}{2}
        a \overset{a}{\rightarrow} \sqrt{} \;&&\; (A_0), x = a, v = a \\
        a \cdot (bb + a) \cdot b \overset{a}{\rightarrow} (bb + a) \cdot b \;&&\; (T.^{\sqrt{}}), x = a, v = a, y = (bb + a) \cdot b \\
        a \cdot (bb + a) \cdot b + a \cdot (b + (b + bb)) \overset{a}{\rightarrow} (bb + a) \cdot b \;&&\; (T_{+R}), x = a \cdot (bb + a) \cdot b, v = a \\
        &&\; y = a \cdot (b + (b + bb)), x' = (bb + a) \cdot b
    \end{alignat*}

    Dritte Ableitung:
    \begin{alignat*}{2}
        a \overset{a}{\rightarrow} \sqrt{} \;&&\; (A_0), x = a, v = a \\
        bb + a \overset{a}{\rightarrow} \sqrt{} \;&&\; (T^{\sqrt{}}_{+L}), y = a, v = a, x = bb \\
        (bb + a) \cdot b \overset{a}{\rightarrow} b \;&&\; (T.^{\sqrt{}}), x = (bb + a), v = a, y = b
    \end{alignat*}

    \subsection{}
    Der Prozessgraph für \(t_5\) ist auf \fref{fig:t5-graph} sichtbar. Der Prozessgraph für \(t_6\) ist auf \fref{fig:t6-graph} sichtbar.

    Die Knoten \(f\) in beiden Graphen, \((c + d)f\) in beiden Graphen, sowie \(cf + df\) im Graph von \(t_5\) und die beiden Wurzelknoten sind jeweils bisimilar.
    \begin{figure}
        \begin{tikzpicture}[node distance=1cm]
            \node (t5-start) {\(b(cf + df) + a(c + d)f\)};
            % left side
            \node (t5-b-left) [below left=of t5-start] {\(cf + df\)};
            \node (t5-f1-left) [below left=of t5-b-left] {\(f\)};
            \node (t5-f2-left) [below right=of t5-b-left] {\(f\)};
            \node (t5-finish-left) [below=2 of t5-b-left] {\(\sqrt{}\)};
            % right side
            \node (t5-a-right) [below right=of t5-start] {\((c + d)f\)};
            \node (t5-f-right) [below=of t5-a-right] {\(f\)};
            \node (t5-finish-right) [below=of t5-f-right] {\(\sqrt{}\)};

            \path[->] (t5-start) edge node[above left] {b} (t5-b-left)
                (t5-b-left) edge node[above left] {c} (t5-f1-left)
                (t5-b-left) edge node[above right] {d} (t5-f2-left)
                (t5-f1-left) edge node[below left] {f} (t5-finish-left)
                (t5-f2-left) edge node[below right] {f} (t5-finish-left)
                (t5-start) edge node[above right] {a} (t5-a-right)
                (t5-a-right) edge[bend right] node[left] {c} (t5-f-right)
                (t5-a-right) edge[bend left] node[right] {d} (t5-f-right)
                (t5-f-right) edge node[right] {f} (t5-finish-right);
        \end{tikzpicture}
        \caption{Prozessgraph für \(t_5\)}
        \label{fig:t5-graph}
    \end{figure}

    \begin{figure}
        \begin{tikzpicture}[node distance=1cm]
            \node (t6-start) {\((a + b)(c + d)f\)};
            \node (t6-c-plus-d-f) [below=of t6-start] {\((c + d)f\)};
            \node (t6-f) [below=of t6-c-plus-d-f] {\(f\)};
            \node (t6-finish) [below=of t6-f] {\(\sqrt{}\)};

            \path[->] (t6-start) edge[bend right] node[left] {a} (t6-c-plus-d-f)
                (t6-start) edge[bend left] node[right] {b} (t6-c-plus-d-f)
                (t6-c-plus-d-f) edge[bend right] node[left] {c} (t6-f)
                (t6-c-plus-d-f) edge[bend left] node[right] {d} (t6-f)
                (t6-f) edge node[right] {f} (t6-finish);
        \end{tikzpicture}
        \caption{Prozessgraph für \(t_6\)}
        \label{fig:t6-graph}
    \end{figure}
\section{} %12.4
    \subsection{}
    \begin{alignat*}{2}
        a \overset{a}{\rightarrow} \sqrt{} \;&&\; (A_0), x = a, v = a \\
        a \cdot (c + d)f \overset{a}{\rightarrow} (c + d)f \;&&\; (T.^{\sqrt{}}), x = a, v = a, y = (c + d)f \\
        b(cf + df) + a(c + d)f \overset{a}{\rightarrow} (c + d)f \;&&\; (T_{+L}), x = b(cf + df), y = a(c + d)f, \\
        &&\; y' = (c + d)f, v = a
    \end{alignat*}

    Wie aus dieser Ableitung sichtbar wird, ist der gegebene Übergang korrekt.

    \subsection{} %12.4.2 - eventually
        \begin{alignat*}{2}
            t_7 &=& (d(c + d))(ab + (a + a)\underline{(b + b)}) \\
            &=& (d(c + d))(ab + \underline{(a + a)}b) \\
            &=& (d(c + d))\underline{(ab + ab)} \\
            &=& (d(c + d))ab
        \end{alignat*}

    \subsection{}
        \begin{alignat*}{2}
            t_8 &=& ((d + d)(a + a + c))(a + \underline{(b + b)}) \\
            &=& ((d + d)(\underline{a + a} + c))(a + b) \\
            &=& (\underline{(d + d)}(a + c))(a + b) \\
            &=& (d(a + c))(a + b)
        \end{alignat*}

        \(t_7\) und \(t_8\) sind nicht äquivalent, da \((a \cdot b) \neq (a + b)\) und \((c + d) \neq (a + c)\).

\end{document}

