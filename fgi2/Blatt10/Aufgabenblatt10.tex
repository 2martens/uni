\documentclass[10pt,a4paper,oneside,ngerman,numbers=noenddot]{scrartcl}
\usepackage[T1]{fontenc}
\usepackage[utf8x]{inputenc}
\usepackage[ngerman]{babel}
\usepackage{amsmath}
\usepackage{amsfonts}
\usepackage{amssymb}
\usepackage{paralist}
\usepackage{gauss}
\usepackage{pgfplots}
\usepackage[locale=DE,exponent-product=\cdot,detect-all]{siunitx}
\usepackage{tikz}
\usetikzlibrary{automata,matrix,fadings,calc,positioning,decorations.pathreplacing,arrows,decorations.markings,petri,shapes}
\usepackage{polynom}
\usepackage{multirow}
\usepackage[german]{fancyref}
\usepackage{morefloats}
\polyset{style=C, div=:,vars=x}
\pgfplotsset{compat=1.8}
\pagenumbering{arabic}
% ensures that paragraphs are separated by empty lines
\parskip 12pt plus 1pt minus 1pt
\parindent 0pt
% define how the sections are rendered
\def\thesection{10.\arabic{section})}
\def\thesubsection{\arabic{subsection}.}
\def\thesubsubsection{(\alph{subsubsection})}
% some matrix magic
\makeatletter
\renewcommand*\env@matrix[1][*\c@MaxMatrixCols c]{%
  \hskip -\arraycolsep
  \let\@ifnextchar\new@ifnextchar
  \array{#1}}
\makeatother

\tikzset{
    place/.style={
        circle,
        thick,
        draw=black,
        fill=white,
        minimum size=6mm,
        font=\bfseries
    },
    transitionH/.style={
        rectangle,
        thick,
        draw=black,
        fill=white,
        minimum width=8mm,
        inner ysep=4pt,
        font=\bfseries
    },
    transitionV/.style={
        rectangle,
        thick,
        fill=black,
        minimum height=8mm,
        inner xsep=2pt
    }
}

\renewcommand{\vec}[1]{\underline{#1}}

\begin{document}
\author{Benjamin Kuffel, Jim Martens\\Gruppe 6}
\title{Hausaufgaben zum 05. Januar}
\maketitle

\setcounter{section}{2}
\section{} %10.3
\subsection{}
Um die T-Invarianten bestimmen zu können, stellen wir zunächst die Inzidenzmatrix auf.

Inzidenzmatrix \(\Delta_N\) vom Netz \(N_{10.3}\):
\[
	\begin{vmatrix}
		-1 & -1 & 3 & 2 & -1 \\
		0 & -1 & 3 & 0 & 0 \\
		0 & 1 & -3 & 0 & 0 \\
		1 & 2 & -6 & -2 & 1
	\end{vmatrix}
\]

Eine T-Invariante \(\textbf{j}\) muss diese Gleichung erfüllen: \(\Delta_N \cdot \textbf{j} = \vec{0}\)

Daraus ergibt sich folgendes lineares Gleichungssystem:
\begin{alignat*}{7}
	\text{I} && \; -j_a &-& j_b &+& 3j_c &+& 2j_d &-& j_e &= 0 \\
	\text{II} && && -j_b &+& 3j_c && && &= 0 \\
	\text{III} && && j_b &-& 3j_c && && &= 0 \\
	\text{IV} && \; j_a &+& 2j_b &-& 6j_c &-& 2j_d &+& j_e &= 0
\end{alignat*}

Dieses Gleichungssystem kann schnell reduziert werden. Es wird \(\text{I} = \text{I} + \text{IV}\), sowie \(\text{II} = \text{II} + \text{III}\) ausgeführt.

\begin{alignat*}{7}
	\text{I} && \; 0j_a &+& j_b &-& 3j_c &+& 0j_d &+& 0j_e &= 0 \\
	\text{II} && && 0j_b &+& 0j_c && && &= 0 \\
	\text{III} && && j_b &-& 3j_c && && &= 0 \\
	\text{IV} && \; j_a &+& 2j_b &-& 6j_c &-& 2j_d &+& j_e &= 0
\end{alignat*}

Somit stellt II keine Bedingung mehr auf und I und III sind identisch. Wir verbleiben also mit III und IV. III kann so umgestellt werden: \(j_b = 3j_c\). Im Folgenden können wir also \(j_b\) durch \(3j_c\) ersetzen.

\begin{alignat*}{7}
	\text{I} && && 3j_c &-& 3j_c && && &= 0 \\
	\text{II} && && && && && 0 &= 0 \\
	\text{III} && && 3j_c &-& 3j_c && && &= 0 \\
	\text{IV} && \; j_a &+& 2(3j_c) &-& 6j_c &-& 2j_d &+& j_e &= 0
\end{alignat*}

Aus IV ergibt sich nach Auflösen und Zusammenfassen folgende Gleichung: \(j_a - 2j_d + j_e = 0\). Nach Umstellen nach \(j_d\) ergibt sich daraus \(2j_d = j_a + j_e\). Aus dieser Gleichung ergibt sich, dass die Summe von \(j_a\) und \(j_e\) eine gerade Zahl sein muss. Dies ist dann der Fall, wenn beide Summanden gerade oder ungerade sind. Wenn die Summanden gerade sind, dann muss einer der beiden 2 sein oder sie können solange durch 2 geteilt werden bis dies gilt.

Aus diesen Gleichungen ergeben sich folgende Bedingungen für die einzelnen Werte:
\begin{itemize}
	\item \(j_c\) beliebig
	\item \(j_b = 3j_c\)
	\item \(j_d\) beliebig
	\item \(0 < j_a < 2j_d\) beliebig unter dieser Bedingung
	\item \(j_e = 2j_d - j_a\)
\end{itemize}

Als Menge der T-Invarianten ergibt sich daher: \(\{(a,3c,c,d,e)^{tr} | a,c,d,e \in \mathbb{N}^+ \wedge a < 2d \wedge e = 2d -a\}\).

\subsection{}
Es sei \(\psi = (1, 3, 1, 1, 1)^{tr}\). Die Startmarkierung \(\textbf{m}_0\) sei \((3, 3, 0, 0)^{tr}\). Daraus ergibt sich die Schaltfolge \(w = b^3caed\). Mit den erreichten Zwischenmarkierungen sieht die Schaltfolge dann so aus:
\begin{alignat*}{2}
	\textbf{m}_0 &\overset{b}{\rightarrow} (2, 2, 1, 2)^{tr} \overset{b}{\rightarrow} (1, 1, 2, 4)^{tr} \overset{b}{\rightarrow} (0, 0, 3, 6)^{tr} \overset{c}{\rightarrow} (3, 3, 0, 0)^{tr} \overset{a}{\rightarrow} (2, 3, 0, 1)^{tr} \\
	&\overset{e}{\rightarrow} (1, 3, 0, 2)^{tr} \overset{d}{\rightarrow} (3, 3, 0, 0)^{tr}
\end{alignat*}
\section{} %10.4
\section{} %10.5
\end{document}
