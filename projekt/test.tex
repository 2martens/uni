%!TEX encoding = UTF-8 Unicode
\documentclass[12pt]{scrartcl}
\usepackage[utf8]{inputenc} % Unicode funktioniert unter Windows, Linux und Mac
\usepackage[T1]{fontenc}
\usepackage{times}
\usepackage[ngerman]{babel}
\usepackage[german]{fancyref}
\usepackage{csquotes}
\usepackage[
backend=biber,
bibstyle=ieee,
citestyle=ieee
]{biblatex}
%\usepackage{ngerman}
\usepackage{graphicx}
\usepackage[hidelinks]{hyperref}\urlstyle{rm}
\usepackage{times}
\usepackage[scaled]{helvet}
\usepackage{a4wide}
\usepackage{rotating}
\usepackage{listings}\lstset{breaklines=true,breakatwhitespace=true,frame=leftline,captionpos=b,xleftmargin=6ex,tabsize=4,numbers=left,numberstyle=\ttfamily\footnotesize,basicstyle=\ttfamily\footnotesize}
\sloppy
\setlength{\parindent}{0em}
\setlength{\parskip}{1.2ex plus 0.5ex minus 0.5ex}
\pagestyle{plain}
\addbibresource{/home/jim/Documents/Studium/WS2017_18/Project/literatur.bib}

\begin{document}
\hyphenation{in-te-res-sant in-te-res-san-te}

\newpage
\thispagestyle{empty}
\begin{center}\Large
Universität Hamburg \par
Fachbereich Informatik
\vfill
Seminararbeit
\vfill
{\Large\textsf{\textbf{Vergleich von IPsec und OpenVPN}}\par}
\vfill
von
\par\bigskip
Mustafa Eris, Jim Martens, Benjamin Scholz \par
Betreuer: Hannes Federrath \par
%Matrikelnummern 6420323 \par
%Studiengang BSc. Informatik
\end{center}

\cite{Mueller2017}

\newpage

\printbibliography

\end{document}
