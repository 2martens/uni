%!TEX encoding = UTF-8 Unicode
\documentclass[12pt]{scrartcl}
%\usepackage[applemac]{inputenc} % Mac-Umlaute direkt verwenden öäüß
%\usepackage[isolatin]{inputenc} % PC-Umlaute direkt verwenden 
\usepackage[utf8]{inputenc} % Unicode funktioniert unter Windows, Linux und Mac
\usepackage[T1]{fontenc}
\usepackage[ngerman]{babel}
\usepackage{csquotes}
\usepackage[
backend=biber,
bibstyle=ieee,
citestyle=ieee
]{biblatex}
%\usepackage{ngerman}
\usepackage{graphicx}
\usepackage[hidelinks]{hyperref}\urlstyle{rm}
\usepackage{times}
\usepackage[scaled]{helvet}
\usepackage{a4wide}
\usepackage{rotating}
\usepackage{listings}\lstset{breaklines=true,breakatwhitespace=true,frame=leftline,captionpos=b,xleftmargin=6ex,tabsize=4,numbers=left,numberstyle=\ttfamily\footnotesize,basicstyle=\ttfamily\footnotesize}
\sloppy
\setlength{\parindent}{0em}
\setlength{\parskip}{1.2ex plus 0.5ex minus 0.5ex}
\pagestyle{plain}
\addbibresource{semFrieden.bib}

\begin{document}

\newpage
\thispagestyle{empty}
\begin{center}\Large
Universität Hamburg \par
Fachbereich Informatik
\vfill
Seminararbeit
\vfill
{\Large\textsf{\textbf{Haager Landkriegsordnung}}\par}
\vfill
vorgelegt von 
\par\bigskip
Jim Martens \par
Matrikelnummern 6420323 \par
Studiengang BSc. Informatik
\end{center}

\newpage
\section*{Zusammenfassung}

\newpage
\tableofcontents

\newpage
\section{Vorbemerkung}


\section{Weg zur Friedenskonferenz}
\begin{itemize}
	\item Gründe für Zustandekommen\cite{Buss1992}
	\item Begriff des Kombattant essentiell für westliches Werteverständnis
	\item Entwicklung seit dem Mittelalter
	\item zunehmende Kodifizierung des geltenden (Kriegs-)rechts seit dem 18. Jhd.
	\item Brüsseler Konferenz 1870
	\item Oxford Manual
\end{itemize}
\section{Erste Friedenskonferenz}
\begin{itemize}
	\item Begriff des Kombattanten\cite{Buss1992}
	\item Levée en masse\cite{Scott1920}
	\item Chemiewaffen\cite{Scott1920}
\end{itemize}
\section{Zweite Friedenskonferenz}
\begin{itemize}
	\item Levée en masse (Deutscher Änderungsantrag)
\end{itemize}
\section{Auswertung}
\begin{itemize}
	\item gelungene Kodifikation des vorher bereits geltenden Kriegsgewohnheitsrechts
	\item Bestimmungen zum Kombattantenstatus und Behandlung von Kombattanten auch im 1. WK weitgehend eingehalten
	\item Allbeteiligungsklausel nicht so problematisch, wie es scheint
\end{itemize}
\cite{Gasser1991}
\cite{Lingen2014}
\cite{Buss1992}
\cite{Fraenkel1968}
\cite{Heffter1951}
\cite{Reich2010}
\cite{Scott1920}
\cite{Scott1921}



\section{Schlussbemerkungen}


%%%%%%%%%%%%%%%%%%%%%%%%%%%%%%%%%%%%%%%%%%%%%%%%%%%%%%%%%%%%%%%%%%%%%%
\newpage

\printbibliography
\addcontentsline{toc}{section}{Literaturverzeichnis}


%%%%%%%%%%%%%%%%%%%%%%%%%%%%%%%%%%%%%%%%%%%%%%%%%%%%%%%%%%%%%%%%%%%%%%
\newpage
\addcontentsline{toc}{section}{Anhang}

\end{document}
