%!TEX encoding = UTF-8 Unicode
\documentclass[12pt]{scrartcl}
%\usepackage[applemac]{inputenc} % Mac-Umlaute direkt verwenden öäüß
%\usepackage[isolatin]{inputenc} % PC-Umlaute direkt verwenden 
\usepackage[utf8]{inputenc} % Unicode funktioniert unter Windows, Linux und Mac
\usepackage[T1]{fontenc}
\usepackage[ngerman]{babel}
\usepackage{csquotes}
\usepackage[
backend=biber,
bibstyle=ieee,
citestyle=ieee
]{biblatex}
%\usepackage{ngerman}
\usepackage{graphicx}
\usepackage[hidelinks]{hyperref}\urlstyle{rm}
\usepackage{times}
\usepackage[scaled]{helvet}
\usepackage{a4wide}
\usepackage{rotating}
\usepackage{listings}\lstset{breaklines=true,breakatwhitespace=true,frame=leftline,captionpos=b,xleftmargin=6ex,tabsize=4,numbers=left,numberstyle=\ttfamily\footnotesize,basicstyle=\ttfamily\footnotesize}
\sloppy
\setlength{\parindent}{0em}
\setlength{\parskip}{1.2ex plus 0.5ex minus 0.5ex}
\pagestyle{plain}
\addbibresource{semFrieden.bib}

\begin{document}

\newpage
\thispagestyle{empty}
\begin{center}\Large
Universität Hamburg \par
Fachbereich Informatik
\vfill
Seminararbeit
\vfill
{\Large\textsf{\textbf{Haager Landkriegsordnung}}\par}
\vfill
vorgelegt von 
\par\bigskip
Jim Martens \par
Matrikelnummern 6420323 \par
Studiengang BSc. Informatik
\end{center}

\newpage
\section*{Zusammenfassung}
Die Haager Landkriegsordnung (kurz: HLKO) ging aus den Haager Friedenskonferenzen (1899 und 1907) hervor und steht am Ende eines langen Weges, der im 16. Jahrhundert begann. Zentrale Punkte der HLKO sind der Kombattantenstatus, die Regeln des humanitären Völkerrechts bzgl. der erlaubten feindlichen Aktivitäten und das Verhalten im Falle einer Besatzung. Bis heute hat die HLKO eine weitreichende Bedeutung. Insbesondere die Martens'sche Klausel findet sich in zahlreichen internationalen Verträgen.

\newpage
\tableofcontents

\newpage
\section{Vorbemerkung}
Aus den Haager Friedenskonferenzen ging die Haager Landkriegsordnung hervor. Die Konferenzen gelten als Meilenstein in der Kodifikation von geltendem Kriegsvölkerrecht. Doch sie sind keineswegs eine plötzliche Erscheinung gewesen, sondern stehen am Ende eines Weges, der sich über mehrere Jahrhunderte erstreckt. In diesem Paper wird die Haager Landkriegsordnung hauptsächlich unter dem Punkt des Kombattantenstatus betrachtet. Denn die Unterscheidung zwischen rechtmäßig Kämpfenden und der unschuldigen Zivilbevölkerung ist Grundlage unseres heutigen westlichen Werteverständnisses.

Im Folgenden wird daher der historische Weg mit den wichtigsten Meilensteinen erläutert, um anschließend beide Konferenzen vorzustellen, wobei natürlich auch die übrigen Regelungen, wie z.B. zu erlaubten Waffengattungen, Erwähnung finden werden. Beide Konferenzen werden schließlich unter dem Aspekt des Kombattantenstatus ausgewertet, um mit Schlussbemerkungen zu schließen.

Der Weg zu den Konferenzen orientiert sich an Buß\cite{Buss1992}.
\section{Weg zu den Friedenskonferenzen}
Der Weg zu den Haager Friedenskonferenzen beginnt nicht erst im 19. Jahrhundert, sondern bereits weit früher im Mittelalter. Eine Kodifikation des Kriegsrechts setzt voraus, dass die Einhaltung dieses Rechts auch sichergestellt werden kann. Dazu ist Disziplin in der kämpfenden Truppe notwendig.

Die erste Möglichkeit einer solchen Kontrolle auf Einhaltung hatten die Römer mit ihren hochdisziplinierten Truppen. Allerdings wurde die Möglichkeit nicht genutzt. Die gegnerische Zivilbevölkerung war häufig genauso Ziel militärischer Handlungen wie die gegnerischen Soldaten. Ein paar hundert Jahre später gab es mit dem Rittertum ausgebildete Kämpfer, die jedoch Einzelkämpfer waren und keineswegs die Zivilbevölkerung schonten.

Erst mit den sogenannten Schweizer Haufen gelang eine Disziplinierung einer Armee, sodass Anweisungen von Vorgesetzten durchgesetzt werden konnten. So war die Plünderung und Schändung der Zivilbevölkerung (hauptsächlich Frauen und Kinder) untersagt.

Francisco de Vitoria hat im Jahre 1532 anhand der Kolonialisierung Amerikas durch die Spanier, die er für legal hielt, erklärt, dass ein Krieg für beide Parteien rechtmäßig sein kann. Zudem hält er fest, dass ein rechtmäßiger Krieg den Einsatz jeglicher Mittel erlaubt, um ihn schnellstmöglich siegreich zu beenden. Ist der Krieg jedoch beiderseitig gerechtfertigt, würde dies in einer Eskalation der Mittel enden, wodurch eine beiderseitige Mäßigung notwendig ist. Seine Ausführungen bilden zudem die Grundlage für die spätere Separierung zwischen Kämpfenden und der Zivilbevölkerung.

Der nächste große Schritt ereignete sich mit dem Unabhängigkeitskampf der späteren Niederlande gegen Spanien. Wilhelm von Oranien leitete dabei den Widerstand und schuf ein stehendes Söldnerheer, dass diszipliniert war, regelmäßig entlohnt wurde und als erstes Heer nach noch heute gültigen Grundsätzen aufgebaut war. 

Einen Einschnitt gab es mit dem Dreißigjährigen Krieg, in dem überwiegend undisziplinierte Söldnerheere aufeinander trafen. Außerdem wurde überwiegend eine Ermattungsstrategie verfolgt, bei der offene Feldschlachten vermieden und eher Belagerungen und kleinere Gefechte ausgetragen wurden. Im Rahmen dieser Strategie dauerten Kriege auch länger und die Versorgung der Soldaten erfolgte meist aus dem Kampfgebiet. Durch diese kontinuierlichen Plünderungen zur Ernährung der Truppe war ein effektiver Schutz der Zivilbevölkerung, wie er zuvor teilweise in Kriegsartikeln bereits gefasst war, nicht möglich.

Während dieser Zeit hat Hugo Grotius ein dreibändiges Werk (De iure bella ac pacis) verfasst, in welchem er das "`freiwillige"' Völkerrecht in einer realistischen Weise darstellt und im dritten Buch auf ethische, moralische und religiöse Grundsätze hinweist. Obgleich dieses Rechtsbuch einen wichtigen Beitrag zum späteren Völkerrecht darstellt, so entspricht es doch in keinem Maße der Kodifikation in den Haager Konferenzen.

Eine Zäsur stellte der Westfälische Frieden von 1648 dar. Von da an wurden stehende Heere die Normalität und die Souveränität der einzelnen Staaten garantiert. Durch diese Entwicklung, die mit dem stehenden Heer der Oranier (spätere Utrechter Union und Niederlande) begann, wurde eine effektive Unterscheidung zwischen Kämpfenden und Zivilbevölkerung erstmals flächendeckend möglich. Dieser Frieden bildet daher in vielerlei Hinsicht die Grundlage dafür, dass 250 Jahre später die Haager Friedenskonferenzen stattfinden konnten.

In der Folgezeit gab es noch weitere Ereignisse, doch für den geschichtlichen Hintergrund sollte dies ausreichen. Im Folgenden wird eine zeitlich nähere Einordnung vorgenommen, die in der zweiten Hälfte 19. Jahrhundert verbleibt.

Der Lieber Code ist im Rahmen des Sezessionskriegs zwischen der USA und der Konföderation entstanden. Dieser ist in vielen Punkten speziell auf den amerikanischen Bürgerkrieg ausgerichtet, enthält aber eine detaillierte Auflistung dessen, was zu der Zeit als geltendes Kriegsrecht angenommen wurde. Er diente daher als Grundlage für die weiteren Verhandlungen in den kommenden Jahrzehnten.

Die Brüsseler Deklaration von 1874 schließlich beinhaltet die hauptsächliche Vorarbeit zu den Haager Friedenskonferenzen und stellt auch den letzten Schritt vor der ersten Konferenz dar, auf die im nächsten Abschnitt näher eingegangen wird. Allerdings scheitert die Konferenz an der Frage der Levée en masse.

\section{Erste Friedenskonferenz}

\subsection{Oxford Manual}

Das Scheitern der Brüsseler Konferenz veranlasste Wissenschaftler dazu sich mit dem Problem der kriegsrechtlichen Kodifikation zu beschäftigen. Dabei wurde der Versuch unternommen den Inhalt der kriegsrechtlichen Bestimmungen der Brüsseler Deklaration derart umzuformulieren, dass sie der europäischen Praxis der Kriegführung entsprachen. Die Hauptarbeit wurde dabei vom Institut für internationales Recht durchgeführt. Diesem Institut gehörten ausgewählte Wissenschaftler an, die ohne amtlichen Auftrag arbeiteten. Bei der Annahme des Oxford Manual (nach dem Abstimmungsort Oxford benannt) waren namhafte Völkerrechtler anwesend. Die Abstimmung fand am 9. September 1880 in Oxford statt.\cite{Buss1992}

In Bezug auf den Kombattantenstatus vertreten die Wissenschaftler die Position der Staaten und auch bei der umstrittenen Levée en masse folgt das Institut weitgehend den Großmächten. Lediglich auf unbesetztem Gebiet war demnach die unorganisierte Volkserhebung rechtmäßig. Auf besetztem Gebiet stellt das Oxford Manual eine Gehorsamspflicht der Bevölkerung gegenüber dem Besatzer fest.\cite{Buss1992}

\subsection{Vorarbeiten zur ersten Friedenskonferenz}

Im Auftrag des Zaren Nikolaus II. verschickte der russische Außenminister Graf Mouravieff ein diplomatisches Rundschreiben, das zu einer internationalen Konferenz einlädt. Zusätzlich zu allen in Petersburg diplomatisch vertretenen Staaten ging die Einladung an Luxemburg, Montenegro und Siam. Fast alle Staten nahmen die Einladung an, sodass insgesamt 26 Teilnehmer zu verzeichnen waren. Damit war die Zahl der zu einer Konferenz bereiten Staaten fast doppelt so hoch, wie 25 Jahre zuvor in Brüssel.\cite{Buss1992}

Auf Einladung der niederländischen Regierung begann die Konferenz am 18. Mai 1899 in Den Haag. Sie war nicht ausdrücklich auf die Kodifikation ausgerichtet, sondern vielmehr wollten die an der Konferenz teilnehmenden Staaten Instrumente zur friedlichen Streitbeilegung schaffen. Insofern war auch Rüstungsbeschränkung Ziel der Konferenz.

\subsection{Struktur der Konferenz}

Die Konferenz bestand aus 3 Kommissionen, wobei sich jede Kommission mit einem Teil der Themen beschäftigte. Jede Kommission konnte ihrerseits in weitere Unterkommissionen unterteilt sein. Die erste Kommission beschäftigte sich mit der wohl schwierigsten Frage, nämlich der Rüstungsbeschränkung im Landkrieg und im Seekrieg. In der Kommission gab es zwei Unterkommissionen: Eine diskutierte über militärische Fragen (bzw. landbezogene Fragen), während die andere sich mit Fragen des Seekriegs auseinandersetzte.\cite{Scott1920}

Die zweite Kommission unter dem Vorsitz von Herrn Martens, einem russischen Delegierten, beschäftigte sich einerseits mit der Ausweitung der Regeln der Genfer Konvention von 1864 auf den Seekrieg und der Revision der Brüsseler Deklaration von 1874. Es wurden zwei Unterkommissionen gebildet, wobei sich die erste mit den Fragen zur Genfer Konvention und die zweite mit der Revision der Brüsseler Deklaration beschäftigte.\cite{Scott1920}

Eine etwas andere Struktur findet sich in der dritten Kommission. Diese wurde nicht in Unterkommissionen unterteilt. Stattdessen wurde ein Komitee gegründet, um die zugrundeliegenden Schriftstücke zu analysieren. Insgesamt beschäftigte sich die dritte Kommission mit Fragen zur Schlichtung von Disputen zwischen Staaten.\cite{Scott1920}

Für die Untersuchung der Haager Landkriegsordnung (kurz: HLKO) ist insbesondere die zweite Unterkommission der zweiten Kommission für Interesse. Die folgenden Abschnitte werden sich daher auf diesen Bereich konzentrieren.

\subsection{Martens'sche Klausel}

In der elften Sitzung der zweiten Unterkommission der zweiten Kommission wurden die Artikel 9 und 10 der Brüsseler Deklaration diskutiert. Diese beschäftigen sich mit der Rolle von Kombattanten und Nichtkombattanten. Vor dem Beginn der Diskussion verlas der Präsident Martens eine Deklaration, die für den erfolgreichen Abschluss der Arbeit der Unterkommission eine zentrale Bedeutung hatte. Im Verlauf der Sitzung wurde beschlossen, dass diese Deklaration in das endgültige Protokoll der Konferenz oder das begleitende Abkommen aufgenommen werden soll. Später erhielt diese Deklaration den Namen Martens'sche Klausel. Der genaue Wortlaut der Deklaration ist als Referenz 
im Folgenden aus dem Protokoll der entsprechenden Sitzung\cite{Scott1920} zitiert:

\begin{quotation}
	The Conference is unanimous in thinking that it is extremely desirable that
the usages of war should be defined and regulated. In this spirit it has adopted
a great number of provisions which have for their object the determination of the 
rights and of the duties of belligerents and populations and for their end
a softening of the evils of war so far as military necessities permit. It has not,
however, been possible to agree forthwith on provisions embracing all the cases
which occur in practice.

On the other hand, it could not be intended by the Conference that the
cases not provided for should, for want of a written provision, be left to the
arbitrary judgment of the military commanders.

Until a perfectly complete code of the laws of war is issued, the Conference
thinks it right to declare that in cases not included in the present arrangement,
populations and belligerents remain under the protection and empire of the
principles of international law, as they result from the usages established 
between civilized nations, from the laws of humanity, and the requirements of the
public conscience.

It is in this sense especially that Articles 9 and 10 adopted by the Conference
must be understood.
\end{quotation}

Um die Wichtigkeit dieser Deklaration verstehen zu können, muss die Diskussion in der entsprechenden Sitzung ebenso begutachtet werden. Im nächsten Abschnitt wird exemplarisch anhand der Diskussion in der elften Sitzung das Spannungsverhältnis aufgezeigt. Da eine umfassende Beschäftigung mit jener Sitzung jedoch den Umfang dieses Papers bei weitem übersteigt, wird es bei einer übersichtartigen Behandlung verbleiben. Für den kompletten Dikussionsverlauf sei auf das Protokoll verwiesen.

\subsection{Schlüsselpunkte}

Aufgrund der Erfahrungen mit der Brüsseler Deklaration wurden die Artikel nicht in numerischer Reihenfolge bearbeitet. Stattdessen wurde mit den Artikeln begonnen, bei denen die geringsten Meinungsverschiedenheiten vermutet wurden. Als Folge daraus wurde die Rolle der Kombattanten und Nichtkombattanten erst in der bereits oben angesprochenen elften Sitzung (von insgesamt zwölf) behandelt. Anhand dieser Diskussion kann beispielhaft das Spannungsverhältnis auf der Konferenz und insbesondere in der zweiten Unterkommission der zweiten Kommission sichtbar gemacht werden.

Zunächst werden die Artikel 9 und 10 ohne Änderungen einstimmig nach ein paar Reden angenommen. Artikel 11 über die Rolle der Nichtkombattanten in Heeresverbänden wird ebenso schnell angenommen. Danach beginnt jedoch die Diskussion über die zusätzlichen Artikel von dem englischen Delegierten Sir John Ardagh und der schweizerischen Delegation.

Sir John Ardaghs Artikel:
\begin{quotation}
	Nothing in this chapter shall be considered as tending to lessen or abolish 
the right belonging to the population of an invaded country to fulfill its
duty of offering by all lawful means, the most energetic patriotic resistance
against the invaders.
\end{quotation}

Artikel der schweizerischen Delegation:
\begin{quotation}
	No acts of retaliation shall be exercised against the population of the
occupied territory for having openly taken up arms against the invader.
\end{quotation}

Sir John Ardagh wird vom Vorsitzenden Martens gefragt, ob das Einfügen des Artikels in das Protokoll ausreicht. Ardagh zieht es jedoch vor, dass der Artikel nach Artikel 11 eingefügt wird. Allerdings besteht er nicht darauf, sollte die Unterkommission seinem Wunsch widersprechen. In jenem Fall möchte er jedoch, dass über seinen Artikel abgestimmt wird. Die schweizerische Delegation schließt sich mit einer eloquenten Rede dem Vorschlag von Ardagh an und zieht ihre Änderungsvorschläge für Artikel 9 und 10 zurück. Anschließend ergreift der deutsche Delegierte Schwarzhoff das Wort und hält eine berühmt gewordene Rede. Zentraler Punkt seiner Rede ist, dass die in Artikel 9 und 10 gefassten Bestimmungen nicht ausgeweitet werden sollten. Als Reaktion darauf zieht die schweizerische Delegation ihren Artikel zurück und empfiehlt die Annahme des Artikels von Ardagh. Ardagh besteht darauf, dass sein Artikel als separater Artikel eingefügt und dass darüber abgestimmt wird.

Léon Bourgeois fasst die Situation zusammen und stellt fest, dass im Prinzip alle Ardagh zustimmen, wenngleich die genaue Formulierung zu Problemen führen kann. Nach einigen weiteren Beiträgen beschließt die Unterkommission einstimmig, dass die Deklaration des Präsidenten in das finale Protokoll der Konferenz aufgenommen werden soll. Als Folge darauf wird Ardagh gebeten seinen Artikel zurückzuziehen, da er in der Essenz mit der Deklaration des Präsidenten übereinstimmt. Nachdem ihm angeboten wird, dass der Artikel neben der Deklaration des Vorsitzenden erscheint, zieht er seinen Artikel zurück. 

Der Vorsitzende Martens verkündet, dass der Artikel sowie alle dazu geäußerten Bedenken in den Aufzeichnungen festgehalten werden. Diese Vorgehensweise wird einstimmig angenommen.

Anhand dieser hier auf das Wesentliche reduzierten Diskussion wird das bereits angesprochene Spannungsverhältnis deutlich. Auf der einen Seite stehen die kleinen Staaten, die häufig auf die eigene Bevölkerung angewiesen sind und auf der anderen Seite die Großmächte, die über große Heere verfügen. Doch darüber hinaus gibt es ein weiteres Spannungsverhältnis zwischen den nationalen Einzelinteressen und dem Willen der Delegierten eine gemeinsame Lösung zu finden. Insbesondere das zweite Verhältnis wird an dieser Diskussion sehr deutlich. Obwohl der deutsche Delegierte große Bedenken sogar gegenüber Artikel 10 hat, schweigt er dazu, damit eine Lösung gefunden werden kann. In der gleichen Weise ist auch das Zurückziehen der schweizerischen Änderungsanträge zu verstehen.

Dieser Wille zur Einigung insbesondere in dieser Unterkommission ist unter anderem mit dem Scheitern der Brüsseler Deklaration zu begründen. Die Delegierten wollen durch möglichst einstimmige Abstimmungen der erarbeiteten Ordnung größtmögliches Gewicht beimessen. Die Intention aller Anwesenden war demnach klar darauf ausgerichtet zu einer konsensfähigen Einigung zu kommen. Doch es wird in anderen Sitzungen der gleichen Unterkommission ebenso deutlich, dass von den Delegierten nicht immer auf ihre Regierungen geschlossen werden kann. Dies wird später in der Analyse gerade im Hinblick auf den ersten Weltkrieg von Interesse sein. Auch die Bedeutung der Martens'schen Klausel zeigt sich in dieser Diskussion. Ohne diese Deklaration hätte es womöglich keine Einigung gegeben.

\section{Zweite Friedenskonferenz}

\subsection{Vorbereitung der zweiten Konferenz}

Die Vorbereitungen für die zweite Konferenz begannen 1904 auf Initiative des US-amerikanischen Präsidenten Theodore Roosevelt. Die offizielle Einladung wurde schließlich 1906 von Russland an 44 Staaten (Signatarmächte der ersten Konferenz und weitere Staaten) verschickt. Die weiteren Staaten kommen bis auf eine Ausnahme aus Süd- und Mittelamerika.

Sie umfassen Argentinien, Bolivien, Brasilien, Chile, die Dominikanische Republik, Equador, Guatemala, Haiti, Kolumbien, Nicaragua, Panama, Paraguay, El Salvador, Uruguay, Venezuela und die europäische Ausnahme Norwegen, welches 1905 unabhängig wurde. Insgesamt waren 174 Delegierte auf der Konferenz vertreten.\cite{Buss1992}

In dem Rundschreiben Russlands vom April 1906 werden die Ziele der zweiten Konferenz benannt\cite{Scott-V1-1921}:

\begin{itemize}
	\item Verbesserungen der friedlichen Streitbeilegung bzgl. des Schiedsgerichts
	\item Erweiterungen der Konvention von 1899 bzgl. des Landkriegs
	\item Diskussion der Erneuerung einer ausgelaufenen Deklaration von 1899
	\item Erstellung einer Konvention bzgl. des Seekriegs
	\item Erweiterungen einer Konvention von 1899 bzgl. der Anpassung von Regeln der Genfer Konvention von 1864 für den Seekrieg
\end{itemize}

Anhand der hier sichtbaren Zielsetzung wird klar, dass die zweite Friedenskonferenz für die Landkriegsordnung eine untergeordnete Rolle spielt.

\subsection{Struktur der zweiten Konferenz}

Auch die zweite Konferenz war in verschiedene Kommissionen und Unterkommissionen unterteilt. In den folgenden Zeilen werden die Aufgabengebiete der einzelnen Kommissionen kurz umrissen.

Die erste Kommission beschäftigte sich mit den Verbesserungen für die Regeln der friedlichen Streitbeilegung und den Belohnungen im Seekrieg (engl.: "`maritime prizes"').

In der zweiten Kommission wurden die Verbesserungen für die Landkriegsordnung, einige Deklarationen von 1899, die Rechte und Pflichten von neutralen Staaten auf dem Land und der Beginn von Feindseligkeiten behandelt.

Die dritte und vierte Kommission befassten sich mit zahlreichen Fragen der maritimen Kriegführung, die für diese Analyse keine herausragende Bedeutung haben. 

Aufgrund der hohen Teilnehmeranzahl wurden jedoch weitergehende Regelungen nötig im Vergleich zu 1899. So wurden in der zweiten Plenarsitzung zwölf Artikel als Geschäftsordnung einstimmig angenommen. Außerdem wurde die Redezeit jeder einzelnen Person auf 10 Minuten am Stück begrenzt, was zur damaligen Zeit eine übliche Regelung in vielen Parlamenten war.\cite{Scott-V1-1921}

Durch die Größe der Konferenz und damit auch der zweiten Kommission ist es daher im Rahmen dieses Papers nicht möglich auf die Diskussionen einzugehen. Daher wird im nächsten Abschnitt direkt mit der Analyse der fertigen Landkriegsordnung begonnen.

\section{Haager Landkriegsordnung}

Die Landkriegsordnung war eine von 13 Konventionen, die am Ende der Friedenskonferenzen als Ergebnis stand. In Langform heißt sie "`Convention regarding the laws and customs of war on land"'\cite{Scott-V1-1921}. In dem einleitenden Text findet sich auch die Martens'sche Klausel wieder. Die eigentlichen Regulierungen finden sich im Anhang zur Konvention. Die Konvention selber enthält Artikel zu den Formalia und ist an sich nicht interessant. Lediglich ein Artikel sticht dort hervor und dies ist Artikel 2, der im weiteren Verlauf den Namen Allbeteiligungsklausel bekam. In kompletter Länge lautet er folgendermaßen:

\begin{quotation}
	The provisions contained in the Regulations referred to in Article 1, as well as in the present Convention, do not apply except between contracting Powers, and then only if all the belligerents are parties to the Convention.
\end{quotation}

Dieser Artikel wird für die Analyse der Auswirkungen noch eine wichtige Rolle spielen. Zunächst sind jedoch die eigentlichen Regulierungen von Interesse. Diese Regulierungen werden auch Haager Landkriegsordnung (kurz: HLKO) genannt. Im weiteren Verlauf wird der Begriff der Landkriegsordnung verwendet. Die HLKO unterteilt sich in drei Abschnitte: Der erste Abschnitt befasst sich mit den Kriegsteilnehmern (engl.: belligerents), der zweite Abschnitt mit feindlichen Auseinandersetzungen, sprich den militärischen Aktivitäten selbst. Der dritte Abschnitt schließlich beschäftigt sich mit den Regeln einer Besatzung. Im Rahmen dieser Analyse wird auf alle Abschnitte eingegangen, der Fokus wird allerdings auf die ersten beiden Abschnitte gelegt. Der erste Abschnitt beginnt im ersten Kapitel auch zugleich mit dem Kombattantenstatus. 

Artikel 1 und 2 entsprechen den bereits besprochenen Artikeln 9 und 10 der Brüsseler Deklaration. Aufgrund ihrer Signifikanz folgen sie hier in kompletter Länge.

Artikel 1 befasst sich mit dem Kombattantenstatus im Allgemeinen.
\begin{quotation}
	The laws, rights, and duties of war apply not only to armies, but also to militia and volunteer corps fulfilling the following conditions:
\begin{enumerate}
\item That they be commanded by a person responsible for his subordinates;
\item That they have a fixed distinctive emblem recognizable at a distance;
\item That they carry arms openly; and
\item That they conduct their operations in accordance with the laws and customs of war.
\end{enumerate}
In countries where militia or volunteer corps constitute the army, or form part of it, they are included under the denomination  ``army''.
\end{quotation}

Artikel 2 erweitert diese Einteilung für unbesetztes Gebiet und schließt dort auch die Levée en masse mit ein.
\begin{quotation}
	The population of a territory which has not been occupied who, on the
approach of the enemy, spontaneously take up arms to resist the invading troops
without having had time to organize themselves in accordance with Article 1,
shall be regarded as belligerents if they carry arms openly and if they respect
the laws and customs of war.
\end{quotation}

Aus diesen beiden Artikel geht sehr klar hervor, dass ein asymmetrischer Kampf nicht gestattet ist. Die Gründe dafür sind vielfältig, aber der deutsche Delegierte auf der ersten Friedenskonferenz hat es sehr gut auf den Punkt gebracht\footnote{siehe der Unterabschnitt über Schlüsselpunkte bei der ersten Haager Friedenskonferenz}: Wenn eine Armee friedliche Zivilisten von potentiellen Untergrundkämpfern nicht unterscheiden kann, wird sie alleine aus Eigenschutz alle Zivilisten unter Generalverdacht stellen. Ein fast perfide aktuelles Beispiel zu dieser über 100 Jahre alten Aussage ist der Krieg in Afghanistan. Auf der einen Seite die technisch hochgerüsteten Westarmeen und auf der anderen Seite die technologisch schwächeren Taliban. Diese kämpfen jedoch keine offene Feldschlacht, sondern greifen aus dem Hinterhalt einzelne Patrouillen an. Das Ergebnis ist ein Generalverdacht gegen alle Zivilisten und daraus resultierend viele zivile Kollateralschäden. Aber Afghanistan ist keine Ausnahme. Vielmehr sind der asymmetrische Konflikt zur Regel und die klassischen Kriege des 19. Jahrhunderts die Ausnahme geworden.

Die HLKO hat demnach mit diesen beiden Artikeln immer noch eine unvorstellbare Brisanz für die heutige Welt. Doch sie besteht aus mehr als nur dem Kombattantenstatus. Im zweiten Kapitel befasst sie sich mit der Behandlung von Kriegsgefangenen.\cite{Scott-V1-1921} Viele der Regelungen sind mittlerweile veraltet, aber ein paar zentrale Aspekte behalten weiterhin Relevanz:

\begin{itemize}
	\item Kriegsgefangene dürfen nicht am Krieg teilnehmen (inklusive kriegsrelevanter Arbeit)
	\item sie müssen wie die eigene Armee versorgt werden
	\item sie sind nach Friedensschluss freizulassen
\end{itemize}

Kapitel 3 beschäftigt sich mit den Verwundeten und Kranken. Der einzige Artikel dieses Kapitels verweist für die entsprechenden Regelungen auf die Genfer Konvention. Im zweiten werden die erlaubten bzw. verbotenen militärischen Aktivitäten behandelt. Dabei ist gleich der erste Artikel -- Artikel 22 -- von besonderem Interesse. Dieser fasst die Grundlage aller weiteren Regulierungen gut zusammen:

\begin{quotation}
	The right of belligerents to adopt means of injuring the enemy is not unlimited.
\end{quotation}

Diese wichtige Erkenntnis der HLKO ist das Ergebnis eines langen Erkenntnisprozesses, der mit Francisco de Vitoria 1532 begann. Der direkt folgende Artikel 23 ist sodann auch eine Bestätigung eben jenes Grundsatzes, in dem eine bestimmte Liste von Waffen und Aktionen verboten wird. Wohlwissend dass eine abschließende Nennung aller potentiell neuen Grausamkeiten nicht möglich ist, erlaubt der Artikel eine weitere Einschränkung über das beschriebene Maß hinaus.\cite{Scott-V1-1921} Die folgenden Waffen oder Maßnahmen werden u.a. explizit verboten:

\begin{itemize}
	\item Nutzung von Gift oder vergifteten Waffen
	\item hinterhältige Ermordung oder Verwundung von Angehörigen der feindlichen Nation oder Armee
	\item Tötung oder Verwundung eines wehrlosen Feindes
	\item Verwendung von Waffen oder Material welches unnötigen Schaden verursacht
	\item Zerstörung oder Aneignung von feindlichem Eigentum solange es nicht zwingend militärisch notwendig ist
\end{itemize}

Diese und weitere Regeln in nachfolgenden Artikeln des zweiten Abschnittes können dem humanitären Völkerrecht zugerechnet werden, welches sich um eine Schadensbegrenzung im Falle eines Krieges bemüht. So werden Offiziere angehalten bei bevorstehenden Bombardierungen, welche nicht mit einem Sturmangriff verbunden sind, die Behörden der betroffenen Stadt im Voraus über die Bombardierung zu informieren.\cite{Scott-V1-1921}

Im dritten Abschnitt wird die Besatzung behandelt. Hier wird der humanitäre Ansatz sehr deutlich, was sich zum Beispiel an Artikel 46 zeigt\cite{Scott-V1-1921}:

\begin{quotation}
	Family honor and rights, the lives of persons, and private property, as well as religious convictions and practice, must be respected.

Private property cannot be confiscated.
\end{quotation}

Die Enteignung von Privateigentum wird hier explizit verboten. Ein weiterer brandaktueller Artikel ist Artikel 50:

\begin{quotation}
	No general penalty, pecuniary or otherwise, shall be inflicted upon the population on account of the acts of individuals for which they cannot be regarded
as jointly and severally responsible.
\end{quotation}

Dieser Artikel verbietet ziemlich klar die Kollektivbestrafung. Damit bietet der Artikel auch einen guten Rückbezug zur Definition von Kombattanten. Diese Regelung zum Schutz vor Kollektivbestrafung ist nur möglich, weil es der Bevölkerung des besetzten Gebietes untersagt ist gegen den Besatzer zu kämpfen. Es gibt noch weitere Regelungen, aber das waren die wesentlichen Artikel der HLKO.

\section{Auswertung}
Nachdem die Konferenzen selber und die resultierende Haager Landkriegsordnung präsentiert wurden, geht es jetzt an die Auswertung der HLKO. Der wohl interessanteste Konflikt dafür ist der wenig später stattfindene Erste Weltkrieg. Anhand der Verwendung von Giftgas im Krieg mag man schnell zum Schluss kommen, dass die HLKO nutzlos war und sich niemand daran gehalten hat. Ein weiteres beliebtes Argument ist die Allbeteiligungsklausel: Es waren Nicht-Signatarstaaten Kriegsteilnehmer und daher ist die HLKO wirklos. Ganz so einfach ist es dann aber doch nicht.

Es stimmt sicherlich, dass die Ordnung nicht wirklich eingehalten worden ist. Gerade die Bestimmungen zu verbotenen Waffengattungen wurden bspw. im Fall des Giftgases gebrochen. Aber es gibt auch die andere Seite der Haager Landkriegsordnung. Die Bestimmungen zum Kombattantenstatus wurden weitgehend eingehalten\cite{Buss1992} und das obwohl es später auch Kriegsteilnehmer gab, die die HLKO nicht unterschrieben hatten. Auf der anderen Seite waren zu Beginn des Krieges die Kriegsteilnehmer allesamt Vertragsparteien der HLKO\cite{Buss1992}. Dennoch wurden Regelungen gebrochen. Die Allbeteiligungsklausel hat demnach keine wirkliche Argumentationskraft.

Von dem Giftgas abgesehen sind im 1. WK auch viele neue Waffengattungen zum Einsatz gekommen, von denen in der HLKO keine Rede war und die aufgrund ihrer Neuheit auch keinen anderen restriktiven Regulierungen unterlagen. 

Der mitunter größte Knackpunkt der HLKO bleibt jedoch das fehlende Verifizierungsregime. Allerdings sind diese Regelungen, anders als etwa der spätere Nichtverbreitungsvertrag für Nuklearwaffen, auch nicht so einfach zu überprüfen. Da sie sich mit den Kriegsaktivitäten beschäftigen, müsste zunächst einmal Krieg herrschen, damit überhaupt verifiziert werden kann. Dann müsste es bei jeder Armeeeinheit einen Zuständigen für die Verifizierung geben, der aufpasst, dass alles mit rechten Dingen zugeht. Wie praktikabel dies ist kann hinterfragt werden. Auf jeden Fall müsste es jedoch eine im Nachgang zum Krieg stattfindene Verurteilung von Kriegsverbrechen geben. Aus diesem Anreiz heraus ist dann 2002 der Internationale Strafgerichtshof entstanden. Allerdings sind noch nicht alle Staaten dem dazugehörigen Vertrag beigetreten. Ein promintenter Vertreter dieser Gruppe sind die USA\footnote{Signatarstaat, aber noch nicht ratifiziert}.

\section{Schlussbemerkungen}
Im Verlauf dieses Papers ist deutlich geworden, dass bei vielen Punkten in der HLKO eine Aktualität gegeben ist, die zunächst unwahrscheinlich erscheint. Viele ihrer Regelungen könnten 1:1 in die heutige Zeit ohne Bedeutungsverlust transferiert werden. Einige weitere müssten lediglich angepasst werden. Diese Tatsache zeigt in besonderer Weise, was den Delegierten 1899 und 1907 gelungen ist. Sie haben nicht nur damals geltendes Kriegsgewohnheitsrecht kodifiziert, sondern auch ein mehrheitlich zeitloses Werk geschaffen, welches zur Pflichtlektüre eines jeden Militärs gehören sollte. Diese Tatsache ist zugleich aber auch eine herbe Enttäuschung, denn offenbar hat sich in den letzten 100 Jahren so wenig zum Positiven geändert, dass die gleichen Regelungen immer noch Anwendung finden können.

Es hat fast 100 Jahre seit dem Ende der zweiten Haager Friedenskonferenz gedauert, bis 2002 der internationale Strafgerichtshof eingerichtet wurde. Noch immer sind nicht alle Staaten dem Vertrag von Rom (engl.: Rome Statute) beigetreten. Viele Kriegsverbrechen der letzten Zeit werden nicht verfolgt und der Drohnenkrieg lässt die Regelung zum Kombattantenstatus verschwimmen.

Der asymmetrische Konflikt ist nicht nur ärgerlich für die betroffenen Armeen, sondern schadet in immenser Weise der Zivilbevölkerung auf beiden Seiten. Durch Kollateralschäden bei Drohnenangriffen werden Antipathien gegen die USA erzeugt. Wird dies kombiniert mit schlechter Infrastruktur und geringen Chancen im Leben, dann ergibt sich ein toxisches Gemisch, welches sich in Richtung Terrorismus entwickeln kann. Dieser wiederum trifft die Zivilbevölkerung auf der anderen Seite. Als Reaktion darauf folgt der nächste Drohnenangriff. Daran wird perfekt deutlich, warum ein asymmetrischer Konflikt in seinen Konsequenzen wiederum symmetrisch ist. Die Spirale der Gewalt gegen Zivilbevölkerung kann nur durchbrochen werden, wenn eine Seite den ersten Schritt tut. Dies kann im Prinzip nur die organisierte staatliche Seite sein, die ein Gewaltmonopol über ihre Beteiligung am Konflikt hat.

Allerdings mangelt es angesichts dieser Situation nicht an Gesetzen oder Regeln, sondern an der Umsetzung eben jener. Daher muss mit Hochdruck an der Motivsuche für etwaige Aktionen gearbeitet werden. Wenn die Motive für terroristische Aktionen oder ähnliche asymmetrische Handlungen gefunden sind, dann muss daran gearbeitet werden diese Motive zu invalidieren. Das beste Rezept dazu ist nicht Militär, sondern ein gut funktionierender Staat, der nicht korrupt ist und jungen Leuten Perspektiven bietet. In solch einem Staat werden es Terrororganisationen sehr viel schwerer haben Unterstützer zu finden, als in einem sogenannten "`failed state"'.

Vor diesem Hintergrund und bezugnehmend auf die Intentionen der Haager Friedenskonferenzen kann nur an alle Zuständigen appelliert werden keine militärischen Operationen gutzuheißen, denn diese schaden am Ende allen Beteiligten. Sie verbrennen Geld und zerstören Familien auf beiden Seiten. Stattdessen sollte sich wieder auf die in der HLKO gefassten Werte zurückbesonnen und gemeinsam nach einer Lösung gesucht werden.
%%%%%%%%%%%%%%%%%%%%%%%%%%%%%%%%%%%%%%%%%%%%%%%%%%%%%%%%%%%%%%%%%%%%%%
\newpage

\printbibliography
\addcontentsline{toc}{section}{Literaturverzeichnis}


%%%%%%%%%%%%%%%%%%%%%%%%%%%%%%%%%%%%%%%%%%%%%%%%%%%%%%%%%%%%%%%%%%%%%%
%\newpage
%\addcontentsline{toc}{section}{Anhang}

\end{document}
