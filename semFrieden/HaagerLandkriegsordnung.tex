%!TEX encoding = UTF-8 Unicode
\documentclass[12pt]{scrartcl}
%\usepackage[applemac]{inputenc} % Mac-Umlaute direkt verwenden öäüß
%\usepackage[isolatin]{inputenc} % PC-Umlaute direkt verwenden 
\usepackage[utf8]{inputenc} % Unicode funktioniert unter Windows, Linux und Mac
\usepackage[T1]{fontenc}
\usepackage[ngerman]{babel}
\usepackage{csquotes}
\usepackage[
backend=biber,
bibstyle=ieee,
citestyle=ieee
]{biblatex}
%\usepackage{ngerman}
\usepackage{graphicx}
\usepackage[hidelinks]{hyperref}\urlstyle{rm}
\usepackage{times}
\usepackage[scaled]{helvet}
\usepackage{a4wide}
\usepackage{rotating}
\usepackage{listings}\lstset{breaklines=true,breakatwhitespace=true,frame=leftline,captionpos=b,xleftmargin=6ex,tabsize=4,numbers=left,numberstyle=\ttfamily\footnotesize,basicstyle=\ttfamily\footnotesize}
\sloppy
\setlength{\parindent}{0em}
\setlength{\parskip}{1.2ex plus 0.5ex minus 0.5ex}
\pagestyle{plain}
\addbibresource{semFrieden.bib}

\begin{document}

\newpage
\thispagestyle{empty}
\begin{center}\Large
Universität Hamburg \par
Fachbereich Informatik
\vfill
Seminararbeit
\vfill
{\Large\textsf{\textbf{Haager Landkriegsordnung}}\par}
\vfill
vorgelegt von 
\par\bigskip
Jim Martens \par
Matrikelnummern 6420323 \par
Studiengang BSc. Informatik
\end{center}

\newpage
\section*{Zusammenfassung}


\newpage
\tableofcontents

\newpage
\section{Vorbemerkung}


\section{Weg zur Friedenskonferenz}
\section{Erste Friedenskonferenz}
\section{Zweite Friedenskonferenz}
\section{Auswertung}
\cite{Gasser1991}
\cite{Lingen2014}
\cite{Buss1992}
\cite{Fraenkel1968}
\cite{Heffter1951}
\cite{Reich2010}
\cite{Scott1920}
\cite{Scott1921}



\section{Schlussbemerkungen}


%%%%%%%%%%%%%%%%%%%%%%%%%%%%%%%%%%%%%%%%%%%%%%%%%%%%%%%%%%%%%%%%%%%%%%
\newpage

\printbibliography
\addcontentsline{toc}{section}{Literaturverzeichnis}


%%%%%%%%%%%%%%%%%%%%%%%%%%%%%%%%%%%%%%%%%%%%%%%%%%%%%%%%%%%%%%%%%%%%%%
\newpage
\addcontentsline{toc}{section}{Anhang}

\end{document}
