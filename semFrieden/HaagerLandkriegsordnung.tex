%!TEX encoding = UTF-8 Unicode
\documentclass[12pt]{scrartcl}
%\usepackage[applemac]{inputenc} % Mac-Umlaute direkt verwenden öäüß
%\usepackage[isolatin]{inputenc} % PC-Umlaute direkt verwenden 
\usepackage[utf8]{inputenc} % Unicode funktioniert unter Windows, Linux und Mac
\usepackage[T1]{fontenc}
\usepackage[ngerman]{babel}
\usepackage{csquotes}
\usepackage[
backend=biber,
bibstyle=ieee,
citestyle=ieee
]{biblatex}
%\usepackage{ngerman}
\usepackage{graphicx}
\usepackage[hidelinks]{hyperref}\urlstyle{rm}
\usepackage{times}
\usepackage[scaled]{helvet}
\usepackage{a4wide}
\usepackage{rotating}
\usepackage{listings}\lstset{breaklines=true,breakatwhitespace=true,frame=leftline,captionpos=b,xleftmargin=6ex,tabsize=4,numbers=left,numberstyle=\ttfamily\footnotesize,basicstyle=\ttfamily\footnotesize}
\sloppy
\setlength{\parindent}{0em}
\setlength{\parskip}{1.2ex plus 0.5ex minus 0.5ex}
\pagestyle{plain}
\addbibresource{semFrieden.bib}

\begin{document}

\newpage
\thispagestyle{empty}
\begin{center}\Large
Universität Hamburg \par
Fachbereich Informatik
\vfill
Seminararbeit
\vfill
{\Large\textsf{\textbf{Haager Landkriegsordnung}}\par}
\vfill
vorgelegt von 
\par\bigskip
Jim Martens \par
Matrikelnummern 6420323 \par
Studiengang BSc. Informatik
\end{center}

\newpage
\section*{Zusammenfassung}

\newpage
\tableofcontents

\newpage
\section{Vorbemerkung}
Aus den Haager Friedenskonferenzen ging die Haager Landkriegsordnung hervor. Die Konferenzen gelten als Meilenstein in der Kodifikation von geltendem Kriegsvölkerrechts. Doch sie sind keineswegs eine plötzliche Erscheinung gewesen, sondern stehen am Ende eines Weges, der sich über mehrere Jahrhunderte erstreckt. In diesem Paper wird die Haager Landkriegsordnung hauptsächlich unter dem Punkt des Kombattantenstatus betrachtet. Denn die Unterscheidung zwischen rechtmäßig Kämpfenden und der unschuldigen Zivilbevölkerung ist Grundlage unseres heutigen westlichen Werteverständnisses.

Im Folgenden wird daher der historische Weg mit den wichtigsten Meilensteinen erläutert, um anschließend beide Konferenzen vorzustellen, wobei natürlich auch die übrigen Regelungen, wie z.B. zu erlaubten Waffengattungen, Erwähnung finden werden. Beide Konferenzen werden schließlich unter dem Aspekt des Kombattantenstatus ausgewertet, um mit Schlussbemerkungen zu schließen.

Der Weg zu den Konferenzen orientiert sich an Buß\cite{Buss1992}.
\section{Weg zu den Friedenskonferenzen}
Der Weg zu den Haager Friedenskonferenzen beginnt nicht erst im 19. Jahrhundert, sondern bereits weit früher im Mittelalter. Eine Kodifikation des Kriegsrechts setzt voraus, dass die Einhaltung dieses Rechts auch sichergestellt werden kann. Dazu ist Disziplin in der kämpfenden Truppe notwendig, was wiederum andere Voraussetzungen hatte.

Die erste Möglichkeit einer solchen Kontrolle auf Einhaltung hatten die Römer mit ihren hochdisziplinierten Truppen. Allerdings wurde die Möglichkeit nicht genutzt. Die gegnerische Zivilbevölkerung war häufig genauso Ziel militärischer Handlungen wie die gegnerischen Soldaten. Ein paar hundert Jahre später gab es mit dem Rittertum ausgebildete Kämpfer, die jedoch Einzelkämpfer waren und keineswegs die Zivilbevölkerung schonten.

Erst mit den sogenannten Schweizer Haufen gelang eine Disziplinierung einer Armee, sodass Anweisungen von Vorgesetzten durchgesetzt werden konnten. So war die Plünderung und Schändung der Zivilbevölkerung (hauptsächlich Frauen und Kinder) untersagt.

Francisco de Vitoria hat im Jahre 1532 anhand der Kolonialisierung Amerikas durch die Spanier, die er für legal hielt, erklärt, dass ein Krieg für beide Parteien rechtmäßig sein kann. Zudem hält er fest, dass ein rechtmäßiger Krieg den Einsatz jeglicher Mittel erlaubt, um ihn schnellstmöglich siegreich zu beenden. Ist der Krieg jedoch beiderseitig gerechtfertigt, würde dies in einer Eskalation der Mittel enden, wodurch eine beiderseitige Mäßigung notwendig ist. Seine Ausführungen bilden zudem die Grundlage für die spätere Separierung zwischen Kämpfenden und der Zivilbevölkerung.

Der nächste große Schritt ereignete sich mit dem Unabhängigkeitskampf der späteren Niederlande gegen Spanien. Wilhelm von Oranien leitete dabei den Widerstand und schuf ein stehendes Söldnerheer, dass diszipliniert war, regelmäßig entlohnt wurde und als erstes Heer nach noch heute gültigen Grundsätzen aufgebaut war. 

Einen Einschnitt gab es mit dem Dreißigjährigen Krieg, in dem überwiegend undisziplinierte Söldnerheere aufeinander trafen. Außerdem wurde überwiegend eine Ermattungsstrategie verfolgt, bei der offene Feldschlachten vermieden und eher Belagerungen und kleinere Gefechte ausgetragen wurden. Im Rahmen dieser Strategie dauerten Kriege auch länger und die Versorgung der Soldaten erfolgte meist aus dem Kampfgebiet. Durch diese kontinuierlichen Plünderungen zur Ernährung der Truppe war ein effektiver Schutz der Zivilbevölkerung, wie er zuvor teilweise in Kriegsartikeln bereits gefasst war, nicht möglich.

Während dieser Zeit hat Hugo Grotius ein dreibändiges Werk (De iure bella ac pacis) verfasst, in welchem er das "`freiwillige"' Völkerrecht in einer realistischen Weise darstellt und im dritten Buch auf ethische, moralische und religiöse Grundsätze hinweist. Obgleich dieses Rechtsbuch einen wichtigen Beitrag zum späteren Völkerrecht darstellt, so entspricht es doch in keinem Maße der Kodifikation in den Haager Konferenzen.

Eine Zäsur stellte der Westfälische Frieden von 1648 dar. Von da an wurden stehende Heere die Normalität und die Souveränität der einzelnen Staaten garantiert. Durch diese Entwicklung, die mit dem stehenden Heer der Oranier (spätere Utrechter Union und Niederlande) begann, wurde eine effektive Unterscheidung zwischen Kämpfenden und Zivilbevölkerung erstmals flächendeckend möglich. Dieser Frieden bildet daher in vielerlei Hinsicht die Grundlage dafür, dass 250 Jahre später die Haager Friedenskonferenzen stattfinden konnten.

In der Folgezeit gab es noch weitere Ereignisse, doch für den geschichtlichen Hintergrund sollte dies ausreichen. Im Folgenden wird eine zeitlich nähere Einordnung vorgenommen, die in der zweiten Hälfte 19. Jahrhundert verbleibt.

Der Lieber Code ist im Rahmen des Sezessionskriegs zwischen der USA und der Konföderation entstanden. Dieser ist in vielen Punkten speziell auf den amerikanischen Bürgerkrieg ausgerichtet, enthält aber eine detaillierte Auflistung dessen, was zu der Zeit als geltendes Kriegsrecht angenommen wurde. Er diente daher als Grundlage für die weiteren Verhandlungen in den kommenden Jahrzehnten.

Die Brüsseler Deklaration von 1874 schließlich beinhaltet die hauptsächliche Vorarbeit zu den Haager Friedenskonferenzen und stellt auch den letzten Schritt vor der ersten Konferenz dar, auf die im nächsten Abschnitt näher eingegangen wird. Allerdings scheitert die Konferenz an der Frage der Levée en masse.

\section{Erste Friedenskonferenz}

\subsection{Oxford Manual}

Das Scheitern der Brüsseler Konferenz veranlasste Wissenschaftler dazu sich mit dem Problem der kriegsrechtlichen Kodifikation zu beschäftigen. Dabei wurde der Versuch unternommen den Inhalt der kriegsrechtlichen Bestimmungen der Brüsseler Deklaration derart umzuformulieren, dass sie der europäischen Praxis der Kriegführung entsprachen. Die Hauptarbeit wurde dabei vom Institut für internationales Recht durchgeführt. Diesem Institut gehörten ausgewählte Wissenschaftler an, die ohne amtlichen Auftrag arbeiteten. Bei der Annahme des Oxford Manual (nach dem Abstimmungsort Oxford benannt) waren namhafte Völkerrechtler anwesend. Die Abstimmung fand am 9. September 1880 in Oxford statt.\cite{Buss1992}

In Bezug auf den Kombattantenstatus vertreten die Wissenschaftler die Position der Staaten und auch bei der umstrittenen Levée en masse folgt das Institut weitgehend den Großmächten. Lediglich auf unbesetztem Gebiet war demnach die unorganisierte Volkserhebung rechtmäßig. Auf besetztem Gebiet stellt das Oxford Manual eine Gehorsamspflicht der Bevölkerung gegenüber dem Besatzer fest.\cite{Buss1992}

\subsection{Vorarbeiten zur ersten Friedenskonferenz}

Im Auftrag des Zaren Nikolaus II. verschickte der russische Außenminister Graf Mouravieff ein diplomatisches Rundschreiben, dass zu einer internationalen Konferenz einlädt. Zusätlich zu allen in Petersburg diplomatisch vertretenen Staaten ging die Einladung an Luxemburg, Montenegro und Siam. Fast alle Staten nahmen die Einladung an, sodass insgesamt 26 Teilnehmer zu verzeichnen waren. Damit war die zu einer Konferenz bereiten Staaten fast doppelt so hoch, wie 25 Jahre zuvor in Brüssel.\cite{Buss1992}

Auf Einladung der niederländischen Regierung begann die Konferenz am 18. Mai 1899 in Den Haag. Sie war nicht ausdrücklich auf die Kodifikation ausgerichtet, sondern vielmehr wollten die an der Konferenz teilnehmenden Staaten Instrumente zur friedlichen Streitbeilegung schaffen. Insofern war auch Rüstungsbeschränkung Ziel der Konferenz.

\subsection{Struktur der Konferenz}

Die Konferenz bestand aus 3 Kommissionen, wobei sich jede Kommission mit einem Teil der Themen beschäftigte. Jede Kommission konnte ihrerseits in weitere Unterkommissionen unterteilt sein. Die erste Kommission beschäftigte sich mit der wohl schwierigsten Frage, nämlich der Rüstungsbeschränkung im Landkrieg und im Seekrieg. In der Kommission gab es zwei Unterkommissionen: Eine diskutierte über militärische Fragen (bzw. landbezogene Fragen), während die andere sich mit Fragen des Seekriegs auseinandersetzte.\cite{Scott1920}

Die zweite Kommission unter dem Vorsitz von Herrn Martens, einem russischen Delegierten, beschäftigte sich einerseits mit der Ausweitung der Regeln der Genfer Konvention von 1864 auf den Seekrieg und der Revision der Brüsseler Deklaration von 1874. Es wurden zwei Unterkommissionen gebildet, wobei sich die erste mit den Fragen zur Genfer Konvention und die zweite mit der Revision der Brüsseler Deklaration beschäftigte.\cite{Scott1920}

Eine etwas andere Struktur findet sich in der dritten Kommission. Diese wurde nicht in Unterkommissionen unterteilt. Stattdessen wurde ein Komitee gegründet, um die zugrundeliegenden Schriftstücke zu analysieren. Insgesamt beschäftigte sich die dritte Kommission mit Fragen zur Schlichtung von Disputen zwischen Staaten.\cite{Scott1920}

Für die Untersuchung der Haager Landkriegsordnung (kurz: HLKO) ist insbesondere die zweite Unterkommission der zweiten Kommission für Interesse. Die folgenden Abschnitte werden sich daher auf diesen Bereich konzentrieren.

\begin{itemize}
	\item Begriff des Kombattanten\cite{Buss1992}
	\item Levée en masse\cite{Scott1920}
	\item Chemiewaffen\cite{Scott1920}
\end{itemize}
\section{Zweite Friedenskonferenz}
\begin{itemize}
	\item Levée en masse (Deutscher Änderungsantrag)
\end{itemize}
\section{Auswertung}
\begin{itemize}
	\item gelungene Kodifikation des vorher bereits geltenden Kriegsgewohnheitsrechts
	\item Bestimmungen zum Kombattantenstatus und Behandlung von Kombattanten auch im 1. WK weitgehend eingehalten
	\item Allbeteiligungsklausel nicht so problematisch, wie es scheint
\end{itemize}
\cite{Gasser1991}
\cite{Lingen2014}
\cite{Fraenkel1968}
\cite{Heffter1951}
\cite{DeutschesReich2010}
\cite{Scott1921}



\section{Schlussbemerkungen}


%%%%%%%%%%%%%%%%%%%%%%%%%%%%%%%%%%%%%%%%%%%%%%%%%%%%%%%%%%%%%%%%%%%%%%
\newpage

\printbibliography
\addcontentsline{toc}{section}{Literaturverzeichnis}


%%%%%%%%%%%%%%%%%%%%%%%%%%%%%%%%%%%%%%%%%%%%%%%%%%%%%%%%%%%%%%%%%%%%%%
\newpage
\addcontentsline{toc}{section}{Anhang}

\end{document}
