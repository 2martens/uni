\documentclass[10pt,a4paper,oneside,ngerman,numbers=noenddot]{scrartcl}
\usepackage[T1]{fontenc}
\usepackage[utf8]{inputenc}
\usepackage[ngerman]{babel}
\usepackage{amsmath}
\usepackage{amsfonts}
\usepackage{amssymb}
\usepackage{bytefield}
\usepackage{paralist}
\usepackage{gauss}
\usepackage{pgfplots}
\usepackage{textcomp}
\usepackage[locale=DE,exponent-product=\cdot,detect-all]{siunitx}
\usepackage{tikz}
\usepackage{algpseudocode}
\usepackage{algorithm}
\usepackage{mathtools}
\usepackage{hyperref}
%\usepackage{algorithmic}
%\usepackage{minted}
\usetikzlibrary{automata,matrix,fadings,calc,positioning,decorations.pathreplacing,arrows,decorations.markings}
\usepackage{polynom}
\polyset{style=C, div=:,vars=x}
\pgfplotsset{compat=1.8}
\pagenumbering{arabic}
%\def\thesection{\arabic{section})}
%\def\thesubsection{(\alph{subsection})}
%\def\thesubsubsection{(\roman{subsubsection})}
\makeatletter
\renewcommand*\env@matrix[1][*\c@MaxMatrixCols c]{%
  \hskip -\arraycolsep
  \let\@ifnextchar\new@ifnextchar
  \array{#1}}
\makeatother
\parskip 12pt plus 1pt minus 1pt
\parindent 0pt

\DeclarePairedDelimiter\abs{\lvert}{\rvert}%
\DeclarePairedDelimiter{\ceil}{\lceil}{\rceil}

%switch starred and non-starred (auto-size)
\makeatletter
\let\oldabs\abs
\def\abs{\@ifstar{\oldabs}{\oldabs*}}
\makeatother

\hypersetup{
    colorlinks,
    citecolor=black,
    filecolor=black,
    linkcolor=black,
    urlcolor=black
}

\begin{document}
\author{Jim Martens}
\title{Ergebnis der CG-Vorbesprechung}
\maketitle

\section{Tagesordnung}

TOP 5 ``70 Jahre Befreiung'' vor TOP 3 ``Bestätigung RIS-Referenten'' behandeln

\section{Abstimmungen}

\begin{description}
	\item[Bestätigung RBCS-ReferentInnen] dafür
	\item[Bestätigung der Wahlordnung Queer-Referat] 1. dafür, 2. dagegen, 3. dagegen
	\item[Bestätigung der RIS-Referenten] dagegen
	\item[Nachwahl Ausschuss gegen Rechts] dafür, außer es ist Elvis und Co.
	\item[70 Jahre Befreiung] dafür
	\item[Vollständige Genehmigung der Grundordnung] dafür, falls auf Tagesordnung
	\item[Internationale Solidarität konkret] mit Änderungsanträgen (siehe unten): dafür, sonst: dagegen
	\item[Vorbereitung StuPa-Wahl] dafür
	\item[Allgemeine Stimmung im StuPa] dagegen
	\item[Dies Academicus] mit Ersetzungsantrag von Geoffrey: dafür, sonst: dagegen
	\item[Sitzungsraum] in aktueller Form: dagegen. Erst einmal ESA W 221 ausprobieren. Idee eines gleichbleibenden Sitzungsraumes jedoch gut.
\end{description}

\section{Internationale Solidarität}

Zeile 26 "`der NATO-Staaten"' streichen und durch "`des Kapitals"' ersetzen\\
Zeile 90 ersetzen durch "`alle offenen und versteckten Bildungsgebühren (z.B. Verwaltungskostenbeitrag, Laborkittel, Prepbestecke, unbezahlte Praktika)"'

\end{document}