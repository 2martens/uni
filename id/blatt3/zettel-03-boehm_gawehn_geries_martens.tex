\documentclass[a4paper,10pt]{scrartcl}

% Hier die Nummer des Blatts und Autoren angeben.
\newcommand{\blatt}{3}
\newcommand{\autor}{Florian B\"{o}hm, Christopher Gawehn, Ulrike Geries, Jim Martens}

\usepackage{hci}
\usepackage[utf8]{inputenc}
\usepackage{float}
\usepackage[official]{eurosym}

\begin{document}
% Seitenkopf mit Informationen
\kopf
\renewcommand{\figurename}{Figure}

\aufgabe{1}

\begin{tikzpicture}[>=stealth]
				\begin{axis}[
					ymin=0,ymax=12,
					x=1cm,
					y=1cm,
					axis x line=middle,
					axis y line=middle,
					axis line style=->,
					xlabel={ID},
					ylabel={Zeit in sec.},
					xmin=0,xmax=5
				]
				
				 \node at (axis cs: 2,7.68) {.};
				 \node at (axis cs: 2,7.81) {.};
				 \node at (axis cs: 2,6.16) {.};
				 \node at (axis cs: 2,6.2) {.};
				 \node at (axis cs: 2,4.35) {.};
				 \node at (axis cs: 2,3.81) {.};
				 
				 \node at (axis cs: 3,9.82) {.};
				 \node at (axis cs: 3,9.03) {.};
				 \node at (axis cs: 3,7.99) {.};
				 \node at (axis cs: 3,8.12) {.};
				 \node at (axis cs: 3,5.65) {.};
				 \node at (axis cs: 3,4.85) {.};
				 
				 \node at (axis cs: 4,10.41) {.};
				 \node at (axis cs: 4,10.82) {.};
				 \node at (axis cs: 4,9.75) {.};
				 \node at (axis cs: 4,10.4) {.};
				 \node at (axis cs: 4,6.79) {.};
				 \node at (axis cs: 4,5.92) {.};
				 
				\end{axis}
	
\end{tikzpicture}

Eine lineare Regression konnte von uns wegen mangelndem Wissen nicht durchgeführt werden. Wir waren auch nicht imstande eine vernünftige Erklärung zu finden.

\end{document}