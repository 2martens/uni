\documentclass[a4paper,10pt]{scrartcl}

% Hier die Nummer des Blatts und Autoren angeben.
\newcommand{\blatt}{3}
\newcommand{\autor}{Jim Martens}

\usepackage{hci}
\usepackage[utf8]{inputenc}
\usepackage{float}
\usepackage[official]{eurosym}

\begin{document}
% Seitenkopf mit Informationen
\kopf
\renewcommand{\figurename}{Figure}

\aufgabe{1}

Das Keystroke-Level Model bietet die Möglichkeit die Dauer von Benutzerinteraktionen mit einem Interface zu schätzen. Wenngleich die damit errechnete Dauer nur ein Durchschnittswert ist, so kann sie als Richtlinie gelten bei einem Vergleich von Alternativen.

Das KLM kennt auch mentale Aktionen die nur vorsichtig genutzt werden sollten, da deren Dauer nicht eindeutig ermittelt werden kann. Es kommt daher bei der Verwendung mehr auf die Anzahl solcher Aktionen, als auf deren tatsächliche Dauer an.

\end{document}