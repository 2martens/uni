\documentclass[a4paper,10pt]{scrartcl}

% Hier die Nummer des Blatts und Autoren angeben.
\newcommand{\blatt}{6}
\newcommand{\autor}{Florian B\"{o}hm, Christopher Gawehn, Ulrike Geries, Jim Martens}

\usepackage{hci}
\usepackage[utf8]{inputenc}
\usepackage{float}
\usepackage[official]{eurosym}

\begin{document}
% Seitenkopf mit Informationen
\kopf
\renewcommand{\figurename}{Figure}

\aufgabe{1}

Erste Skizze:\\

\begin{tikzpicture}[
	goal/.style={rectangle,draw,fill=yellow!40,align=left},
	plan/.style={align=left},
	level 0/.style={sibling distance=17em},
	level 1/.style={sibling distance=15em},
	level 2/.style={sibling distance=8em},
	level 3/.style={sibling distance=7em},
	level 4/.style={sibling distance=10em},
	nextLevel/.style={level distance=40ex},
  	nextLevel2/.style={level distance=30ex},
  	nextLevel3/.style={level distance=30ex}]
	
	\coordinate
	  child[grow=up] {node[goal,anchor=south] (start) {0. Bearbeitung hochladen}}
	  child[grow=down,level distance=0ex]
      [edge from parent fork down]
      % sub goals
      child {node[goal] (one) {1. Raum Interak- \\ tionsdesign öffnen}
      	child[nextLevel2] {node[goal]{1.1 Commsy \\ öffnen}}
	    child[nextLevel2] {node[goal]{1.2 Raum \\ betreten}}
      }
      child {node[goal] (two) {2. Aufgabe \\ erstellen}
		child[nextLevel2] {node[goal]{2.1 Aufga- \\ ben öffnen}}
	    child[nextLevel2] {node[goal] (twoTwo) {2.2 Neuen Ein- \\ trag erstellen}
	    	child[nextLevel3] {node[goal] (twoTwoOne) {2.2.1 Meta- \\ daten eintra- \\ gen}
				child[nextLevel3] {node[goal]{2.2.1.1 Titel \\ eintragen}}
				child[nextLevel3] {node[goal]{2.2.1.2 Fällig am \\ eintragen}}
				child[nextLevel3] {node[goal]{2.2.1.3 Geplante Zeit \\ eintragen}}
				child[nextLevel3] {node[goal]{2.2.1.4 Status \\ eintragen}}
			}
			child[nextLevel3] {node[goal]{2.2.2 Dateien \\ hochladen}}
			child[nextLevel3] {node[goal]{2.2.3 Katego- \\ rien auswählen}}
			child[nextLevel3] {node[goal]{2.2.4 \\ Zugriffs- \\ rechte ein- \\ stellen}}
			child[nextLevel3] {node[goal]{2.2.5 Auf \\ Erstell- \\ button \\ klicken}}
	    }
      };

	  \node[plan] [below right=0.4 and -2.0 of start] {\underline{Plan 0:} \\
	  	DO 1.-2.
	  };
	  \node[plan] [below right=0.4 and -1.4 of one] {\underline{Plan 1:} \\
	  	DO 1.1.-1.2.
	  };
	  \node[plan] [below right=0.4 and -1 of two] {\underline{Plan 2:} \\
	  	DO 2.1.-2.2.
	  };
	  \node[plan] [below right=0.4 and -1 of twoTwo] {\underline{Plan 2.2:} \\
	  	DO 2.2.1.-2.2.5
	  };
	  \node[plan] [below right=0.4 and -1 of twoTwoOne] {\underline{Plan 2.2.1:} \\
	  	DO 2.2.1.1-2.2.1.4
	  };
\end{tikzpicture}

\newpage

heuristisch verbessert:

\begin{tikzpicture}[
	goal/.style={rectangle,draw,fill=yellow!40,align=left},
	plan/.style={align=left},
	level 0/.style={sibling distance=17em},
	level 1/.style={sibling distance=15em},
	level 2/.style={sibling distance=8em},
	level 3/.style={sibling distance=7em},
	level 4/.style={sibling distance=10em},
	nextLevel/.style={level distance=40ex},
  	nextLevel2/.style={level distance=30ex},
  	nextLevel3/.style={level distance=30ex}]
	
	\coordinate
	  child[grow=up] {node[goal,anchor=south] (start) {0. Bearbeitung hochladen}}
	  child[grow=down,level distance=0ex]
      [edge from parent fork down]
      % sub goals
      child {node[goal] (one) {1. Raum Interak- \\ tionsdesign öffnen}
      	child[nextLevel2] {node[goal]{1.1 Commsy \\ öffnen}}
	    child[nextLevel2] {node[goal]{1.2 Raum \\ betreten}}
      }
      child {node[goal] (two) {2. Aufgabe \\ erstellen}
		child[nextLevel2] {node[goal]{2.1 Aufga- \\ ben öffnen}}
	    child[nextLevel2] {node[goal] (twoTwo) {2.2 Neuen Ein- \\ trag erstellen}
	    	child[nextLevel3] {node[goal] (twoTwoOne) {2.2.1 Meta- \\ daten eintra- \\ gen}
				child[nextLevel3] {node[goal]{2.2.1.1 Titel \\ eintragen}}
				child[nextLevel3] {node[goal]{2.2.1.2 Fällig am \\ eintragen}}
				child[nextLevel3] {node[goal]{2.2.1.3 Geplante Zeit \\ eintragen}}
				child[nextLevel3] {node[goal]{2.2.1.4 Status \\ eintragen}}
			}
			child[nextLevel3] {node[goal]{2.2.2 Dateien \\ hochladen}}
			child[nextLevel3] {node[goal]{2.2.3 Katego- \\ rien auswählen}}
			child[nextLevel3] {node[goal]{2.2.4 \\ Zugriffs- \\ rechte ein- \\ stellen}}
			child[nextLevel3] {node[goal]{2.2.5 Auf \\ Erstell- \\ button \\ klicken}}
	    }
      };

	  \node[plan] [below right=0.4 and -2.0 of start] {\underline{Plan 0:} \\
	  	DO 1.-2.
	  };
	  \node[plan] [below right=0.4 and -1.4 of one] {\underline{Plan 1:} \\
	  	DO 1.1.-1.2.
	  };
	  \node[plan] [below right=0.4 and -1 of two] {\underline{Plan 2:} \\
	  	DO 2.1.-2.2.
	  };
	  \node[plan] [below right=0.4 and -1 of twoTwo] {\underline{Plan 2.2:} \\
	  	DO 2.2.1.-2.2.5
	  };
	  \node[plan] [below right=0.4 and -1 of twoTwoOne] {\underline{Plan 2.2.1:} \\
	  	DO 2.2.1.1-2.2.1.4
	  };
\end{tikzpicture}

\end{document}