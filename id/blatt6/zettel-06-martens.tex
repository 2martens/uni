\documentclass[a4paper,10pt]{scrartcl}

% Hier die Nummer des Blatts und Autoren angeben.
\newcommand{\blatt}{6}
\newcommand{\autor}{Jim Martens}

\usepackage{hci}
\usepackage[utf8]{inputenc}
\usepackage{float}
\usepackage[official]{eurosym}

\begin{document}
% Seitenkopf mit Informationen
\kopf
\renewcommand{\figurename}{Figure}

\aufgabe{1}

Erste Skizze:\\

\begin{tikzpicture}[
	goal/.style={rectangle,draw,fill=yellow!40,align=left},
	plan/.style={align=left},
	level 1/.style={sibling distance=7.5em},
	nextLevel/.style={level distance=40ex},
  	nextLevel2/.style={level distance=30ex},
  	nextLevel3/.style={level distance=18ex}]
	
	\coordinate
	  child[grow=up] {node[goal,anchor=south] (start) {0. Recorder programmieren}}
	  child[grow=down,level distance=0ex]
      [edge from parent fork down]
      % sub goals
      child {node[goal] (one) {1. Geräte \\ einschalten}
      	child[nextLevel2] {node[goal]{1.1 Fernseher \\ einschalten}}
	    child[nextLevel2] {node[goal]{1.2 Recorder \\ einschalten}}
      }
      child {node[goal]{2. Programm \\ öffnen}}
      child {node[goal]{4. Aufnahme \\ auswählen}}
      child {node[goal] (three) {3. Sendung \\ auswählen}
		child[nextLevel2] {node[goal]{3.1 Tag \\ auswählen}}
		child[nextLevel2] {node[goal]{3.2 Zur Uhrzeit \\ navigieren}}
		child[nextLevel2] {node[goal]{3.3 Sendung \\ selektieren}}
      }
      child {node[goal]{5. Programmie- \\ rung speichern}};

	\node[plan] [below right=0.4 and -2.0 of start] {\underline{Plan 0:} \\
	  	IF ein Gerät aus DO 1. \\
		DO 2.-5.
	  };
     \node[plan] [below left=0.4 and -1.0 of one] {\underline{Plan 1:} \\
      		IF Fernseher aus DO 1.1 \\
      		IF Recorder aus DO 1.2};
     \node[plan] [below left=0.4 and -1.0 of three] {\underline{Plan 3:} \\
     	DO 3.1-3.3};
\end{tikzpicture}

\newpage

heuristisch verbessert:\\

\begin{tikzpicture}[
	goal/.style={rectangle,draw,fill=yellow!40,align=left},
	plan/.style={align=left},
	level 1/.style={sibling distance=7.5em},
	nextLevel/.style={level distance=40ex},
  	nextLevel2/.style={level distance=30ex},
  	nextLevel3/.style={level distance=18ex}]
	
	\coordinate
	  child[grow=up] {node[goal,anchor=south] (start) {0. Recorder programmieren}}
	  child[grow=down,level distance=0ex]
      [edge from parent fork down]
      % sub goals
      child {node[goal] (one) {1. Geräte \\ einschalten}
      	child[nextLevel2] {node[goal]{1.1 Fernseher \\ einschalten}}
	    child[nextLevel2] {node[goal]{1.2 Recorder \\ einschalten}}
      }
      child {node[goal]{2. Programm \\ öffnen}}
      child {node[goal]{4. Aufnahme \\ auswählen}}
      child {node[goal] (three) {3. Sendungen \\ auswählen}
		child[nextLevel2] {node[goal]{3.1 Tag \\ auswählen}}
		child[nextLevel2] {node[goal]{3.2 Zur Uhrzeit \\ navigieren}}
		child[nextLevel2] {node[goal]{3.3 Sendung \\ selektieren}}
      }
      child {node[goal]{5. Programmie- \\ rung speichern}};

	\node[plan] [below right=0.4 and -2.0 of start] {\underline{Plan 0:} \\
	  	IF ein Gerät aus DO 1. \\
		DO 2.-5.
	  };
     \node[plan] [below left=0.4 and -1.0 of one] {\underline{Plan 1:} \\
      		IF Fernseher aus DO 1.1 \\
      		IF Recorder aus DO 1.2};
     \node[plan] [below left=0.4 and -1.0 of three] {\underline{Plan 3:} \\
     	FOR all Sendungen DO 3.1-3.3};
\end{tikzpicture}

\end{document}