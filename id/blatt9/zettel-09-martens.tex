\documentclass[a4paper,10pt]{scrartcl}

% Hier die Nummer des Blatts und Autoren angeben.
\newcommand{\blatt}{9}
\newcommand{\autor}{Jim Martens}

\usepackage{hci}
\usepackage[utf8]{inputenc}
\usepackage{float}
\usepackage[official]{eurosym}
\usepackage[parfill]{parskip}

\begin{document}
% Seitenkopf mit Informationen
\kopf
\renewcommand{\figurename}{Figure}

\aufgabe{1}

	\begin{enumerate}
		\item 
		Die App läuft auf einem touchfähigen Smartphone ohne Tasten. Dabei geschieht die Ausgabe sowohl über das Display als auch in Form von Strahlung oder WLAN, die eine Aktion beim Fernseher durchführt. Die App muss sowohl bei Helligkeit als auch bei Dunkelheit funktionieren. Ebenso darf sie für das Ausführen kein Internet benötigen.
		\item 
		Die App wird maximal von Besitzern eines touchfähigen Smartphones verwendet werden. Eine weitere Einschränkung ergibt sich aus der Umsetzung der Interaktion mit dem Fernseher. Wenn die Kommunikation über ein Netzwerk verläuft (WLAN), dann muss es ein moderner Fernseher mit Internetbefähigung sein. Läuft die Kommunikation traditionell über die Strahlung wie auch bei einer Fernbedienung, dann kann jeder Fernseher verwendet werden.
		
		Von den übrigbleibenden Personen kommen wiederum nur jene in Betracht, denen die Bedienung mit der Fernbedienung zu kompliziert ist (das Verändern der Lautstärke selber oder das Verändern des Programms bzw. Senders ist ja nicht sehr kompliziert oder komplex).
		
		Mit der App kann man die Lautstärke verändern und einen Sender aus einer Liste an Sendern auswählen. Die interessanten Sender (öffentliche und bekannte private Sender) sind oben angeordnet, sodass diese schnell ausgewählt werden können. Die weniger interessanten Sender (Werbesender und Nischensender) sind weiter unten angeordnet, sodass man keine unnötige Zeit verschwendet, um zu interessanten Sendern zu kommen. Ebenfalls verändert sich die Sortierung im Laufe der Zeit. Je häufiger ein Sender genutzt wird, desto weiter nach oben in der Liste wandert der Sender. Dadurch wird die Nutzungszeit immer weiter reduziert und damit optimiert.

		Außerdem entwickelt sich somit jede Installation der App nach einer Zeit zu einer individuell angepassten Version.
		\setcounter{enumi}{3}
		\item HTA für Lautstärke ändern: \\
\begin{tikzpicture}[
	goal/.style={rectangle,draw,fill=yellow!40,align=left},
	plan/.style={align=left},
	level 1/.style={sibling distance=7.7em},
	nextLevel/.style={level distance=40ex},
  	nextLevel2/.style={level distance=30ex},
  	nextLevel3/.style={level distance=18ex}]
	
	\coordinate
	  child[grow=up] {node[goal,anchor=south] (start) {0. Lautstärke ändern}}
	  child[grow=down,level distance=0ex]
      [edge from parent fork down]
      % sub goals
      child {node[goal] (one) {1. Smartphone \\ einschalten}}
      child {node[goal]{2. App aufrufen}}
      child {node[goal] (three) {3. Lautstärke- \\ regler anpassen}};

	\node[plan] [below right=0.4 and -1.5 of start] {\underline{Plan 0:} \\
	  	DO 1.-3.
	  };
\end{tikzpicture}
		\newpage
		HTA für Programmauswahl: \\
		
\begin{tikzpicture}[
	goal/.style={rectangle,draw,fill=yellow!40,align=left},
	plan/.style={align=left},
	level 1/.style={sibling distance=7.7em},
	nextLevel/.style={level distance=40ex},
  	nextLevel2/.style={level distance=30ex},
  	nextLevel3/.style={level distance=18ex}]
	
	\coordinate
	  	child[grow=up] {node[goal,anchor=south] (start) {0. Programm ändern}}
	  	child[grow=down,level distance=0ex]
      	[edge from parent fork down]
      	% sub goals
      	child {node[goal] (one) {1. Smartphone \\ einschalten}}
      	child {node[goal]{2. App aufrufen}}
      	child {node[goal] (three) {3. Sender \\ auswählen}
			child[nextLevel2] {node[goal] {3.1. Auf \\ Sender \\ klicken}}
			child[nextLevel2] {node[goal] {3.2. Neuen Sender \\ auswählen aus \\ Liste}}
      	};

	\node[plan] [below right=0.4 and -1.5 of start] {\underline{Plan 0:} \\
	  	DO 1.-3.
	  };
	  \node[plan] [below left=0.4 and -1.0 of three] {\underline{Plan 3:} \\
     	DO 3.1-3.2};
\end{tikzpicture}
				
		
	\end{enumerate}

\end{document}