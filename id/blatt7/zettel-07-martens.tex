\documentclass[a4paper,10pt]{scrartcl}

% Hier die Nummer des Blatts und Autoren angeben.
\newcommand{\blatt}{7}
\newcommand{\autor}{Jim Martens}

\usepackage{hci}
\usepackage[utf8]{inputenc}
\usepackage{float}
\usepackage[official]{eurosym}

\begin{document}
% Seitenkopf mit Informationen
\kopf
\renewcommand{\figurename}{Figure}

\aufgabe{2}

Mich interessiert, ob es einen Zusammenhang gibt zwischen dem Besuch der Vorlesung und dem Studiengang?

\paragraph{Hypothese} Da Interaktionsdesign in den Bereich Mensch-Computer-Interaktion einzuordnen ist, sind MCI-Studierende häufiger in der Vorlesung als Studierende von BSc. Informatik oder SSE. Personen die andere als die eben genannten Fächer studieren werden entfernt.

\paragraph{Ergebnisse} Es gibt keinen signifikanten Zusammenhang zwischen dem Studiengang und dem Besuchen der Vorlesung nach einem ANOVA-Test. Die Daten sind nicht einmal normalverteilt. Daher lohnt sich auch kein Vergleich zwischen je zwei Studiengängen.

\begin{figure}[hb]
	\caption{Häufigkeit des Vorlesungsbesuch je untersuchtem Studiengang. 1 bedeutet "`immer"', 5 "`überhaupt nicht"'}
	\includegraphics[scale=0.8]{zettel-07-plot}
\end{figure}

Um diese Grafik besser nachvollziehen zu können folgen hier die Mittelwerte und Standardabweichungen je Studiengang.

\begin{tabular}{ccc}
	& Mittelwert & Standardabweichung \\
	BSc. Informatik & $2.333$ & $1.345$ \\
	MCI & 2 & $1.319$ \\
	SSE & $2.1$ & $1.197$ 
\end{tabular}

\end{document}