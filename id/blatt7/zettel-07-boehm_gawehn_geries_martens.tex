\documentclass[a4paper,10pt]{scrartcl}

% Hier die Nummer des Blatts und Autoren angeben.
\newcommand{\blatt}{7}
\newcommand{\autor}{Florian B\"{o}hm, Christopher Gawehn, Ulrike Geries, Jim Martens}

\usepackage{hci}
\usepackage[utf8]{inputenc}
\usepackage{float}
\usepackage[official]{eurosym}

\begin{document}
% Seitenkopf mit Informationen
\kopf
\renewcommand{\figurename}{Figure}

\aufgabe{1}

Im Folgenden wird der Zusammenhang zwischen Geschlecht und der Beurteilung des kognitiven Aufwands betrachtet.

\paragraph{Hypothese} Es gibt keinen Zusammenhang zwischen Geschlecht und der Beurteilung des kognitiven Aufwands. 

\paragraph{Ergebnisse} Die Hypothese kann bestätigt werden, da es keine signifikanten Unterschiede in den erhobenen Daten gibt.\\

Zunächst wurde ein Shapiro-Wilk Test angewandt, um die Daten auf eine Normalverteilung zu testen. Das Ergebnis war, dass die Daten normalverteilt sind (\texttt{p = 0.05075}).

Da es beim Geschlecht nur zwei Gruppen gab und jeder nur einer angehörte (in den Auswertungen gab es nur die Werte "`male"' und "`female"', wurde ein unpaired t test vorgenommen. Das Ergebnis war, dass es keinen signifikanten Unterschied zwischen den Geschlechtern gibt (\texttt{t(24.848) = -0.6244; p = 0.5381}). Zum leichteren Verständnis ein Diagramm, dass den Mittelwert und die Verteilung zeigt.

\begin{figure}[hb]
	\caption{Beurteilung des kognitiven Aufwands nach Geschlecht}
	\includegraphics[scale=0.8]{zettel-07-gruppe-plot}
\end{figure}

\newpage 
Diesem Diagramm liegen diese Werte zugrunde:

\begin{tabular}{ccc}
	& Mittelwert & Standardabweichung \\
	Male & $14.667$ & $3.512$ \\
	Female & $15.41667$ & $3.204$
\end{tabular}

\end{document}