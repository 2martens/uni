\documentclass[a4paper,10pt]{scrartcl}

% Hier die Nummer des Blatts und Autoren angeben.
\newcommand{\blatt}{2}
\newcommand{\autor}{Jim Martens}

\usepackage{hci}
\usepackage[utf8]{inputenc}
\usepackage{float}
\usepackage[official]{eurosym}

\begin{document}
% Seitenkopf mit Informationen
\kopf
\renewcommand{\figurename}{Figure}

\aufgabe{1}

Im Thought Paper von Card, Moran und Newell geht es um den "`Model Human Processor"'. Dieser stellt ein Framework dar, um die Resultate der Psychologie für andere Bereiche nutzbar zu machen. Ein zentraler Punkt ist die Berechnung der Dauer für gewisse Tätigkeiten (lesen, erkennen, etc.), wie dies unter anderem mit Fitt's Law geschieht. Dabei gibt es Zyklen bestimmter Länge, mit denen die Dauer beschrieben werden kann.

Dieses Wissen könnte beispielsweise für Präsentationen genutzt werden, um dynamisch die Dauer zu bestimmen, die eine Folie sichtbar sein muss, damit die Zuschauer alle Informationen erfassen können. Aber auch in anderen Anwendungsgebieten kann der "`Model Human Processor"' eingesetzt werden.

\end{document}