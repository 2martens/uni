\documentclass[a4paper,10pt]{scrartcl}

% Hier die Nummer des Blatts und Autoren angeben.
\newcommand{\blatt}{2}
\newcommand{\autor}{Jim Martens}

\usepackage{hci}
\usepackage[utf8]{inputenc}
\usepackage{float}
\usepackage[official]{eurosym}

\begin{document}
% Seitenkopf mit Informationen
\kopf
\renewcommand{\figurename}{Figure}

\aufgabe{1}

Im Thought Paper von Card, Moran und Newell geht es um den "`Model Human Processor"'. Dieser stellt ein Framework dar, um die Resultate der Psychologie für andere Bereiche nutzbar zu machen. Ein zentraler Punkt ist die Berechnung der Dauer für gewisse Tätigkeiten (lesen, erkennen, etc.), wie dies unter anderem mit Fitt's Law geschieht. Dabei gibt es Zyklen bestimmter Länge, mit denen die Dauer beschrieben werden kann.

Dieses Wissen könnte beispielsweise für Präsentationen genutzt werden, um dynamisch die Dauer zu bestimmen, die eine Folie sichtbar sein muss, damit die Zuschauer alle Informationen erfassen können. Aber auch in anderen Anwendungsgebieten kann der "`Model Human Processor"' eingesetzt werden.

\aufgabe{2}

Für die folgende Schätzaufgabe wird ein Preis von $0.0325$ \euro{} pro GB Speicher (ungefähr heutiger Stand) angenommen. Außerdem sind alle Kommazahlen in englischer Darstellung geschrieben (Punkt statt Komma).

\aufgabe{2.1}

Versucht man alle Informationen zu speichern, die auf einen Mensch sekündlich einströmen und dies für das gesamte Leben, dann ergibt sich zunächst einmal eine erhebliche Menge an Speicher.

Laut der Vorlesung fließen sekündlich rund $1.5$ GBit auf den Menschen ein. Dies sind demnach $0.1875$ GB (Gigabyte) pro Sekunde. Daraus ergeben sich in einer Stunde 675 GB. In einem Tag sind es bereits 16200 GB. Auf ein Jahr (365.25 Tage) sind es 5917050 GB. In einem ganzen Leben sind es somit (ein Leben = 90 Jahre) 532534500 GB.

Selbst bei den sehr geringen Preisen im Moment für ein GB, die in den nächsten Jahren noch weiter fallen werden (Moore's Law), sind dies dann $17,307,371.25$ \euro{}. In Worten sind das 17 Millionen 307 Tausend 371 Euro und 25 Cent.

Dies ist also erheblich weniger als die großen IT-Konzerne (Microsoft, Google, Apple) jährlich and Umsatz machen.

\aufgabe{2.2}

Geht man nun nur noch von der Menge der Informationen aus, die von den Rezeptoren wahrgenommen wird (rund 15 MBit pro Sekunde), dann ergeben sich $1.875$ MB pro Sekunde. Auf eine Stunde hochgerechnet ergeben sich 6750 MB. In GB sind dies dann $6.75$ GB. In einem ganzen Tag fallen somit 162 GB an. Auf ein ganzes Jahr gerechnet sind dies $59170.5$ GB. Für ein gesamtes Leben werden somit 5325345 GB an Speicher benötigt.

Mit den jetzigen Kosten multipliziert ergeben sich daraus $173,073.7125$ \euro{}. Dies sind somit 173 Tausend 73 Euro und (aufgerundet) 72 Cent. Diese Menge and Geld ist zwar noch immer recht viel, aber bereits greifbar.

\aufgabe{2.3}

Betrachtet man schließlich nur die Informationen, die tatsächlich das Bewusstsein erreichen, so sind dies $12.5$ Byte pro Sekunde. In einer Stunde sind dies 45000 B und somit 45 KB. Auf einen Tag gerechnet sind das 1080 KB und somit $1.08$ MB. In einem Jahr ergeben sich auf diese Weise $394.47$ MB. Auf ein ganzes Leben hochgerechnet sind dies damit $35502.3$ MB und damit $35.5023$ GB. Viele Festplatten haben somit mehr Speicher zur Verfügung, als zur Speicherung aller von einem Menschen in einem Leben bewusst wahrgenommenen Informationen nötig wäre.

Mit dem aktuellen Preis multipliziert ergibt sich ein Kostenvolumen von $1,153.82475$ \euro{}. Dies sind in Worten 1 Tausend 153 Euro und (aufgerundet) 83 Cent. Diese Menge Geld kann man leicht verdienen. 

\end{document}