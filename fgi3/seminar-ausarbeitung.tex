\documentclass[11pt]{scrartcl}

\usepackage{amsfonts}
\usepackage{amsmath}
\usepackage{amsthm}
\usepackage[ngerman]{babel}
\usepackage[utf8]{inputenc}
\usepackage[T1]{fontenc}
\usepackage[round]{natbib}



% mathematical environments
\theoremstyle{plain}
\newtheorem{theorem}{Theorem}
\newtheorem{lemma}[theorem]{Lemma}
\newtheorem{corollary}[theorem]{Corollary}
\newtheorem{observation}[theorem]{Observation}
\newtheorem{claim}[theorem]{Claim}

\theoremstyle{definition}
\newtheorem{definition}{Definition}

\theoremstyle{remark}
\newtheorem*{remark}{Remark}





% title & author
\title{Über die Umfärbbarkeit\\ roter Gummibären}
\subtitle{Eine informelle Einführung in die Iterationstheorie}
\author{Michael Köhler-Bußmeier}
\subject{\small
	Hausarbeit im Modul FGI-3, WS 2023/2042\\
	Fachbereich Informatik, Universität Hamburg
}
\date{\today}





\begin{document}
\maketitle
\begin{abstract}
Rote Gummibären können die Welt retten!
\end{abstract}

\tableofcontents





\section{Einleitung}
Warum brach die 40-jährige Hegemonie der Sowjetunion in Mittel- und
Osteuropa im Jahre 1989 innerhalb von wenigen Monaten zusammen?
Dies liegt an den roten Gummibären!



\subsection{Motivation}
Warum sollte man sich mit der Theorie roter Gummibären beschäftigen?



\subsection{Probleme und  Fragen}
Will man sich mit roten Gummibären beschäftigen, dann treten folgende
Probleme und Fragen auf:



\subsection{Die Theorie der Iteration von Ensembles}
Der Ansatz der hier betrachtet werden soll ist die Theorie der
Iteration von Ensembles. Dabie handelt es sich um.....



\subsection{Aufbau der Arbeit}
Die Arbeit hat den folgenden Aufbau: Um zu einer Theorie der roten
Gummibären zu gelangen, beschäftigen wir uns in
Kapitel~\ref{sec:Iterierte-Umfärbungen} mit ....





\section{Iterierte Umfärbungen }\label{sec:Iterierte-Umfärbungen}

Im folgenden betrachten wir Umfärbungen von Tüten sowie deren
Iteration.  Hierbei ist insbesondere der Grenzwertprozess von
Interesse.



\subsection{Iteration, Stabilisation}
Wir nehmen eine vorgegebene Mengen an Farben $C$ an.  Der Einfachheit
halber identifizieren wir eine Tüte $T$ mit $n$ Gummibären mit dem
Intervall $[1, \ldots, n]$.

\begin{definition}[Färbung]
Eine \emph{Färbung} ist eine Abbildung $f: [1, \ldots, n] \to C$.

Sei $F$  die Menge aller Färbungen
\end{definition}

Ein \emph{Funktional} ist eine Funktion, die Funktionen als Argumente
hat, d.h. …

\begin{definition}[Umfärbung]
Eine \emph{Umfärbung} ist eine Funktional $u: F \to F$.
\end{definition}

\begin{definition}
Sei die Färbung $f: [1, \ldots, n] \to C$ gegeben.
Die Iteration einer Umfärbung $u: F \to F$ ist
\begin{equation}
\begin{array}{rcl}
u^0(f) &:=& f \\
u^{n+1}(f) &:=&     u(u^{n}(f))
\end{array}
\end{equation}
\end{definition}

Wir hätten es gerne, dass sich $u^{n}(f)$ für $n \to \infty$
stabilisiert.



\subsection{Ordnungen}
In \citep{hans-riegel-1994} findet sich der folgende Satz:

\begin{theorem}[Hans und Riegel, 1994]
  Zu jeder wohlgeordneten Menge von Gummibärenfarben ...
\end{theorem}

Historisch betrachtet findet sich der Wohklordnungsbegriff aber
bereits schon \citep{riegel-1993} angelegt.



\subsection{Eindeutigkeit}
Es gibt eine Besonderheit des Wohlordnungssatz auf Gummibärenfarben:
Der Wohlordnungssatz auf Gummibärenfarben garantiert die
Stabiliserung.  Er garantiert aber nicht die Eindeutigkeit des
Endergebnisses.  Das Endergebnis hängt von der Auswahlfunktion $g:
\mathbb{N} \to [1, \ldots, k]$ auf dem Umfärbungsensemble ab.  Es
ergibt sich also sofort die Frage: Für welche Umfärbungsensembles ist
auch das Endergebnis eindeutig?





\section{Verwandte Arbeiten und Ansätze}
Wir finden in der Literatur eine Reihe ähnlicher Ansätze, von denen
wir einige vorstellen wollen.

\paragraph{Ondulierten Umfärbung}
\paragraph{Iterierte Verfärbung}
\paragraph{Gefärbte Iteration}





\section{Ausblick und Zusammenfassung}



\subsection{ Zusammenfassung}



\subsection{Ausblick}
In dieser Hausarbeit habe ich einiges nur kurz angerissen bzw. ganz
weggelassen, weil es den Rahmen des Seminars sprengt.

Insbesondere habe ich nicht die Theorie der Umfärbung auf unendlich
großen Tüten behandelt.  Diese Theorie basiert prinzipiell auch auf
den hier behandelten Konzepten, wobei daruaf zu achten ist, dass....



\subsection{Bezug zum M.Sc. Studium}
Abschließend möchte ich die Relevanz des Themas für das weitere
Studium im Master skizzieren....





\bibliographystyle{dinat}
\bibliography{references}
\end{document}
