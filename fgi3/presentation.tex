\RequirePackage{pdf14}
\documentclass{beamer}
\usepackage[T1]{fontenc}
\usepackage[utf8]{inputenc}
\usepackage[ngerman]{babel}
%\usepackage{paralist}
%\useoutertheme{infolines}
\usepackage{graphicx}
\usepackage{hyperref}
\usepackage{listings}
\usepackage{color}
\usepackage[german=quotes]{csquotes}
\usetheme{Warsaw}
\usecolortheme{crane}
\pagenumbering{arabic}
\def\thesection{\arabic{section})}
\def\thesubsection{\alph{subsection})}
\def\thesubsubsection{(\roman{subsubsection})}
\setbeamertemplate{navigation symbols}{}
\graphicspath{ {src/} {/home/jim/Pictures/Studium/KBS/} }

\MakeOuterQuote{"}

\definecolor{mygreen}{rgb}{0,0.6,0}
\definecolor{mygray}{rgb}{0.5,0.5,0.5}
\definecolor{mymauve}{rgb}{0.58,0,0.82}

\lstset{ %
  backgroundcolor=\color{white},   % choose the background color; you must add \usepackage{color} or \usepackage{xcolor}
  basicstyle=\footnotesize,        % the size of the fonts that are used for the code
  breakatwhitespace=false,         % sets if automatic breaks should only happen at whitespace
  breaklines=true,                 % sets automatic line breaking
  captionpos=b,                    % sets the caption-position to bottom
  commentstyle=\color{mygray},    % comment style
  deletekeywords={},            % if you want to delete keywords from the given language
  escapeinside={\%*}{*)},          % if you want to add LaTeX within your code
  extendedchars=true,              % lets you use non-ASCII characters; for 8-bits encodings only, does not work with UTF-8
  keepspaces=true,                 % keeps spaces in text, useful for keeping indentation of code (possibly needs columns=flexible)
  keywordstyle=\color{blue},       % keyword style
  language=PHP,                 % the language of the code
  morekeywords={class, function, return, protected, public, private, const, static, new, extends, namespace, null},            % if you want to add more keywords to the set
  numbers=left,                    % where to put the line-numbers; possible values are (none, left, right)
  numbersep=5pt,                   % how far the line-numbers are from the code
  numberstyle=\tiny\color{mygray}, % the style that is used for the line-numbers
  rulecolor=\color{black},         % if not set, the frame-color may be changed on line-breaks within not-black text (e.g. comments (green here))
  showspaces=false,                % show spaces everywhere adding particular underscores; it overrides 'showstringspaces'
  showstringspaces=false,          % underline spaces within strings only
  showtabs=false,                  % show tabs within strings adding particular underscores
  stepnumber=2,                    % the step between two line-numbers. If it's 1, each line will be numbered
  stringstyle=\color{mygreen},     % string literal style
  tabsize=2,                       % sets default tabsize to 2 spaces
  title=\lstname                   % show the filename of files included with \lstinputlisting; also try caption instead of title
}

\hypersetup{
	pdfauthor=Jim Martens,
	pdfstartview=Fit
}

\expandafter\def\expandafter\insertshorttitle\expandafter{%
	\raggedleft \insertframenumber\,/\,\inserttotalframenumber\;}

\begin{document}
\author{Jim Martens}
\title{Stabilizing Concensus with the power of two choices}
\date{16. Januar 2017}

	% Introduction
	\begin{frame}
		\titlepage
	\end{frame}

    \begin{frame}
        \tableofcontents
    \end{frame}

    \section{Problem}
    \begin{frame}{Problem}
        \begin{itemize}
            \item mehrere Prozesse sollen sich auf einen Wert einigen
            \vfill
            \item dies soll unabhängig von der initialen Belegung funktionieren
            \vfill
            \item der konsensuale Wert soll Teil der initialen Wertemenge sein
        \end{itemize}
    \end{frame}

    \begin{frame}{Problem - Beispielanwendung}
        \begin{itemize}
            \item Parteien haben zum Mehrwertsteuersatz unterschiedliche Forderungen
            \vfill
            \item sie sollen sich nach der Wahl in überschaubarer Zeit einigen
            \vfill
            \item der Konsens soll der Forderung einer der beteiligten Parteien
                  entsprechen (vereinfacht)
            \vfill
            \item ein Ergebnis soll unabhängig davon, welche Partei was fordert,
                  erreicht werden
        \end{itemize}
    \end{frame}

    \begin{frame}{Problem - Beispielanwendung 2}
        \begin{itemize}
            \item Publisher, Entwickler und Kunden haben eine Position zum
                  Releasetermin eines Spiels
            \vfill
            \item in Verhandlungen sollen sie sich auf einen Termin einigen
            \vfill
            \item dieser Termin soll einer der vorgeschlagenen Termine sein
            \vfill
            \item ein Ergebnis soll unabhängig davon, wer was fordert,
                  erreicht werden
        \end{itemize}
    \end{frame}
%Beweis einer oberen Laufzeitschranke hilft enorm
    \section{Motivation}
    \begin{frame}{Motivation}
        \begin{itemize}
            \item nützlich für Konsensfindung zwischen vielen Leuten
            \vfill
            \item hat obere Laufzeitschranke
            \vfill
            \item Stabilisierung trotz Gegner möglich
            \vfill
            \item schönes Problem, "Wer möchte nicht gerne Konsenslösungen
                  herbeiführen?"
        \end{itemize}
    \end{frame}

    \section{Lösung}
    \subsection{Allgemein}

    \begin{frame}{Median-Verfahren}
        \begin{itemize}
            \item Ergebnis Teil der Anfangsmenge
            \vfill
            \item Ausreißer werden ignoriert
            \vfill
            \item in der Mitte stehender Wert in nach Größe sortierter
                  Zahlenreihe
        \end{itemize}
    \end{frame}

    \begin{frame}{Algorithmus}
        \begin{itemize}
            \item jeder Prozess wählt in jedem Zeitschritt 2 zufällige
                  Prozesse aus Gesamtmenge der Prozesse (inklusive sich selbst)
                  aus
            \vfill
            \item neuer Wert des Prozesses ist Median von sich selbst und den
                  Werten, der zwei ausgewählten Prozesse
        \end{itemize}
    \end{frame}

    \begin{frame}{Ergebnisse}
        n Prozesse
        \vfill
        \begin{tabular}{c|c|c}
            Eingabe & mit Gegner & ohne Gegner\\
            \hline
            worst-case 2 Werte & \(O(\log n)\) & \(O(\log n)\) \\
            \hline
            worst-case m Werte & \(O(\log m \log \log n + \log n)\) & \(O(\log n)\)
        \end{tabular}
    \end{frame}

    \subsection{Notation}
    \begin{frame}{Notation}
        \begin{itemize}
            \item Bälle statt Prozesse (von 1 bis n nummeriert)
            \vfill
            \item Bins statt Werte (von 1 bis n nummeriert)
            \vfill
            \item \(b_{0,1}\) bis \(b_{0,n}\) initiale Zuweisungen
            \vfill
            \item \(t \in \mathbb{N}\) Zeitschritte
            \vfill
            \item \(m_t\) ist Bin, welche den Median-Ball in Schritt
                  \(t\) enthält
        \end{itemize}
    \end{frame}

    \begin{frame}{Notation}
        \begin{itemize}
            \item in der \(i\)-ten Iteration: \(I_{i,1}\) bis \(I_{i,n}\) und
                  \(J_{i,1}\) bis \(J_{i,n}\) sind 2n zufällige Variablen
            \vfill
            \item \(j = 1,...,n\)
            \vfill
            \item Zuweisung der \(i\)-ten Iteration:
                  \(b_{i,j} = median(b_{i-1,j}, b_{i-1,I_{i,j}}, b_{i-1,J_{i,j}})\)
            \vfill
            \item "mit hoher Wahrscheinlichkeit" meint Wahrscheinlichkeit von
                  mindestens \(1 - n^{-c}\) für \(c > 1\)
        \end{itemize}
    \end{frame}

    \subsection{Zwei Bins mit Gegner}
    \begin{frame}{Zwei Bins mit Gegner}
        \begin{itemize}
            \item Grundlage, da Fälle mit mehr Bins auf diesen reduziert werden
            \vfill
            \item Gegner kann in jeder Runde Werte von bis zu\(\sqrt{n}\) Bällen
                  ändern
            \vfill
            \item trotz Gegner wird stabiler Zustand mit \(n - \sqrt{n}\) Bällen
                  in einer Bin erreicht
            \vfill
            \item Fall mit zwei Bins gleicht der "majority rule"
        \end{itemize}
    \end{frame}

    \begin{frame}{Zwei Bins mit Gegner}
        \begin{block}{Theorem 10}
            Für jede initiale Verteilung der Bälle sind mit hoher Wahrscheinlichkeit
            \(O(\log n)\) Runden der "majority rule" ausreichend, um
            \(n - O(\sqrt{n})\) Bälle in die gleiche Bin zu bekommen.
            Dies gilt selbst dann, wenn ein Gegner die zufälligen Entscheidungen
            von bis zu \(O(\sqrt{n})\) Bällen manipulieren darf.
        \end{block}
    \end{frame}
    \begin{frame}{Zwei Bins mit Gegner}
        Sei \(L_t\) die Anzahl an Bällen in der linken Bin zum Zeitpunkt \(t\)
        und \(R_t\) die Anzahl an Bällen in der rechten Bin zum Zeitpunkt \(t\).
        Sei \(X_t = min(L_t, R_t)\) und \(Y_t = max(L_t, R_t)\). Angenommen \(n\)
        ist gerade. Das Ungleichgewicht für Schritt \(t\) ist
        \(\Delta_t = (Y_t - X_t) /2\). Das markierte Ungleichgewicht für Schritt
        \(t\) ist \(\Psi_t = (R_t - L_t)/2\).
        Basierend auf dem Ungleichgewicht wird zwischen drei Fällen unterschieden.
    \end{frame}

    \begin{frame}{Zwei Bins mit Gegner}
        Fall 1: \(\Delta_0 \geq n/3\)

        \begin{block}{Lemma 11}
            Wenn \(\Delta_0 \geq cn\) gilt, wobei \(c > 0\) eine beliebige Konstante
            ist, dann reichen mit hoher Wahrscheinlichkeit \(O(\log \log n)\)
            Runden der "majority rule", um einen stabilen Konsens zu erreichen.
        \end{block}

        Fall 2: \(c \sqrt{n \ln n} \leq \Delta_0 < n/3\) für genügend großes \(c\)
        \begin{block}{Lemma 12}
            Wenn es einen Schritt \(t\) mit \(c\sqrt{n\ln n} \leq \Delta_0 < n/3\)
            für ein genügend großes \(c\) gibt, dann reichen mit hoher Wahrscheinlichkeit
            \(O(\log n)\) zusätzliche Runden der "majority rule", um in Fall 1
            zu kommen.
        \end{block}
    \end{frame}

    \begin{frame}{Zwei Bins mit Gegner}
        Fall 3: \(\Delta_0 \leq c\sqrt{n \ln n}\)\\
        Wenn \(Pr[Y \geq x] \geq Pr[Z \geq x]\) für zwei Zufallsvariablen Y und Z
        und beliebige \(x\) gilt, dann wird Z stochastisch von Y dominiert. Dies
        wird als \(Y \succeq Z\) geschrieben.
        \begin{block}{Lemma 13}
            Angenommen es gibt keinen Gegner und es gilt \(L_t \leq R_t \). Für
            zwei beliebige markierte Ungleichgewichte \(\Psi_t\) und \(\Psi_t^{'}\),
            für die \(\Psi_t > \Psi_t^{'}\) gilt, folgt
            \(\Psi_{t+1} \succeq \Psi_{t+1}^{'}\).
        \end{block}
    \end{frame}

    \begin{frame}{Zwei Bins mit Gegner}
        Im nächsten Lemma wird das "Central Limit Theorem" angewendet, um zu zeigen,
        dass mit konstanter Wahrscheinlichkeit ein genügend großes Ungleichgewicht
        ungeachtet des vorherigen Ungleichgewichtes existiert.

        \begin{block}{Lemma 14}
            Angenommen es gibt keinen Gegner. Sei \(\varepsilon > 0\) eine beliebige
            Konstante und sei \(c\) die Konstante von Lemma 12. Für jedes
            \(\Psi_t \geq 0\) gilt
            \(Pr[\Psi_{t+1} \geq c\sqrt{n}] \geq \frac{1}{\sqrt{2\pi}(1 + 4c/\sqrt{3})}e^{-8c^{2}/3}-\varepsilon\).
        \end{block}
        \begin{block}{Lemma 15}
            Angenommen es gibt keinen Gegner. Wenn \(\Delta_t \geq c\sqrt{n}\) gilt,
            wobei \(c\) die Konstante von Lemma 12 ist, dann gilt auch
            \(Pr[\Delta_{t+1} \geq (4/3)\Delta_t] \geq 1 - exp(-\Theta(\Delta^2_t / n))\).
        \end{block}
    \end{frame}

    \begin{frame}{Zwei Bins mit Gegner}
        \begin{block}{Lemma 16}
            Wenn initial \(\Delta_0 < c\sqrt{n \ln n}\) für das \(c\) aus Lemma 12
            gilt, dann gilt mit hoher Wahrscheinlichkeit nach \(O(\log n)\)
            Schritten \(\Delta_t \geq c\sqrt{n \ln n}\).
        \end{block}
    \end{frame}

    \section{Abschluss}
    \begin{frame}{Abschluss}
        \begin{itemize}
            \item Fälle 2 und 3 werden auf den ersten zurückgeführt
            \vfill
            \item Szenarien mit mehr als 2 Bällen werden auf Szenario mit 2 Bällen
                  zurückgeführt
            \vfill
            \item Lösung des Konsensproblems mit einfachem Algorithmus bewiesen
        \end{itemize}
    \end{frame}

    \begin{frame}{Fragen}
        \centering
        Fragen?
    \end{frame}
    % weiter machen mit Notation und Herleitung/Erklärung der Ergebnisse
\end{document}
