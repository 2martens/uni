\documentclass[a4paper,11pt]{scrartcl} % Dokumenttyp
\usepackage[utf8x]{inputenc} % utf8
\usepackage[T1]{fontenc} % Schriftart
\usepackage{lmodern} % Schriftart
\usepackage[ngerman]{babel} % Deutsche Formatierung
%\usepackage{polyglossia} % babel-Ersatz für XeLaTex
\usepackage{graphicx} % Grafiken
\usepackage{tabularx} % Tabellen
\usepackage{fancyhdr} % Kopfzeile
\usepackage{wrapfig} % toc-wrapper
\usepackage{wasysym} % für die Checkboxes
\usepackage{multicol} % für zweispaltigen Text
\usepackage{multirow} % für zweizeiligen Text
\usepackage[tocflat]{tocstyle} % um Punkte-Linie und di Einrückung im Inhaltsverzeichnis zu unterdrücken
\usepackage[left=2.5cm,right=2.5cm,top=2cm,bottom=1cm,includeheadfoot]{geometry} % Seitenränder

\pagestyle{fancyplain} % Kopfzeile anzeigen
\setlength{\headheight}{30pt} % Abstände
\pagenumbering{gobble} % Seitenzahlen unterdrücken
\usetocstyle{noonewithdot} % Punkte-Linie im Inhaltsverzeichnis unterdrücken
\graphicspath{{images/}} % Pfad für Bild-Dateien TODO Bild-Dateien im Unterordner ``images'' ablegen
%\setdefaultlanguage[babelshorthands=true]{german} % Deutsche Formatierung

\hyphenation{da-rauf}
\hyphenation{Gru-ppe}

\newcommand{\AGName}{JuStu \& Minderjährigen-AG} % TODO "Beispiel-AG" durch AG-Name ersetzen
\newenvironment{myitemize}{\begin{itemize}\itemsep -2pt}{\end{itemize}} % Zeilenabstand in Aufzählungen geringer

\fancyhead[C]{\AGName} % Inhalt Kopfzeile mittig
\fancyhead[R]{} % Inhalt Kopfzeile rechts leer
\fancyhead[L]{} % Inhalt Kopfzeile rechts leer
\fancypagestyle{firststyle}{\fancyhead[C]{}\fancyhead[L]{\huge\textbf{\AGName}}} % Inhalt Kopfzeile Seite 1

\begin{document}
\thispagestyle{firststyle}

\begin{wrapfigure}{r}{130pt}
\vspace{-78pt}
  \fbox{
  \begin{minipage}{140pt}
   \tableofcontents
  \end{minipage}
  }
\end{wrapfigure}

\section{Kurzbeschreibung} % TODO Text ersetzen
	Juniorstudierende studieren neben der Schule bereits vor. Minderjährige Studierende sind normale Studierende, die jedoch noch minderjährig sind. Dieses Paper beschreibt, wie mit Juniorstudierenden und minderjährigen Studierenden in der Gruppe umgegangen werden sollte.
    
\section{Zusammenfassung}

\begin{tabular}{c|c|c}
erlaubt & nicht erlaubt & "`Grauzone"' \\
\hline
Studienberatung & \multirow{2}{3cm}{Alkohol (unter 16 Jahre)} & \\
&  & \\
& & \\
\hline
\multirow{3}{3cm}{nicht harter Alkohol\\(16-18 Jahre)} & \multirow{2}{3cm}{harter Alkohol (mehr als 37,5 Vol-\%)} & OWE\\
& & \\
& &  \\
\hline
OWE-Vorstellung & Nachtclubs & Abendprogramm\\
\hline
Gremien/VV & sonstige Drogen & \\
\hline
OE-Ausklang & Kiez-Tour & \\
\hline
Bibführung & Rauchen & \\
\hline
OERB & & \\
\hline
Lernen Lernen & & \\
\hline
Rallye & & \\
\hline
Verantwortung & & \\
\hline
AG-Vorstellung & & 
\end{tabular}
    
\section{Beschreibung lang} 
% TODO Optional Struktur ändern (subsections ändern, subsubsections einfügen etc.) 
% Anmerkung: "subsection*{Titel}" erscheint nicht in der Inhaltsangabe; "\\" erzwingt einen Zeilenumbruch

	\subsection{Behandlung minderjähriger Studierender}

Im Gegensatz zu einem Juniorstudierenden absolviert ein minderjähriger Studierender das volle Programm. Da diese Studierende ein volles Studium absolvieren, haben sie auch bereits eine Hochschulreife erlangt und sind in den meisten Fällen 17 Jahre alt. Damit können sie an allen Einheiten und Modulen der OE-Woche teilnehmen. Bei ihnen gibt es nur ein paar rechtliche Einschränkungen (dazu unten mehr).

	\subsection{Behandlung von Juniorstudierenden}

Beim ersten Kennenlernen solltest du fragen, ob Juniorstudierende anwesend sind. Ansonsten wird der erste Mittwoch ganz normal durchgeführt. Nach dem letzten Programmpunkt bittest du den Juniorstudierenden, noch ein paar Minuten bei dir zu bleiben. Beginne das Gespräch mit einer einleitenden Frage (Beispiel: Was erwartest du dir von der OE-Woche? Wie hat es dir bisher gefallen?). Der Juniorstudierende soll nicht das Gefühl haben, er bräuchte Einzelnachhilfe. Der Sinn des Gesprächs ist, gemeinsam den Stundenplan durchzusprechen. Erkläre ihm kurz die Module und warum sie auch sinnvoll für einen Juniorstudierenden sind:

\newpage

\begin{myitemize}
	\item Sehr wichtig:
		\begin{myitemize}
			\item AG-Vorstellung
				\begin{myitemize}
					\item \textit{Grund}: Die Teilnahme an AGen ist erwünscht. Zudem sind sie ein förderliches soziales Umfeld und erlauben ein Vertiefen persönlicher Interessen.
				\end{myitemize}
			\item OWE-Vorstellung (wenn über 16 Jahre)
				\begin{myitemize}
					\item \textit{Grund}: Das OWE macht Spaß! Kontakte knüpfen ist auch für Juniorstudierende wichtig
				\end{myitemize}
			\item Gremien/VV
				\begin{myitemize}
					\item \textit{Grund}: Mache den Juniorstudierenden klar, dass ein Juniorstudium nicht nur dazu da ist, dass sie fachlich in Informatik gefördert werden. Es soll auch einen Einblick in universitäre Strukturen bieten und genau dazu sind auch diese Punkte im Stundenplan.
				\end{myitemize}
			\item OE-Ausklang
				\begin{myitemize}
					\item \textit{Grund}: macht Spaß! Kontakte knüpfen ist auch für Juniorstudierende wichtig
				\end{myitemize}
			\item Bibliotheksführung
				\begin{myitemize}
					\item \textit{Grund}: Auch Juniorstudierende können einen Bibliotheksausweis erhalten und sollten diesen nutzen.
				\end{myitemize}
			\item OERB
				\begin{myitemize}
					\item \textit{Grund}: Während der OERB werden auch Themen wie Anmeldung am Rechner (Wichtig für SE), Drucken, Jabber, Mafiaseite usw. behandelt. Um am Informatikum arbeiten zu können, sollte sich ein Juniorstudent hier auskennen. Es wird einige Themen geben, die der Juniorstudent in der Zeit erfährt, während die anderen STiNE-Anmeldung machen (VPN, Linux usw.), denn Juniorstudierende müssen sich nicht über STiNE für Veranstaltungen anmelden. Er wird sich also garantiert nicht langweilen.
				\end{myitemize}
			\item Lernen lernen
				\begin{myitemize}
					\item \textit{Grund}: Lernen an der Uni funktioniert anders als in der Schule. Es ist wichtig dies einmal in einem ungezwungenen Umfeld üben zu können. Keine Angst, alle anderen haben genau so wenig Vorerfahrung mit dem Stoff!
				\end{myitemize}
			\item Studienberatung 2
				\begin{myitemize}
					\item \textit{Grund}: Extra für Juniorstudierende (siehe unten)
				\end{myitemize}
			\item Rallye
				\begin{myitemize}
					\item \textit{Grund}: Vorlesungen werden auch für Juniorstudierende am Hauptcampus stattfinden, daher sollte man sich dort auskennen. Außerdem macht es Spaß!
				\end{myitemize}
			\item Verantwortung (heißt im Stundenplan anders)
		\end{myitemize}
	\item Nicht so wichtig:
		\begin{myitemize}
			\item Studienbüro
				\begin{myitemize}
					\item \textit{Grund}: Juniorstudierende haben einen festen Ansprechpartner für alles. Diesen haben sie auch schon kennen gelernt und in diesem Zusammenhang auch das Studienbüro. Schaden kann es natürlich trotzdem nicht.
				\end{myitemize}
			
			\item Studienberatung 3 -- kann aber durchaus als Überbrückung bis zum Ausklang am Freitag genutzt werden
		\end{myitemize}
	\newpage
	\item Nicht wichtig
		\begin{myitemize}
			\item OWE-Vorstellung (für Studierende unter 16 Jahren)
		\end{myitemize}
	\item Nicht unproblematisch
		\begin{myitemize}
			\item Abendprogramm
				\begin{myitemize}
					\item \textit{Grund}: Einige Eltern werden möglicherweise ein Problem damit haben, wenn das Kind jeden Abend bis spät unterwegs ist. Hier sollte man auf wichtige Punkte wie Rallyeausklang und OE-Ausklang hinweisen. Wenn es gar nicht möglich ist teilzunehmen, dann ist das auch nicht so schlimm. Es wird noch andere Gelegenheiten geben die Kommilitonen besser kennen zu lernen.
				\end{myitemize}
		\end{myitemize}		
\end{myitemize}

Das Durchgehen des Stundenplans dient nicht dazu, die Juniorstudierenden von bestimmten Punkten auszuschließen oder einen Freibrief zu erteilen, an gewissen Modulen nicht teilzunehmen. Es ist vielmehr dafür gedacht ein Verständnis zu schaffen, warum jedes Modul auch für ihn wichtig ist. Das kann Konflikte bei der Durchführung der Einheiten später vermeiden.

	\subsection{Eigene Studienberatung 2}
	
In der Studienberatung 2 werden häufig Fragen geklärt, die für die Juniorstudierenden nicht ganz so interessant sind. Dafür gibt es viele Fragen, die nur die Juniorstudierenden interessieren (wie bekomme ich einen Bibliotheksausweis? Kann ich den Hochschulsport auch zu Studierendenpreisen nutzen?). Genau dafür ist die eigene Studienberatung 2 da. Sollten vor der Studienberatung 2 Fragen auftauchen, die ihr nicht beantworten könnt, dann vertröstet sie auf die Studienberatung 2. Da können alle Fragen geklärt werden, die nur die Juniorstudierenden betreffen.	

	\subsection{OERB}

Bei der OERB gab es in den letzten Jahren teilweise Unmut, da die Einführung in STiNE die meiste Zeit in Anspruch nimmt und für Juniorstudierende keine Bedeutung hat. Juniorstudierende haben bei STiNE zwar einen Account, dieser dient aber nur zur Einsicht in die Noten und Daten. Über STiNE kann keine Modulanmeldung erfolgen, es können keine Dokumente abgerufen werden, keine persönlichen Daten geändert werden... Das heißt eine Einführung in STiNE macht keinen Sinn. Stattdessen solltet ihr als Tutoren darauf vorbereitet sein, dem Juniorstudierenden etwas über VPN, Tunneln, Linux, Jabber, Mafiaseite, ... zu erzählen und ihn damit zu beschäftigen (siehe Handout der OERB-AG dazu in der OE-Woche).

\subsection{Sonstiges} % TODO Text ersetzen; (chrono)logisch

Weise öfter einmal darauf hin, dass alle Angebote auch für Juniorstudierende gedacht sind. Es gab schon Juniorstudierende, die sich nicht in AGen oder sogar ins c.t getraut haben, weil sie dachten, so etwas ist nur für "`richtige"' Studierende. Beachtet aber, dass Juniorstudierende nicht den Status "`Studierender"' haben, sondern offiziell weiter Schüler sind. Somit sind sie auf einer VV und bei der Stupa-Wahl nicht stimmberechtigt. An vielem Anderen dürfen sie natürlich trotzdem teilnehmen und dafür gilt es zu begeistern. Seid also stets einladend - wir möchten die Juniorstudierenden integrieren und nirgendwo ausschließen.

	\subsection{Ausgeschlossene Veranstaltungen}
	
Die Kiez-Tour ist nicht für Minderjährige geeignet.

	\subsection{OWE}
	
Das OWE ist für Minderjährige unter 16 Jahren nicht geeignet. Für 16-18 Jahre alte Studierende (noch nicht volljährig) wird die Einverständniserklärung der Eltern benötigt. Sollten minderjährige Studenten auf das OWE mitkommen, so können geschlechtsspezifische Schlafräume angeboten werden.


\section{Rechtliches}		
		
	\subsection*{Kurz:}

Kein Alkohol, kein Tabak/Rauchen, keine Drogen, keine Gewalt, kein Geschlechtsverkehr, kein Glücksspiel und vor allem: Keine Panik!

		\subsection*{Lang:}
		
Die folgenden Informationen sind rechtlich abgesichert. Lest sie durch, denn es ist sehr relevant. In kursiv stehen Bemerkungen dazu, was das für praktische Auswirkungen auf die OE hat.

\subsection{Aufsichtspflicht}

Grundsätzlich bedürfen Minderjährige stets der Aufsicht. Dabei richtet sich der Umfang nach dem Alter und der Verständigkeit des Kindes. Für die Aufsicht gibt es folgende Grundsätze, mit deren Hilfe der Aufsichtspflicht entsprochen werden kann.

\begin{enumerate}
	\item Zunächst müssen richtige Anweisungen gegeben werden
	\item Deren Einhaltung und Ausführung sind zu überwachen
	\item Bei Bedarf muss der Minderjährige zur Ordnung gerufen werden
	\item Schließlich müssen im Fall von Missverhalten daraus Konsequenzen gezogen werden.
\end{enumerate}

Wenn Aufsichtspflichtige sich nachweisbar daran halten, dann können sie sich kaum der Verletzung der Aufsichtspflicht schuldig machen. Dabei ist es unmöglich zu verlangen, dass unter allen Umständen Schäden vermieden werden. Stattdessen sind Aufsichtspflichtige gefordert, nach bestem Wissen und Gewissen und den oben genannten Grundsätzen das zu tun, was sie für notwendig halten, um einen möglichen Schaden zu verhindern.

Insgesamt hängt die Beobachtung, Belehrung und Aufklärung vom Alter, der Eigenart und dem Charakter des konkreten Minderjährigen ab. Also danach, was Eltern vernünftigerweise in der konkreten Situation an erforderlichen und zumutbaren Maßnahmen treffen müssen, um Schädigungen durch ihre Kinder zu verhindern.

\textit{Es kann davon ausgegangen werden, dass Minderjährige (die meisten 17 Jahre alt), die sich vernünftig benehmen, durchaus unbeobachtet gelassen werden können.}

\subsection{Weiterdelegierung der Aufsichtspflicht}

Wenn es mit den Eltern abgesprochen und am besten schriftlich festgehalten wurde, kann die Aufsicht auch an eine andere verantwortungsvolle und volljährige Person für einen bestimmten Anlass übertragen werden. Für diese Person gelten dann die gleichen Regeln, wie für die ursprünglich eingesetzte Person.

\textit{Gerade dieser Punkt ist für die OE von immenser Bedeutung, denn in den unterschiedlichen Veranstaltungen sind teilweise unterschiedliche Personen zuständig.}

\subsection{Haftung}

\subsubsection*{Minderjähriger kommt zu Schaden}

Kommt der Minderjährige oder sein Eigentum zu Schaden, können sowohl der Minderjährige selber als auch die Eltern einen Schadensersatzanspruch gegen den Aufsichtspflichtigen haben. Der Anspruch muss jedoch nur einmal bezahlt werden, da beide (Minderjähriger und Eltern) als Gesamtgläubiger gelten.

Damit ein Schadensersatzanspruch entstehen kann, muss zunächst ein Schaden (z.B. Verletzung) eingetreten sein. Desweiteren muss die Aufsichtsperson fahrlässig ihre Aufsichtspflicht verletzt haben. Dies kann auch durch das Unterlassen einer gebotenen Handlung geschehen. Das Maß der Aufsichtspflicht wurde bereits im Abschnitt "`Aufsichtspflicht"' erläutert.

Wenn sich herausstellt, dass der Schaden durch eine Verletzung der Aufsichtspflicht entstanden ist, dann muss die Aufsichtsperson den entstandenen Schaden ersetzen.

\subsubsection*{Dritte kommen zu Schaden}

Alle vom Minderjährigen verursachten Schäden müssen grundsätzlich vom Minderjährigen beglichen werden, sofern er die nötige Einsichtsfähigkeit hatte. Diese Einsicht ist bei 17-jährigen Studierenden grundsätzlich zu unterstellen.

Zusätzlich kann aber auch die Aufsichtsperson haften. Diese Haftung tritt jedoch nur ein, wenn eine Aufsichtspflichtverletzung vorzuwerfen ist. Das eine solche Verletzung nicht vorliegt, muss die Aufsichtsperson beweisen.

Der Minderjährige muss eine rechtswidrige unerlaubte Handlung begangen haben, also das Eigentum oder die Gesundheit eines anderen vorsätzlich oder fahrlässig geschädigt haben. Der Schaden muss einem Dritten entstehen (alle Personen außer dem Minderjährigen und dem Aufsichtspflichtigen).

Wenn ein Minderjähriger einem Dritten einen Schaden zufügt, greifen zunächst zwei Vermutungen: Der Aufsichtspflichtige hat seine Aufsichtspflicht schuldhaft verletzt und zwischen dem Schaden und der Verletzung der Aufsichtspflicht besteht ein ursächlicher Zusammenhang.

Die Aufsichtsperson kann jedoch bezüglich beider Vermutungen einen Entlastungsbeweis führen. Das erforderliche Ausmaß der Aufsichtspflicht wurde in dem Abschnitt "`Aufsichtspflicht"' abgehandelt. Dieser Entlastungsbeweis wird bei einem 17-jährigen wohl nur bei sehr grober Verletzung der Aufsichtspflicht nicht erfolgreich sein. Sofern der Minderjährige sorgsam instruiert ist und dem Alter entsprechend beaufsichtigt wird, kann eine solche Haftung also vermieden werden.

Wenn sowohl der Minderjährige als auch die Aufsichtsperson haften, hat der Dritte gegen beide einen Schadensersatzanspruch. Allerdings kann er sich aussuchen, gegen wen dieser Anspruch geltend gemacht wird. Im Ergebnis muss der Schaden nur einmal ersetzt werden.

\textit{Durch die Beweislastumkehr sollten solche Schäden von Beginn an vermieden werden. Allerdings ist davon auszugehen, dass es in den meisten Fällen nicht so weit kommt.}

\subsection{Jugendschutz}

Die folgenden Regelungen beziehen sich nur auf Jugendliche zwischen 16 und 18 Jahren, die im Folgenden Minderjährige genannt werden. Wenn eine über 18 Jahre alte Person die Aufsichtspflicht von den Eltern übernommen hat, gilt sie als erziehungsbeauftragte Person im Sinne des Jugendschutzgesetzes.

\subsubsection*{Gaststätten}

Minderjährige dürfen sich ohne Begleitung einer erziehungsbeauftragen Person nur zwischen 5 und 24 Uhr in Gaststätten aufhalten. In Begleitung einer erziehungsbeauftragten Person darf sich ein Minderjähriger unbeschränkt in Gaststätten aufhalten.

\textit{Das Ausklingenlassen der Rallye in einem Restaurant ist mit Zustimmung der Eltern kein Problem, da die Minderjährigen selbst ohne Begleitung bis 24 Uhr in einer Gaststätte bleiben dürfen.}

\subsubsection*{Disco/Tanzveranstaltungen}

In Diskotheken darf sich ein Minderjähriger ohne Begleitung einer erziehungsbeauftragten Person nur bis 24 Uhr aufhalten. In Begleitung einer erziehungsbeauftragten Person darf er sich dort zeitlich unbegrenzt aufhalten.

\textit{Wenn eine klar definierte Aufsichtsperson die Minderjährigen bei der Kieztour begleitet, dann könnte auch das machbar sein. Sicherheitshalber sollte jedoch von solchen Dingen abgesehen werden.}

\subsubsection*{Alkohol}

Branntwein, branntweinhaltige Getränke oder Lebensmittel, die Branntwein in nicht nur geringfügiger Menge enthalten, dürfen an Jugendliche unter 18 Jahren grundsätzlich nicht abgegeben werden. Ihnen darf auch der Verzehr dieser Getränke in der Öffentlichkeit nicht gestattet werden. Als Branntwein gelten durch Destillation hergestellte Spirituosen, die durch Destillation einen Mindestalkoholgehalt von 37,5 Vol-\% und von maximal 86 Vol-\% aufweisen. Als Abgabe gilt nicht nur das Verkaufen, sondern auch das kostenlose Überlassen. 

Als Öffentlichkeit ist insofern auch die Jugendherberge, beziehungsweise die Zimmer anzusehen. Das Jugendschutzgesetz gilt nur in privaten Räumen nicht. Zudem gilt in den meisten Herbergen ohnehin ein Alkoholverbot.

Bier und Wein dürfen von Jugendlichen über 16 aber unter 18 Jahren getrunken werden.

\subsubsection*{Rauchen}

Ein Minderjähriger darf in der Öffentlichkeit Tabakwaren jeglicher Art, unabhängig davon, ob sie zum Rauchen bestimmt sind oder nicht (z.B. Kautabak, Schnupftabak) nicht erwerben oder konsumieren, also auch nicht rauchen. Dieses Verbot gilt ohne Ausnahme, auch bei Begleitung durch eine erziehungsbeauftragte Person und für den Erwerb von Tabakwaren im Auftrag einer erwachsenen Person.

\subsubsection*{Drogen}

Die Vorschriften des Betäubungsmittelgesetzes sind zu beachten. Drogenkonsum ist daher in jedem Fall zu unterbinden. Wer z.B. als Person über 21 Jahre Betäubungsmittel unerlaubt an eine Person unter 18 Jahren abgibt oder sie verabreicht oder zum unmittelbaren Verbrauch überlässt, wird mit Freiheitsstrafe nicht unter einem Jahr bestraft.

\subsubsection*{Strafvorschriften}

Wer an einen Jugendlichen alkoholische Getränke oder Tabakwaren abgibt, handelt ordnungswidrig im Sinne des Jugendschutzgesetzes. Diese Ordnungswidrigkeit kann mit einem Bußgeld geahndet werden.
	
\end{document}
