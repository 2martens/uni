\documentclass[a4paper,11pt]{scrartcl} % Dokumenttyp
\usepackage[utf8x]{inputenc} % utf8
\usepackage[T1]{fontenc} % Schriftart
\usepackage{lmodern, textcomp} % Schriftart
\usepackage[ngerman]{babel} % Deutsche Formatierung
\usepackage{graphicx} % Grafiken
\usepackage{amsmath}
\usepackage{tabularx} % Tabellen
\usepackage{fancyhdr} % Kopfzeile
\usepackage{wrapfig} % toc-wrapper
\usepackage{wasysym} % für die Checkboxes
\usepackage{multicol} % für zweispaltigen Text
\usepackage[tocflat]{tocstyle} % um Punkte-Linie und di Einrückung im Inhaltsverzeichnis zu unterdrücken
\usepackage[left=2.5cm,right=2.5cm,top=2cm,bottom=1cm,includeheadfoot]{geometry} % Seitenränder
\usepackage[hidelinks]{hyperref}
\usepackage{longtable}
\usepackage{framed}
\usepackage{hyperref}
\usepackage{url}

\pagestyle{fancyplain} % Kopfzeile anzeigen
\setlength{\headheight}{30pt} % Abstände
\pagenumbering{gobble} % Seitenzahlen unterdrücken
\usetocstyle{noonewithdot} % Punkte-Linie im Inhaltsverzeichnis unterdrücken
\graphicspath{{images/}} % Pfad für Bild-Dateien TODO Bild-Dateien im Unterordner ``images'' ablegen

\newcommand{\AGName}{Diskussionsleitfaden} % TODO "Beispiel-AG" durch AG-Name ersetzen
\newenvironment{myitemize}{\begin{itemize}\itemsep -2pt}{\end{itemize}} % Zeilenabstand in Aufzählungen geringer

\fancyhead[C]{\AGName} % Inhalt Kopfzeile mittig
\fancyhead[R]{} % Inhalt Kopfzeile rechts leer
\fancyhead[L]{} % Inhalt Kopfzeile rechts leer
\fancypagestyle{firststyle}{\fancyhead[C]{}\fancyhead[L]{\huge\textbf{\AGName}}} % Inhalt Kopfzeile Seite 1

\hyphenation{inte-res-san-te}

\begin{document}
\thispagestyle{firststyle}

\begin{wrapfigure}{r}{130pt}
\vspace{-45pt}
  \fbox{
  \begin{minipage}{140pt}
   \tableofcontents
  \end{minipage}
  }
\end{wrapfigure}

\textbf{Version:} \#3, \today % TODO Version hochzählen, falls Paper aktualisiert wird

\section{Kurzbeschreibung} % TODO Text ersetzen
    Dieser Diskussionleitfaden soll euch beim Führen der Diskussion
    im Anschluss an den HoPo-Vortrag helfen. Die Ersties sollten nach 
    der Einheit die folgende Message mitnehmen:
    Die Hochschulpolitik ist relevant für die Studierenden und kann
    direkten Einfluss auf ihr Studium haben.
  
 
\section{Situationsbeschreibung} 
% TODO Optional Struktur ändern (subsections ändern, subsubsections einfügen etc.) 
% Anmerkung: "subsection*{Titel}" erscheint nicht in der Inhaltsangabe; "\\" erzwingt einen Zeilenumbruch
Die Ersties kommen gerade vom HoPo-Vortrag mit Arne Köhn und sollten daher noch im Thema sein.
Der Vortrag handelt über den Wandel in der Hochschulpolitik in den letzten Jahren und soll
im Prinzip zeigen, dass sich viele Dinge ändern und dass nichts in Stein gemeißelt ist.

An sich gibt es im Vortrag keinen sehr polarisierenden Inhalt, es ist mehr ein Informationsvortrag.
Eine Diskussion sollte dennoch möglich sein.

\section{Motivation}

Für den Fall, dass euch nicht einfällt, wie ihr die Diskussion motivieren könnt, haben wir hier für
euch etwas vorbereitet.

Ein guter Anfang ist die Ersties nach ihrer Meinung über den Vortrag zu fragen. Das kann idealerweise
in einer Runde gemacht werden, sodass jeder zu Wort kommt. Dabei solltet ihr bereits eine Grundstimmung
feststellen können. Ebenfalls fallen hierbei womöglich bereits interessante Diskussionsthemen, die im
Anschluss aufgegriffen werden können.

Ein kleines Beispiel zum Verständnis.

\begin{quotation}
  Ersties = \{Alex, Michael, Julian, Anna, Laura\}\\
  Runde ist im Gange...\\
  Michael: "`[...]vor allem finde ich es interessant, dass die Unimittel effektiv gekürzt werden[...]"'\\
  ...und geht weiter\\
  Tutor: "`Michael hat ja schon die effektive Kürzung angesprochen. Was denkt ihr denn darüber?"'
\end{quotation}

Auf die Art könntet ihr relativ leicht von der Abfragerunde in eine Diskussion übergehen. Welche
Überleitung sich konkret eignet, hängt natürlich von eurer Gruppe ab.

Ein weiterer Teil der Motivation besteht daraus, nicht explizit eine Diskussion anzukündigen.
Im Stundenplan ist diese zwar vermerkt und daher nicht geheim, aber die Einheit mit
"`Wir diskutieren jetzt über Hochschulpolitik"' anzufangen, ist denkbar ungeeignet. Die meisten
der Ersties werden frisch aus der Schule kommen und dort Politikunterricht gehabt haben. Verordnete
(und angekündigte) Diskussionen verfehlen meistens ihren Zweck und die Motivation ist bei den meisten
gleich im Keller.

Schließlich noch eine Hilfestellung an unsere ¬Hochschulpolitik-Kenner, um kompetent für Hochschulpolitik
werben zu können.

\begin{itemize}
  \item durch Engagement in FSR, Studienkommission, Fakultätsrat wurden die Inf-Studiengänge reformiert
  \item durch Engagement im Akademischen Senat (AS) wird eine Rahmen-Prüfungsordnung und eine Grundordnung erarbeitet
  \item durch Engagement im StuPa und mittelbar im AStA wirds sich für eine Ausfinanzierung der Uni eingesetzt
  \item durch Engagement im Fakultätsrat wurde und wird sich gegen eine schädliche Zerschlagung der MIN-Fakultät gewehrt
  \item durch Engagement bei Dies Academici (wenn das der Plural ist) wurde sich gegen ein zu einengendes FÜS-Konzept gewehrt
  \item durch Engagement bei einem MIN-Dies wurde sich für eine pragmatisch positive Studienreform eingesetzt
\end{itemize}

\section{Moderationstipps}

Wir wollen eine sich natürlich entwickelnde Diskussion erreichen - keine erzwungene. Daher solltet
ihr euch in der Moderation auf die Einhaltung der Diskussionsregeln beschränken und lediglich Inputs
geben, wenn die Diskussion ins Stocken gerät.

Sollten die Ersties das Themenfeld Hochschulpolitik verlassen und stattdessen über die "`große"'
Politik reden, dann ist das ebenso wünschenswert. Denn wer sich für nationale und internationale
Politik interessiert, der ist dann wahrscheinlich auch für Hochschulpolitik begeisterungsfähig.

Für den Fall, dass die Diskussion in Richtung Flüchtlinge geht (durchaus plausibel), solltet ihr
auf das Diskussionsklima achten (wie eigentlich immer). Wenn das Klima angenehm bleibt, dann
kann ruhig darüber gesprochen werden. Sollte die Diskussion jedoch aus dem Ruder laufen,
solltet ihr sie auf ein anderes Thema lenken.

Warum dieser Hinweis an dieser Stelle? Aufgrund der Zentralen Erstaufnahme direkt beim Ikum
ist das Thema Flüchtlinge (internationale, nationale und lokale Politik) allgegenwärtig.
Als Diskussionsthema kann es daher durchaus vorkommen. Wir hoffen natürlich, dass kein Erstie
rassistische Positionen vertritt.

\section{Diskussionsthemen}

Es folgt eine Liste von Diskussionsthemen, über die unterschiedliche Meinungen vorherrschen können
und die sich daher für eine Diskussion eignen.

\begin{myitemize}
  \item Engagement in Gremien: Vorteile erst für nachkommende "`Generationen"' - ein Problem?
  \item Zivilklausel - gut oder schlecht?
  \item Drittmittel - Beeinflussung der Lehre?
\end{myitemize}

Ansonsten kann man natürlich auch passend über die OE selber reden. Schließlich ist das im Grunde
ein Paradebeispiel von Selbstverwaltung auf unterster Ebene. Die Studis organisieren ihre OE selber.

\section{Troubleshooting}

\subsection{Ersties haben kein Bock}

Sollte deine Gruppe unmotiviert sein, versuche ein Gespräch über die OE zu führen. Nebenbei kannst
du darauf hinweisen, dass sich viele Studis in ihrer Freizeit für die OE einsetzen, um den Ersties
eine gute OE zu ermöglichen. Solch ein Gespräch sollte auch deshalb funktionieren, da es nicht explizit
um "`Politik"' geht, sondern um Engagement.

\subsection{"`Es ändert sich doch eh nichts"'}

Wenn jemand aus deiner Gruppe der Meinung ist, dass ein Engagement ohnehin nichts bewirkt, dann kannst
du auf die OE, aber auch auf den FSR hinweisen. In beiden Bereichen ändert sich bei entsprechendem
Engagement durchaus etwas. Aber auch bspw. bei Berufungskommissionen kann man bedingt einen Einfluss darauf
haben, wer für die Stelle in Betracht kommt.

\subsection{"`Da sind so Spinner..."'}

Sollte jemand die Position vertreten, dass in der Selbstverwaltung (in Teilen) nur "`Spinner"' tätig sind,
dann kannst du diese Person darauf hinweisen, dass sich daran nur etwas ändern kann, wenn ¬"`Spinner"'
in die Selbstverwaltung gehen.

\end{document}