\documentclass[a4paper,11pt]{scrartcl} % Dokumenttyp
\usepackage[utf8x]{inputenc} % utf8
\usepackage[T1]{fontenc} % Schriftart
\usepackage{lmodern, textcomp} % Schriftart
\usepackage[ngerman]{babel} % Deutsche Formatierung
\usepackage{graphicx} % Grafiken
\usepackage{tabularx} % Tabellen
\usepackage{fancyhdr} % Kopfzeile
\usepackage{wrapfig} % toc-wrapper
\usepackage{wasysym} % für die Checkboxes
\usepackage{multicol} % für zweispaltigen Text
\usepackage[tocflat]{tocstyle} % um Punkte-Linie und di Einrückung im Inhaltsverzeichnis zu unterdrücken
\usepackage[left=2.5cm,right=2.5cm,top=2cm,bottom=1cm,includeheadfoot]{geometry} % Seitenränder
\usepackage[hidelinks]{hyperref}
\usepackage{longtable}
\usepackage{framed}
\usepackage{hyperref}
\usepackage{url}

\pagestyle{fancyplain} % Kopfzeile anzeigen
\setlength{\headheight}{30pt} % Abstände
\pagenumbering{gobble} % Seitenzahlen unterdrücken
\usetocstyle{noonewithdot} % Punkte-Linie im Inhaltsverzeichnis unterdrücken
\graphicspath{{images/}} % Pfad für Bild-Dateien TODO Bild-Dateien im Unterordner ``images'' ablegen

\newcommand{\AGName}{Diskussionsleitfaden} % TODO "Beispiel-AG" durch AG-Name ersetzen
\newenvironment{myitemize}{\begin{itemize}\itemsep -2pt}{\end{itemize}} % Zeilenabstand in Aufzählungen geringer

\fancyhead[C]{\AGName} % Inhalt Kopfzeile mittig
\fancyhead[R]{} % Inhalt Kopfzeile rechts leer
\fancyhead[L]{} % Inhalt Kopfzeile rechts leer
\fancypagestyle{firststyle}{\fancyhead[C]{}\fancyhead[L]{\huge\textbf{\AGName}}} % Inhalt Kopfzeile Seite 1

\begin{document}
\thispagestyle{firststyle}

\begin{wrapfigure}{r}{130pt}
\vspace{-45pt}
  \fbox{
  \begin{minipage}{140pt}
   \tableofcontents
  \end{minipage}
  }
\end{wrapfigure}

\textbf{Version:} \#1, \today % TODO Version hochzählen, falls Paper aktualisiert wird

\section{Kurzbeschreibung} % TODO Text ersetzen
    Dieser Diskussionleitfaden soll euch beim Führen der Diskussion
    im Anschluss an den HoPo-Vortrag helfen. Die Ersties sollten nach 
    der Einheit die folgende Message mitnehmen:
    Die Hochschulpolitik ist relevant für die Studierenden und kann
    direkten Einfluss auf ihr Studium haben.
  
 
\section{Situationsbeschreibung} 
% TODO Optional Struktur ändern (subsections ändern, subsubsections einfügen etc.) 
% Anmerkung: "subsection*{Titel}" erscheint nicht in der Inhaltsangabe; "\\" erzwingt einen Zeilenumbruch
Die Ersties kommen gerade vom HoPo-Vortrag mit Arne Köhn und sollten daher noch im Thema sein.
Der Vortrag handelt über den Wandel in der Hochschulpolitik in den letzten Jahren und soll
im Prinzip zeigen, dass sich viele Dinge ändern und dass nichts in Stein gemeißelt ist.

An sich gibt es im Vortrag keinen sehr polarisierenden Inhalt, es ist mehr ein Informationsvortrag.
Eine Diskussion sollte dennoch möglich sein.

\section{Diskussionsthemen}

Es folgt eine Liste von Diskussionsthemen, über die unterschiedliche Meinungen vorherrschen können
und die sich daher für eine Diskussion eignen.

\begin{myitemize}
  \item Engagement in Gremien: Vorteile erst für nachkommende "`Generationen"' - ein Problem?
  \item Zivilklausel - gut oder schlecht?
  \item Drittmittel - Beeinflussung der Lehre?
\end{myitemize}

Ansonsten kann man natürlich auch passend über die OE selber reden. Schließlich ist das im Grunde
ein Paradebeispiel von Selbstverwaltung auf unterster Ebene. Die Studis organisieren ihre OE selber.

\section{Troubleshooting}

\subsection{Ersties haben kein Bock}

Sollte deine Gruppe unmotiviert sein, versuche ein Gespräch über die OE zu führen. Nebenbei kannst
du darauf hinweisen, dass sich viele Studis in ihrer Freizeit für die OE einsetzen, um den Ersties
eine gute OE zu ermöglichen. Solch ein Gespräch sollte auch deshalb funktionieren, da es nicht explizit
um "`Politik"' geht, sondern um Engagement.

\subsection{"`Es ändert sich doch eh nichts"'}

Wenn jemand aus deiner Gruppe der Meinung ist, dass ein Engagement ohnehin nichts bewirkt, dann kannst
du auf die OE, aber auch auf den FSR hinweisen. In beiden Bereichen ändert sich bei entsprechendem
Engagement durchaus etwas. Aber auch bspw. bei Berufungskommissionen kann man bedingt einen Einfluss darauf
haben, wer für die Stelle in Betracht kommt.

\subsection{"`Da sind so Spinner..."'}

Sollte jemand die Position vertreten, dass in der Selbstverwaltung (in Teilen) nur "`Spinner"' tätig sind,
dann kannst du diese Person darauf hinweisen, dass sich daran nur etwas ändern kann, wenn ¬"`Spinner"'
in die Selbstverwaltung gehen.

\end{document}