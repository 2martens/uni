\documentclass[10pt,a4paper,oneside,ngerman,numbers=noenddot]{scrartcl}
\usepackage[T1]{fontenc}
\usepackage[utf8]{inputenc}
\usepackage[ngerman]{babel}
\usepackage{amsmath}
\usepackage{amsfonts}
\usepackage{amssymb}
\usepackage{paralist}
\usepackage[locale=DE,exponent-product=\cdot,detect-all]{siunitx}
\usepackage{tikz}
\usetikzlibrary{matrix,fadings,calc,positioning,decorations.pathreplacing,arrows,decorations.markings}
\usepackage{multirow}
\pagenumbering{arabic}
% ensures that paragraphs are separated by empty lines
\parskip 12pt plus 1pt minus 1pt
\parindent 0pt
% define how the sections are rendered
\def\thesection{\arabic{section})}
\def\thesubsection{\alph{subsection})}
\def\thesubsubsection{(\roman{subsubsection})}

\begin{document}
\author{Jim 2martens}
\title{OE-Tutor-Skript}
\maketitle

\tableofcontents

\section{Mittwoch 1}

	\subsection{erste 5 Minuten}
	Begrüßung - Buttons schön und gut, aber doch nicht DIE Lösung - Namen lernen auf spielerische Art und Weise - Namen können so besser gemerkt werden.
	
	Vorschlag - Runde,  bei der jeder ein Adjektiv und seinen Namen sagt - Adjektiv beginnt mit dem ersten Buchstaben des Vornamens.
	
	Frage - kennt ihr die Namen der anderen? 
	-> Ja: Namen zwar bekannt - aber keine Ahnung über die Personen dahinter. Vorschlag - Runde, bei der jeder drei wahre und eine falsche Eigenschaft (keine offensichtlichen) nennt - die anderen müssen raten.
	-> Nein: OK - % TODO: anderes Spiel
	
	Vorstellung - studiere Bachelor Informatik im 5. Semester und mache zum zweiten Mal OE-Tutor. OE-Woche kein Frontalunterricht - was sind eure Vorstellungen von Informatik/SSE?
	
	\subsection{Was ist Informatik/SSE?}
	
	

\section{Donnerstag 1}

\section{Montag}

\section{Dienstag}

\section{Mittwoch 2}

\section{Donnerstag 2}

\section{Freitag}

\end{document}