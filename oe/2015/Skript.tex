\documentclass[10pt,a4paper,oneside,ngerman,numbers=noenddot]{scrartcl}
\usepackage[T1]{fontenc}
\usepackage[utf8]{inputenc}
\usepackage[ngerman]{babel}
\usepackage{amsmath}
\usepackage{amsfonts}
\usepackage{amssymb}
\usepackage{paralist}
\usepackage[locale=DE,exponent-product=\cdot,detect-all]{siunitx}
\usepackage{tikz}
\usetikzlibrary{matrix,fadings,calc,positioning,decorations.pathreplacing,arrows,decorations.markings}
\usepackage{multirow}
\pagenumbering{arabic}
% ensures that paragraphs are separated by empty lines
\parskip 12pt plus 1pt minus 1pt
\parindent 0pt
% define how the sections are rendered
\def\thesection{\arabic{section})}
\def\thesubsection{\alph{subsection})}
\def\thesubsubsection{(\roman{subsubsection})}

\begin{document}
\author{Jim 2martens}
\title{OE-Tutor-Skript}
\maketitle

\tableofcontents

\section{Freitag 1}

	\subsection{erste 5 Minuten}
	Begrüßung - Buttons schön und gut, aber doch nicht DIE Lösung - Namen lernen auf spielerische Art und Weise - Namen können so besser gemerkt werden.
	
	Vorschlag - Runde,  bei der jeder ein Adjektiv und seinen Namen sagt - Adjektiv beginnt mit dem ersten Buchstaben des Vornamens. (spontan auf Ballwurf ändern für müde Gruppen)
	
	\textit{Nach den Kennlernspielen:} Kennen uns gegenseitig - was ist eigenlich Informatik?
	
	\subsection{Was ist Informatik/SSE?}
	
	Gedanken aufschreiben - großes Blatt Papier. 
	
	\textit{Aufschreiben fertiggestellt:} Sieht gut aus - Runde, in der jeder beantwortet, warum er Informatik/SSE in HH studiert und was für Erwartungen er hat?
	
	Schauen ob Erwartungen sich bestätigen werden. 
	
	\subsection{Studienberatung 1}
	
	Organisatorisches - Namensliste - jeder trägt sich bitte ein - wird für weiteren Verlauf benötigt. Stundenplan ausgeben - durchgehen. STiNE-Nutzername, Passwort und TAN-Liste EXTREM wichtig für OERB.
	
	Erklärung Studienberatung - was wisst ihr bereits - was wisst ihr nicht - was solltet ihr wissen?
	
	Fragen aufschreiben - Stifte/Karten ausgeben.
	
	\textit{Aufschreiben der Fragen fertig:} Fragen sortieren - nach Themen sortieren (Studienplan, Module, Veranstaltungen, Prüfungen, Materialien, Infoquellen und Ansprechpartner, Fortgeschrittenes Studium). Unpassendes auf Sonstiges.
	
	Aufteilen in 6 gleich große Teilgruppen - alle Fragen anschauen lassen - Materialien verteilen (FSB und PO, Modultabelle, KVV-Kurzfassung, OE-Bits, auf Tetris an Wand aufmerksam machen).
	
	Mit Material vertraut machen lassen - Arbeit einleiten. \textit{Währenddessen:} Sonstiges-Stapel anschauen und wenn nötig Gruppen helfen.
	
	Zusammenfassung - passende sonstige Fragen klären.
	
	Stundenplan austeilen - mithilfe der Unterlagen ausfüllen lassen.	
\section{Montag}
	\subsection{OERB}
		Zu OERB gehen - anwesend sein - helfen, wo nötig.
	\subsection{Studienbüro}
		Anwesend sein.
	\subsection{AG-Vorstellungen}
		Anwesend sein.
	\subsection{Gremienorientierung}
		AStA erklären - FSR einführen - Spielregeln erklären.
		
		\textit{Spiel durchführen}
		
		\textit{Falls noch Zeit übrig oder Gruppe keine Lust mehr:} Über Gremienstruktur sprechen
	
\section{Dienstag}
	\subsection{Gremienvorstellung}
		Anwesend sein - wenn nötig, FSR erklären.
	\subsection{HoPo-Vortrag}
		Anwesend sein.
	\subsection{HoPo-Diskussion}
		Anhand des Diskussionsleitfadens diskutieren.
	\subsection{Vorlesungsvorbereitung}
		Grundlagen austeilen und durchgehen - Vorlesungsängste austeilen und durchgehen - Petrinetze austeilen und NICHT durchgehen.
	\subsection{Studienberatung 2}
		Offene Fragen klären - wiederholen. Themenblock Fortgeschrittenes Studium durchgehen.
	
\section{Donnerstag}
	\subsection{Bib}
		Anwesend sein.
	\subsection{Vorlesungsnachbereitung}
		Feedback über VL einholen - Nachbereitung motivieren - Vorlesungsstoffnachbereitung - Pause
	\subsection{Skriptnachbereitung}
		Erneutes Motivieren - Skriptarbeit - Feedback zur Gruppenarbeit

\section{Freitag2}
	\subsection{Seminar}
		Anwesend sein.
	\subsection{Seminar-Nachbesprechung}
		Feedback zu Seminar einholen - Über Themen aus Seminar diskutieren
	\subsection{Studienberatung 3}
		Offene Fragen klären - wiederholen. Rund ums Studium - Nachtreffen organisieren
	\subsection{Feedback}
		Nummern in Fragebögen ausfüllen - Fragebögen austeilen - eigenen ausfüllen - einsammeln

		persönliches Feedback
\end{document}