\documentclass[10pt,a4paper,oneside,ngerman,numbers=noenddot]{scrartcl}
\usepackage[T1]{fontenc}
\usepackage[utf8]{inputenc}
\usepackage[ngerman]{babel}
\usepackage{amsmath}
\usepackage{amsfonts}
\usepackage{amssymb}
\usepackage{paralist}
\usepackage[locale=DE,exponent-product=\cdot,detect-all]{siunitx}
\usepackage{tikz}
\usetikzlibrary{matrix,fadings,calc,positioning,decorations.pathreplacing,arrows,decorations.markings}
\usepackage{polynom}
\polyset{style=C, div=:,vars=x}
\pagenumbering{arabic}
\def\thesection{\arabic{section})}
\def\thesubsection{\alph{subsection})}
\def\thesubsubsection{(\roman{subsubsection})}

\begin{document}
\author{Jim Martens}
\title{Hausaufgaben zum 10./11. Januar}
\maketitle
\section{} %1
\subsection{} %a
$H_{1}$ ist keine Untergruppe, da das neutrale Element der Multiplikation fehlt.

$H_{2}$ ist keine Untergruppe, da sie nicht abgeschlossen ist. $4 \cdot 4 = 16 = 3$ (mod $13$) ist nicht in $H_{2}$ enthalten.

$H_{3}$ ist eine Untergruppe von $G$, da das neutrale Element $1$ vorhanden ist, die Menge abgeschlossen ist ($12 \cdot 12 = 144 = 1$ (mod $13$) und jedes Element ein Inverses hat ($1$ ist zu sich selbst invers und $12$ ist zu sich selbst invers).
\subsection{} %b
$H = <3> = \{1,3,9\}$
$a=1 \rightarrow 1H = H = 3H = 9H$ \\
$a=2 \rightarrow 2H = \{2,6,5\} = 6H$ \\
$a=4 \rightarrow 4H = \{4,12,10\} = 10H = 12H$ \\
$a=5 \rightarrow 5H = \{5,2,6\}$ \\
$a=7 \rightarrow 7H = \{7,8,11\} = 8H = 11H$
\section{} %2
\subsection{} %a
$H = \{id, (12)\}$\\
\\
Linksnebenklassen:\\
$id \circ H = \{id \circ id, id \circ (1,2)\} = \{id, (1,2)\}$\\
$(1,2) \circ H = \{(1,2) \circ id, (1,2) \circ (1,2)\} = \{(1,2), id\}$\\
$(1,3) \circ H = (1,2,3) \circ H = \{(1,3) \circ id, (1,3) \circ (1,2)\} = \{(1,3), (1,2,3)\}$\\
$(2,3) \circ H = (1,3,2) \circ H = \{(2,3) \circ id, (2,3) \circ (1,2)\} = \{(2,3), (1,3,2)\}$\\
\\
Rechtsnebenklassen:\\
$H \circ id = \{id \circ id, (1,2) \circ id\} = \{id, (1,2)\}$\\
$H \circ (1,2) = \{id \circ (1,2), (1,2) \circ (1,2)\} = \{(1,2), id\}$\\
$H \circ (1,3) = H \circ (1,3,2) = \{id \circ (1,3), (1,2) \circ (1,3)\} = \{(1,3), (1,3,2)\}$\\
$H \circ (2,3) = H \circ (1,2,3) = \{id \circ (2,3), (1,2) \circ (2,3)\} = \{(2,3), (1,2,3)\}$

\subsection{} %b
$S_{6}$ enthält $6! = 720$ Elemente. $H$ ist eine mögliche Untergruppe von $G$, da $H$ $360$ Elemente hat und dies ein Teiler von $720$ ist. Nach dem Satz von Lagrange kommen nur Untergruppen in Frage, deren Mächtigkeit ein Teiler der Mächtigkeit der "Obergruppe" ist.
%todo
\subsection{} %c
$G = \{1,5,11,13,17,19,23,25,29,31,35,37,41\}$

Die Linksnebenklasse $gH$ ist gleich der Rechtsnebenklasse $Hg$, weil sowohl Addition als auch Multiplikation assoziativ sind.
\section{} %3
\subsection{} %a
\begin{alignat*}{2}
x + y &=& (4x^{2} - x + 2) + (2x^{3} + x^{2} - 3x + 2) \\
&=& 2x^{3} + 5x^{2} - 4x + 4 \\
x \cdot y &=& (4x^{2} - x + 2) \cdot (2x^{3} + x^{2} - 3x + 2) \\
&=& 8x^{5} + 4x^{4} - 12x^{3} + 8x^{2} - 2x^{3} -x^{2} + 3x^{2} -2x + 4x^{3} + 2x^{2} -6x + 4 \\
&=& 8x^{5} + 4x^{4} - 10x^{3} + 12x^{2} - 8x +4
\end{alignat*}
\subsection{} %b
\begin{alignat*}{2}
Koeffizient(a(x) \cdot b(x),x^{7}) &=& -2x^{7} + 6x^{7} - 18x^{7} + 9x^{7} - 7x^{7} + 40x^{7} + 6x^{7} + 2x^{7} \\
&=& 36x^{7}
\end{alignat*}
Der Koeffizient des Produktes von $a(x) \cdot b(x)$ für $x^{7}$ lautet $36$.
\subsection{} %c
\begin{alignat*}{2}
a(x) + b(x) &=& 3x^{4} + 4x^{3} + x^{2} + 3x + 5 \\
a(x) \cdot b(x) &=& 2x^{7} + 4x^{5} + 2x^{3} + x^{6} + 2x^{5} + x^{2} + 4x^{5} + 3x^{3} + 4x + x^{4} + 2x^{2} + 1 \\
&=& 2x^{7} + x^{6} + 0x^{5} + x^{4} + 0x^{3} + 3x^{2} + 1 \\
&=& 2x^{7} + x^{6} + x^{4} + 3x^{2} + 1
\end{alignat*}
\section{} %4
\subsection{} %a
\polylongdiv{x^5 + 2x^4 + 3x^3 + x^2 + 4x + 2}{x^2 + 4x +3}

\begin{equation*}
x^{5} + 2x^{4} + 3x^{3} + x^{2} + 4x + 2 = (x^{3} -2x + 9) \cdot (x^{2} + 4x + 3) + (-26x - 25)
\end{equation*}
\subsection{} %b
Der normierte größte gemeinsame Teiler der beiden Polynome ergibt sich wie folgt:\\

\hspace{-2.5cm}
\polylongdiv{6x^5 + 7x^4 - 7x^3 - 22x^2 - 25x - 15}{3x^4 + 2x^3 - 6x^2 - 6x -9}\\
\\
\polylongdiv{3x^4 + 2x^3 - 6x^2 - 6x -9}{3x^3 - 4x^2 - x - 6}\\
\\
\polylongdiv{3x^3 - 4x^2 - x - 6}{3x^2 + 2x + 3}\\
\\

\begin{alignat*}{4}
6x^{5} + 7x^{4} - 7x^{3} - 22x^{2} - 25x - 15 &=& (2x + 1) &\cdot & (3x^{4} + 2x^{3} - 6x^{2} - 6x - 9) &+& (3x^{3} - 4x^{2} - x - 6) \\
3x^{4} + 2x^{3} - 6x^{2} - 6x - 9 &=& (x + 2) &\cdot & (3x^{3} - 4x^{2} - x - 6) &+& (3x^{2} - 2x + 3) \\
3x^{3} - 4x^{2} - x - 6 &=& (x - 2) &\cdot & (3x^{2} - 2x + 3) &+& 0
\end{alignat*}
\\
Der größte normierte Teiler ist demzufolge $x^2 + \frac{2}{3}x + 1$.
\end{document}
