\documentclass[10pt,a4paper,oneside,ngerman,numbers=noenddot]{scrartcl}
\usepackage[T1]{fontenc}
\usepackage[utf8]{inputenc}
\usepackage[ngerman]{babel}
\usepackage{amsmath}
\usepackage{amsfonts}
\usepackage{amssymb}
\usepackage{paralist}
\usepackage[locale=DE,exponent-product=\cdot,detect-all]{siunitx}
\usepackage{tikz}
\usetikzlibrary{matrix,fadings,calc,positioning,decorations.pathreplacing,arrows,decorations.markings}
\pagenumbering{arabic}
\def\thesection{\arabic{section})}
\def\thesubsection{\alph{subsection})}
\def\thesubsubsection{(\arabic{subsubsection})}

\begin{document}
\author{Jim Martens}
\title{Hausaufgaben zum 13./14. Dezember}
\maketitle
\section{} %1
\subsection{} %a
Ja, die Graphen sind isomorph.

In Graph G gibt es vier Knoten mit je vier abgehenden Kanten und vier Knoten mit je drei abgehenden Kanten.

In Graph G' gibt es ebenfalls vier Knoten mit vier abgehenden Kanten und auch vier Knoten mit je drei abgehenden Kanten.

Die Knoten seien von oben links beginnend im Uhrzeigersinn aufsteigend nummeriert. Die äußeren vier Knoten im Graph G haben die Bezeichnungen 1,2,3 und 4. Die inneren vier Knoten (auch oben links beginnend) haben die Bezeichnungen 5,6,7 und 8.

Im Graph G' werden die Knoten in gleicher Weise mit Buchstaben bezeichnet. Somit ergeben sich für die äußeren Knoten a,b,c und d, sowie e,f,g, und h für die inneren Knoten (jeweils oben links beginnend).

Damit gebe es folgende Abbildung $f: G \rightarrow G'$:\\

\begin{alignat*}{2}
f(1) &=& b \\
f(2) &=& a \\
f(3) &=& c \\
f(4) &=& d \\
f(5) &=& f \\
f(6) &=& e \\
f(7) &=& g \\
f(8) &=& h
\end{alignat*}
\subsection{} %b
Alle drei Graphen sind isomorph zueinander.

\section{} %2
\subsection{} %a
G hat $\frac{10 \cdot 9}{2} = 45$ Kanten.
\subsection{} %b
Es gibt $\binom{10}{3} = 120$ Kreise mit der Länge $3$ in G.
\subsection{} %c
Es gibt $\binom{10}{4} = 210$ Kreise mit der Länge $4$ in G.
\subsection{} %d
Es gibt $\binom{10}{4}$ Möglichkeiten vier Knoten aus zehn zu nehmen. Für jede dieser vier Knoten gibt es zwei Möglichkeiten die Diagonale zu wählen. Es ergeben sich daher $\binom{10}{4} \cdot 2 = 210 \cdot 2 = 420$ Möglichkeiten einen solchen Teilgraphen aus G zu nehmen.
\section{} %3
\subsection{} %a
%\begin{figure}[h!]
$n=4$:\\
\begin{tikzpicture}[shorten >=1pt,node distance=1.1cm,on grid]
%h1
\node (v1) {$v_{1}$};
\node (v2) [below=of v1] {$v_{2}$};
\node (v3) [right=of v2] {$v_{3}$};
\node (v4) [right=of v1] {$v_{4}$};

\path[every node/.style={font=\scriptsize}] 
	(v1) edge (v2)
	(v1) edge (v3)
	(v1) edge (v4)
	(v2) edge (v3)
	(v2) edge (v4)
	(v3) edge (v4);
	
%h2
\node (v5) [right=2 of v4] {$v_{5}$} 
edge [bend right] (v1)
edge [bend left=50] (v2)
edge (v4)
edge [bend left] (v3);
\node (v6) [below=of v5] {$v_{6}$} 
edge [bend right=50] (v1)
edge [bend right] (v4)
edge (v3)
edge [bend left] (v2);
\node (v7) [right=of v6] {$v_{7}$} 
edge [bend left=60] (v1)
edge [bend left] (v4)
edge [bend left] (v3)
edge [bend left] (v2);
\node (v8) [right=of v5] {$v_{8}$} 
edge [bend right] (v1)
edge [bend right=60] (v2)
edge [bend right] (v4)
edge [bend right] (v3);

\path[every node/.style={font=\scriptsize}] 
	(v5) edge (v6)
	(v5) edge (v7)
	(v5) edge (v8)
	(v6) edge (v7)
	(v6) edge (v8)
	(v7) edge (v8);
\end{tikzpicture}

$n=6$:\\
\begin{tikzpicture}[shorten >=1pt,node distance=1.1cm,on grid]
%h1
\node (v1) {$v_{1}$};
\node (v2) [below left=of v1] {$v_{2}$};
\node (v3) [below right=of v2] {$v_{3}$};
\node (v4) [right=of v3] {$v_{4}$};
\node (v5) [above right=of v4] {$v_{5}$};
\node (v6) [above left=of v5] {$v_{6}$};

\path[every node/.style={font=\scriptsize}] 
	(v1) edge (v2)
	(v1) edge (v6)
	(v1) edge (v4)
	(v2) edge (v3)
	(v2) edge (v5)
	(v3) edge (v4)
	(v3) edge (v6)
	(v4) edge (v5)
	(v5) edge (v6);
	
%h2
\node (v7) [right=2.5 of v6] {$v_{7}$}
edge (v6)
edge [bend right=10] (v5)
edge (v4)
edge [bend right] (v1)
edge [bend right=70] (v2)
edge [bend right=30] (v3);
\node (v8) [below left=of v7] {$v_{8}$}
edge (v5)
edge [bend right=10] (v6)
edge [bend left=10] (v4)
edge [bend right=87] (v2)
edge [bend right=10] (v1)
edge [bend left=10] (v3);
\node (v9) [below right=of v8] {$v_{9}$}
edge (v4)
edge [bend left=10] (v5)
edge (v6)
edge [bend left] (v3)
edge [bend left=70] (v2)
edge [bend left=30] (v1);
\node (v10) [right=of v9] {$v_{10}$}
edge [bend left] (v3)
edge [bend left] (v4)
edge [bend left] (v5)
edge [bend left=60] (v2)
edge [bend right=15] (v6)
edge [bend left=40] (v1);
\node (v11) [above right=of v10] {$v_{11}$}
edge [bend right=15] (v5)
edge (v6)
edge (v4)
edge [bend right=85] (v2)
edge [bend right=45] (v1)
edge [bend left=45] (v3);
\node (v12) [above left=of v11] {$v_{12}$}
edge [bend right] (v1)
edge [bend right] (v6)
edge [bend right] (v5)
edge [bend right=60] (v2)
edge [bend left=15] (v4)
edge [bend right=40] (v3);

\path[every node/.style={font=\scriptsize}] 
	(v7) edge (v8)
	(v7) edge (v9)
	(v7) edge (v10)
	(v7) edge (v11)
	(v7) edge (v12)
	(v8) edge (v9)
	(v8) edge (v10)
	(v8) edge (v11)
	(v8) edge (v12)
	(v9) edge (v10)
	(v9) edge (v11)
	(v9) edge (v12)
	(v10) edge (v11)
	(v10) edge (v12)
	(v11) edge (v12);
\end{tikzpicture}
%\end{figure}
\subsection{} %b
Es ist zu zeigen, dass $|E(G)| = \frac{3}{2}n^{2} + n$ gilt.\\
In der Zusammenhangskomponente $H_{1}$ hat jeder Knoten den Grad $3$. Bei $n$ Knoten ergibt dies $\frac{3n}{2}$ Kanten in $H_{1}$. In der Zusammenhangskomponente $H_{2}$ gibt es auch $n$ Knoten und jeder Knoten ist mit jedem verbunden. Es ergeben sich also dort $\frac{n \cdot (n-1)}{2}$ Kanten. In $G$ gibt es nun zusätzlich zwischen jedem Knoten aus $H_{1}$ und $H_{2}$ eine Kante, wodurch sich $\frac{n \cdot n}{2}$ Kanten für die Verbindungen ergeben. Zusammengefasst ergibt sich:\\
\begin{alignat*}{2}
|E(G)| &=& \frac{3n + n \cdot (n-1) + n \cdot n}{2} \\
&=& \frac{3n + n^{2} -n + n^{2}}{2} \\
&=& \frac{2n + 2n^{2}}{2} \\
&=& n + n^{2}
\end{alignat*}
Die zu zeigende Formel gilt augenscheinlich nicht.
\subsection{} %c
Für $n=4$ startet man bei $v_{4}$, geht zu $v_{1}$, dann $v_{2}$, $v_{3}$, weiter zu $v_{6}$, $v_{7}$, $v_{8}$, $v_{5}$, dann wieder zurück zu $v_{4}$.
\subsection{} %d
$G$ besitzt keine Eulersche Linie, da der Grad jedes Knotens $2n-1$ beträgt. Für die Existenz einer Eulerschen Linie muss jeder Knoten einen geraden Grad besitzen.
\section{} %4
\subsection{} %a
$P(M)$ enthält $2^{n} = 2^{4} = 16$ Elemente. Es gilt:\\
\begin{alignat*}{2}
P(M) &=& \{\emptyset, \{a\}, \{b\}, \{c\}, \{d\}, \{a,b\}, \{a,c\}, \{a,d\}, \{b,c\}, \{b,d\}, \{c,d\}, \\
&& \{a,b,c\}, \{a,b,d\}, \{a,c,d\}, \{b,c,d\}, \{a,b,c,d\}\}
\end{alignat*}
\subsection{} %b
\begin{tikzpicture}[shorten >=1pt,node distance=1.1cm,on grid]
\node (empty) {$\emptyset$};
\node (a) [above left=1.5 and 2.0 of empty] {$\{a\}$};
\node (b) [right=1.25 of a] {$\{b\}$};
\node (c) [right=1.25 of b] {$\{c\}$};
\node (d) [right=1.25 of c] {$\{d\}$};
\node (ab) [above left=1.5 and 0.5 of a] {$\{a,b\}$};
\node (ac) [right=1.0 of ab] {$\{a,c\}$};
\node (ad) [right=2.0 of ab] {$\{a,d\}$};
\node (bc) [right=3.0 of ab] {$\{b,c\}$};
\node (bd) [right=4.0 of ab] {$\{b,d\}$};
\node (cd) [right=5.0 of ab] {$\{c,d\}$};
\node (abc) [above=3.0 of a] {$\{a,b,c\}$};
\node (abd) [above=3.0 of b] {$\{a,b,d\}$};
\node (acd) [above=3.0 of c] {$\{a,c,d\}$};
\node (bcd) [above=3.0 of d] {$\{b,c,d\}$};
\node (abcd) [above=6.0 of empty] {$\{a,b,c,d\}$};

\path[every node/.style={font=\scriptsize}]
(empty) edge (a)
(empty) edge (b)
(empty) edge (c)
(empty) edge (d)
(a) edge (ab)
(a) edge (ac)
(a) edge (ad)
(b) edge (ab)
(b) edge (bc)
(b) edge (bd)
(c) edge (ac)
(c) edge (bc)
(c) edge (cd)
(d) edge (ad)
(d) edge (bd)
(d) edge (cd)
(ab) edge (abc)
(ac) edge (abc)
(bc) edge (abc)
(ab) edge (abd)
(ad) edge (abd)
(bd) edge (abd)
(ac) edge (acd)
(ad) edge (acd)
(cd) edge (acd)
(bc) edge (bcd)
(bd) edge (bcd)
(cd) edge (bcd)
(abc) edge (abcd)
(abd) edge (abcd)
(acd) edge (abcd)
(bcd) edge (abcd);
\end{tikzpicture}
\subsection{} %c
Solch ein Graph kam auf dem Übungsblatt (Präsenz- und Hausaufgaben) nicht vor. Entweder hat keiner der Graphen $16$ Knoten und/oder die Knoten haben nicht alle den Grad $4$. Von daher kann die Eigenschaft der Isomorphie von vornherein verneint werden.
\end{document}
