\documentclass[10pt,a4paper,oneside,ngerman,numbers=noenddot]{scrartcl}
\usepackage[T1]{fontenc}
\usepackage[utf8]{inputenc}
\usepackage[ngerman]{babel}
\usepackage{amsmath}
\usepackage{amsfonts}
\usepackage{amssymb}
\usepackage{paralist}
\usepackage{gauss}
\usepackage[locale=DE,exponent-product=\cdot,detect-all]{siunitx}
\usepackage{tikz}
\usetikzlibrary{matrix,fadings,calc,positioning,decorations.pathreplacing,arrows,decorations.markings}
\usepackage{polynom}
\polyset{style=C, div=:,vars=x}
\pagenumbering{arabic}
\def\thesection{\arabic{section})}
\def\thesubsection{\alph{subsection})}
\def\thesubsubsection{(\roman{subsubsection})}
\makeatletter
\renewcommand*\env@matrix[1][*\c@MaxMatrixCols c]{%
  \hskip -\arraycolsep
  \let\@ifnextchar\new@ifnextchar
  \array{#1}}
\makeatother

\begin{document}
\author{Jim Martens}
\title{Hausaufgaben zum 24./25. Januar}
\maketitle
\section{} %1
\subsection{} %a
\begin{equation*}
x = 
\begin{bmatrix} 
2 \\ 
-1 \\ 
3 
\end{bmatrix} + t \cdot
\begin{bmatrix}
5 \\
-2 \\
-1
\end{bmatrix}
\end{equation*}
\subsection{} %b
\subsubsection{} %(i)
\begin{equation*}
x =
\begin{bmatrix}
5 \\
1 \\
2
\end{bmatrix} + s \cdot
\begin{bmatrix}
-8 \\
0 \\
2
\end{bmatrix} + t \cdot
\begin{bmatrix}
-3 \\
-2 \\
1
\end{bmatrix}
\end{equation*}
\subsubsection{} %(ii)
\begin{alignat*}{2}
\underset{0D}{\rightarrow} &=&
\underset{0A}{\rightarrow} + \underset{AB}{\rightarrow} + \underset{AC}{\rightarrow} \\
&=& 
\begin{bmatrix}
-6 \\
-3 \\
5
\end{bmatrix} \\
x &=&
\begin{bmatrix}
-6 \\
-3 \\
5
\end{bmatrix} + s \cdot
\begin{bmatrix}
3 \\
4 \\
-1
\end{bmatrix} + t \cdot
\begin{bmatrix}
8 \\
2 \\
-2
\end{bmatrix}
\end{alignat*}
\subsection{} %c
\begin{alignat*}{3}
&& \begin{bmatrix}
x_{1} \\
x_{2} \\
x_{3}
\end{bmatrix} &=& 
\begin{bmatrix}
5 \\
1 \\
2
\end{bmatrix} + s \cdot
\begin{bmatrix}
-8 \\
0 \\
2
\end{bmatrix} + t \cdot
\begin{bmatrix}
-3 \\
-2 \\
1
\end{bmatrix} \\
\Rightarrow & I & x_{1} &=& 5 - 8s - 3t \\
& II & x_{2} &=& 1 - 2t \\
& III & x_{3} &=& 2 + 2s + t \\
\overset{II}{\Rightarrow} && x_{2} &=& 1 - 2t & \;| -1 \\
&& x_{2} - 1 &=& -2t & \;| \cdot -\frac{1}{2} \\
&& -\frac{1}{2}x_{2} + \frac{1}{2} &=& t & \\
\overset{III}{\Rightarrow} && x_{3} &=& 2 + 2s + t & \\
&& x_{3} &=& 2 + 2s + (-\frac{1}{2}x_{2} + \frac{1}{2}) & \;| \text{Kl. aufl. + Zus.} \\
&& x_{3} &=& 2s - \frac{1}{2}x_{2} + \frac{5}{2} & \;| + \frac{1}{2}x_{2}, - \frac{5}{2} \\
&& x_{3} + \frac{1}{2}x_{2} - \frac{5}{2} &=& 2s & \;| \cdot \frac{1}{2} \\
&& \frac{1}{2}x_{3} + \frac{1}{4}x_{2} - \frac{5}{4} &=& s & \\
\overset{I}{\Rightarrow} && x_{1} &=& 5 - 8s - 3t & \\
&& x_{1} &=& 5 - 8(\frac{1}{2}x_{3} + \frac{1}{4}x_{2} - \frac{5}{4}) - 3(-\frac{1}{2}x_{2} + \frac{1}{2}) & \;| \text{Kl. aufl.} \\
&& x_{1} &=& 5 - 4x_{3} - 2x_{2} + 10 + \frac{3}{2}x_{2} - \frac{3}{2} & \;| \text{Zus.} \\
&& x_{1} &=& - 4x_{3} - \frac{1}{2}x_{2} + \frac{27}{2} & \;| +4x_{3} + \frac{1}{2}x_{2} - \frac{27}{2} \\
&& x_{1} + \frac{1}{2}x_{2} + 4x_{3} - \frac{27}{2} &=& 0 &
\end{alignat*}
\\
Schnittpunkt mit der $x_{1}$-Achse:\\
\begin{alignat*}{2}
x_{1} + \frac{1}{2}x_{2} + 4x_{3} - \frac{27}{2} &=& 0 & \;| x_{2} = 0, x_{3} = 0 \\
x_{1} - \frac{27}{2} &=& 0 & \;| + \frac{27}{2} \\
x_{1} &=& \frac{27}{2} &
\end{alignat*}
\\
Schnittpunkt mit der $x_{2}$-Achse:\\
\begin{alignat*}{2}
x_{1} + \frac{1}{2}x_{2} + 4x_{3} - \frac{27}{2} &=& 0 & \;| x_{1} = 0, x_{3} = 0 \\
\frac{1}{2}x_{2} - \frac{27}{2} &=& 0 & \;| + \frac{27}{2} \\
\frac{1}{2}x_{2} &=& \frac{27}{2} & \;| \cdot 2 \\
x_{2} &=& 27 &
\end{alignat*}
\\
Schnittpunkt mit der $x_{3}$-Achse:\\
\begin{alignat*}{2}
x_{1} + \frac{1}{2}x_{2} + 4x_{3} - \frac{27}{2} &=& 0 & \;| x_{1} = 0, x_{2} = 0 \\
4x_{3} - \frac{27}{2} &=& 0 & \;| + \frac{27}{2} \\
4x_{3} &=& \frac{27}{2} & \;| \cdot \frac{1}{4} \\
x_{3} &=& \frac{27}{8} &
\end{alignat*}
\section{} %2
\begin{alignat*}{3}
3x_{1} - 2x_{2} + 8x_{3} - 10 &=& 0 & \;| + 2x_{2} \\
2x_{2} &=& -10 + 3x_{1} + 8x_{3} & \;| \cdot \frac{1}{2} \\
x_{2} &=& -5 + \frac{3}{2}x_{1} + 4x_{3} &
\end{alignat*}
\begin{alignat*}{5}
\Rightarrow & x_{1} &=&& 0 &\,+\,& 1x_{1} &\,+\,& 0x_{3} \\
& x_{2} &=&& -5 &\,+\,& \frac{3}{2}x_{1} &\,+\,& 4x_{3}\\
& x_{3} &=&& 0 &\,+\,& 0x_{1} &\,+\,& 1x_{3}
\end{alignat*}
Daraus ergibt sich:\\
\begin{alignat*}{2}
\begin{bmatrix}
x_{1} \\
x_{2} \\
x_{3}
\end{bmatrix} &=& 
\begin{bmatrix}
0 \\
-5 \\
0
\end{bmatrix} + x_{1}
\begin{bmatrix}
1 \\
\frac{3}{2} \\
0
\end{bmatrix} + x_{3}
\begin{bmatrix}
0 \\
4 \\
1
\end{bmatrix} \\
\intertext{$x_{1} = s, x_{3} = t$}
x &=& 
\begin{bmatrix}
0 \\
-5 \\
0
\end{bmatrix} + s
\begin{bmatrix}
1 \\
\frac{3}{2} \\
0
\end{bmatrix} + t
\begin{bmatrix}
0 \\
4 \\
1
\end{bmatrix}
\end{alignat*}
\section{} %3
\subsection{} %a
\begin{alignat*}{3}
& a \cdot b &=& (2 \cdot 3) + (-1 \cdot 2) + (5 \cdot (-3)) & \;|\text{Kl. aufl.} \\
& &=& 6 - 2 - 15 & \;|\text{Zus.} \\
& &=& -11 & \\
\intertext{Einsetzen von $z$}
& a \cdot b &=& (2 \cdot 3) + (-1 \cdot 2) + (5 \cdot z) & \;|\text{Kl. aufl.} \\
& &=& 6 - 2 + 5z & \;|\text{Zus.} \\
& &=& 4 + 5z & \\
\Rightarrow & 0 &=& 4 + 5z & \;| -4 \\
& 5z &=& -4 & \;| \cdot \frac{1}{5} \\
& z &=& -\frac{4}{5} &
\end{alignat*}
$z$ ist $-\frac{4}{5}$.
\subsection{} %b
\begin{alignat*}{2}
|a| &=& \sqrt{2^{2} + (-1)^{2} + 5^{2}} \\
&=& \sqrt{4 + 1 + 25} \\
&=& \sqrt{30} \\
&\approx & 5,48
\end{alignat*}
\\
Errechnung von d:\\
\begin{alignat*}{2}
& \begin{bmatrix}
d_{1} \\
d_{2} \\
d_{3}
\end{bmatrix} &=& \frac{1}{\sqrt{30}} \cdot
\begin{bmatrix}
2 \\
-1 \\
5
\end{bmatrix} \\
\Rightarrow I: & d_{1} &=& \frac{2}{\sqrt{30}} \\
II: & d_{2} &=& -\frac{1}{\sqrt{30}} \\
III: & d_{3} &=& \frac{5}{\sqrt{30}}
\end{alignat*}
\subsection{} %c
\begin{alignat*}{2}
|P_{1}P_{2}| &=& \sqrt{(0-4)^{2} + (3-2)^{2} + (1-1)^{2}} \\
&=& \sqrt{-4^{2} + {1}^{2}} \\
&=& \sqrt{16 + 1} \\
&=& \sqrt{17} \\
&\approx & 4,12
\end{alignat*}
\subsection{} %d
\begin{alignat*}{2}
cos \phi &=& \frac{u \cdot v}{|u| \cdot |v|} \\
&=& \frac{2}{4 \cdot \sqrt{5}} \\
&=& \frac{1}{2 \cdot \sqrt{5}} \\
&\approx & 0,22 \\
arccos(0,22) &\approx &  77,29^{\circ}
\end{alignat*}
\section{} %4
\subsection{} %a
Es gilt: \\
\begin{alignat*}{2}
x \cdot u &=& 0 \\
x \cdot v &=& 0 \\
\begin{bmatrix}
x_{1} \\
x_{2} \\
x_{3}
\end{bmatrix} \cdot 
\begin{bmatrix}
1 \\
2 \\
3
\end{bmatrix} &=& 0 \\
\begin{bmatrix}
x_{1} \\
x_{2} \\
x_{3}
\end{bmatrix} \cdot 
\begin{bmatrix}
-4 \\
-7 \\
5
\end{bmatrix} &=& 0 \\
\intertext{In LGS umformen}
I: 1x_{1} + 2x_{2} + 3x_{3} &=& 0 \\
II: -4x_{1} - 7x_{2} + 5x_{3} &=& 0 \\
\intertext{II = II + 4I}
1x_{1} + 2x_{2} + 3x_{3} &=& 0 \\
0x_{1} + 1x_{2} + 17x_{3} &=& 0 \\
\intertext{II = II - $17x_{3}$}
x_{2} &=& -17x_{3} \\
\intertext{In I einsetzen}
1x_{1} + 2 \cdot (-17x_{3}) + 3x_{3} &=& 0 \\
x_{1} - 31x_{3} &=& 0 \\
\intertext{I = I + $31x_{3}$}
x_{1} &=& 31x_{3} \\
\intertext{Sei $x_{3} = t, t \in \mathbb{R}$}
x_{1} &=& 31t \\
\intertext{Daraus ergibt sich für $x_{2}$}
x_{2} &=& -17t \\
\intertext{Daraus folgt für $x$}
x &=& \begin{bmatrix}
31t \\
-17t \\
t
\end{bmatrix}, \; t \in \mathbb{R} \\
&=& t \cdot \begin{bmatrix}
31 \\
-17 \\
1
\end{bmatrix}, \; t \in \mathbb{R}
\end{alignat*}
\subsection{} %b
Es gilt:\\
\begin{alignat*}{2}
x \cdot u &=& 0 \\
\begin{bmatrix}
x_{1} \\
x_{2} \\
x_{3}
\end{bmatrix} \cdot
\begin{bmatrix}
2 \\
4 \\
1
\end{bmatrix} &=& 0 \\
\intertext{In LGS umwandeln}
2x_{1} + 4x_{2} + x_{3} &=& 0 \\
\intertext{Umstellen nach $x_{3}$}
x_{3} &=& -2x_{1} - 4x_{2} \\
\intertext{Sei $x_{1} = s, s \in \mathbb{R}$ und $x_{2} = t, t \in \mathbb{R}$}
x_{3} &=& -2s - 4t \\
\intertext{Daraus ergibt sich für $x$}
x &=& \begin{bmatrix}
s \\
t \\
-2s - 4t
\end{bmatrix}, \; s,t \in \mathbb{R} \\
&=& s \cdot \begin{bmatrix}
1 \\
0 \\
-2
\end{bmatrix} + t \cdot 
\begin{bmatrix}
0 \\
1 \\
-4
\end{bmatrix}, \; s,t \in \mathbb{R}
\end{alignat*}
\subsection{} %c
Es gilt: \\
\begin{alignat*}{2}
x \cdot u &=& 0 \\
x \cdot v &=& 0 \\
\begin{bmatrix}
x_{1} \\
x_{2} \\
x_{3} \\
x_{4}
\end{bmatrix} \cdot 
\begin{bmatrix}
1 \\
2 \\
-1 \\
2
\end{bmatrix} &=& 0 \\
\begin{bmatrix}
x_{1} \\
x_{2} \\
x_{3} \\
x_{4}
\end{bmatrix} \cdot 
\begin{bmatrix}
1 \\
1 \\
4 \\
2
\end{bmatrix} &=& 0 \\
\intertext{In LGS umformen}
I: 1x_{1} + 2x_{2} - 1x_{3} + 2x_{4} &=& 0 \\
II: 1x_{1} + 1x_{2} + 4x_{3} + 2x_{4} &=& 0 \\
\intertext{II = II - I}
1x_{1} + 2x_{2} - 1x_{3} + 2x_{4} &=& 0 \\
0x_{1} - 1x_{2} + 5x_{3} + 0x_{4} &=& 0 \\
\intertext{II = II + $x_{2}$}
x_{2} &=& 5x_{3} \\
\intertext{In I einsetzen}
x_{1} + 2 \cdot (5x_{3}) - x_{3} + 2x_{4} &=& 0 \\
x_{1} + 9x_{3} + 2x_{4} &=& 0 \\
\intertext{Nach $x_{1}$ umstellen}
x_{1} &=& -9x_{3} - 2x_{4} \\
\intertext{Sei $x_{3} = s, s \in \mathbb{R}$ und $x_{4} = t, t \in \mathbb{R}$}
x_{1} &=& -9s - 2t \\
\intertext{Daraus ergibt sich für $x_{2}$}
x_{2} &=& 5s \\
\intertext{Daraus folgt für $x$}
x &=& \begin{bmatrix}
-9s - 2t \\
5s \\
s \\
t
\end{bmatrix}, \; s,t \in \mathbb{R} \\
&=& s \cdot \begin{bmatrix}
-9 \\
5 \\
1 \\
0
\end{bmatrix} + t \cdot 
\begin{bmatrix}
-2 \\
0 \\
0 \\
1
\end{bmatrix}, \; s,t \in \mathbb{R}
\end{alignat*}
\subsection{} %d
\begin{alignat*}{2}
|u| &=& \sqrt{1^{2} + 2^{2} + (-1)^{2} + 2^{2}} \\
&=& \sqrt{1 + 4 + 1 + 4} \\
&=& \sqrt{10} \\
&\approx & 3,16 \\
|v| &=& \sqrt{1^{2} + 1^{2} + 4^{2} + 2^{2}} \\
&=& \sqrt{1 + 1 + 16 + 4} \\
&=& \sqrt{22} \\
&\approx & 4,69
\end{alignat*}
\end{document}
