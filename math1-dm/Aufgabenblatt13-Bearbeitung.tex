\documentclass[10pt,a4paper,oneside,ngerman,numbers=noenddot]{scrartcl}
\usepackage[T1]{fontenc}
\usepackage[utf8]{inputenc}
\usepackage[ngerman]{babel}
\usepackage{amsmath}
\usepackage{amsfonts}
\usepackage{amssymb}
\usepackage{paralist}
\usepackage{gauss}
\usepackage[locale=DE,exponent-product=\cdot,detect-all]{siunitx}
\usepackage{tikz}
\usetikzlibrary{matrix,fadings,calc,positioning,decorations.pathreplacing,arrows,decorations.markings}
\usepackage{polynom}
\polyset{style=C, div=:,vars=x}
\pagenumbering{arabic}
\def\thesection{\arabic{section})}
\def\thesubsection{\alph{subsection})}
\def\thesubsubsection{(\roman{subsubsection})}
\makeatletter
\renewcommand*\env@matrix[1][*\c@MaxMatrixCols c]{%
  \hskip -\arraycolsep
  \let\@ifnextchar\new@ifnextchar
  \array{#1}}
\makeatother

\begin{document}
\author{Jim Martens}
\title{Hausaufgaben zum 31. Januar/1. Februar}
\maketitle
\section{} %1
\subsection{} %a
In ein LGS umgeformt, ergibt sich dieses Bild:\\
\begin{alignat*}{3}
x_{1} &=& \frac{5}{3} &-& \frac{2}{3}t \\
x_{2} &=& - \frac{4}{3} &+& \frac{1}{3}t \\
x_{3} &=& 0 &+& 1t \\
\intertext{Daraus ergeben sich diese Vektoren}
\begin{bmatrix}
x_{1} \\
x_{2} \\
x_{3}
\end{bmatrix} &=& 
\begin{bmatrix}
\frac{5}{3} \\
-\frac{4}{3} \\
0
\end{bmatrix} &+&
\begin{bmatrix}
-\frac{2}{3}t \\
\frac{1}{3}t \\
1t
\end{bmatrix} \\
\intertext{t vor die Matrix ziehen}
&=& \begin{bmatrix}
\frac{5}{3} \\
-\frac{4}{3} \\
0
\end{bmatrix} &+& t \cdot 
\begin{bmatrix}
-\frac{2}{3} \\
\frac{1}{3} \\
1
\end{bmatrix}
\end{alignat*}
\subsection{} %b
In ein LGS umgeformt, ergibt sich dieses Bild:\\
\begin{alignat*}{4}
x_{1} &=& 2 &-& \frac{1}{2}s &-& \frac{1}{2}t \\
x_{2} &=& 0 &+& 1s &+& 0t \\
x_{3} &=& 0 &+& 0s &+& 1t \\
\intertext{Daraus ergeben sich diese Vektoren}
\begin{bmatrix}
x_{1} \\
x_{2} \\
x_{3}
\end{bmatrix} &=& 
\begin{bmatrix}
2 \\
0 \\
0
\end{bmatrix} &+&
\begin{bmatrix}
\frac{1}{2}s \\
1s \\
0
\end{bmatrix} &+&
\begin{bmatrix}
-\frac{1}{2}t \\
0 \\
1t
\end{bmatrix} \\
\intertext{Koeffizienten vor die Matrix ziehen}
&=& \begin{bmatrix}
2 \\
0 \\
0
\end{bmatrix} &+& s \cdot
\begin{bmatrix}
\frac{1}{2} \\
1 \\
0
\end{bmatrix} &+& t \cdot 
\begin{bmatrix}
-\frac{1}{2} \\
0 \\
1
\end{bmatrix}
\end{alignat*}
\subsection{} %c
In ein LGS umgeformt, ergibt sich dieses Bild:\\
\begin{alignat*}{5}
x_{1} &=& -3 &+& 5r &+& 3s &+& t \\
x_{2} &=& 1 &-& 2r &-& 3s &+& 0 \\
x_{3} &=& 0 &+& r &+& 0 &+& 0 \\
x_{4} &=& 0 &+& 0 &+& s &+& 0 \\
x_{5} &=& -2 &+& 0 &+& 0 &+& 3t \\
x_{6} &=& 0 &+& 0 &+& 0 &+& t \\
\intertext{Umgewandelt in Vektoren ergibt sich dies:}
\begin{bmatrix}
x_{1} \\
x_{2} \\
x_{3} \\
x_{4} \\
x_{5} \\
x_{6}
\end{bmatrix} &=& 
\begin{bmatrix}
-3 \\
1 \\
0 \\
0 \\
-2 \\
0
\end{bmatrix} &+&
\begin{bmatrix}
5r \\
-2r \\
r \\
0 \\
0 \\
0
\end{bmatrix} &+& 
\begin{bmatrix}
3s \\
-3s \\
0 \\
s \\
0 \\
0
\end{bmatrix} &+&
\begin{bmatrix}
t \\
0 \\
0 \\
0 \\
3t \\
t
\end{bmatrix} \\
\intertext{Koeffizienten vor Vektor ziehen}
&=& 
\begin{bmatrix}
-3 \\
1 \\
0 \\
0 \\
-2 \\
0
\end{bmatrix} &+& r \cdot
\begin{bmatrix}
5 \\
-2 \\
1 \\
0 \\
0 \\
0
\end{bmatrix} &+& s \cdot 
\begin{bmatrix}
3 \\
-3 \\
0 \\
1 \\
0 \\
0
\end{bmatrix} &+& t \cdot 
\begin{bmatrix}
1 \\
0 \\
0 \\
0 \\
3 \\
1
\end{bmatrix}
\end{alignat*}
\section{} %2
\subsection{} %a
\begin{alignat*}{2}
u_{1} &=& \begin{bmatrix}
0 \\
0 \\
0 \\
0
\end{bmatrix} \in U \\
u_{2} &=& \begin{bmatrix}
1 \\
1 \\
1 \\
1
\end{bmatrix} \in U \\
u_{3} &=& \begin{bmatrix}
2 \\
2 \\
2 \\
2
\end{bmatrix} \not\in U \\
u_{4} &=& \begin{bmatrix}
1 \\
-1 \\
1 \\
1
\end{bmatrix} \not\in U
\end{alignat*}
U ist kein Unterraum von $V = \mathbb{R}^{4}$, da $u_{1} + u_{1} = u_{2} \not\in U$. Demnach ist $U$ nicht abgeschlossen bezüglich der Addition.
\subsection{} %b
\begin{alignat*}{2}
u_{1} &=& \begin{bmatrix}
0 \\
0 \\
0 \\
0
\end{bmatrix} \in U \\
u_{2} &=& \begin{bmatrix}
1 \\
1 \\
1 \\
1
\end{bmatrix} \in U \\
u_{3} &=& \begin{bmatrix}
2 \\
0 \\
0 \\
1
\end{bmatrix} \not\in U \\
u_{4} &=& \begin{bmatrix}
0 \\
0 \\
0 \\
-1
\end{bmatrix} \not\in U
\end{alignat*}
1. Der Nullvektor ist in $U$ enthalten. \\
2. $u,v \in U \; c,d \in \mathbb{R}^{\geq 0}$
\begin{alignat*}{2}
u &=& \begin{bmatrix}
u_{1} \\
u_{2} \\
u_{3} \\
u_{4}
\end{bmatrix} = 
\begin{bmatrix}
u_{1} \\
u_{2} \\
u_{3} \\
u_{1} + c
\end{bmatrix} \\
v &=& \begin{bmatrix}
v_{1} \\
v_{2} \\
v_{3} \\
v_{4}
\end{bmatrix} =
\begin{bmatrix}
v_{1} \\
v_{2} \\
v_{3} \\
v_{1} + d
\end{bmatrix} \\
u + v &=& \begin{bmatrix}
u_{1} + v_{1} \\
u_{2} + v_{2} \\
u_{3} + v_{3} \\
(u_{1} + c) + (v_{1} + d)
\end{bmatrix} \\
&=& \begin{bmatrix}
u_{1} + v_{1} \\
u_{2} + v_{2} \\
u_{3} + v_{3} \\
u_{1} + v_{1} + c + d
\end{bmatrix}
\end{alignat*}
Wenn zwei Vektoren $u,v$ in $U$ enthalten sind, dann ist es auch ihre Summe.\\
3. $u \in U, t \in \mathbb{R}, c \in \mathbb{R}^{\geq 0}$
\begin{alignat*}{2}
u &=& \begin{bmatrix}
u_{1} \\
u_{2} \\
u_{3} \\
u_{4}
\end{bmatrix} =
\begin{bmatrix}
u_{1} \\
u_{2} \\
u_{3} \\
u_{1} + c
\end{bmatrix} \\
t \cdot u &=& t \cdot \begin{bmatrix}
u_{1} \\
u_{2} \\
u_{3} \\
u_{1} + c
\end{bmatrix} \\
&=& \begin{bmatrix}
tu_{1} \\
tu_{2} \\
tu_{3} \\
t(u_{1} + c)
\end{bmatrix} \\
&=& \begin{bmatrix}
tu_{1} \\
tu_{2} \\
tu_{3} \\
tu_{1} + tc
\end{bmatrix}
\end{alignat*}
Wenn ein Vektor in $U$ ist, dann ist es auch das Produkt mit einem beliebigen Skalar.

Daher ist $U$ ein Unterraum von $V = \mathbb{R}^{4}$.\hfill $\Box$

\subsection{} %c
\begin{alignat*}{2}
u_{1} &=& \begin{bmatrix}
0 \\
0 \\
0 \\
0
\end{bmatrix} \in U \\
u_{2} &=& \begin{bmatrix}
1 \\
1 \\
1 \\
4
\end{bmatrix} \in U \\
u_{3} &=& \begin{bmatrix}
2 \\
0 \\
0 \\
1
\end{bmatrix} \not\in U \\
u_{4} &=& \begin{bmatrix}
0 \\
0 \\
0 \\
-1
\end{bmatrix} \not\in U
\end{alignat*}
1. Der Nullvektor ist in $U$ enthalten. \\
2. $u,v \in U$
\begin{alignat*}{2}
u &=& \begin{bmatrix}
u_{1} \\
u_{2} \\
u_{3} \\
u_{4}
\end{bmatrix} = 
\begin{bmatrix}
u_{1} \\
u_{2} \\
u_{3} \\
2u_{1} + u_{2} + u_{3}
\end{bmatrix} \\
v &=& \begin{bmatrix}
v_{1} \\
v_{2} \\
v_{3} \\
v_{4}
\end{bmatrix} =
\begin{bmatrix}
v_{1} \\
v_{2} \\
v_{3} \\
2v_{1} + v_{2} + v_{3}
\end{bmatrix} \\
u + v &=& \begin{bmatrix}
u_{1} + v_{1} \\
u_{2} + v_{2} \\
u_{3} + v_{3} \\
(2u_{1} + u_{2} + u_{3}) + (2v_{1} + v_{2} + v_{3})
\end{bmatrix} \\
&=& \begin{bmatrix}
u_{1} + v_{1} \\
u_{2} + v_{2} \\
u_{3} + v_{3} \\
2u_{1} + 2v_{1} + u_{2} + v_{2} + u_{3} + v_{3}
\end{bmatrix} \\
&=& \begin{bmatrix}
u_{1} + v_{1} \\
u_{2} + v_{2} \\
u_{3} + v_{3} \\
2(u_{1} + v_{1}) + (u_{2} + v_{2}) + (u_{3} + v_{3})
\end{bmatrix}
\end{alignat*}
Wenn zwei Vektoren in $U$ sind, dann ist es auch ihre Summe.\\
3. $u \in U, t \in \mathbb{R}$
\begin{alignat*}{2}
u &=& \begin{bmatrix}
u_{1} \\
u_{2} \\
u_{3} \\
u_{4}
\end{bmatrix} =
\begin{bmatrix}
u_{1} \\
u_{2} \\
u_{3} \\
2u_{1} + u_{2} + u_{3}
\end{bmatrix} \\
t \cdot u &=& t \cdot \begin{bmatrix}
u_{1} \\
u_{2} \\
u_{3} \\
2u_{1} + u_{2} + u_{3}
\end{bmatrix} \\
&=& \begin{bmatrix}
tu_{1} \\
tu_{2} \\
tu_{3} \\
t(2u_{1} + u_{2} + u_{3})
\end{bmatrix} \\
&=& \begin{bmatrix}
tu_{1} \\
tu_{2} \\
tu_{3} \\
2tu_{1} + tu_{2} + tu_{3}
\end{bmatrix}
\end{alignat*}
Wenn ein Vektor in $U$ ist, dann ist es auch das Produkt mit einem beliebigen Skalar.

Daher ist $U$ ein Unterraum von $V = \mathbb{R}^{4}$.\hfill $\Box$
\subsection{} %d
\begin{alignat*}{2}
u_{1} &=& \begin{bmatrix}
0 \\
-1 \\
0 \\
0
\end{bmatrix} \in U \\
u_{2} &=& \begin{bmatrix}
1 \\
-1 \\
-2 \\
0
\end{bmatrix} \in U \\
u_{3} &=& \begin{bmatrix}
0 \\
0 \\
0 \\
0
\end{bmatrix} \not\in U \\
u_{4} &=& \begin{bmatrix}
0 \\
0 \\
0 \\
-1
\end{bmatrix} \not\in U
\end{alignat*}
$U$ ist kein Unterraum, da der Nullvektor nicht enthalten ist.\hfill $\Box$
\section{} %3
\subsection{} %a
In einem LGS ergibt sich: \\
\begin{alignat*}{2}
c_{1} + 3c_{2} - c_{3} + 6c_{4} &=& 0 \\
c_{1} - c_{2} + 3c_{3} + 2c_{4} &=& 0 \\
2c_{1} + 0c_{2} + c_{3} + 0c_{4} &=& 0 \\
3c_{1} + c_{2} + 0c_{3} + 4c_{4} &=& 0 \\
\intertext{In Matrixform bringen}
\begin{bmatrix}[cccc|c]
1 & 3 & -1 & 6 & 0 \\
1 & -1 & 3 & 2 & 0 \\
2 & 0 & 1 & 0 & 0 \\
3 & 1 & 0 & 4 & 0
\end{bmatrix} \\
\intertext{II = II - I, III = III - 2I, IV = IV - 3I}
\begin{bmatrix}[cccc|c]
1 & 3 & -1 & 6 & 0 \\
0 & -4 & 4 & -4 & 0 \\
0 & -6 & 3 & -12 & 0 \\
0 & -8 & 3 & -14 & 0
\end{bmatrix} \\
\intertext{II = II$ \cdot -\frac{1}{4}$}
\begin{bmatrix}[cccc|c]
1 & 3 & -1 & 6 & 0 \\
0 & 1 & -1 & 1 & 0 \\
0 & -6 & 3 & -12 & 0 \\
0 & -8 & 3 & -14 & 0
\end{bmatrix} \\
\intertext{III = III + 6II, IV = IV + 8II}
\begin{bmatrix}[cccc|c]
1 & 3 & -1 & 6 & 0 \\
0 & 1 & -1 & 1 & 0 \\
0 & 0 & -3 & -6 & 0 \\
0 & 0 & -3 & -8 & 0
\end{bmatrix} \\
\intertext{III = III$ \cdot -\frac{1}{3}$}
\begin{bmatrix}[cccc|c]
1 & 3 & -1 & 6 & 0 \\
0 & 1 & -1 & 1 & 0 \\
0 & 0 & 1 & 2 & 0 \\
0 & 0 & -3 & -8 & 0
\end{bmatrix} \\
\intertext{IV = IV + 3III}
\begin{bmatrix}[cccc|c]
1 & 3 & -1 & 6 & 0 \\
0 & 1 & -1 & 1 & 0 \\
0 & 0 & 1 & 2 & 0 \\
0 & 0 & 0 & -2 & 0
\end{bmatrix} \\
\intertext{IV = IV$ \cdot -\frac{1}{2}$}
\begin{bmatrix}[cccc|c]
1 & 3 & -1 & 6 & 0 \\
0 & 1 & -1 & 1 & 0 \\
0 & 0 & 1 & 2 & 0 \\
0 & 0 & 0 & 1 & 0
\end{bmatrix} \\
\overset{IV}{\Rightarrow} c_{4} &=& 0 \\
\overset{III}{\Rightarrow} c_{3} + 2c_{4} &=& 0 \\
c_{3} + 2 \cdot 0 &=& 0 \\
c_{3} &=& 0 \\
\overset{II}{\Rightarrow} c_{2} - c_{3} + c_{4} &=& 0 \\
c_{2} - 0 + 0 &=& 0 \\
c_{2} &=& 0 \\
\overset{I}{\Rightarrow} c_{1} + 3_c{2} - c_{3} + 6c_{4} &=& 0 \\
c_{1} + 3 \cdot 0 - 0 + 6 \cdot 0 &=& 0 \\
c_{1} &=& 0
\end{alignat*}
Es gibt keine $c_{1}, c_{2}, c_{3}, c_{4}$ für die $c_{1}v_{1} + c_{2}v_{2} + c_{3}v_{3} + c_{4}v_{4} = 0$ gilt und mindestens ein $c_{i}$ ungleich Null ist.
\subsection{} %b
In einem LGS ergibt sich: \\
\begin{alignat*}{2}
c_{1} + 3c_{2} + c_{3} - 5c_{4} &=& 0 \\
c_{1} - c_{2} + 3c_{3} - c_{4} &=& 0 \\
2c_{1} + 0c_{2} - c_{3} + 8c_{4} &=& 0 \\
3c_{1} + c_{2} + 0c_{3} + 7c_{4} &=& 0 \\
\intertext{In Matrixform bringen}
\begin{bmatrix}[cccc|c]
1 & 3 & 1 & -5 & 0 \\
1 & -1 & 3 & -1 & 0 \\
2 & 0 & -1 & 8 & 0 \\
3 & 1 & 0 & 7 & 0
\end{bmatrix} \\
\intertext{II = II - I, III = III - 2I, IV = IV - 3I}
\begin{bmatrix}[cccc|c]
1 & 3 & 1 & -5 & 0 \\
0 & -4 & 2 & +4 & 0 \\
0 & -6 & -3 & 18 & 0 \\
0 & -8 & -3 & 22 & 0
\end{bmatrix} \\
\intertext{II = II$ \cdot -\frac{1}{4}$}
\begin{bmatrix}[cccc|c]
1 & 3 & 1 & -5 & 0 \\
0 & 1 & -\frac{1}{2} & -1 & 0 \\
0 & -6 & -3 & 18 & 0 \\
0 & -8 & -3 & 22 & 0
\end{bmatrix} \\
\intertext{III = III + 6II, IV = IV + 8II}
\begin{bmatrix}[cccc|c]
1 & 3 & 1 & -5 & 0 \\
0 & 1 & -\frac{1}{2} & -1 & 0 \\
0 & 0 & -6 & 12 & 0 \\
0 & 0 & -7 & 14 & 0
\end{bmatrix} \\
\intertext{III = III$ \cdot -\frac{1}{6}$}
\begin{bmatrix}[cccc|c]
1 & 3 & 1 & -5 & 0 \\
0 & 1 & -\frac{1}{2} & -1 & 0 \\
0 & 0 & 1 & -2 & 0 \\
0 & 0 & -7 & 14 & 0
\end{bmatrix} \\
\intertext{IV = IV + 7III}
\begin{bmatrix}[cccc|c]
1 & 3 & 1 & -5 & 0 \\
0 & 1 & -\frac{1}{2} & -1 & 0 \\
0 & 0 & 1 & -2 & 0 \\
0 & 0 & 0 & 0 & 0
\end{bmatrix} \\
\overset{IV}{\Rightarrow} 0c_{4} &=& 0 \\
c_{4} &=& t, t \in \mathbb{R} \\
\overset{III}{\Rightarrow} c_{3} - 2c_{4} &=& 0 \\
c_{3} - 2t &=& 0 \\
c_{3} &=& 2t \\
\overset{II}{\Rightarrow} c_{2} - \frac{1}{2}c_{3} - c_{4} &=& 0 \\
c_{2} - \frac{1}{2}(2t) - t &=& 0 \\
c_{2} - 2t &=& 0 \\
c_{2} &=& 2t \\
\overset{I}{\Rightarrow} c_{1} + 3_c{2} + c_{3} - 5c_{4} &=& 0 \\
c_{1} + 3 \cdot 2t + 2t - 5 \cdot t &=& 0 \\
c_{1} + 6t + 2t - 5t &=& 0 \\
c_{1} + 3t &=& 0 \\
c_{1} &=& -3t
\end{alignat*}
Es gibt unendlich viele $c_{1}, c_{2}, c_{3}, c_{4}$ für die $c_{1}v_{1} + c_{2}v_{2} + c_{3}v_{3} + c_{4}v_{4} = 0$ gilt und mindestens ein $c_{i}$ ungleich Null ist.
\subsection{} %c
Wenn es mehr als die triviale Lösung für das lineare Gleichungssystem $c_{1}v_{1} + c_{2}v_{2} + c_{3}v_{3} + c_{4}v_{4} = 0$ gibt, dann sind die Vektoren linear voneinander abhängig.
Gibt es nur die triviale Lösung, dann sind sie linear voneinander unabhängig.
\section{} %4
\subsection{} %a
Wenn die Vektoren $v_{1}, v_{2}, v_{3}$ linear unabhängig sind, dann gibt es nur die triviale Lösung für die Gleichung $c_{1}v_{1} + c_{2}v_{2} + c_{3}v_{3} = 0$.

In ein LGS überführt, ergibt sich dies:\\
\begin{alignat*}{2}
2c_{1} - 2c_{2} - 5c_{3} &=& 0 \\
4c_{1} + 3c_{2} + 18c_{3} &=& 0 \\
-2c_{1} + 3c_{2} + 9c_{3} &=& 0 \\
-4c_{1} + 3c_{2} + 6c_{3} &=& 0 \\
\intertext{In Matrixform ergibt sich dies}
\begin{bmatrix}[ccc|c]
2 & -2 & -5 & 0 \\
4 & 3 & 18 & 0 \\
-2 & 3 & 9 & 0 \\
-4 & 3 & 6 & 0
\end{bmatrix} \\
\intertext{I = I$ \cdot \frac{1}{2}$}
\begin{bmatrix}[ccc|c]
1 & -1 & -\frac{5}{2} & 0 \\
4 & 3 & 18 & 0 \\
-2 & 3 & 9 & 0 \\
-4 & 3 & 6 & 0
\end{bmatrix} \\
\intertext{II = II - 4I, III = III + 2I, IV = IV + 4I}
\begin{bmatrix}[ccc|c]
1 & -1 & -\frac{5}{2} & 0 \\
0 & 7 & 28 & 0 \\
0 & 1 & 4 & 0 \\
0 & -1 & -4 & 0
\end{bmatrix}  \\
\intertext{II = II$ \cdot \frac{1}{7}$}
\begin{bmatrix}[ccc|c]
1 & -1 & -\frac{5}{2} & 0 \\
0 & 1 & 4 & 0 \\
0 & 1 & 4 & 0 \\
0 & -1 & -4 & 0
\end{bmatrix} \\
\intertext{III = III - II, IV = IV + II}
\begin{bmatrix}[ccc|c]
1 & -1 & -\frac{5}{2} & 0 \\
0 & 1 & 4 & 0 \\
0 & 0 & 0 & 0 \\
0 & 0 & 0 & 0
\end{bmatrix} \\
\overset{IV}{\Rightarrow} 0c_{3} &=& 0 \\
c_{3} &=& t, t \in \mathbb{R} \\
\overset{II}{\Rightarrow} c_{2} + 4c_{3} &=& 0 \\
c_{2} + 4t &=& 0 \\
c_{2} &=& -4t \\
\overset{I}{\Rightarrow} c_{1} - c_{2} - \frac{5}{2}c_{3} &=& 0 \\
c_{1} + 4t - \frac{5}{2}t &=& 0 \\
c_{1} + \frac{3}{2}t &=& 0 \\
c_{1} &=& -\frac{3}{2}t
\end{alignat*}
Es gibt unendlich viele Lösungen der Gleichung, daher sind die drei Vektoren voneinander linear abhängig.
\subsection{} %b
Wie in a).

In ein LGS überführt, ergibt sich dies:\\
\begin{alignat*}{2}
2c_{1} - 2c_{2} - 4c_{3} &=& 0 \\
4c_{1} + 3c_{2} - 1c_{3} &=& 0 \\
-2c_{1} + 3c_{2} + 6c_{3} &=& 0 \\
-4c_{1} + 3c_{2} + 7c_{3} &=& 0 \\
\intertext{In Matrixform ergibt sich dies}
\begin{bmatrix}[ccc|c]
2 & -2 & -4 & 0 \\
4 & 3 & -1 & 0 \\
-2 & 3 & 6 & 0 \\
-4 & 3 & 7 & 0
\end{bmatrix} \\
\intertext{I = I$ \cdot \frac{1}{2}$}
\begin{bmatrix}[ccc|c]
1 & -1 & -2 & 0 \\
4 & 3 & -1 & 0 \\
-2 & 3 & 6 & 0 \\
-4 & 3 & 7 & 0
\end{bmatrix} \\
\intertext{II = II - 4I, III = III + 2I, IV = IV + 4I}
\begin{bmatrix}[ccc|c]
1 & -1 & -2 & 0 \\
0 & 7 & 7 & 0 \\
0 & 1 & 2 & 0 \\
0 & -1 & -1 & 0
\end{bmatrix}  \\
\intertext{II = II$ \cdot \frac{1}{7}$}
\begin{bmatrix}[ccc|c]
1 & -1 & -2 & 0 \\
0 & 1 & 1 & 0 \\
0 & 1 & 2 & 0 \\
0 & -1 & -1 & 0
\end{bmatrix} \\
\intertext{III = III - II, IV = IV + II}
\begin{bmatrix}[ccc|c]
1 & -1 & -2 & 0 \\
0 & 1 & 1 & 0 \\
0 & 0 & 1 & 0 \\
0 & 0 & 0 & 0
\end{bmatrix} \\
\overset{III}{\Rightarrow} c_{3} &=& 0 \\
\overset{II}{\Rightarrow} c_{2} + c_{3} &=& 0 \\
c_{2} + 0 &=& 0 \\
c_{2} &=& 0 \\
\overset{I}{\Rightarrow} c_{1} - c_{2} - 2c_{3} &=& 0 \\
c_{1} - 0 - 2 \cdot 0 &=& 0 \\
c_{1} &=& 0
\end{alignat*}
Es gibt nur die triviale Lösung, daher sind die drei Vektoren voneinander linear unabhängig.
\subsection{} %c
Wie in a).

In ein LGS überführt, ergibt sich dies:\\
\begin{alignat*}{2}
1c_{1} - 1c_{2} + 2c_{3} - 1c_{4} &=& 0 \\
1c_{1} + 0c_{2} + 2c_{3} + 5c_{4} &=& 0 \\
1c_{1} + 1c_{2} + 0c_{3} + 13c_{4} &=& 0 \\
1c_{1} + 1c_{2} - 1c_{3} + 14c_{4} &=& 0 \\
\intertext{In Matrixform ergibt sich dies}
\begin{bmatrix}[cccc|c]
1 & -1 & 2 & -1 & 0 \\
1 & 0 & 2 & 5 & 0 \\
1 & 1 & 0 & 13 & 0 \\
1 & 1 & -1 & 14 & 0
\end{bmatrix} \\
\intertext{II = II - I, III = III - I, IV = IV - I}
\begin{bmatrix}[cccc|c]
1 & -1 & 2 & -1 & 0 \\
0 & 1 & 0 & 6 & 0 \\
0 & 2 & 0 & 14 & 0 \\
0 & 2 & -3 & 15 & 0
\end{bmatrix}  \\
\intertext{III = III - 2II, IV = IV - 2II}
\begin{bmatrix}[cccc|c]
1 & -1 & 2 & -1 & 0 \\
0 & 1 & 0 & 6 & 0 \\
0 & 0 & 0 & 2 & 0 \\
0 & 0 & -3 & 3 & 0
\end{bmatrix} \\
\intertext{III $\curvearrowright$ IV, IV $\curvearrowright$ III}
\begin{bmatrix}[cccc|c]
1 & -1 & 2 & -1 & 0 \\
0 & 1 & 0 & 6 & 0 \\
0 & 0 & -3 & 3 & 0 \\
0 & 0 & 0 & 2 & 0
\end{bmatrix} \\
\intertext{III = III$ \cdot -\frac{1}{3}$}
\begin{bmatrix}[cccc|c]
1 & -1 & 2 & -1 & 0 \\
0 & 1 & 0 & 6 & 0 \\
0 & 0 & 1 & -1 & 0 \\
0 & 0 & 0 & 2 & 0
\end{bmatrix} \\
\intertext{IV = IV$ \cdot \frac{1}{2}$}
\begin{bmatrix}[cccc|c]
1 & -1 & 2 & -1 & 0 \\
0 & 1 & 0 & 6 & 0 \\
0 & 0 & 1 & -1 & 0 \\
0 & 0 & 0 & 1 & 0
\end{bmatrix} \\
\overset{IV}{\Rightarrow} c_{4} &=& 0 \\
\overset{III}{\Rightarrow} c_{3} - c_{4} &=& 0 \\
c_{3} - 0 &=& 0 \\
c_{3} &=& 0 \\
\overset{II}{\Rightarrow} c_{2} + 0c_{3} + 6c_{4} &=& 0 \\
c_{2} + 0 + 6 \cdot 0 &=& 0 \\
c_{2} &=& 0 \\
\overset{I}{\Rightarrow} c_{1} - c_{2} + 2c_{3} - c_{4} &=& 0 \\
c_{1} - 0 + 2 \cdot 0 - 0 &=& 0 \\
c_{1} &=& 0
\end{alignat*}
Es gibt nur die triviale Lösung, daher sind die vier Vektoren voneinander linear unabhängig.
\subsection{} %d
Es gibt fünf Vektoren bei vier Dimensionen im Vektorraum. Daher sind diese Vektoren linear voneinander abhängig.
\end{document}
