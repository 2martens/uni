\documentclass[10pt,a4paper,oneside,ngerman,numbers=noenddot]{scrartcl}
\usepackage[T1]{fontenc}
\usepackage[utf8]{inputenc}
\usepackage[ngerman]{babel}
\usepackage{amsmath}
\usepackage{amsfonts}
\usepackage{amssymb}
\usepackage{paralist}
\usepackage{gauss}
\usepackage[locale=DE,exponent-product=\cdot,detect-all]{siunitx}
\usepackage{tikz}
\usetikzlibrary{matrix,fadings,calc,positioning,decorations.pathreplacing,arrows,decorations.markings}
\usepackage{polynom}
\polyset{style=C, div=:,vars=x}
\pagenumbering{arabic}
\def\thesection{\arabic{section})}
\def\thesubsection{\alph{subsection})}
\def\thesubsubsection{(\roman{subsubsection})}
\makeatletter
\renewcommand*\env@matrix[1][*\c@MaxMatrixCols c]{%
  \hskip -\arraycolsep
  \let\@ifnextchar\new@ifnextchar
  \array{#1}}
\makeatother

\begin{document}
\author{Jim Martens}
\title{Hausaufgaben zum 17./18. Januar}
\maketitle
\section{} %1
\subsection{} %a
\[
\begin{bmatrix}[ccc|c]
2 & 1 & 1 & 1 \\
1 & -1 & 1 & 4 \\
3 & 1 & 2 & 4
\end{bmatrix}
\]
$\overset{II \curvearrowright I, I \curvearrowright II}{\leadsto}$
\[
\begin{bmatrix}[ccc|c]
1 & -1 & 1 & 4 \\
2 & 1 & 1 & 1 \\
3 & 1 & 2 & 4
\end{bmatrix}
\]
$\overset{II = II - 2I, III = III - 3I}{\leadsto}$
\[
\begin{bmatrix}[ccc|c]
1 & -1 & 1 & 4 \\
0 & 3 & -1 & -7 \\
0 & 4 & -1 & -8
\end{bmatrix}
\]
$\overset{II = \frac{1}{3}II}{\leadsto}$
\[
\begin{bmatrix}[ccc|c]
1 & -1 & 1 & 4 \\
0 & 1 & -\frac{1}{3} & -\frac{7}{3} \\
0 & 4 & -1 & -8
\end{bmatrix}
\]
$\overset{III = III - 4II}{\leadsto}$
\[
\begin{bmatrix}[ccc|c]
1 & -1 & 1 & 4 \\
0 & 1 & -\frac{1}{3} & -\frac{7}{3} \\
0 & 0 & \frac{1}{3} & \frac{4}{3}
\end{bmatrix}
\]
$\overset{III = 3III}{\leadsto}$
\[
\begin{bmatrix}[ccc|c]
1 & -1 & 1 & 4 \\
0 & 1 & -\frac{1}{3} & -\frac{7}{3} \\
0 & 0 & 1 & 4
\end{bmatrix}
\]
\begin{alignat*}{3}
\overset{III}{\Rightarrow} & x_{3} &\,=\,& 4 && \\
\overset{II}{\Rightarrow} & x_{2} - \frac{4}{3} &\,=\,& - \frac{7}{3} && \;| +\frac{4}{3} \\
\Leftrightarrow & x_{2} &\,=\,& -1 & \\
\overset{I}{\Rightarrow} & x_{1} - (-1) + 4 &\,=\,& 4 && \;|\text{Zus.} \\
\Leftrightarrow & x_{1} + 5 &\,=\,& 4 && \;| -5 \\
\Leftrightarrow & x_{1} &\,=\,& -1 &&
\end{alignat*}
Es gibt genau eine Lösung.
\subsection{} %b
\[
\begin{bmatrix}[ccc|c]
2 & 1 & 1 & 2 \\
1 & -1 & 1 & 3 \\
3 & 0 & 2 & 5
\end{bmatrix}
\]
$\overset{II \curvearrowright I, I \curvearrowright II}{\leadsto}$
\[
\begin{bmatrix}[ccc|c]
1 & -1 & 1 & 3 \\
2 & 1 & 1 & 2 \\
3 & 0 & 2 & 5
\end{bmatrix}
\]
$\overset{II = II - 2I, III = III - 3I}{\leadsto}$
\[
\begin{bmatrix}[ccc|c]
1 & -1 & 1 & 3 \\
0 & -1 & -1 & -4 \\
0 & -3 & -1 & -4
\end{bmatrix}
\]
$\overset{II = -1II}{\leadsto}$
\[
\begin{bmatrix}[ccc|c]
1 & -1 & 1 & 3 \\
0 & 1 & 1 & 4 \\
0 & -3 & -1 & -4
\end{bmatrix}
\]
$\overset{III = III + 3II}{\leadsto}$
\[
\begin{bmatrix}[ccc|c]
1 & -1 & 1 & 3 \\
0 & 1 & 1 & 4 \\
0 & 0 & 2 & 8
\end{bmatrix}
\]
$\overset{III = \frac{1}{2}III}{\leadsto}$
\[
\begin{bmatrix}[ccc|c]
1 & -1 & 1 & 3 \\
0 & 1 & 1 & 4 \\
0 & 0 & 1 & 4
\end{bmatrix}
\]
\begin{alignat*}{3}
\overset{III}{\Rightarrow} & x_{3} &\,=\,& 4 && \\
\overset{II}{\Rightarrow} & x_{2} + x_{3} &\,=\,& 4 && \\
\Rightarrow & x_{2} + 4 &\,=\,& 4 && \;| -4 \\
\Leftrightarrow & x_{2} &\,=\,& 0 & \\
\overset{I}{\Rightarrow} & x_{1} - x_{2} + x_{3} &\,=\,& 3 && \\
\Rightarrow & x_{2} - 0 + 4 &\,=\,& 3 && \;| -4 \\
\Leftrightarrow & x_{1} &\,=\,& -1 &&
\end{alignat*}
Es gibt genau eine Lösung.
\subsection{} %c
\[
\begin{bmatrix}[ccc|c]
2 & 1 & 1 & -1 \\
1 & -1 & 1 & 3 \\
3 & 0 & 2 & 0
\end{bmatrix}
\]
$\overset{II \curvearrowright I, I \curvearrowright II}{\leadsto}$
\[
\begin{bmatrix}[ccc|c]
1 & -1 & 1 & 3 \\
2 & 1 & 1 & -1 \\
3 & 0 & 2 & 0
\end{bmatrix}
\]
$\overset{II = II - 2I, III = III - 3I}{\leadsto}$
\[
\begin{bmatrix}[ccc|c]
1 & -1 & 1 & 3 \\
0 & 3 & -1 & -7 \\
0 & 3 & -1 & -9
\end{bmatrix}
\]
$\overset{II = \frac{1}{3}II}{\leadsto}$
\[
\begin{bmatrix}[ccc|c]
1 & -1 & 1 & 3 \\
0 & 1 & -\frac{1}{3} & -\frac{7}{3} \\
0 & 3 & -1 & -9
\end{bmatrix}
\]
$\overset{III = III - 3II}{\leadsto}$
\[
\begin{bmatrix}[ccc|c]
1 & -1 & 1 & 3 \\
0 & 1 & -\frac{1}{3} & -\frac{7}{3} \\
0 & 0 & 0 & -2
\end{bmatrix}
\]
\begin{alignat*}{3}
\overset{III}{\Rightarrow} & 0x_{3} &\,=\,& -2 && 
\end{alignat*}
Es gibt keine Lösung, da die Gleichung $0x_{3} = - 2$ durch kein $x_{3}$ erfüllt wird.
\subsection{} %d
\[
\begin{bmatrix}[ccc|c]
3 & \frac{3}{2} & \frac{3}{2} & 6 \\
2 & 1 & 1 & 4 \\
4 & 2 & 2 & 8
\end{bmatrix}
\]
$\overset{II \curvearrowright I, I \curvearrowright II}{\leadsto}$
\[
\begin{bmatrix}[ccc|c]
2 & 1 & 1 & 4 \\
3 & \frac{3}{2} & \frac{3}{2} & 6 \\
4 & 2 & 2 & 8
\end{bmatrix}
\]
$\overset{I = \frac{1}{2}I}{\leadsto}$
\[
\begin{bmatrix}[ccc|c]
1 & \frac{1}{2} & \frac{1}{2} & 2 \\
3 & \frac{3}{2} & \frac{3}{2} & 6 \\
4 & 2 & 2 & 8
\end{bmatrix}
\]
$\overset{II = II - 3I, III = III - 4I}{\leadsto}$
\[
\begin{bmatrix}[ccc|c]
1 & \frac{1}{2} & \frac{1}{2} & 2 \\
0 & 0 & 0 & 0 \\
0 & 0 & 0 & 0
\end{bmatrix}
\]
\begin{alignat*}{3}
\overset{III}{\Rightarrow} & 0x_{3} &\,=\,& 0 && \\
\Leftrightarrow & x_{3} &\,=\,& t, t \in \mathbb{R} && \\
\overset{II}{\Rightarrow} & 0x_{2} &\,=\,& 0 && \\
\Leftrightarrow & x_{2} &\,=\,& s, s \in \mathbb{R} && \\
\overset{I}{\Rightarrow} & x_{1} + \frac{1}{2}s + \frac{1}{2}t &\,=\,& 2 && \;|-\frac{1}{2}s, -\frac{1}{2}t \\
\Leftrightarrow & x_{1} &\,=\,& 2 - \frac{1}{2}s - \frac{1}{2}t  &&
\end{alignat*}
Es gibt unendlich viele Lösungen.
\section{} %2
$x_{1}$, $x_{2}$ und $x_{5}$ sind die führenden Variablen. Die restlichen Variablen sind die freien Variablen.
\\
\\
\begin{alignat*}{3}
\overset{IV}{\Rightarrow} & 0x_{6} &\,=\,& 0 && \\
\Leftrightarrow & x_{6} &\,=\,& t, t \in \mathbb{R} \\
\overset{III}{\Rightarrow} & x_{5} - 3t &\,=\,& -2 && | + 3t \\
\Leftrightarrow & x_{5} &\,=\,& 3t - 2 & \\
\overset{II}{\Rightarrow} & x_{2} + 2x_{3} + 3x_{3} &\,=\,& 1 && \\
\Leftrightarrow & x_{3} &\,=\,& r, r \in \mathbb{R} & \\
& x_{4} &\,=\,& s, s \in \mathbb{R} && \\
\Rightarrow & x_{2} + 2r + 3s &\,=\,& 1 && | -2r, -3s \\
\Leftrightarrow & x_{2} &\,=\,& -2r - 3s + 1 && \\
\overset{I}{\Rightarrow} & x_{1} + 2(-2r - 3s + 1) - r + 3s - (3t - 2) + 2t &\,=\,& 1 && | \text{Kl. aufl.} \\
\Leftrightarrow & x_{1} - 4r - 6s + 2 - r + 3s -3t + 2 + 2t &\,=\,& 1 && | \text{Zus.} \\
\Leftrightarrow & x_{1} - t - 5r - 3s + 4 &\,=\,& 1 && | +t, +5r, +3s, -4 \\
\Leftrightarrow & x_{1} &\,=\,& 5r + 3s + t - 3 &&
\end{alignat*}
\section{} %3
Prüfe, ob $u$ und $w$ Linearkombinationen von $v_{1}$, $v_{2}$ und $v_{3}$ sind:\\
\[
\begin{bmatrix}[ccc|cc]
1 & 0 & -1 & 1 & -2 \\
0 & -1 & 4 & 3 & 2 \\
0 & 1 & 2 & 6 & 4 \\
3 & 2 & 1 & 15 & 1
\end{bmatrix}
\]
$\overset{IV = IV - 3I}{\leadsto}$
\[
\begin{bmatrix}[ccc|cc]
1 & 0 & -1 & 1 & -2 \\
0 & -1 & 4 & 3 & 2 \\
0 & 1 & 2 & 6 & 4\\
0 & 2 & 4 & 12 & 7
\end{bmatrix}
\]
$\overset{II \curvearrowright III, III \curvearrowright II}{\leadsto}$
\[
\begin{bmatrix}[ccc|cc]
1 & 0 & -1 & 1 & -2 \\
0 & 1 & 2 & 6 & 4\\
0 & -1 & 4 & 3 & 2 \\
0 & 2 & 4 & 12 & 7
\end{bmatrix}
\]
$\overset{III = III + II, IV = IV - 2II}{\leadsto}$
\[
\begin{bmatrix}[ccc|cc]
1 & 0 & -1 & 1 & -2 \\
0 & 1 & 2 & 6 & 4 \\
0 & 0 & 6 & 9 & 6 \\
0 & 0 & 0 & 0 & -1
\end{bmatrix}
\]
$\overset{III = \frac{1}{6}III}{\leadsto}$
\[
\begin{bmatrix}[ccc|cc]
1 & 0 & -1 & 1 & -2 \\
0 & 1 & 2 & 6 & 4 \\
0 & 0 & 1 & \frac{3}{2} & 1 \\
0 & 0 & 0 & 0 & -1
\end{bmatrix}
\]
$u$:\\
\begin{alignat*}{3}
\overset{IV}{\Rightarrow} & 0v_{3} &\,=\,& 0 && \\
\Leftrightarrow & v_{3} &\,=\,& t, t \in \mathbb{R} && \\
\overset{III}{\Rightarrow} & v_{3} &\,=\,& \frac{3}{2} && \\
\overset{II}{\Rightarrow} & v_{2} + 2(\frac{3}{2}) &\,=\,& 6 && \;| \text{Kl. aufl.} \\
\Leftrightarrow & v_{2} + 3 &\,=\,& 6 && \;| -3 \\
\Leftrightarrow & v_{2} &\,=\,& 3 && \\
\overset{I}{\Rightarrow} & v_{1} + 0 \cdot 3 - \frac{3}{2} &\,=\,& 1 && \;| \text{Zus.}\\
\Leftrightarrow & v_{1} - \frac{3}{2} &\,=\,& 1 && \;| + \frac{3}{2} \\
\Leftrightarrow & v_{1} &\,=\,& \frac{5}{2} &&
\end{alignat*}
Der Vektor $u$ ist eine Linearkombination von $v_{1}$, $v_{2}$ und $v_{3}$.

$w$:\\
\begin{alignat*}{3}
\overset{IV}{\Rightarrow} & 0v_{3} &\,=\,& -1 &&
\end{alignat*}
Der Vektor $w$ ist keine Linearkombination von $v_{1}$, $v_{2}$ und $v_{3}$, da es kein $v_{3}$ gibt, für das die Gleichung $0v_{3} = -1$ gilt.

\section{} %4
\subsection{} %a
\subsubsection{} %i
\begin{alignat*}{2}
A^{-1} &=& \frac{1}{4 \cdot 2 - 1 \cdot 3} \begin{bmatrix} 2 & -1 \\ -3 & 4 \end{bmatrix} \\
&=& \frac{1}{5} \begin{bmatrix} 2 & -1 \\ -3 & 4 \end{bmatrix}
\end{alignat*}
\subsubsection{} %ii
\begin{alignat*}{2}
\begin{bmatrix}[c|c]
A & E_{2}
\end{bmatrix} &=& 
\begin{bmatrix}[cc|cc]
4 & 1 & 1 & 0 \\
3 & 2 & 0 & 1
\end{bmatrix} \\
\intertext{$I = \frac{1}{4}I$}
&=&
\begin{bmatrix}[cc|cc]
1 & \frac{1}{4} & \frac{1}{4} & 0 \\
3 & 2 & 0 & 1
\end{bmatrix} \\
\intertext{$II = II - 3I$}
&=&
\begin{bmatrix}[cc|cc]
1 & \frac{1}{4} & \frac{1}{4} & 0 \\
0 & \frac{5}{4} & -\frac{3}{4} & 1
\end{bmatrix} \\
\intertext{$II = \frac{4}{5}II$}
&=&
\begin{bmatrix}[cc|cc]
1 & \frac{1}{4} & \frac{1}{4} & 0 \\
0 & 1 & -\frac{3}{5} & \frac{4}{5}
\end{bmatrix} \\
\intertext{$I = I - \frac{1}{4}II$}
&=&
\begin{bmatrix}[cc|cc]
1 & 0 & \frac{2}{5} & -\frac{1}{5} \\
0 & 1 & -\frac{3}{5} & \frac{4}{5}
\end{bmatrix} \\
\end{alignat*}
\subsection{} %b
\begin{alignat*}{2}
\begin{bmatrix}[c|c]
A & E_{3}
\end{bmatrix} &=& 
\begin{bmatrix}[ccc|ccc]
2 & 4 & -2 & 1 & 0 & 0 \\
4 & 9 & -3 & 0 & 1 & 0 \\
-2 & -3 & 7 & 0 & 0 & 1
\end{bmatrix} \\
\intertext{$I = \frac{1}{2}I$}
&=&
\begin{bmatrix}[ccc|ccc]
1 & 2 & -1 & \frac{1}{2} & 0 & 0 \\
4 & 9 & -3 & 0 & 1 & 0 \\
-2 & -3 & 7 & 0 & 0 & 1
\end{bmatrix} \\
\intertext{$II = II - 4I, III = III + 2I$}
&=&
\begin{bmatrix}[ccc|ccc]
1 & 2 & -1 & \frac{1}{2} & 0 & 0 \\
0 & 1 & 1 & -2 & 1 & 0 \\
0 & 1 & 5 & 1 & 0 & 1
\end{bmatrix} \\
\intertext{$III = III - II$}
&=&
\begin{bmatrix}[ccc|ccc]
1 & 2 & -1 & \frac{1}{2} & 0 & 0 \\
0 & 1 & 1 & -2 & 1 & 0 \\
0 & 0 & 4 & 3 & -1 & 1
\end{bmatrix} \\
\intertext{$III = \frac{1}{4}III$}
&=&
\begin{bmatrix}[ccc|ccc]
1 & 2 & -1 & \frac{1}{2} & 0 & 0 \\
0 & 1 & 1 & -2 & 1 & 0 \\
0 & 0 & 1 & \frac{3}{4} & -\frac{1}{4} & \frac{1}{4}
\end{bmatrix} \\
\intertext{$II = II - III, I = I + III$}
&=&
\begin{bmatrix}[ccc|ccc]
1 & 2 & 0 & \frac{5}{4} & -\frac{1}{4} & \frac{1}{4} \\
0 & 1 & 0 & -\frac{11}{4} & \frac{5}{4} & -\frac{1}{4} \\
0 & 0 & 1 & \frac{3}{4} & -\frac{1}{4} & \frac{1}{4}
\end{bmatrix} \\
\intertext{$I = I - 2II$}
&=&
\begin{bmatrix}[ccc|ccc]
1 & 0 & 0 & \frac{27}{4} & -\frac{11}{4} & \frac{3}{4} \\
0 & 1 & 0 & -\frac{11}{4} & \frac{5}{4} & -\frac{1}{4} \\
0 & 0 & 1 & \frac{3}{4} & -\frac{1}{4} & \frac{1}{4}
\end{bmatrix}
\end{alignat*}
\begin{alignat*}{2}
AA^{-1} &=&
\begin{bmatrix}
2 & 4 & -2 \\
4 & 9 & -3 \\
-2 & -3 & 7
\end{bmatrix} \cdot
\begin{bmatrix}
\frac{27}{4} & -\frac{11}{4} & \frac{3}{4} \\
-\frac{11}{4} & \frac{5}{4} & -\frac{1}{4} \\
\frac{3}{4} & -\frac{1}{4} & \frac{1}{4}
\end{bmatrix} \\
&=& \begin{bmatrix}
1 & 0 & 0 \\
0 & 1 & 0 \\
0 & 0 & 1
\end{bmatrix}
\end{alignat*}
\begin{alignat*}{3}
Ax &=& b^{T} && \;| \cdot A^{-1} \\
(A^{-1}A)x &=& A^{-1} \cdot b && \\
x &=& A^{-1} \cdot b && \\
&=& 
\begin{bmatrix}
\frac{27}{4} & -\frac{11}{4} & \frac{3}{4} \\
-\frac{11}{4} & \frac{5}{4} & -\frac{1}{4} \\
\frac{3}{4} & -\frac{1}{4} & \frac{1}{4}
\end{bmatrix} \cdot 
\begin{bmatrix}
-2 \\
4 \\
2
\end{bmatrix} \\
&=& 
\begin{bmatrix}
-23 \\
10 \\
-4
\end{bmatrix}
\end{alignat*}
\subsection{} %c
\begin{alignat*}{2}
\begin{bmatrix}[c|c]
B & E_{3}
\end{bmatrix} &=& 
\begin{bmatrix}[ccc|ccc]
-1 & -2 & 3 & 1 & 0 & 0 \\
6 & 4 & -1 & 0 & 1 & 0 \\
-10 & -12 & 13 & 0 & 0 & 1
\end{bmatrix} \\
\intertext{$I = -I$}
&=&
\begin{bmatrix}[ccc|ccc]
1 & 2 & -3 & -1 & 0 & 0 \\
6 & 4 & -1 & 0 & 1 & 0 \\
-10 & -12 & 13 & 0 & 0 & 1
\end{bmatrix} \\
\intertext{$II = -6I, III = III + 10I$}
&=&
\begin{bmatrix}[ccc|ccc]
1 & 2 & -3 & -1 & 0 & 0 \\
0 & -8 & 17 & 0 & 1 & 0 \\
0 & 8 & -17 & 0 & 0 & 1
\end{bmatrix} \\
\intertext{$II = -\frac{1}{8}II$}
&=&
\begin{bmatrix}[ccc|ccc]
1 & 2 & -3 & -1 & 0 & 0 \\
0 & 1 & -\frac{17}{8} & 0 & -\frac{1}{8} & 0 \\
0 & 8 & -17 & 0 & 0 & 1
\end{bmatrix} \\
\intertext{$III = III - 8II$}
&=&
\begin{bmatrix}[ccc|ccc]
1 & 2 & -3 & -1 & 0 & 0 \\
0 & 1 & -\frac{17}{8} & 0 & -\frac{1}{8} & 0 \\
0 & 0 & 0 & 0 & 0 & 1
\end{bmatrix}
\end{alignat*}
Auf der linken Seite der Blockmatrix kann unmöglich die Einheitsmatrix erreicht werden. Daher hat die Matrix B keine inverse Matrix.
\end{document}
