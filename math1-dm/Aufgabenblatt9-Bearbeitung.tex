\documentclass[10pt,a4paper,oneside,ngerman,numbers=noenddot]{scrartcl}
\usepackage[T1]{fontenc}
\usepackage[utf8]{inputenc}
\usepackage[ngerman]{babel}
\usepackage{amsmath}
\usepackage{amsfonts}
\usepackage{amssymb}
\usepackage{paralist}
\usepackage[locale=DE,exponent-product=\cdot,detect-all]{siunitx}
\usepackage{tikz}
\usetikzlibrary{matrix,fadings,calc,positioning,decorations.pathreplacing,arrows,decorations.markings}
\pagenumbering{arabic}
\def\thesection{\arabic{section})}
\def\thesubsection{\alph{subsection})}
\def\thesubsubsection{(\roman{subsubsection})}

\begin{document}
\author{Jim Martens}
\title{Hausaufgaben zum 20./21. Dezember}
\maketitle
\section{} %1
\subsection{} %a
Für $a$ hat man sechs Möglichkeiten in $\mathbb{Z}_{7}$, da $a \neq 0$ gilt. Für $b$ hat man sieben Möglichkeiten in $\mathbb{Z}_{7}$, da es für $b$ keine Einschränkungen gibt. Daher gibt es $6 \cdot 7 = 42$ Matrizen der angegebenen Form in $M$. Somit ist die Ordnung von $M$ gleich $42$.
\subsection{} %b
Zur Bestimmung des Inversen zu A mache man sich $\mathbb{Z}_{7}$ zu nutze. Es gilt $2 \cdot 2 \cdot 2 = 2 \cdot 4 = 8 = 1$. Ferner gilt $4 \cdot 1 + 3 \cdot 1 = 7 = 0$.

Damit ergibt sich als inverse Matrix $\begin{pmatrix} 4 & 3 \\ 0 & 1 \end{pmatrix}$.
\subsection{} %c
Da nur Teiler der Ordnung von $M$ für Halbgruppen in Frage kommen, kann man alle Ordnungen der Elemente auslassen, die keine Teiler von der Ordnung von $M$ sind.
\begin{alignat*}{2}
B^{2} = B \cdot B &=& \begin{pmatrix} 4 & 1 \\ 0 & 1 \end{pmatrix} \cdot \begin{pmatrix} 4 & 1 \\ 0 & 1 \end{pmatrix} \\
&=& \begin{pmatrix} 2 & 5 \\ 0 & 1 \end{pmatrix} \\
B^{3} = B^{2} \cdot B &=& \begin{pmatrix} 2 & 5 \\ 0 & 1 \end{pmatrix} \cdot \begin{pmatrix} 4 & 1 \\ 0 & 1 \end{pmatrix} \\
&=& \begin{pmatrix} 1 & 0 \\ 0 & 1 \end{pmatrix} \\
\intertext{$B$ hat demzufolge die Ordnung $3$}
C^{2} = C \cdot C &=& \begin{pmatrix} 3 & 3 \\ 0 & 1 \end{pmatrix} \cdot \begin{pmatrix} 3 & 3 \\ 0 & 1 \end{pmatrix} \\
&=& \begin{pmatrix} 2 & 5 \\ 0 & 1 \end{pmatrix} \\
C^{3} = C^{2} \cdot C &=& \begin{pmatrix} 2 & 5 \\ 0 & 1 \end{pmatrix} \cdot \begin{pmatrix} 3 & 3 \\ 0 & 1 \end{pmatrix} \\
&=& \begin{pmatrix} 6 & 4 \\ 0 & 1 \end{pmatrix} \\
C^{6} = C^{3} \cdot C^{3} &=& \begin{pmatrix} 6 & 4 \\ 0 & 1 \end{pmatrix} \cdot \begin{pmatrix} 6 & 4 \\ 0 & 1 \end{pmatrix} \\
&=& \begin{pmatrix} 1 & 0 \\ 0 & 1 \end{pmatrix} \\
\intertext{$C$ hat demzufolge die Ordnung $6$}
D^{2} = D \cdot D &=& \begin{pmatrix} 1 & 4 \\ 0 & 1 \end{pmatrix} \cdot \begin{pmatrix} 1 & 4 \\ 0 & 1 \end{pmatrix} \\
&=& \begin{pmatrix} 1 & 1 \\ 0 & 1 \end{pmatrix} \\
D^{3} = D^{2} \cdot D &=& \begin{pmatrix} 1 & 1 \\ 0 & 1 \end{pmatrix} \cdot \begin{pmatrix} 1 & 4 \\ 0 & 1 \end{pmatrix} \\
&=& \begin{pmatrix} 1 & 5 \\ 0 & 1 \end{pmatrix} \\
D^{6} = D^{3} \cdot D^{3} &=& \begin{pmatrix} 1 & 5 \\ 0 & 1 \end{pmatrix} \cdot \begin{pmatrix} 1 & 5 \\ 0 & 1 \end{pmatrix} \\
&=& \begin{pmatrix} 1 & 3 \\ 0 & 1 \end{pmatrix} \\
D^{7} = D^{6} \cdot D &=& \begin{pmatrix} 1 & 3 \\ 0 & 1 \end{pmatrix} \cdot \begin{pmatrix} 1 & 4 \\ 0 & 1 \end{pmatrix} \\
&=& \begin{pmatrix} 1 & 0 \\ 0 & 1 \end{pmatrix} \\
\intertext{$D$ hat demzufolge die Ordnung $7$}
\end{alignat*}
\subsection{} %d
Ja und zwar $\begin{pmatrix} 6 & 6 \\ 0 & 1 \end{pmatrix}$.
\section{} %2
\subsection{} %a
\begin{alignat*}{2}
s \ast y &=& z \\
z^{-1} &=& z \\
x \ast r &=& y \\
y^{-1} &=& y \\
x \ast y &=& r \\
r^{-1} &=& t
\end{alignat*}
\subsection{} %b
$G$ ist nicht kommutativ, da nicht alle Operationen kommutativ ausführbar sind. Als Beispiel sei hier $w \ast r = z$ und $r \ast w = y$ genannt.

$G$ ist nicht zyklisch, da es kein Element gibt, mit dem alle Elemente erzeugt werden können.

Die Elemente von $G$ haben folgende Ordnungen:
\begin{alignat*}{2}
i &:& 1 \\
r &:& 4 \\
s &:& 2 \\
t &:& 4 \\
w &:& 2 \\
x &:& 2 \\
y &:& 2 \\
z &:& 2
\end{alignat*}
\subsection{} %c
Da es nur zwei Elemente von $G$ mit der Ordnung $4$ gibt ($r$ und $t$), fällt die Wahl beider Untergruppen leicht.
\section{} %3
\subsection{} %a
Es wird der folgende Ausdruck vereinfacht:
\begin{alignat*}{2}
a^{-1}(bd^{-1})^{-1}bc(b^{-1}cdc)^{-1}ab^{-1} \\
a^{-1}d(b^{-1}b)(cc^{-1})d^{-1}c^{-1}bab^{-1} \\
a^{-1}(dd^{-1})c^{-1}bab^{-1} \\
a^{-1}c^{-1}bab^{-1} \\
\intertext{Wenn $G$ abelsch ist, dann gilt weiter:}
(a^{-1}a)c^{-1}(bb^{-1}) \\
c^{-1}
\end{alignat*}
\subsection{} %b
\textbf{Behauptung:} Jede zyklische Gruppe ist abelsch.\\
\textbf{Beweis:}\\
Sei $a \in G$ der Erzeuger der Gruppe $G$. Dann gilt für ein beliebiges $b \in G$: $b=a^{k}$.
Es ist zu zeigen, dass $c \cdot d = d \cdot c$ für alle $c,d \in G$ gilt. Da $a$ das Erzeugerelement ist, ergibt sich $a^{m} \cdot a^{n} = a^{m+n} = a^{n} \cdot a^{m}$, womit gezeigt ist, dass jede zyklische Gruppe G abelsch ist. \hfill $\Box$
\subsection{} %c

\subsubsection{} %(i)
Wahr, da es für jedes $n \in \mathbb{N}$ die zyklische Gruppe $(\mathbb{Z}_{n},+)$ gibt.
\subsubsection{} %(ii)
Wahr, da die Elementordnungen gleich häufig vorkommen.
\section{} %4
\textbf{Behauptung:} Außer der zyklischen Gruppe $G = <a> = \{1,a,a^{2},a^{3}\}$ gibt es bis auf Isomorphie nur eine weitere Gruppe der Ordnung $4$.\\
\textbf{Beweis:}\\
Sei $G = \{1,a,b,c\}$ eine nicht zyklische Gruppe. $G$ hat die Ordnung $4$. Da $G$ keine zyklische Gruppe sein soll, bleiben die Ordnungen $1$ und $2$. Das neutrale Element $1$ hat die Ordnung $1$. Somit haben $a,b,c$ die Ordnung $2$. Dies gilt, da die Ordnung eines Elements einer Gruppe, die Gruppenordnung teilen muss.

Somit ergibt eine Verknüpfung eines Elements mit sich selbst immer das neutrale Element. Die erste Spalte und Zeile sind ohnehin klar. Bleiben $6$ Felder übrig. Je Feld sind aber bereits $3$ Zeichen belegt, sodass nur eins gewählt werden kann. Es ergibt sich diese Gruppentafel:

\begin{tabular}{c||c|c|c|c}
  & 1 & a & b & c \\
\hline
\hline
1 & 1 & a & b & c \\
\hline
a & a & 1 & c & b \\
\hline
b & b & c & 1 & a \\
\hline
c & c & b & a & 1
\end{tabular}

Somit wurde die Behauptung bestätigt. \hfill $\Box$

Eine bereits bekannte nicht-zyklische Gruppe der Ordnung 4 wäre z. B.: $G = \{1,a,b,c\}$ konkret definiert als $\mathbb{Z}_{6} \setminus \{1,5\} = \{0,2,3,4\}$ definiert über die Operation $+$.
\end{document}
