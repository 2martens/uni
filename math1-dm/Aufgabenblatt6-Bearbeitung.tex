\documentclass[10pt,a4paper,oneside,ngerman,numbers=noenddot]{scrartcl}
\usepackage[T1]{fontenc}
\usepackage[utf8]{inputenc}
\usepackage[ngerman]{babel}
\usepackage{amsmath}
\usepackage{amsfonts}
\usepackage{amssymb}
\usepackage{paralist}
\usepackage[locale=DE,exponent-product=\cdot,detect-all]{siunitx}
\usepackage{tikz}
\usetikzlibrary{matrix,fadings,calc,positioning,decorations.pathreplacing,arrows,decorations.markings}
\pagenumbering{arabic}
\def\thesection{\arabic{section})}
\def\thesubsection{\alph{subsection})}
\def\thesubsubsection{(\arabic{subsubsection})}

\begin{document}
\author{Jim Martens}
\title{Hausaufgaben zum 29./30. November}
\maketitle
\section{} %1
\subsection{} %a
Das Produkt AB existiert, da A gleich viele Spalten (3) wie B Zeilen (3) hat.
\begin{alignat*}{2}
A \cdot B &=& \begin{pmatrix} 2 & 0 & 1 \\ 1 & 0 & -1 \\ 7 & 6 & 3 \\ -1 & 2 & 4 \end{pmatrix} \cdot \begin{pmatrix} 3 & 2 & -1 \\ 1 & 0 & 2 \\ 1 & 1 & 0 \end{pmatrix} \\
&=& \begin{pmatrix} 7 & 5 & -2 \\ 2 & 1 & -1 \\ 30 & 17 & 5 \\ 3 & 2 & 5 \end{pmatrix}
\end{alignat*}

BA existiert nicht, da B weniger Spalten (3) als A Zeilen (4) hat.
AC existiert nicht, da A mehr Spalten (3) als C Zeilen (1) hat.
AD existiert, da A gleich viele Spalten (3) wie D Zeilen (3) hat.

\begin{alignat*}{2}
A \cdot D &=& \begin{pmatrix} 2 & 0 & 1 \\ 1 & 0 & -1 \\ 7 & 6 & 3 \\ -1 & 2 & 4 \end{pmatrix} \cdot \begin{pmatrix} 2 \\ 3 \\ -2 \end{pmatrix} \\
&=& \begin{pmatrix} 2 \\ 0 \\ 26 \\ -4 \end{pmatrix}
\end{alignat*}

AA existiert nicht, da A weniger Spalten (3) hat als Zeilen (4).
BB existiert, da B gleich viele Spalten (3) und Zeilen (3) hat.

\begin{alignat*}{2}
A \cdot D &=& \begin{pmatrix} 3 & 2 & -1 \\ 1 & 0 & 2 \\ 1 & 1 & 0 \end{pmatrix} \cdot \begin{pmatrix} 3 & 2 & -1 \\ 1 & 0 & 2 \\ 1 & 1 & 0 \end{pmatrix} \\
&=& \begin{pmatrix} 10 & 5 & 1 \\ 5 & 4 & -1 \\ 4 & 2 & 1 \end{pmatrix}
\end{alignat*}

CD existiert, da C gleich viele Spalten (3) wie D Zeilen (3) hat.

\begin{alignat*}{2}
C \cdot D &=& \begin{pmatrix} 1 & 2 & -2 \end{pmatrix} \cdot \begin{pmatrix} 2 \\ 3 \\ -2 \end{pmatrix} \\
&=& \begin{pmatrix} 12 \end{pmatrix}
\end{alignat*}

DC existiert, da D gleich viele Spalten (1) wie C Zeilen (1) hat.

\begin{alignat*}{2}
D \cdot C &=& \begin{pmatrix} 2 \\ 3 \\ -2 \end{pmatrix} \cdot \begin{pmatrix} 1 & 2 & -2 \end{pmatrix} \\
&=& \begin{pmatrix} 2 & 4 & -4 \\ 3 & 6 & -6 \\ -2 & -4 & 4 \end{pmatrix}
\end{alignat*}

\subsection{} %b
Um das Element zu berechnen, das in AB in der dritten Zeile und zweiten Spalte steht, benötigen wir die dritte Zeile von A und die zweite Spalte von B. Diese packen wir in eigene Matrizen und multiplizieren diese.

\begin{alignat*}{2}
A_{3j} \cdot B_{i2} &=& \begin{pmatrix} 1 & 2 & 3 & 4 \end{pmatrix} \cdot \begin{pmatrix} -2 \\ 2 \\ 3 \\ 1 \end{pmatrix} \\
&=& \begin{pmatrix} 15 \end{pmatrix}
\end{alignat*}

Für die vierte Spalte von AB benötigen wir A und die vierte Spalte von B. Wir multiplizieren also A und die vierte Spalte von B, die in einer eigenen Matrix steht.

\begin{alignat*}{2}
A \cdot B_{i4} &=& \begin{pmatrix} 3 & 4 & 5 & 6 \\ 2 & 3 & 4 & 5 \\ 1 & 2 & 3 & 4 \\ 4 & 7 & 7 & 4 \end{pmatrix} \cdot \begin{pmatrix} 4 \\ 4 \\ -3 \\ 0 \end{pmatrix} \\
&=& \begin{pmatrix} 13 \\ 8 \\ 3 \\ 23 \end{pmatrix}
\end{alignat*}

\section{} %2
\subsection{} %a
Um das Distributivgesetz zu bestätigen, wird zunächst $A(B_{1} + B_{2})$ gerechnet und anschließend $AB_{1} + AB_{2}$. Das Ergebnis muss gleich sein, um das Gesetz zu bestätigen.

\begin{alignat*}{2}
A \cdot (B_{1} + B_{2}) &=& \begin{pmatrix} 5 & 7 \\ 9 & -1 \\ 8 & 2 \end{pmatrix} \cdot \left( \begin{pmatrix} 1 & 2 \\ 3 & 6 \end{pmatrix} + \begin{pmatrix} 1 & -2 \\ 3 & 2 \end{pmatrix} \right)\\
&=& \begin{pmatrix} 5 & 7 \\ 9 & -1 \\ 8 & 2 \end{pmatrix} \cdot \begin{pmatrix} 2 & 0 \\ 6 & 8 \end{pmatrix} \\
&=& \begin{pmatrix} 52 & 56 \\ 12 & -8 \\ 28 & 16 \end{pmatrix}
\end{alignat*}

Nun folgt die Berechnung des zweiten Teils.

\begin{alignat*}{2}
AB_{1} + AB_{2} &=& \left( \begin{pmatrix} 5 & 7 \\ 9 & -1 \\ 8 & 2 \end{pmatrix} \cdot \begin{pmatrix} 1 & 2 \\ 3 & 6 \end{pmatrix} \right) + \left(\begin{pmatrix} 5 & 7 \\ 9 & -1 \\ 8 & 2 \end{pmatrix} \cdot \begin{pmatrix} 1 & -2 \\ 3 & 2 \end{pmatrix} \right)\\
&=& \begin{pmatrix} 26 & 52 \\ 6 & 12 \\ 14 & 28 \end{pmatrix} + \begin{pmatrix} 26 & 4 \\ 6 & -20 \\ 14 & -12 \end{pmatrix} \\
&=& \begin{pmatrix} 52 & 56 \\ 12 & -8 \\ 28 & 16 \end{pmatrix}
\end{alignat*}

Beide Seiten der Gleichung ergeben dasselbe Ergebnis, somit ist das Distributivgesetz mit den angegebenen Matrizen bestätigt.

\subsection{} %b
Um die Gleichung $(AB)^{T} = B^{T}A^{T}$ zu bestätigen, wird zunächst $(AB)^{T}$ berechnet und anschließend $B^{T}A^{T}$. Ergeben beide Berechnungen das gleiche Ergebnis, so ist die Gleichung damit bestätigt.

\begin{alignat*}{2}
(AB)^{T} &=& \left( \begin{pmatrix} 1 & 3 \\ 2 & 6 \end{pmatrix} \cdot \begin{pmatrix} 2 & -1 & 5 \\ 3 & 2 & 4 \end{pmatrix} \right)^{T} \\
&=& \begin{pmatrix} 11 & 5 & 17 \\ 22 & 10 & 34 \end{pmatrix}^{T} \\
&=& \begin{pmatrix} 11 & 22 \\ 5 & 10 \\ 17 & 34 \end{pmatrix}
\end{alignat*}

Nun folgt die Berechnung des zweiten Teils.

\begin{alignat}{2}
B^{T}A^{T} &=& \begin{pmatrix} 2 & -1 & 5 \\ 3 & 2 & 4 \end{pmatrix}^{T} \cdot \begin{pmatrix} 1 & 3 \\ 2 & 6 \end{pmatrix}^{T} \notag\\
&=& \begin{pmatrix} 2 & 3 \\ -1 & 2 \\ 5 & 4 \end{pmatrix} \cdot \begin{pmatrix} 1 & 2 \\ 3 & 6 \end{pmatrix} \label{eq:transponiert} \\
&=& \begin{pmatrix} 11 & 22 \\ 5 & 10 \\ 17 & 34 \end{pmatrix} \notag
\end{alignat}

Beide Seiten der Gleichung ergeben das gleiche Ergebnis, somit ist die Gleichung $(AB)^{T} = B^{T}A^{T}$ bestätigt.

\subsection{} %c
Wie in \eqref{eq:transponiert} sichtbar, hat $A^{T}$ weniger Spalten als $B^{T}$ Zeilen hat. Somit kann $A^{T}B^{T}$ gar nicht berechnet werden.
\section{} %3
Beweis des Distributivgesetzes $A(B_{1} + B_{2}) = AB_{1} + AB_{2}$.

Es werden die beiden Seiten der Gleichung zunächst einzeln gerechnet und im Anschluss verglichen. Bei gleichem Ergebnis stimmt die Aussage.

Es seien $A,B_{1},B_{2}$ folgende Matrizen:
\begin{alignat*}{2}
A &=& \begin{pmatrix} a_{11} & a_{12} & a_{13} \\ a_{21} & a_{22} & a_{23} \\ a_{31} & a_{32} & a_{33} \end{pmatrix} \\
B_{1} = B &=& \begin{pmatrix} b_{11} & b_{12} & b_{13} \\ b_{21} & b_{22} & b_{23} \\ b_{31} & b_{32} & b_{33} \end{pmatrix} \\
B_{2} = C &=& \begin{pmatrix} c_{11} & c_{12} & c_{13} \\ c_{21} & c_{22} & c_{23} \\ c_{31} & c_{32} & c_{33} \end{pmatrix} \\
\intertext{Berechnung des linken Teils der Gleichung}
B + C &=& \begin{pmatrix} b_{11} + c_{11} & b_{12} + c_{12} & b_{13} + c_{13} \\ b_{21} + c_{21} & b_{22} + c_{22} & b_{23} + c_{23} \\ b_{31} + c_{31} & b_{32} + c_{32} & b_{33} + c_{33} \end{pmatrix} \\
% todo
A \cdot (B+C) &=& \begin{pmatrix} \sum\limits_{i=1}^{3} a_{1i} \cdot (b_{i1} + c_{i1}) & \sum\limits_{i=1}^{3} a_{1i} \cdot (b_{i2} + c_{i2}) & \sum\limits_{i=1}^{3} a_{1i} \cdot (b_{i3} + c_{i3})  \\ \sum\limits_{i=1}^{3} a_{2i} \cdot (b_{i1} + c_{i1}) & \sum\limits_{i=1}^{3} a_{2i} \cdot (b_{i2} + c_{i2}) & \sum\limits_{i=1}^{3} a_{2i} \cdot (b_{i3} + c_{i3}) \\ \sum\limits_{i=1}^{3} a_{3i} \cdot (b_{i1} + c_{i1}) & \sum\limits_{i=1}^{3} a_{3i} \cdot (b_{i2} + c_{i2}) & \sum\limits_{i=1}^{3} a_{3i} \cdot (b_{i3} + c_{i3}) \end{pmatrix} \\
\intertext{Berechnung des rechten Teils der Gleichung}
AB &=& \begin{pmatrix} \sum\limits_{i=1}^{3} a_{1i} \cdot b_{i1} & \sum\limits_{i=1}^{3} a_{1i} \cdot b_{i2} & \sum\limits_{i=1}^{3} a_{1i} \cdot b_{i3} \\ \sum\limits_{i=1}^{3} a_{2i} \cdot b_{i1} & \sum\limits_{i=1}^{3} a_{2i} \cdot b_{i2} & \sum\limits_{i=1}^{3} a_{2i} \cdot b_{i3} \\ \sum\limits_{i=1}^{3} a_{3i} \cdot b_{i1} & \sum\limits_{i=1}^{3} a_{3i} \cdot b_{i2} & \sum\limits_{i=1}^{3} a_{3i} \cdot b_{i3} \end{pmatrix} \\
AC &=& \begin{pmatrix} \sum\limits_{i=1}^{3} a_{1i} \cdot c_{i1} & \sum\limits_{i=1}^{3} a_{1i} \cdot c_{i2} & \sum\limits_{i=1}^{3} a_{1i} \cdot c_{i3} \\ \sum\limits_{i=1}^{3} a_{2i} \cdot c_{i1} & \sum\limits_{i=1}^{3} a_{2i} \cdot c_{i2} & \sum\limits_{i=1}^{3} a_{2i} \cdot c_{i3} \\ \sum\limits_{i=1}^{3} a_{3i} \cdot c_{i1} & \sum\limits_{i=1}^{3} a_{3i} \cdot c_{i2} & \sum\limits_{i=1}^{3} a_{3i} \cdot c_{i3} \end{pmatrix} \\
AB + AC &=& \begin{pmatrix} \sum\limits_{i=1}^{3} a_{1i} \cdot b_{i1} + a_{1i} \cdot c_{i1} & \sum\limits_{i=1}^{3} a_{1i} \cdot b_{i2} + a_{1i} \cdot c_{i2} & \sum\limits_{i=1}^{3} a_{1i} \cdot b_{i3} + a_{1i} \cdot c_{i3} \\ \sum\limits_{i=1}^{3} a_{2i} \cdot b_{i1} + a_{2i} \cdot c_{i1} & \sum\limits_{i=1}^{3} a_{2i} \cdot b_{i2} + a_{2i} \cdot c_{i2} & \sum\limits_{i=1}^{3} a_{2i} \cdot b_{i3} + a_{2i} \cdot c_{i3} \\ \sum\limits_{i=1}^{3} a_{3i} \cdot b_{i1} + a_{3i} \cdot c_{i1} & \sum\limits_{i=1}^{3} a_{3i} \cdot b_{i2} + a_{3i} \cdot c_{i2} & \sum\limits_{i=1}^{3} a_{3i} \cdot b_{i3} + a_{3i} \cdot c_{i3} \end{pmatrix} \\
\intertext{Durch Ausklammern des a ergibt sich:}
&=& \begin{pmatrix} \sum\limits_{i=1}^{3} a_{1i} \cdot (b_{i1} + c_{i1}) & \sum\limits_{i=1}^{3} a_{1i} \cdot (b_{i2} + c_{i2}) & \sum\limits_{i=1}^{3} a_{1i} \cdot (b_{i3} + c_{i3})  \\ \sum\limits_{i=1}^{3} a_{2i} \cdot (b_{i1} + c_{i1}) & \sum\limits_{i=1}^{3} a_{2i} \cdot (b_{i2} + c_{i2}) & \sum\limits_{i=1}^{3} a_{2i} \cdot (b_{i3} + c_{i3}) \\ \sum\limits_{i=1}^{3} a_{3i} \cdot (b_{i1} + c_{i1}) & \sum\limits_{i=1}^{3} a_{3i} \cdot (b_{i2} + c_{i2}) & \sum\limits_{i=1}^{3} a_{3i} \cdot (b_{i3} + c_{i3}) \end{pmatrix}
\end{alignat*}

Auf beiden Seiten kommt das gleiche Ergebnis heraus. Also ist das Distributivgesetz $A(B_{1} + B_{2}) = AB_{1} + AB_{2}$ damit bewiesen. \hfill $\Box$
\section{} %4
\subsection{} %a
Beweis der Aussage (6), Skript Seite 61. Es ist zu zeigen, dass für jede Abbildung $f: A \rightarrow B$ und jedes $B' \subseteq B$ Folgendes gilt: 
\begin{equation*}
f(f^{-1}(B')) \subseteq B'
\end{equation*}

%todo, siehe Tutorium Di
Es gelte $b \in f(f^{-1}(B'))$. Daraus ergeben sich zwei mögliche Fälle:

\begin{alignat*}{1}
\intertext{Fall 1}
\exists a \in f^{-1}(B'): f(a)=b \\
\Rightarrow f(a)=b \in B'
\intertext{Fall 2}
\nexists a \in f^{-1}(B'): f(a)=b
\end{alignat*}

Anhand von Fall 1 ist klar, dass es für jedes Urbild von einem $b \in B'$ ein Bild $b' \in B'$ gibt. Fall 2 behandelt die $b \in B'$, die kein Urbild haben. Die Menge $f(f^{-1}(B'))$ enthält nur solche $b \in B'$, die ein Urbild haben und ist demzufolge eine Teilmenge von $B'$.

Also gilt die Aussage (6), Skript Seite 61. \hfill $\Box$

\subsection{} %b
Es gebe die Mengen $A$ und $B'$ mit $A = \{1\}$ und $B' = \{5, 6\}$. Es gebe die folgende Abbildungsvorschrift $f(1) = 5$.

Die Urbildmenge von $B'$ ist in diesem Fall $A$. Die Bildmenge von $A$ ist in diesem Fall $\{5\}$. Es gilt $\{5\} \neq B'$.
\end{document}
