\documentclass[10pt,a4paper,oneside,ngerman,numbers=noenddot]{scrartcl}
\usepackage[T1]{fontenc}
\usepackage[utf8]{inputenc}
\usepackage[ngerman]{babel}
\usepackage{amsmath}
\usepackage{amsfonts}
\usepackage{amssymb}
\usepackage{paralist}
\usepackage[locale=DE,exponent-product=\cdot,detect-all]{siunitx}
\usepackage{tikz}
\usetikzlibrary{matrix,fadings,calc,positioning,decorations.pathreplacing,arrows}
\pagenumbering{arabic}
\def\thesection{3.\arabic{section})}
\def\thesubsection{(\alph{subsection})}
\def\thesubsubsection{(\arabic{subsubsection})}
\renewcommand{\labelenumi}{(\roman{enumi})}
\hyphenation{Nach-komma-stel-len}

\begin{document}
\author{Jim Martens (Matrikelnummer 6420323) \and Marlo Kornblum (Matrikelnummer 6427301)}
\title{Rechnerstrukturen Aufgabenblatt 3}
\maketitle

\section{}%3.1
\subsection{}
\begin{alignat}{3}
&\; 1385_{10} &-&\: 532_{10} &=&\: x\\
\intertext{Bildung des 10-Komplements des Subtrahenden:}
\Leftrightarrow &\; 1385_{10} &+&\: K_{10}(532)_{10} &=&\: x\\
\Leftrightarrow &\; 1385_{10} &+&\: 9467_{10} &=&\: 10852_{10}\\
\intertext{Streichen der führenden Eins}
\Leftrightarrow &\; 1385_{10} &-&\: 532_{10} &=&\: \underline{\underline{0852_{10}}}
\end{alignat}
\subsection{}
\begin{alignat}{3}
&\; 372_{10} &-&\: 687_{10} &=&\: x\\
\intertext{Bildung des 10-Komplements des Subtrahenden:}
\Leftrightarrow &\; 372_{10} &+&\: K_{10}(0687)_{10} &=&\: x\\
\Leftrightarrow &\; 372_{10} &+&\: 9313_{10} &=&\: 9685_{10}\\
\intertext{Bildung des 10-Komplements des Ergebnisses}
\Leftrightarrow &\; K_{10}(9685)_{10} &=&\: \underline{\underline{-0315_{10}}}
\end{alignat}
\subsection{}
\begin{alignat}{3}
&\; 1385_{10} &-&\: 532_{10} &=&\: x\\
\intertext{Umwandlung in Dualzahlen:}
&\; 10101101001_{2} &-&\: 1000010100_{2} &=&\: x\\
\intertext{Bildung des 2-Komplements des Subtrahenden:}
\Leftrightarrow &\; 10101101001_{2} &+&\: K_{2}(001000010100)_{2} &=&\: x\\
\Leftrightarrow &\; 10101101001_{2} &+&\: 110111101100_{2} &=&\: 101101010101_{2}\\
\intertext{Streichen der führenden Eins}
\Leftrightarrow &\; 10101101001_{2} &-&\: 001000010100_{2} &=&\: \underline{\underline{001101010101_{2}}}
\end{alignat}
\subsection{}
\begin{alignat}{3}
&\; 372_{10} &-&\: 687_{10} &=&\: x\\
\intertext{Umwandlung in Dualzahlen:}
&\; 101110100_{2} &-&\: 1010101111_{2} &=&\: x\\
\intertext{Bildung des 2-Komplements:}
\Leftrightarrow &\; 101110100_{2} &+&\: K_{2}(001010101111)_{2} &=&\: x\\
\Leftrightarrow &\; 000101110100_{2} &+&\: 110101010001_{2} &=&\: 111011000101_{2}\\
\intertext{Bildung des 2-Komplements des Ergebnisses}
\Leftrightarrow &\; K_{2}(111011000101)_{2} &=&\: \underline{\underline{000100111011_{2}}}
\end{alignat}
\section{}%3.2 _todo
\subsection{}
$(6,9242 \mid 4)_{10}$
\subsection{}
$(-1,100101 \mid -10)_{2}$
\subsection{}
$(-2,D4A \mid B)_{16}$
\section{}%3.3
\subsection{}
\begin{tikzpicture}
\draw (0,0) -- +(8,0); %untere Kante
\draw (0,0) -- +(0,0.5); %linke Kante
\draw (0,0.5) -- +(8,0); %obere Kante
\draw (8,0) -- +(0,0.5); %rechte Kante
\draw (0.25,0) -- +(0,0.5); %rechte Kante von Vorzeichen
\draw (2.25,0) -- +(0,0.5); %rechte Kante des Exponenten
\node at ++(0.15,0.2) (sign) {$0$}; %Vorzeichen
\node at ++(1.25,0.2) (exp) {$0000\, 0110$}; %Exponent
\node at ++(5.125,0.2) (mantisse) {$011\, 0110\, 0000\, 0000\, 0000\, 0000$}; %Mantisse
\end{tikzpicture}
\subsection{}
\begin{tikzpicture}
\draw (0,0) -- +(8,0); %untere Kante
\draw (0,0) -- +(0,0.5); %linke Kante
\draw (0,0.5) -- +(8,0); %obere Kante
\draw (8,0) -- +(0,0.5); %rechte Kante
\draw (0.25,0) -- +(0,0.5); %rechte Kante von Vorzeichen
\draw (2.25,0) -- +(0,0.5); %rechte Kante des Exponenten
\node at ++(0.15,0.2) (sign) {$1$}; %Vorzeichen
\node at ++(1.25,0.2) (exp) {$0000\, 0111$}; %Exponent
\node at ++(5.125,0.2) (mantisse) {$010\, 1000\, 1010\, 0000\, 0000\, 0000$}; %Mantisse
\end{tikzpicture}
\section{}%3.4
$8,626 \cdot 10^{5} + 9,9442 \cdot 10^{7}$\\

\subsection{}
Skalierung des kleineren Summanden, bis beide Exponenten gleich sind:
$0,08626 \cdot 10^{7} + 9,9442 \cdot 10^{7}$.\\
Daraus folgt:
\begin{alignat}{2}
&\; 0,08626 \cdot 10^{7} + 9,9442 \cdot 10^{7} &=&\: x\\
\Leftrightarrow &\; 0,08626 \cdot 10^{7} + 9,94420 \cdot 10^{7} &=&\: 10,03046 \cdot 10^{7}
\end{alignat}
Daraus ergibt sich dieses normalisierte Ergebnis:\\
$1,003046 \cdot 10^{8}$\\
Gerundet ergibt sich:\\
$1,0030 \cdot 10^{8}$\\
\subsection{}
Skalierung des kleineren Summanden, bis beide Exponenten gleich sind:
$0,0863 \cdot 10^{7} + 9,9442 \cdot 10^{7}$.\\
Daraus folgt:
\begin{alignat}{2}
&\; 0,0863 \cdot 10^{7} + 9,9442 \cdot 10^{7} &=&\: x\\
\Leftrightarrow &\; 0,0863 \cdot 10^{7} + 9,9442 \cdot 10^{7} &=&\: 10,0305 \cdot 10^{7}
\end{alignat}
Daraus ergibt sich dieses normalisierte Ergebnis:\\
$1,00305 \cdot 10^{8}$\\
Gerundet ergibt sich:\\
$1,0031 \cdot 10^{8}$\\
\subsection{}
Bei Zahlen mit vielen Nachkommastellen ist eine Rundung nach jedem Schritt vorteilhafter, weil dadurch kompletter Präzisionsverlust vermieden werden kann (mehr Nachkommastellen als sinnvoll darstellbar).

Bei Zahlen mit wenigen Nachkommastellen eignet sich das Verfahren mit einmaliger Rundung am Ende besser, da hierbei ohne Genauigkeitsverlust (zu viele Nachkommastellen) ein relativ gutes Ergebnis ermittelt werde kann.

Anhand des Beispiels wird sichtbar, dass die Entscheidung Auswirkungen auf das Ergebnis haben können (in diesem Fall im Wert von $10000$), welche je nach Kontext trivial sind oder bereits katastrophale Ausmaße annehmen können.

Von daher hängt die Wahl des "`perfekten"' Verfahrens davon ab, was man machen muss. 
\section{}%3.5
$5,6538 \cdot 10^{7} * 3,1415 \cdot 10^{4}$\\
Daraus ergibt sich:\\
\begin{alignat}{2}
&\; 5,6538 \cdot 10^{7} * 3,1415 \cdot 10^{4} &=&\: x\\
\Leftrightarrow &\; 5,6538 \cdot 10^{7} * 3,1415 \cdot 10^{4} &=& (5,6538 \cdot 3,1415) \cdot 10^{7+4}\\
\Leftrightarrow &\; 5,6538 \cdot 10^{7} * 3,1415 \cdot 10^{4} &=& 17,7614127 \cdot 10^{11}
\end{alignat}
Daraus ergibt sich dieses normalisierte Ergebnis:\\
$1,77614127 \cdot 10^{12}$\\
Gerundet ergibt sich:\\
$1,7761 \cdot 10^{12}$
\end{document}