\documentclass[10pt,a4paper,oneside,ngerman,numbers=noenddot]{scrartcl}
\usepackage[T1]{fontenc}
\usepackage[utf8]{inputenc}
\usepackage[ngerman]{babel}
\usepackage{amsmath}
\usepackage{amsfonts}
\usepackage{amssymb}
\usepackage{paralist}
\usepackage{gauss}
\usepackage{pgfplots}
\usepackage[locale=DE,exponent-product=\cdot,detect-all]{siunitx}
\usepackage{tikz}
\usetikzlibrary{matrix,fadings,calc,positioning,decorations.pathreplacing,arrows,decorations.markings}
\usepackage{polynom}
\polyset{style=C, div=:,vars=x}
\pgfplotsset{compat=1.8}
\pagenumbering{arabic}
\def\thesection{\arabic{section})}
\def\thesubsection{\alph{subsection})}
\def\thesubsubsection{(\roman{subsubsection})}
\makeatletter
\renewcommand*\env@matrix[1][*\c@MaxMatrixCols c]{%
  \hskip -\arraycolsep
  \let\@ifnextchar\new@ifnextchar
  \array{#1}}
\makeatother

\begin{document}
\author{Jim Martens (6420323)}
\title{Hausaufgaben zum 6. Juni}
\maketitle
\section{} %1
\subsubsection{} %i
\begin{alignat*}{2}
\int \frac{x + 1}{x^{2} - x +6}\,dx &=& \int \frac{x}{x^{2} - x + 6}\,dx + \int \frac{1}{x^{2} -x + 6}\,dx \\
&=& \frac{1}{2}\int \frac{2x - 1 + 1}{x^{2} - x + 6}\,dx + \int \frac{1}{x^{2} -x + 6}\,dx \\
&=& \frac{1}{2}\int \frac{2x - 1}{x^{2} - x + 6}\,dx + \frac{1}{2}\int \frac{1}{x^{2} - x + 6}\,dx + \int \frac{1}{x^{2} -x + 6}\,dx \\
&=& \frac{1}{2} \cdot \ln |x^{2} - x + 6| + \frac{1}{2}\int \frac{1}{x^{2} - x + 6}\,dx + \int \frac{1}{x^{2} -x + 6}\,dx
\end{alignat*}
\begin{alignat*}{2}
x^{2} - x + 6 &=& x^{2} - x + \frac{1}{4} + 6 - \frac{1}{4} \\
&=& \left( x - \frac{1}{2}\right)^{2} + 6 - \frac{1}{4}
\end{alignat*}
\begin{alignat*}{2}
\intertext{Setze $c = 6 - \frac{1}{4}$. Substitution: $t = x - \frac{1}{2}$, $\frac{dt}{dx} = 1$ und $dx = dt$.}
&& \frac{1}{2} \cdot \ln |x^{2} - x + 6| + \frac{1}{2}\int \frac{1}{x^{2} - x + 6}\,dx + \int \frac{1}{x^{2} -x + 6}\,dx\\
&=& \frac{1}{2} \cdot \ln |x^{2} - x + 6| + \frac{1}{2}\int \frac{1}{\left( x - \frac{1}{2}\right)^{2} + c}\,dx + \int \frac{1}{\left( x - \frac{1}{2}\right)^{2} + c}\,dx \\
&=& \frac{1}{2} \cdot \ln |x^{2} - x + 6| + \frac{1}{2}\int \frac{1}{t^{2} + c}\,dt + \int \frac{1}{t^{2} + c}\,dt
\end{alignat*}
\begin{alignat*}{2}
\intertext{Bestimmung Nullstellen des Nennerpolynoms}
0 &=& x^{2} - x + 6 \\
x_{1,2} &=& \frac{1}{2} \pm \sqrt{\left(\frac{1}{2}\right)^{2} - 6} \\
&=& \frac{1}{2} \pm \sqrt{- \frac{23}{4}}
\end{alignat*}
\begin{alignat*}{2}
\intertext{Eine negative Wurzel kann nicht gelöst werden. Demnach gibt es keine Nullstellen. Setzen von $r = \sqrt{c}$. Dann ist $c = r^{2}$.}
&& \frac{1}{2} \cdot \ln |x^{2} - x + 6| + \frac{1}{2}\int \frac{1}{x^{2} - x + 6}\,dx + \int \frac{1}{x^{2} -x + 6}\,dx\\
&=& \frac{1}{2} \cdot \ln |x^{2} - x + 6| + \frac{1}{2}\int \frac{1}{t^{2} + r^{2}}\,dt + \int \frac{1}{t^{2} + r^{2}}\,dt \\
&=& \frac{1}{2} \cdot \ln |x^{2} - x + 6| + \frac{1}{2} \cdot \frac{1}{r} \cdot \arctan \left(\frac{t}{r} \right) + \frac{1}{r} \cdot \arctan \left(\frac{t}{r} \right) \\
&=& \frac{1}{2} \cdot \ln |x^{2} - x + 6| + \frac{1}{2} \cdot \frac{1}{\sqrt{c}} \cdot \arctan \left(\frac{t}{\sqrt{c}} \right) + \frac{1}{\sqrt{c}} \cdot \arctan \left(\frac{t}{\sqrt{c}} \right) \\
\intertext{Resubsitution}
&& \frac{1}{2} \cdot \ln |x^{2} - x + 6| + \frac{1}{2}\int \frac{1}{x^{2} - x + 6}\,dx + \int \frac{1}{x^{2} -x + 6}\,dx \\ 
&=& \frac{1}{2} \cdot \ln |x^{2} - x + 6| + \frac{1}{2} \cdot \frac{1}{\sqrt{c}} \cdot \arctan \left(\frac{x - \frac{1}{2}}{\sqrt{c}} \right) + \frac{1}{\sqrt{c}} \cdot \arctan \left(\frac{x - \frac{1}{2}}{\sqrt{c}} \right)
\end{alignat*}
\begin{alignat*}{2}
\intertext{Probe:}
&& \left(\frac{1}{2} \cdot \ln |x^{2} - x + 6| + \frac{1}{2} \cdot \frac{1}{\sqrt{c}} \cdot \arctan \left(\frac{x - \frac{1}{2}}{\sqrt{c}} \right) + \frac{1}{\sqrt{c}} \cdot \arctan \left(\frac{x - \frac{1}{2}}{\sqrt{c}} \right)\right)' \\
&=& \left(\frac{1}{2} \cdot \ln |x^{2} - x + 6|\right)' + \left(\frac{1}{2} \cdot \frac{1}{\sqrt{c}} \cdot \arctan \left(\frac{x - \frac{1}{2}}{\sqrt{c}} \right)\right)' + \left(\frac{1}{\sqrt{c}} \cdot \arctan \left(\frac{x - \frac{1}{2}}{\sqrt{c}} \right)\right)' \\
&=& \frac{1}{2} \cdot \frac{2x - 1}{x^{2} - x + 6} + \frac{1}{2} \cdot \frac{1}{\sqrt{c}} \cdot \frac{1}{\left(\frac{x - \frac{1}{2}}{\sqrt{c}} \right)^{2} + 1} \cdot \left(\frac{x - \frac{1}{2}}{\sqrt{c}} \right)' + \frac{1}{\sqrt{c}} \cdot \frac{1}{\left(\frac{x - \frac{1}{2}}{\sqrt{c}} \right)^{2} + 1} \cdot \left(\frac{x - \frac{1}{2}}{\sqrt{c}} \right)' \\
&=& \frac{1}{2} \cdot \frac{2x - 1}{x^{2} - x + 6} + \frac{1}{2} \cdot \frac{1}{\sqrt{c}} \cdot \frac{1}{\sqrt{c}} \cdot \frac{1}{\frac{\left(x - \frac{1}{2}\right)^{2}}{c} + 1} \cdot \left(x - \frac{1}{2}\right)' + \frac{1}{\sqrt{c}} \cdot \frac{1}{\sqrt{c}} \cdot \frac{1}{\frac{\left(x - \frac{1}{2}\right)^{2}}{c} + 1} \cdot \left(x - \frac{1}{2}\right)' \\
&=& \frac{1}{2} \cdot \frac{2x - 1}{x^{2} - x + 6} + \frac{1}{2} \cdot \frac{1}{c} \cdot \frac{1}{\frac{\left(x - \frac{1}{2}\right)^{2}}{c} + 1} + \frac{1}{c} \cdot \frac{1}{\frac{\left(x - \frac{1}{2}\right)^{2}}{c} + 1} \\
&=& \frac{1}{2} \cdot \frac{2x - 1}{x^{2} - x + 6} + \frac{1}{2} \cdot \frac{1}{c\left(\frac{\left(x - \frac{1}{2}\right)^{2}}{c} + 1\right)} + \frac{1}{c\left(\frac{\left(x - \frac{1}{2}\right)^{2}}{c} + 1\right)} \\
&=& \frac{1}{2} \cdot \frac{2x - 1}{x^{2} - x + 6} + \frac{1}{2} \cdot \frac{1}{\left(x - \frac{1}{2}\right)^{2} + c} + \frac{1}{\left(x - \frac{1}{2}\right)^{2} + c} \\
&=& \frac{1}{2} \cdot \frac{2x - 1}{x^{2} - x + 6} + \frac{1}{2} \cdot \frac{1}{x^{2} - x + \frac{1}{4} + 6 - \frac{1}{4}} + \frac{1}{x^{2} - x + \frac{1}{4} + 6 - \frac{1}{4}} \\
&=& \frac{1}{2} \cdot \frac{2x - 1}{x^{2} - x + 6} + \frac{1}{2} \cdot \frac{1}{x^{2} - x + 6} + \frac{1}{x^{2} - x + 6} \\
&=& \frac{1}{2} \cdot \frac{2x - 1 + 1}{x^{2} - x + 6} + \frac{1}{x^{2} - x + 6} \\
&=& \frac{1}{2} \cdot \frac{2x}{x^{2} - x + 6} + \frac{1}{x^{2} - x + 6} \\
&=& \frac{x}{x^{2} - x + 6} + \frac{1}{x^{2} - x + 6} \\
&=& \frac{x + 1}{x^{2} - x + 6}
\end{alignat*}
\subsubsection{} %ii
\begin{alignat*}{2}
0 &=& x^{2} - 4x + 4 \\
x_{1,2} &=& 2 \pm \sqrt{2^{2} - 4} \\
&=& 2 \pm \sqrt{0} \\
x_{1} = x_{2} &=& 2 \\
x^{2} - 4x + 4 &=& (x-2)(x-2) = (x-2)^{2}
\end{alignat*}
\begin{alignat*}{2}
\frac{2x + 1}{x^{2} -4x +4} = \frac{2x + 1}{(x-2)^{2}} &=& \frac{A}{(x-2)^{2}} + \frac{B}{x-2} \\
2x + 1 &=& \frac{A(x-2)^{2}}{(x-2)^{2}} + \frac{B(x-2)^{2}}{x-2} \\
&=& A + B(x-2) \\
&=& A + Bx -2B \\
&=& Bx - 2B + A
\end{alignat*}
\begin{alignat*}{2}
\intertext{Vergleichen der Koeffizienten}
B &=& 2 \\
-2B + A &=& 1 \\
-2 \cdot 2 + A &=& 1 \\
-4 + A &=& 1 \\
A &=& 5
\end{alignat*}
\begin{alignat*}{2}
\intertext{Partialbruchzerlegung:}
\frac{2x+1}{x^{2}-4x+4} &=& \frac{5}{(x-2)^{2}} + \frac{2}{x-2}
\end{alignat*}
\begin{alignat*}{2}
\intertext{Integrieren}
&& \int \frac{2x+1}{x^{2}-4x+4}\,dx \\
&=& \int \left(\frac{5}{(x-2)^{2}} + \frac{2}{x-2}\right)\,dx \\
&=& \int \frac{5}{(x-2)^{2}}\,dx + \int \frac{2}{x-2}\,dx \\
&=& 5 \int (x-2)^{-2}\,dx + 2 \int \frac{1}{x-2}\,dx \\
&=& -5 (x-2)^{-1} + 2 \cdot \ln |x-2|
\end{alignat*}
\begin{alignat*}{2}
\intertext{Probe}
&& \left(-5 (x-2)^{-1} + 2 \cdot \ln |x-2|\right)' \\
&=& \left(-5 (x-2)^{-1}\right)' + \left(2 \cdot \ln |x-2|\right)' \\
&=& 5(x-2)^{-2} + \frac{2}{x-2} \\
&=& \frac{5}{(x-2)^{2}} + \frac{2}{x-2} \\
&=& \frac{5 + 2(x-2)}{(x-2)^{2}} \\
&=& \frac{5 + 2x -4)}{(x-2)^{2}} \\
&=& \frac{2x + 1}{x^{2} -4x +4}
\end{alignat*}
\subsubsection{} %iii
\begin{alignat*}{2}
\int \frac{4x + 1}{x^{2} + 4x +8}\,dx &=& \int \frac{4x}{x^{2} + 4x + 8}\,dx + \int \frac{1}{x^{2} + 4x + 8}\,dx \\
&=& 2\int \frac{2x +4 - 4}{x^{2} + 4x + 8}\,dx + \int \frac{1}{x^{2} + 4x + 8}\,dx \\
&=& 2\int \frac{2x + 4}{x^{2} + 4x + 8}\,dx - 8\int \frac{1}{x^{2} + 4x + 8}\,dx + \int \frac{1}{x^{2} + 4x + 8}\,dx \\
&=& 2 \cdot \ln |x^{2} + 4x + 8| - 8\int \frac{1}{x^{2} + 4x + 8}\,dx + \int \frac{1}{x^{2} + 4x + 8}\,dx
\end{alignat*}
\begin{alignat*}{2}
x^{2} + 4x + 8 &=& x^{2} + 4x + \frac{16}{4} + 8 - \frac{16}{4} \\
&=& (x +2)^{2} + 4
\end{alignat*}
\begin{alignat*}{2}
\intertext{Setze $c = 4$. Substitution: $t = x + 2$, $\frac{dt}{dx} = 1$ und $dx = dt$.}
&& 2 \cdot \ln |x^{2} + 4x + 8| - 8\int \frac{1}{x^{2} + 4x + 8}\,dx + \int \frac{1}{x^{2} + 4x + 8}\,dx\\
&=& 2 \cdot \ln |x^{2} + 4x + 8| -8\int \frac{1}{(x+2)^{2} + c}\,dx + \int \frac{1}{(x+2)^{2} + c}\,dx \\
&=& 2 \cdot \ln |x^{2} + 4x + 8| -8\int \frac{1}{t^{2} + c}\,dt + \int \frac{1}{t^{2} + c}\,dt
\end{alignat*}
\begin{alignat*}{2}
\intertext{Bestimmung Nullstellen des Nennerpolynoms}
0 &=& x^{2} + 4x + 8 \\
x_{1,2} &=& -2 \pm \sqrt{(2)^{2} - 8} \\
&=& -2 \pm \sqrt{-4}
\end{alignat*}
\begin{alignat*}{2}
\intertext{Eine negative Wurzel kann nicht gelöst werden. Demnach gibt es keine Nullstellen. Setzen von $r = \sqrt{c}$. Dann ist $c = r^{2}$.}
&& 2 \cdot \ln |x^{2} + 4x + 8| - 8\int \frac{1}{x^{2} + 4x + 8}\,dx + \int \frac{1}{x^{2} + 4x + 8}\,dx\\
&=& 2 \cdot \ln |x^{2} + 4x + 8| -8\int \frac{1}{t^{2} + r^{2}}\,dt + \int \frac{1}{t^{2} + r^{2}}\,dt \\
&=& 2 \cdot \ln |x^{2} + 4x + 8| -8 \cdot \frac{1}{r} \cdot \arctan \left(\frac{t}{r} \right) + \frac{1}{r} \cdot \arctan \left(\frac{t}{r} \right) \\
&=& 2 \cdot \ln |x^{2} + 4x + 8| -8 \cdot \frac{1}{\sqrt{c}} \cdot \arctan \left(\frac{t}{\sqrt{c}} \right) + \frac{1}{\sqrt{c}} \cdot \arctan \left(\frac{t}{\sqrt{c}} \right) \\
\intertext{Resubsitution}
&& 2 \cdot \ln |x^{2} + 4x + 8| - 8\int \frac{1}{x^{2} + 4x + 8}\,dx + \int \frac{1}{x^{2} + 4x + 8}\,dx \\ 
&=& 2 \cdot \ln |x^{2} + 4x + 8| -8 \cdot \frac{1}{\sqrt{c}} \cdot \arctan \left(\frac{x+2}{\sqrt{c}} \right) + \frac{1}{\sqrt{c}} \cdot \arctan \left(\frac{x+2}{\sqrt{c}} \right)
\end{alignat*}
\begin{alignat*}{2}
\intertext{Probe:}
&& \left(2 \cdot \ln |x^{2} + 4x + 8| -8 \cdot \frac{1}{\sqrt{c}} \cdot \arctan \left(\frac{x+2}{\sqrt{c}} \right) + \frac{1}{\sqrt{c}} \cdot \arctan \left(\frac{x+2}{\sqrt{c}} \right)\right)' \\
&=& \left(2 \cdot \ln |x^{2} + 4x + 8|\right)' + \left(-6 \cdot \frac{1}{\sqrt{c}} \cdot \arctan \left(\frac{x+2}{\sqrt{c}} \right)\right)' + \left(\frac{1}{\sqrt{c}} \cdot \arctan \left(\frac{x+2}{\sqrt{c}} \right)\right)' \\
&=& 2 \cdot \frac{2x + 4}{x^{2} + 4x + 8} - 8 \cdot \frac{1}{\sqrt{c}} \cdot \frac{1}{\left(\frac{x + 2}{\sqrt{c}} \right)^{2} + 1} \cdot \left(\frac{x + 2}{\sqrt{c}} \right)' + \frac{1}{\sqrt{c}} \cdot \frac{1}{\left(\frac{x + 2}{\sqrt{c}} \right)^{2} + 1} \cdot \left(\frac{x + 2}{\sqrt{c}} \right)' \\
&=& 2 \cdot \frac{2x + 4}{x^{2} + 4x + 8} - 8 \cdot \frac{1}{\sqrt{c}} \cdot \frac{1}{\sqrt{c}} \cdot \frac{1}{\frac{(x + 2)^{2}}{c} + 1} \cdot (x + 2)' + \frac{1}{\sqrt{c}} \cdot \frac{1}{\sqrt{c}} \cdot \frac{1}{\frac{(x + 2)^{2}}{c} + 1} \cdot (x + 2)' \\
&=& 2 \cdot \frac{2x + 4}{x^{2} + 4x + 8} - 8 \cdot \frac{1}{c} \cdot \frac{1}{\frac{(x + 2)^{2}}{c} + 1} + \frac{1}{c} \cdot \frac{1}{\frac{(x + 2)^{2}}{c} + 1} \\
&=& 2 \cdot \frac{2x + 4}{x^{2} + 4x + 8} - 8 \cdot \frac{1}{c\left(\frac{(x + 2)^{2}}{c} + 1\right)} + \frac{1}{c\left(\frac{(x + 2)^{2}}{c} + 1\right)} \\
&=& 2 \cdot \frac{2x + 4}{x^{2} + 4x + 8} - 8 \cdot \frac{1}{(x + 2)^{2} + c} + \frac{1}{(x + 2)^{2} + c} \\
&=& 2 \cdot \frac{2x + 4}{x^{2} + 4x + 8} - 8 \cdot \frac{1}{x^{2} + 4x + 4 + 4} + \frac{1}{x^{2} + 4x + 4 + 4} \\
&=& 2 \cdot \frac{2x + 4}{x^{2} + 4x + 8} - 2\cdot \frac{4}{x^{2} + 4x + 8} + \frac{1}{x^{2} + 4x + 8} \\
&=& 2 \cdot \frac{2x + 4 - 4}{x^{2} + 4x + 8} + \frac{1}{x^{2} + 4x + 8} \\
&=& 2 \cdot \frac{2x}{x^{2} + 4x + 8} + \frac{1}{x^{2} + 4x + 8} \\
&=& \frac{4x + 1}{x^{2} + 4x + 8}
\end{alignat*}
\section{} %2
\subsection{} %a
\begin{alignat*}{2}
f(x) &=& e^{-x} \\
f'(x) &=& -e^{-x} \\
f''(x) &=& e^{-x} \\
f'''(x) &=& -e^{-x} \\
\intertext{Wendepunkte bestimmen}
f''(x) &=& 0 \\
e^{-x} &=& 0 \\
-x &=& \ln 0 \Rightarrow \text{$\ln 0$ ist nicht definiert, daher kann es keine Wendepunkte für $e^{-x}$ geben.}
\end{alignat*}
\begin{alignat*}{2}
g(x) &=& \frac{1}{1+x} \\
g'(x) &=& -(x+1)^{-2} \\
g''(x) &=& 2(x+1)^{-3} \\
g'''(x) &=& -6(x+1)^{-4} \\
\intertext{Wendepunkte bestimmen}
g''(x) &=& 0 \\
2(x+1)^{-3} &=& 0 \\
\frac{2}{(x+1)^{3}} &=& 0 \Rightarrow \text{Die Funktion wird niemals $0$. Daher kann es keine Wendepunkte geben.}
\end{alignat*}
\begin{alignat*}{2}
h(x) &=& \frac{1}{1+x^{2}} \\
h'(x) &=& -(1+x^{2})^{-2} \cdot 2x \\
h''(x) &=& 2(1+x^{2})^{-3} \cdot 2x \cdot 2x -(1+x^{2})^{-2} \cdot 2 \\
&=& 2(1+x^{2})^{-3} \cdot 4x^{2}  - 2(1+x^{2})^{-2} \\
h'''(x) &=& -6(1+x^{2})^{-4} \cdot 2x \cdot 4x^{2} + 2(1+x^{2})^{-3} \cdot 8x + 4(1+x^{2})^{-3} \cdot 2x \\
&=& -6(1+x^{2})^{-4} \cdot 8x^{3} + 2(1+x^{2})^{-3} \cdot 8x + 4(1+x^{2})^{-3} \cdot 2x \\
\intertext{Wendepunkte bestimmen}
h''(x) &=& 0 \\
2(1+x^{2})^{-3} \cdot 4x^{2}  - 2(1+x^{2})^{-2} &=& 0 \\
\intertext{Es ergibt sich das Ergebnis:}
x = \frac{1}{\sqrt{3}}
\end{alignat*}
\begin{alignat*}{2}
\intertext{Probe}
2(1+\left(\frac{1}{\sqrt{3}}\right)^{2})^{-3} \cdot 4\left(\frac{1}{\sqrt{3}}\right)^{2}  - 2(1+\left(\frac{1}{\sqrt{3}}\right)^{2})^{-2} &=& 0 \\
2(1+\frac{1}{3})^{-3} \cdot 4 \cdot \frac{1}{3}  - 2(1+\frac{1}{3})^{-2} &=& 0 \\
\frac{2 \cdot \frac{4}{3}}{\left(\frac{4}{3}\right)^{3}} - \frac{2}{\left(\frac{4}{3}\right)^{2}} &=& 0 \\
\frac{2 \cdot \frac{4}{3}}{\frac{64}{27}} - \frac{2}{\frac{16}{9}} &=& 0 \\
2 \cdot \frac{4}{3} \cdot \frac{27}{64} - 2 \cdot \frac{9}{16} &=& 0 \\
\frac{4}{3} \cdot \frac{54}{64} - \frac{18}{16} &=& 0 \\
\frac{4}{3} \cdot \frac{27}{32} - \frac{9}{8} &=& 0 \\
\intertext{27 mit 3 kürzen und 32 mit 4 kürzen}
1 \cdot \frac{9}{8} - \frac{9}{8} &=& 0 \\
\frac{9 - 9}{8} &=& 0 \\
0 &=& 0
\end{alignat*}

\begin{tikzpicture}[>=stealth]
\begin{axis}[
	ymin=0,ymax=5,
	x=1cm,
	y=1cm,
	axis x line=middle,
	axis y line=middle,
	axis line style=->,
	xlabel={$x$},
	ylabel={$y$},
	xmin=0,xmax=5
	]

\addplot[no marks, black, -] expression[domain=0:5,samples=100]{e^(-x)} node[pos=0.65,anchor=north]{};
\node at (axis cs: 1,0.8) {f};
\end{axis}
\end{tikzpicture}
\begin{tikzpicture}[>=stealth]
\begin{axis}[
	ymin=0,ymax=5,
	x=1cm,
	y=1cm,
	axis x line=middle,
	axis y line=middle,
	axis line style=->,
	xlabel={$x$},
	ylabel={$y$},
	xmin=0,xmax=5
	]

\addplot[no marks, black, -] expression[domain=0:5,samples=100]{1/(1+x)} node[pos=0.65,anchor=north]{};
\node at (axis cs: 1,0.8) {g};
\end{axis}
\end{tikzpicture}\\
\begin{tikzpicture}[>=stealth]
\begin{axis}[
	ymin=0,ymax=5,
	x=1cm,
	y=1cm,
	axis x line=middle,
	axis y line=middle,
	axis line style=->,
	xlabel={$x$},
	ylabel={$y$},
	xmin=0,xmax=5
	]
\addplot[no marks, black, -] expression[domain=0:5,samples=100]{1/(1+x^2)} node[pos=0.65,anchor=north]{};
\node at (axis cs: 2,0.8) {h};
\draw (axis cs:0.577350269,0.75) circle (2pt);
\end{axis}
\end{tikzpicture}
\subsection{} %b
\begin{alignat*}{2}
\intertext{Integrieren von f}
\int\limits_{0}^{b} e^{-x}\,dx &=& \left[ -e^{-x}\right]_{0}^{b} \\
&=& -e^{-b} + e^{0} \\
&=& -\frac{1}{e^{b}} + 1 \rightarrow 1 \text{ für } b \rightarrow \infty \\
&\Longrightarrow & \lim\limits_{b \rightarrow \infty} \int\limits_{0}^{b} e^{-x}\,dx = 1
\end{alignat*}
\begin{alignat*}{2}
\intertext{Integrieren von g}
\int\limits_{0}^{b} \frac{1}{1+x}\,dx &=& \left[\ln |1+x|\right]_{0}^{b} \\
&=& \ln |1 + b| - \ln |1 + 0| \\
&=& \ln |1 + b| \rightarrow \infty \text{ für } b \rightarrow \infty \\
&\Longrightarrow & \lim\limits_{b \rightarrow \infty} \int\limits_{0}^{b} \frac{1}{1+x}\,dx = \infty
\end{alignat*}
\begin{alignat*}{2}
\intertext{Integrieren von h}
\int\limits_{0}^{b} \frac{1}{1+x^{2}} \,dx &=& \left[ \arctan x \right]_{0}^{b} \\
&=& \arctan b - \arctan 0 \\
&=& \arctan b \rightarrow \frac{\pi}{2} \text{ für } b \rightarrow \infty \\
&\Longrightarrow & \lim\limits_{b \rightarrow \infty} \int\limits_{0}^{b} \frac{1}{1+x^{2}}\,dx = \frac{\pi}{2}
\end{alignat*}
\subsection{} %c
\begin{tikzpicture}[>=stealth]
\begin{axis}[
	ymin=0,ymax=2,
	x=2cm,
	y=2cm,
	axis x line=middle,
	axis y line=middle,
	axis line style=->,
	xlabel={$x$},
	ylabel={$y$},
	xmin=-1,xmax=1
	]
\addplot[no marks, black, -] expression[domain=-1:1,samples=100]{1/sqrt(1-x^2)} node[pos=0.65,anchor=north]{};
\node at (axis cs: 2,0.8) {f};
\end{axis}
\end{tikzpicture}
\begin{alignat*}{2}
\intertext{Integrieren von f}
\int\limits_{-1}^{1} \frac{1}{\sqrt{1-x^{2}}}\,dx &=& \left[\arcsin x\right]_{-1}^{1} \\
&=& \arcsin 1 - \arcsin (-1) \\
&=& \frac{\pi}{2} + \frac{\pi}{2}\\
&=& \pi
\end{alignat*}
\section{} %3
\begin{alignat*}{2}
\intertext{$n=4$}
\int\limits_{0}^{1} \sin x \,dx &\approx & \frac{1}{8}\left(\sin(0) + 2\sin\left(\frac{1}{4}\right) + 2\sin\left(\frac{1}{2}\right) + 2\sin\left(\frac{3}{4}\right) + \sin(1)\right) \\
&\approx & 0.4573009376
\end{alignat*}
\begin{alignat*}{2}
\intertext{$n=5$}
\int\limits_{0}^{1} \sin x \,dx &\approx & \frac{1}{10}\left(\sin(0) + 2\sin\left(\frac{1}{5}\right) +
2\sin\left(\frac{2}{5}\right) + 2\sin\left(\frac{3}{5}\right) + 2\sin\left(\frac{4}{5}\right) + \sin(1)\right) \\
&\approx & 0.4581643460
\end{alignat*}
\begin{alignat*}{2}
\intertext{$n=10$}
\int\limits_{0}^{1} \sin x \,dx &\approx \begin{split} \frac{1}{20}\left(\sin(0) + 2\sin\left(\frac{1}{10}\right) +
2\sin\left(\frac{1}{5}\right) + 2\sin\left(\frac{3}{10}\right) + 2\sin\left(\frac{2}{5}\right) +
2\sin\left(\frac{1}{2}\right)\right.\\ +
\left. 2\sin\left(\frac{3}{5}\right) + 2\sin\left(\frac{7}{10}\right) + 2\sin\left(\frac{4}{5}\right) +
2\sin\left(\frac{9}{10}\right) + \sin(1)\right)\end{split} \\
&\approx & 0.4593145489
\end{alignat*}
\section{} %4
\subsection{} %a
\begin{alignat*}{2}
f(1) &=& 10 \cdot e^{-\frac{2}{5}} \\
&\approx & 6.7032 \\
f(2) &=& 10 \cdot 2 \cdot e^{-frac{2}{5}\cdot 2} \\
&=& 20 \cdot e^{-\frac{4}{5}} \\
&\approx & 8.9866 \\
f(6) &=& 10 \cdot 6 \cdot e^{-\frac{2}{5}\cdot 6}\\
&=& 60 \cdot e^{-\frac{12}{5}} \\
&\approx & 5.4431 \\
f(12) &=& 10 \cdot 12 \cdot e^{-\frac{2}{5}\cdot 12} \\
&=& 120 \cdot e^{-\frac{24}{5}} \\
&\approx & 0.9876 \\
f(24) &=& 10 \cdot 24 \cdot e^{-\frac{2}{5}\cdot 24} \\
&=& 240 \cdot e^{-\frac{48}{5}} \\
&=& 0.0163
\end{alignat*}
\subsection{} %b
\begin{alignat*}{2}
f'(t) &=& 10 \cdot e^{-\frac{2}{5}t} - 10t \cdot e^{-\frac{2}{5}t} \cdot \frac{2}{5} \\
&=& 10 \cdot e^{-\frac{2}{5}t} - \frac{20}{5}t \cdot e^{-\frac{2}{5}t} \\
&=& 10 \cdot e^{-\frac{2}{5}t} - 4t \cdot e^{-\frac{2}{5}t} \\
&=& (10 - 4t) \cdot e^{-\frac{2}{5}t} \\
&=& \frac{10-4t}{e^{\frac{2}{5}t}} \\
f''(x) &=& \left((10 - 4t) \cdot e^{-\frac{2}{5}t}\right)' \\
&=& (10 - 4t)' \cdot e^{-\frac{2}{5}t} + (10 - 4t) \cdot \left(e^{-\frac{2}{5}t}\right)' \\
&=& -4 \cdot e^{-\frac{2}{5}t} + (10 - 4t) \cdot e^{-\frac{2}{5}t} \cdot \left(-\frac{2}{5}\right) \\
\intertext{Berechnung der Nullstelle(n) von $f'(x)$}
f'(x) &=& 0 \\
\intertext{$f'(x)$ mit Zähler ersetzen, da der über Nullstelle bestimmt}
10-4t &=& 0 \\
10 &=& 4t \\
\frac{5}{2} &=& t \\
\intertext{Einsetzen in $f(x)$}
f\left(\frac{5}{2}\right) &=& 10 \cdot \frac{5}{2} \cdot e^{-\frac{2}{5} \cdot \frac{5}{2}} \\
&=& 25 \cdot e^{-1} \\
&\approx & 9.1970
\end{alignat*}
\subsection{} %c
\begin{alignat*}{2}
\int 10t \cdot e^{-\frac{2}{5}t}\,dx &=& -\frac{5}{2}\cdot e^{-\frac{2}{5}t} \cdot 10t - \int -\frac{5}{2} \cdot e^{-\frac{2}{5}t} \cdot 10\,dx \\
&=& -\frac{5}{2}\cdot e^{-\frac{2}{5}t} \cdot 10t + 25 \int e^{-\frac{2}{5}t}\,dx \\
&=& -\frac{5}{2}\cdot e^{-\frac{2}{5}t} \cdot 10t - \frac{5}{2} \cdot e^{-\frac{2}{5}t} \cdot 25   \\
&=& -\frac{5}{2}\cdot e^{-\frac{2}{5}t} \cdot \left( 10t + 25\right) \\
\frac{1}{6} \int\limits_{0}^{6} 10t \cdot e^{-\frac{2}{5}t}\,dx &=& \frac{1}{6} \cdot \left[-\frac{5}{2}\cdot e^{-\frac{2}{5}t} \cdot (10t + 25)\right]_{0}^{6} \\
&=& \frac{1}{6}\left(-\frac{5}{2}\cdot e^{-\frac{2}{5} \cdot 6} \cdot ( 10 \cdot 6 + 25) - \left(-\frac{5}{2}\cdot e^{-\frac{2}{5} \cdot 0} \cdot ( 10 \cdot 0 + 25)\right)\right) \\
&=& \frac{1}{6}\left(-\frac{5}{2}\cdot e^{-\frac{12}{5}} \cdot 85 + \frac{5}{2}\cdot e^{0} \cdot 25\right) \\
&=& \frac{1}{6}\left(-\frac{425}{2}\cdot e^{-\frac{12}{5}} + \frac{125}{2}\right) \\
&\approx & 7.2037
\end{alignat*}
\subsection{} %d
\begin{alignat*}{2}
\intertext{Die Stammfunktion wurde bereits in c) berechnet. Daher setze ich direkt die Werte entsprechend ein.}
\frac{1}{6} \int\limits_{6}^{12} 10t \cdot e^{-\frac{2}{5}t}\,dx &=& \frac{1}{6} \cdot \left[-\frac{5}{2}\cdot e^{-\frac{2}{5}t} \cdot (10t + 25)\right]_{6}^{12} \\
&=& \frac{1}{6}\left(-\frac{5}{2}\cdot e^{-\frac{2}{5} \cdot 12} \cdot ( 10 \cdot 12 + 25) - \left(-\frac{5}{2}\cdot e^{-\frac{2}{5} \cdot 6} \cdot ( 10 \cdot 6 + 25)\right)\right) \\
&=& \frac{1}{6}\left(-\frac{5}{2}\cdot e^{-\frac{24}{5}} \cdot 145 + \frac{5}{2}\cdot e^{-\frac{12}{5}} \cdot 85\right) \\
&=& \frac{1}{6}\left(-\frac{725}{2}\cdot e^{-\frac{24}{5}} + \frac{425}{2}\cdot e^{-\frac{12}{5}}\right) \\
&\approx & 2.7157
\end{alignat*}
\subsection{} %e
\begin{alignat*}{2}
f'(t) &=& \frac{10-4t}{e^{\frac{2}{5}t}} \\
f''(t) &=& -4 \cdot e^{-\frac{2}{5}t} + (10 - 4t) \cdot e^{-\frac{2}{5}t} \cdot \left(-\frac{2}{5}\right) \\
&=& e^{-\frac{2}{5}t} \cdot \left(-4 + (10 - 4t) \cdot \left(-\frac{2}{5}\right) \right) \\
&=& e^{-\frac{2}{5}t} \cdot \left(-4 - 4 + \frac{8}{5}t \right) \\
&=& e^{-\frac{2}{5}t} \cdot \left(-8 + \frac{8}{5}t \right) \\
&=& \frac{-8 + \frac{8}{5}t}{ e^{-\frac{2}{5}t}} \\
f'''(t) &=& \left(e^{-\frac{2}{5}t} \cdot \left(-8 + \frac{8}{5}t \right)\right)' \\
&=& e^{-\frac{2}{5}t} \cdot \left(-\frac{2}{5} \right) \cdot \left(-8 + \frac{8}{5}t \right) + e^{-\frac{2}{5}t} \cdot  \left(-8 + \frac{8}{5}t \right)' \\
&=& e^{-\frac{2}{5}t} \cdot \left(\frac{16}{5} - \frac{16}{25}t \right) + e^{-\frac{2}{5}t} \cdot \left(\frac{8}{5}\right) \\
&=& e^{-\frac{2}{5}t} \cdot \left(\frac{16}{5} - \frac{16}{25}t + \frac{8}{5}\right) \\
&=& e^{-\frac{2}{5}t} \cdot \left(\frac{24}{5} - \frac{16}{25}t\right) \\
\intertext{Nullstelle(n) von $f''(x)$ berechnen}
f''(x) &=& 0 \\
\intertext{$f''(x)$ mit dem Zähler ersetzen, da der für die Nullstelle zuständig ist}
-8 + \frac{8}{5}t &=& 0 \\
\frac{8}{5} &=& 8 \\
t &=& 5 \\
f'''(5) &=& e^{-\frac{2}{5} \cdot 5} \cdot \left(\frac{24}{5} - \frac{16}{25} \cdot 5\right) \\
&=& e^{-2} \cdot \left(\frac{24}{5} - \frac{16}{5}\right) \\
&=& e^{-2} \cdot \frac{8}{5} > 0 \Rightarrow \text{Minimum} \\
f(5) &=& 10 \cdot 5 \cdot e^{-\frac{2}{5} \cdot 5} \\
&=& 50 \cdot e^{-2} \\
&\approx & 6.7668
\end{alignat*}
\begin{tikzpicture}[>=stealth]
\begin{axis}[
	ymin=0,ymax=10,
	x=1em,
	y=1em,
	axis x line=middle,
	axis y line=middle,
	axis line style=->,
	xlabel={$t$},
	ylabel={$f(t)$},
	xmin=0,xmax=24
	]
\addplot[no marks, black, -] expression[domain=0:24,samples=100]{10*x*e^(-(2/5)*x)} node[pos=0.65,anchor=north]{};
\draw (axis cs:5,6.766764162) circle (2pt);
\end{axis}
\end{tikzpicture}
\section{} %5
\subsection{} %a
\begin{alignat*}{2}
h(x) &=& (x^{2}+1)^{\cos(x)} \\
&=& e^{\ln \left((x^{2}+1)^{\cos(x)}\right)}\\
&=& e^{\cos(x) \cdot \ln (x^{2}+1)} \\
h'(x) &=& e^{\cos(x) \cdot \ln (x^{2}+1)} \cdot \left(\cos(x) \cdot \ln (x^{2}+1)\right)' \\
&=& e^{\cos(x) \cdot \ln (x^{2}+1)} \cdot \left((\cos(x))' \cdot \ln (x^{2}+1) + \cos(x) \cdot (\ln (x^{2}+1))'\right) \\
&=& e^{\cos(x) \cdot \ln (x^{2}+1)} \cdot \left(-\sin(x) \cdot \ln (x^{2}+1) + \cos(x) \cdot \frac{(x^{2}+1)'}{x^{2}+1}\right) \\
&=& e^{\cos(x) \cdot \ln (x^{2}+1)} \cdot \left(-\sin(x) \cdot \ln (x^{2}+1) + \cos(x) \cdot \frac{2x}{x^{2}+1}\right) \\
&=& (x^{2}+1)^{\cos(x)} \cdot \left(-\sin(x) \cdot \ln (x^{2}+1) + \cos(x) \cdot \frac{2x}{x^{2}+1}\right)
\end{alignat*}
\subsection{} %b
\begin{alignat*}{2}
t &=& \sqrt{\frac{x}{4} + 3} \\
t^{2} &=& \frac{x}{4} + 3 \\
t^{2} - 3 &=& \frac{x}{4} \\
4t^{2} - 12 &=& x \\
8t &=& \frac{dx}{dt} \\
8t\,dt &=& dx
\end{alignat*}
\begin{alignat*}{2}
\int \sin \left(\sqrt{\frac{x}{4} + 3}\right) &\Rightarrow & \int \sin (t) \cdot 8t\,dt \\
\int \sin (t) \cdot 8t\,dt &=& -\cos(t) \cdot 8t - \int (-\cos (t) \cdot 8)\,dt \\
&=& -\cos(t) \cdot 8t + 8\int \cos(t)\,dt \\
&=& -\cos(t) \cdot 8t + 8 \cdot \sin(t) \\
\int \sin \left(\sqrt{\frac{x}{4} + 3}\right) &=& -\cos \left(\sqrt{\frac{x}{4} + 3} \right) \cdot 8 \cdot \left(\sqrt{\frac{x}{4} + 3} \right) + 8 \cdot \sin \left(\sqrt{\frac{x}{4} + 3}\right)
\end{alignat*}
\subsection{} %c
\begin{alignat*}{2}
g(x) = x^{2}-x-6 &=& 0 \\
x_{1,2} &=& \frac{1}{2} \pm \sqrt{\frac{1}{4} + 6 } \\
&=& \frac{1}{2} \pm \sqrt{\frac{25}{4}} \\
&=& \frac{1}{2} \pm \frac{5}{2} \\
x_{1} &=& \frac{6}{2} = 3 \\
x_{2} &=& -\frac{4}{2} = -2 \\
g(x) = x^{2}-x-6 &=& (x-3)(x+2)
\end{alignat*}
\begin{alignat*}{2}
\frac{3x+2}{x^{2}-x-6} = \frac{1}{(x-3)(x+2)} &=& \frac{A}{x-3} + \frac{B}{x+2} \\
&=& \frac{A(x+2) + B(x-3)}{(x-3)(x+2)} \\
&=& \frac{Ax + Bx + 2A - 3B}{(x-3)(x+2)} \\
&=& \frac{(A+B)x + 2A - 3B}{(x-3)(x+2)} \\
A + B &=& 3 \\
2A - 3B &=& 2 \\
A &=& 3 - B \\
2(3-B) - 3B &=& 2 \\
6 - 2B - 3B &=& 2 \\
-5B &=& -4 \\
5B &=& 4 \\
B &=& \frac{4}{5}\\
A &=& 3 - \frac{4}{5} \\
&=& \frac{11}{5} \\
\frac{3x+2}{x^{2}-x-6} &=& \frac{11}{5} \cdot \frac{1}{x-3} + \frac{4}{5} \cdot \frac{1}{x+2}
\end{alignat*}
\begin{alignat*}{2}
\int \frac{3x+2}{x^{2}-x-6}\,dx &=& \int \frac{11}{5} \cdot \frac{1}{x-3}\,dx + \int \frac{4}{5} \cdot \frac{1}{x+2}\,dx \\
&=& \frac{11}{5} \int \frac{1}{x-3}\,dx + \frac{4}{5} \int \frac{1}{x+2}\,dx \\
&=& \frac{11}{5} \cdot \ln |x-3| + \frac{4}{5} \cdot \ln |x+2|
\end{alignat*}
\subsection{} %d
\begin{alignat*}{2}
g(x) = x^{2} + 8x + 16 &=& (x+4)^{2} \\
\frac{x+1}{(x+4)^{2}} &=& \frac{A}{(x+4)^{2}} + \frac{B}{x+4} \\
&=& \frac{A + B(x+4)}{(x+4)^{2}} \\
&=& \frac{A + Bx + 4B}{(x+4)^{2}} \\
&=& \frac{Bx + A + 4B}{(x+4)^{2}} \\
\Rightarrow B &=&1 \\
A + 4B &=& 1 \\
A &=& -4B + 1 \\
A &=& -4 + 1 \\
A &=& -3 \\
\frac{x+1}{x^{2}+8x+16} &=& -\frac{3}{(x+4)^{2}} + \frac{1}{x+4}
\end{alignat*}
\begin{alignat*}{2}
\int \frac{x+1}{x^{2}+8x+16} \,dx &=& \int -\frac{3}{(x+4)^{2}}\,dx + \int \frac{1}{x+4}\,dx \\
&=& -3\int (x+4)^{-2}\,dx + \int \frac{1}{x+4}\,dx \\
&=& -3\left(-(x+4)^{-1}\right)+ \ln |x+4| \\
&=& 3(x+4)^{-1} + \ln |x+4|
\end{alignat*}
\end{document}
