\documentclass[10pt,a4paper,oneside,ngerman,numbers=noenddot]{scrartcl}
\usepackage[T1]{fontenc}
\usepackage[utf8]{inputenc}
\usepackage[ngerman]{babel}
\usepackage{amsmath}
\usepackage{amsfonts}
\usepackage{amssymb}
\usepackage{paralist}
\usepackage{gauss}
\usepackage[locale=DE,exponent-product=\cdot,detect-all]{siunitx}
\usepackage{tikz}
\usetikzlibrary{matrix,fadings,calc,positioning,decorations.pathreplacing,arrows,decorations.markings}
\usepackage{polynom}
\polyset{style=C, div=:,vars=x}
\pagenumbering{arabic}
\def\thesection{\arabic{section})}
\def\thesubsection{\alph{subsection})}
\def\thesubsubsection{(\roman{subsubsection})}
\makeatletter
\renewcommand*\env@matrix[1][*\c@MaxMatrixCols c]{%
  \hskip -\arraycolsep
  \let\@ifnextchar\new@ifnextchar
  \array{#1}}
\makeatother

\begin{document}
\author{Jim Martens (6420323)}
\title{Hausaufgaben zum 25. April}
\maketitle
\section{} %1
\subsection{} %a
\vspace{11cm}
Die Unstetigkeitsstellen sind $x=2$ und $x=6$.
\subsection{} %b
\vspace{4cm}
Die Funktion $g(x)$ ist periodisch. Betrachtet man die Periode für $x=0$ bis $x=1$, so ist ersichtlich, dass sowohl $x=0$ als auch $x=1$ Unstetigkeitsstellen sind. Nähert man sich an die beiden Stellen von links an, so stimmt der Grenzwert nicht mit dem Funktionswert überein. Nähert man sich von rechts an, stimmt er überein.\\
Stetigkeit erfordert jedoch, dass der Grenzwert existiert und gleich dem Funktionswert ist, unabhängig von der Folge mit der man sich annähert.
Daher ist $g(x)$ in diesen beiden Stellen unstetig. Aufgrund der Periodizität der Funktion ist $g(x)$ an allen Stellen $x \in \mathbb{Z}$ unstetig.\\
\\
Betrachtet man hingegen eine andere Stelle in der Periode, so stimmen Grenzwert und Funktionswert überein, unabhängig davon ob man sich von rechts oder links annähert. Daher ist $g(x)$ in allen Stellen der Periode mit $x \neq 0$ und $x \neq 1$ stetig. Aufgrund der Periodizität der Funktion ist $g(x)$ an allen Stellen $x \not\in \mathbb{Z}$ stetig.
\section{} %2
\subsection{} %a
\begin{alignat*}{2}
\underset{n \rightarrow \infty}{\text{lim}} a_{n} &=& \underset{n \rightarrow \infty}{\text{lim}} \frac{\sqrt{3n^{2}-2n+5}-\sqrt{n}}{\sqrt{n^{2}-n+1}+4n} \\
\intertext{Ausklammern von $n^{2}$ unterhalb der Wurzeln}
&=& \underset{n \rightarrow \infty}{\text{lim}} \frac{\sqrt{n^{2} (3 - \frac{2}{n} + \frac{5}{n^{2}})} - \sqrt{n^{2} \cdot \frac{1}{n}}}{\sqrt{n^{2} (1 -\frac{1}{n} + \frac{1}{n^{2}})} + 4n} \\
\intertext{Wurzelgesetze anwenden}
&=& \underset{n \rightarrow \infty}{\text{lim}} \frac{\sqrt{n^{2}} \cdot \sqrt{3 - \frac{2}{n} + \frac{5}{n^{2}}} - \sqrt{n^{2}} \cdot \sqrt{\frac{1}{n}}}{\sqrt{n^{2}} \cdot \sqrt{1 -\frac{1}{n} + \frac{1}{n^{2}}} + 4n} \\
\intertext{Wurzel auflösen}
&=& \underset{n \rightarrow \infty}{\text{lim}} \frac{n \cdot \sqrt{3 - \frac{2}{n} + \frac{5}{n^{2}}} - n \cdot \sqrt{\frac{1}{n}}}{n \cdot \sqrt{1 -\frac{1}{n} + \frac{1}{n^{2}}} + 4n} \\
\intertext{$n$ ausklammern}
&=& \underset{n \rightarrow \infty}{\text{lim}} \frac{n \cdot (\sqrt{3 - \frac{2}{n} + \frac{5}{n^{2}}} - \sqrt{\frac{1}{n}})}{n \cdot (\sqrt{1 -\frac{1}{n} + \frac{1}{n^{2}}} + 4)} \\
\intertext{$n$ kürzen}
&=& \underset{n \rightarrow \infty}{\text{lim}} \frac{\sqrt{3 - \frac{2}{n} + \frac{5}{n^{2}}} - \sqrt{\frac{1}{n}}}{\sqrt{1 -\frac{1}{n} + \frac{1}{n^{2}}} + 4} \\
\intertext{lim mit Wurzelfunktion vertauschen, da Wurzelfunktion stetig}
&=& \frac{\sqrt{\underset{n \rightarrow \infty}{\text{lim}} (3 - \frac{2}{n} + \frac{5}{n^{2}})} - \sqrt{\underset{n \rightarrow \infty}{\text{lim}}(\frac{1}{n})}}{\sqrt{\underset{n \rightarrow \infty}{\text{lim}} (1 -\frac{1}{n} + \frac{1}{n^{2}})} + 4} \\
\intertext{limes anwenden und Nullfolgen entfernen}
&=& \frac{\sqrt{3} - \sqrt{0}}{\sqrt{1} + 4} \\
\intertext{Zusammenfassen}
&=& \frac{\sqrt{3}}{5}
\end{alignat*}
\subsection{} %b
\begin{alignat*}{2}
&& \underset{n \rightarrow \infty}{\text{lim}} \left( \text{cos} \left( \frac{\sqrt{10n^{2}-n}-n}{2n+3} \right) \right) \\
\intertext{Ausklammern von $n^{2}$ unter der Wurzel}
&=& \underset{n \rightarrow \infty}{\text{lim}} \left( \text{cos} \left( \frac{\sqrt{n^{2} (10-\frac{1}{n})}-n}{2n+3} \right) \right) \\
\intertext{Wurzelgesetze anwenden}
&=& \underset{n \rightarrow \infty}{\text{lim}} \left( \text{cos} \left( \frac{\sqrt{n^{2}} \cdot \sqrt{10-\frac{1}{n}}-n}{2n+3} \right) \right) \\
\intertext{Wurzel auflösen}
&=& \underset{n \rightarrow \infty}{\text{lim}} \left( \text{cos} \left( \frac{n \cdot \sqrt{10-\frac{1}{n}}-n}{2n+3} \right) \right) \\
\intertext{Ausklammern von $n$ in Zähler und Nenner}
&=& \underset{n \rightarrow \infty}{\text{lim}} \left( \text{cos} \left( \frac{n \cdot (\sqrt{10 - \frac{1}{n}} - 1)}{n \cdot (2 + \frac{3}{n})} \right) \right) \\
\intertext{Kürzen von $n$}
&=& \underset{n \rightarrow \infty}{\text{lim}} \left( \text{cos} \left( \frac{\sqrt{10 - \frac{1}{n}} - 1}{2 + \frac{3}{n}} \right) \right) \\
\intertext{cos mit lim vertauschen, da Cosinusfunktion stetig}
&=& \text{cos} \left( \underset{n \rightarrow \infty}{\text{lim}} \left( \frac{\sqrt{10 - \frac{1}{n}} - 1}{2 + \frac{3}{n}} \right) \right) \\
\intertext{lim in Wurzel ziehen, da Wurzelfunktion stetig}
&=& \text{cos} \left( \frac{\sqrt{\underset{n \rightarrow \infty}{\text{lim}} (10 - \frac{1}{n})} - \underset{n \rightarrow \infty}{\text{lim}} (1) }{\underset{n \rightarrow \infty}{\text{lim}} (2 + \frac{3}{n})} \right) \\
&=& \text{cos} \left( \frac{\sqrt{10} - 1}{2} \right)\\
&\approx & 0.47 
\end{alignat*}
\section{} %3
$g \circ f$ kann auch so geschrieben werden $g(f(x))$. Vereinfacht gesagt, liefert $g$ den Funktionswert an der Stelle, die dem Funktionswert von $f$ an der Stelle $x$ entspricht.\\
Das berücksichtigend wissen wir, dass $f$ an der Stelle $x_{0}$ stetig ist. Der Funktionswert für diese Stelle ist $f(x_{0}) = y_{0}$. Wir wissen ferner, dass $g$ an der Stelle $y_{0}$ stetig ist.\\
\\
Da der Funktionswert von $f$ an der Stelle $x_{0}$ der Stelle entspricht, an der $g$ bekanntermaßen stetig ist, werden hier zwei stetige Funktionen nacheinander ausgeführt. Und die Nacheinanderausführung von zwei stetigen Funktionen ist selbst wiederum stetig.\\
Es ist somit ersichtlich, dass $g(f(x_{0}))$ den soeben beschrieben Fall darstellt und damit klarstellt, dass $g \circ f$ ebenfalls an der Stelle $x_{0}$ stetig ist.
\section{} %4
$g(x)$:\\
\begin{alignat*}{2}
\underset{x \rightarrow 0}{\text{lim}} g(x) &\Rightarrow & \underset{x \rightarrow 0}{\text{lim}} (x^{2} \cdot 1 ) \geq \underset{x \rightarrow 0}{\text{lim}} \left(x^{2} \cdot \text{sin} \left(\frac{1}{x}\right) \right) \geq \underset{x \rightarrow 0}{\text{lim}} (x^{2} \cdot -1) \\
&\Rightarrow & 0 \geq \underset{x \rightarrow 0}{\text{lim}} \left(x^{2} \cdot \text{sin} \left(\frac{1}{x} \right) \right) \geq 0
\end{alignat*}
Daraus folgt, dass $g(x)$ für alle $x \in \mathbb{R}$ stetig ist. \\

$h(x)$:\\
\begin{alignat*}{2}
\underset{n \rightarrow \infty}{\text{lim}} h(x_{n}) &=& \underset{n \rightarrow \infty}{\text{lim}} \text{sin} \left( \frac{1}{x_{n}} \right) \\
\intertext{$x_{n}$ sei $\frac{1}{2\pi n}, n \in \mathbb{N}$}
&=& \underset{n \rightarrow \infty}{\text{lim}} \text{sin} \left(\frac{1}{\frac{1}{2 \pi n}}\right) \\
&=& \underset{n \rightarrow \infty}{\text{lim}} \text{sin} (2 \pi n) \\
&=& \text{sin} \left( \underset{n \rightarrow \infty}{\text{lim}} (2 \pi n) \right) = 0
\end{alignat*}
$h(x)$ ist stetig für alle $x = x_{n} = \frac{1}{2\pi n}$. Die Funktion ist nicht stetig für andere $x$.
\end{document}
