\documentclass[10pt,a4paper,oneside,ngerman,numbers=noenddot]{scrartcl}
\usepackage[T1]{fontenc}
\usepackage[utf8]{inputenc}
\usepackage[ngerman]{babel}
\usepackage{amsmath}
\usepackage{amsfonts}
\usepackage{amssymb}
\usepackage{paralist}
\usepackage{gauss}
\usepackage{pgfplots}
\usepackage[locale=DE,exponent-product=\cdot,detect-all]{siunitx}
\usepackage{tikz}
\usetikzlibrary{matrix,fadings,calc,positioning,decorations.pathreplacing,arrows,decorations.markings}
\usepackage{polynom}
\polyset{style=C, div=:,vars=x}
\pgfplotsset{compat=1.8}
\pagenumbering{arabic}
\def\thesection{\arabic{section})}
\def\thesubsection{\alph{subsection})}
\def\thesubsubsection{(\roman{subsubsection})}
\makeatletter
\renewcommand*\env@matrix[1][*\c@MaxMatrixCols c]{%
  \hskip -\arraycolsep
  \let\@ifnextchar\new@ifnextchar
  \array{#1}}
\makeatother

\begin{document}
\author{Jim Martens (6420323)}
\title{Hausaufgaben zum 20. Juni}
\maketitle
\section{} %1
\subsection{} %a
\begin{alignat*}{2}
T_{8}(x) &=&& 1 - \frac{1}{2}x^{2} + \frac{1}{24}x^{4} - \frac{1}{720}x^{6} + \frac{1}{40320}x^{8} \\
T_{9}(x) &=&& 1 - \frac{1}{2}x^{2} + \frac{1}{24}x^{4} - \frac{1}{720}x^{6} + \frac{1}{40320}x^{8} \\
T_{10}(x) &=&& 1 - \frac{1}{2}x^{2} + \frac{1}{24}x^{4} - \frac{1}{720}x^{6} + \frac{1}{40320}x^{8} - \frac{1}{10!}x^{10}\\
T_{11}(x) &=&& 1 - \frac{1}{2}x^{2} + \frac{1}{24}x^{4} - \frac{1}{720}x^{6} + \frac{1}{40320}x^{8} - \frac{1}{10!}x^{10} \\
T_{12}(x) &=&& 1 - \frac{1}{2}x^{2} + \frac{1}{24}x^{4} - \frac{1}{720}x^{6} + \frac{1}{40320}x^{8} - \frac{1}{10!}x^{10} + \frac{1}{12!}x^{12}\\
T_{13}(x) &=&& 1 - \frac{1}{2}x^{2} + \frac{1}{24}x^{4} - \frac{1}{720}x^{6} + \frac{1}{40320}x^{8} - \frac{1}{10!}x^{10} + \frac{1}{12!}x^{12}
\end{alignat*}
\begin{alignat*}{2}
T_{9}(1) &=& 1 - \frac{1}{2} \cdot 1^{2} + \frac{1}{24} \cdot 1^{4} - \frac{1}{720} \cdot 1^{6} + \frac{1}{40320} \cdot 1^{8} \\
&\approx & 0.5403026  \\
T_{11}(1) &=& 1 - \frac{1}{2} \cdot 1^{2} + \frac{1}{24} \cdot 1^{4} - \frac{1}{720} \cdot 1^{6} + \frac{1}{40320} \cdot 1^{8} - \frac{1}{10!} \cdot 1^{10} \\
&\approx & 0.5403023 \\
T_{13}(1) &=& 1 - \frac{1}{2} \cdot 1^{2} + \frac{1}{24} \cdot 1^{4} - \frac{1}{720} \cdot 1^{6} + \frac{1}{40320} \cdot 1^{8} - \frac{1}{10!} \cdot 1^{10} + \frac{1}{12!} \cdot 1^{12} \\
&\approx & 0.5403023
\end{alignat*}
\subsection{} %b
\begin{alignat*}{2}
f(x) &=& \sqrt{1+x} = (1+x)^{\frac{1}{2}} \\
f'(x) &=& \frac{1}{2} \cdot (1+x)^{-\frac{1}{2}} \\
f''(x) &=& -\frac{1}{4} \cdot (1+x)^{-\frac{3}{2}} \\
f'''(x) &=& \frac{3}{8} \cdot (1+x)^{-\frac{5}{2}} \\
f^{(4)}(x) &=& -\frac{15}{16} \cdot (1+x)^{-\frac{7}{2}}
\end{alignat*}
\begin{alignat*}{2}
T_{0}(x) &=&& 1 \\
T_{1}(x) &=&& 1 + \frac{1}{2}x \\
T_{2}(x) &=&& 1 + \frac{1}{2}x - \frac{1}{8}x^{2}\\
T_{3}(x) &=&& 1 + \frac{1}{2}x - \frac{1}{8}x^{2} + \frac{3}{48}x^{3} \\
T_{4}(x) &=&& 1 + \frac{1}{2}x - \frac{1}{8}x^{2} + \frac{3}{48}x^{3} - \frac{15}{384}x^{4}
\end{alignat*}
\begin{alignat*}{2}
g(x) &=& \sqrt[3]{1+x} = (1+x)^{\frac{1}{3}} \\
g'(x) &=& \frac{1}{3} \cdot (1+x)^{-\frac{2}{3}} \\
g''(x) &=& -\frac{2}{9} \cdot (1+x)^{-\frac{5}{3}} \\
g'''(x) &=& \frac{10}{27} \cdot (1+x)^{-\frac{8}{3}} \\
g^{(4)}(x) &=& -\frac{80}{81} \cdot (1+x)^{-\frac{11}{3}}
\end{alignat*}
\begin{alignat*}{2}
T_{0}(x) &=&& 1 \\
T_{1}(x) &=&& 1 + \frac{1}{3}x \\
T_{2}(x) &=&& 1 + \frac{1}{3}x - \frac{1}{9}x^{2} \\
T_{3}(x) &=&& 1 + \frac{1}{3}x - \frac{1}{9}x^{2} + \frac{5}{81}x^{3} \\
T_{4}(x) &=&& 1 + \frac{1}{3}x - \frac{1}{9}x^{2} + \frac{5}{81}x^{3} - \frac{10}{243}x^{4}
\end{alignat*}
\subsection{} %c
\begin{alignat*}{2}
f(x) &=& e^{x} \cdot \sin x \\
f'(x) &=& e^{x} \cdot \cos x \\
f''(x) &=& -e^{x} \cdot \sin x \\
f'''(x) &=& -e^{x} \cdot \cos x \\
f^{(4)}(x) &=& e^{x} \cdot \sin x \\
f^{(5)}(x) &=& e^{x} \cdot \cos x
\end{alignat*}
\begin{alignat*}{2}
T_{5}(x) &=& 0 + x - 0x^{2} - \frac{1}{6}x^{3} + 0x^{4} + \frac{1}{120}x^{5} \\
&=& x - \frac{1}{6}x^{3} + \frac{1}{120}x^{5}
\end{alignat*}
\section{} %2
\subsubsection{} %i
\begin{alignat*}{2}
\lim\limits_{x \rightarrow 1} \left(\frac{x^{3}-3x^{2}+x+2}{x^{2}-5x+6} \right) &=& \frac{1-3+1+2}{1-5+6} \\
&=& \frac{1}{2}
\end{alignat*}
\subsubsection{} %ii
\begin{alignat*}{3}
\lim\limits_{x \rightarrow 2} \left(\frac{x^{3}-3x^{2}+x+2}{x^{2}-5x+6} \right) &=& \lim\limits_{x \rightarrow 2} \left(\frac{3x^{2} - 6x + 1}{2x - 5} \right) &=& \frac{3 \cdot 2^{2} - 6 \cdot 2 + 1}{2 \cdot 2 - 5} \\
&& &=& \frac{12 - 12 + 1}{4 - 5} \\
&& &=& \frac{1}{-1} \\
&& &=& -1
\end{alignat*}
\subsubsection{} %iii
\begin{alignat*}{2}
\lim\limits_{x \rightarrow 0} (1+3x)^{\frac{1}{2x}}
\end{alignat*}
\subsubsection{} %iv
\begin{alignat*}{2}
\lim\limits_{x \rightarrow 0} \left(\frac{1}{e^{x} - 1} - \frac{1}{\sin x} \right)
\end{alignat*}
\section{} %3
\subsection{} %a
\begin{alignat*}{2}
f(x) &=& 3^{x} \\
f'(x) &=& 3^{x} \cdot \ln 3 \\
t'(x) = f'(2) &=& 9 \cdot \ln 3 \\
&\approx & 9.88751 \\
\intertext{Die Steigung der Tangente beträgt $9 \cdot \ln 3$ oder rund $9.88751$.}
t(x) &=& \ln (3) \cdot 9x + b \\
\intertext{Bestimmen des Schnittpunkts mit der y-Achse}
b &=& t(x) - \ln (3) \cdot 9x \\
&=& t(2) - \ln (3) \cdot 18 \\
&=& 9 - \ln (3) \cdot 18 \\
&\approx & -10.77502  \\
t(x) &=& \ln 3 \cdot 9x + 9 - \ln (3) \cdot 18 \\
\intertext{Bestimmen des Schnittpunkts mit der x-Achse}
0 &=& \ln (3) \cdot 9x + 9 - \ln (3) \cdot 18 \\
\ln (3) \cdot 18 - 9 &=& \ln (3) \cdot 9x \\
\frac{\ln (3) \cdot 18 - 9}{\ln 3} &=& 9x \\
\frac{\ln (3) \cdot 18 - 9}{\ln (3) \cdot 9} &=& x \\
x &\approx & 1.08976
\end{alignat*}
\subsection{} %b
\begin{alignat*}{2}
f(x) &=& \sqrt[7]{x+2} \\
&=& (x+2)^{\frac{1}{7}} \\
f'(x) &=& \frac{1}{7} \cdot (x+2)^{-\frac{6}{7}} \\
f''(x) &=& -\frac{6}{49} \cdot (x+2)^{-\frac{13}{7}}
\end{alignat*}
\begin{alignat*}{2}
T_{0} &=&& 2^{\frac{1}{7}} \\
T_{1} &=&& 2^{\frac{1}{7}} + \frac{1}{7} \cdot 2^{-\frac{6}{7}}x \\
T_{2} &=&& 2^{\frac{1}{7}} + \frac{1}{7} \cdot 2^{-\frac{6}{7}}x - \frac{3}{49} \cdot 2^{-\frac{13}{7}}
\end{alignat*}
\subsection{} %c
\begin{alignat*}{2}
x_{n} &=& \frac{1}{2\pi n} \\
\lim\limits_{n\rightarrow \infty} f(x_{n}) = \lim\limits_{n\rightarrow \infty} \cos \left(\frac{1}{x_{n}} \right) &=&  \lim\limits_{n\rightarrow \infty} \cos \left(\frac{1}{\frac{1}{2 \pi n}} \right) \\
&=& \lim\limits_{n\rightarrow \infty} \cos (2 \pi n) = 1
\end{alignat*} \\
Der mögliche Grenzwert 1 stimmt nicht mit dem Funktionswert überein. Daher ist $h(x)$ im Punkt $x_{0}=0$ nicht stetig.
\subsection{} %d
\setcounter{subsubsection}{0}
\subsubsection{} %i
$\mathcal{B}$ enthält alle Folgen, die gegen eine reelle Zahl konvergieren, als auch solche, die zwischen zwei Werten oszillieren.
Lediglich uneigentlich konvergente Folgen sind nicht enthalten.
\subsubsection{} %ii
Aus $i)$ ergibt sich, dass nicht jede Folge in $\mathcal{B}$ konvergiert. Oszillierende Folgen konvergieren nicht, allerdings erfüllen sie die Bedingungen von $\mathcal{B}$.
\section{} %4
\subsection{} %a
\begin{alignat*}{2}
\lim\limits_{x \rightarrow \infty} \left(\frac{a^{x}}{x^{n}}\right) &=& \lim\limits_{x \rightarrow \infty} \left(\frac{a^{x} \cdot \ln a}{n \cdot x^{n-1}}\right) \\
\intertext{Nach n Ableitungen}
&=& \lim\limits_{x \rightarrow \infty} \left(\frac{a^{x} \cdot (\ln a)^{n}}{n! \cdot x^{0}}\right) \\
\intertext{Zähler geht gegen unendlich, Nenner gegen konstante Zahl $n!$. Daher existiert der Grenzwert nicht.}
&=& \lim\limits_{x \rightarrow \infty} \left(\frac{a^{x} \cdot (\ln a)^{n}}{n!}\right) = \infty
\end{alignat*}
\subsection{} %b
\begin{alignat*}{2}
\lim\limits_{x \rightarrow \infty} \left(\frac{x^{r}}{\ln^{k} x} \right) &=& \lim\limits_{x \rightarrow \infty} \left(\frac{rx^{r-1}}{k \cdot \ln^{k-1} x \cdot \frac{1}{x}} \right) \\
&=& \lim\limits_{x \rightarrow \infty} \left(\frac{rx^{r}}{k \cdot \ln^{k-1} x} \right) \\
\intertext{Nach der k-ten Ableitung sieht es so aus}
&=& \lim\limits_{x \rightarrow \infty} \left(\frac{r^{k} \cdot x^{r}}{k! \cdot \ln x} \right) \\
\intertext{Herausziehen der Konstanten}
&=& \frac{r^{k}}{k!} \cdot \lim\limits_{x \rightarrow \infty} \left(\frac{x^{r}}{\ln x} \right) \\
\intertext{Nach Satz 27 im Skript, existiert der Grenzwert $\lim\limits_{x \rightarrow \infty} \left(\frac{x^{r}}{\ln x} \right)$ nicht bzw. ist unendlich. Unendlich mal eine Konstante ist immer noch unendlich.}
&=& \infty
\end{alignat*}
\subsection{} %c
\setcounter{subsubsection}{0}
\subsubsection{} %i
\subsubsection{} %ii
\end{document}
