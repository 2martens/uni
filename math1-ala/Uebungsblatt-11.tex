\documentclass[10pt,a4paper,oneside,ngerman,numbers=noenddot]{scrartcl}
\usepackage[T1]{fontenc}
\usepackage[utf8]{inputenc}
\usepackage[ngerman]{babel}
\usepackage{amsmath}
\usepackage{amsfonts}
\usepackage{amssymb}
\usepackage{paralist}
\usepackage{gauss}
\usepackage{pgfplots}
\usepackage[locale=DE,exponent-product=\cdot,detect-all]{siunitx}
\usepackage{tikz}
\usetikzlibrary{matrix,fadings,calc,positioning,decorations.pathreplacing,arrows,decorations.markings}
\usepackage{polynom}
\polyset{style=C, div=:,vars=x}
\pgfplotsset{compat=1.8}
\pagenumbering{arabic}
\def\thesection{\arabic{section})}
\def\thesubsection{\alph{subsection})}
\def\thesubsubsection{(\roman{subsubsection})}
\makeatletter
\renewcommand*\env@matrix[1][*\c@MaxMatrixCols c]{%
  \hskip -\arraycolsep
  \let\@ifnextchar\new@ifnextchar
  \array{#1}}
\makeatother

\begin{document}
\author{Jim Martens (6420323)}
\title{Hausaufgaben zum 4. Juli}
\maketitle
\section{} %1
\subsection{} %a
\begin{alignat*}{2}
f(x,y,z) &=& 2x^{2} + y^{2} + 4z^{2} - 2yz - 2x - 6y + 8 \\
I f_{x} &=& 4x - 2 = 0\\
II f_{y} &=& 2y - 7z - 6 = 0\\
III f_{z} &=& 8z - 2y = 0 \\
I &\Rightarrow & 4x - 2 = 0 \\
&\Leftrightarrow & 4x = 2 \\
&\Leftrightarrow & x = \frac{1}{2} \\
III &\Rightarrow & 8z - 2y = 0 \\
&\Leftrightarrow & 8z = 2y \\
&\Leftrightarrow & 4z = y \\
II &\Rightarrow & 2 \cdot 4z - 7z - 6 = 0 \\
&\Leftrightarrow & 8z - 7z = 6 \\
&\Leftrightarrow & z = 6 \\
III &\Rightarrow & 4 \cdot 6 = y \\
&\Leftrightarrow & 24 = y \\
\intertext{Die einzige kritische Stelle befindet sich an ($\frac{1}{2}, 24, 6$).}
f_{xx} &=& 4 \\
f_{yx} &=& 0 \\
f_{zx} &=& 0 \\
f_{xy} &=& 0 \\
f_{yy} &=& 2 \\
f_{zy} &=& -7 \\
f_{xz} &=& 0 \\
f_{yz} &=& -2 \\
f_{zz} &=& 8 \\
\intertext{Aufstellen der Hesse-Matrix}
H &=& \begin{pmatrix}4 & 0 & 0 \\
0 & 2 & -7 \\
0 & -2 & 8\end{pmatrix} \\
\bigtriangleup_{1} &=& 4 > 0\\
\bigtriangleup_{2} &=& 8 > 0\\
\bigtriangleup_{3} &=& 64 - 56 = 8 > 0\\
\intertext{Die Hesse-Matrix ist positiv definit. Daher liegt an der kritischen Stelle ein Minimum vor.}
\end{alignat*}
\subsection{} %b
\begin{alignat*}{2}
grad\,f(1,1,1) &=& (4 - 2, 2 - 7 - 6, 8 - 2) \\
&=& (2, -11, 6) \\
||grad\,f(1,1,1)|| &=& \sqrt{2^{2} + 11^{2} + 6^{2}} \\
&=& \sqrt{4 + 121 + 36} \\
&=& \sqrt{161} \\
\intertext{In Richtung von (2, -11, 6) steigt die Temperatur von (1,1,1) aus am stärksten an. In Richtung (-2, 11, -6) sinkt die Temperatur am stärksten. Die Größe des stärksten Anstiegs beträgt $\sqrt{161}$.}
\end{alignat*}
\section{} %2
\subsection{} %a
\begin{alignat*}{2}
A &=& \begin{pmatrix}-i & -1 \\
3 & i \end{pmatrix} \\
B &=& \begin{pmatrix} i \\
1 + i\end{pmatrix} \\
C &=& \begin{pmatrix}-i & i \end{pmatrix} \\
AB &=& \begin{pmatrix}-i \cdot i + (1+i)\cdot (-1) \\
3i + i(1+i) \end{pmatrix} \\
&=& \begin{pmatrix}-i^{2} - 1 - i \\
3i + i + i^{2} \end{pmatrix} \\
&=& \begin{pmatrix}- 1 - i \\
4i - 1 \end{pmatrix} \\
\intertext{AC existiert nicht, da A mehr Spalten hat, als C Zeilen hat.}
BC &=& \begin{pmatrix}i \cdot (-i) & i \cdot i \\
(1+i) \cdot (-i) & (1+i)i \end{pmatrix} \\
&=& \begin{pmatrix}-i^{2} & i^{2} \\
-i - i^{2} & i + i^{2} \end{pmatrix} \\
&=& \begin{pmatrix}1 & -1 \\
-i +1 & i -1 \end{pmatrix} \\
CB &=& \begin{pmatrix}-i \cdot i + i(1+i) \end{pmatrix} \\
&=& \begin{pmatrix}-i^{2} + i + i^{2} \end{pmatrix} \\
&=& \begin{pmatrix}1 + i -1 \end{pmatrix} \\
&=& \begin{pmatrix}i \end{pmatrix}
\end{alignat*}
\subsection{} %b
\begin{alignat*}{2}
\overline{z} &=& \frac{3 + 2i}{4 - 3i} \\
&=& \frac{(3 + 2i)(4+3i)}{(4 - 3i)(4+3i)} \\
&=& \frac{12 + 9i + 8i + 6i^{2}}{16 + 12i - 12i - 9i^{2}} \\
&=& \frac{12 - 6 + 17i}{16 + 9} \\
&=& \frac{6 + 17i}{25} \\
&=& \frac{6}{25} + \frac{17}{25}i \\
z &=& \frac{6}{25} - \frac{17}{25}i \\
a &=& \frac{6}{25} \\
b &=& \frac{17}{25}
\end{alignat*}
\subsection{} %c
\begin{alignat*}{2}
z_{1} &=& -1 - i = (-1, -1) \\
z_{2} &=& \sqrt{2} \cdot \cos \frac{\pi}{4} + \sqrt{2} \cdot i \cdot \sin \frac{\pi}{4} \\
&=& \sqrt{2} \cdot \cos \left(\frac{1}{2} \cdot \frac{\pi}{2} \right) + \sqrt{2} \cdot \sin \left(\frac{1}{2} \cdot \frac{\pi}{2} \right) \\
&=& 1 + i = (1, 1) \\
z_{3} &=& (-1 -i)(1+ i) \\
&=& -1 - i - i - i^{2} \\
&=&  -1 + 1 - 2i = 0 - 2i = (0, -2) \\
z_{4} &=& -1 + i = (-1, 1)
\end{alignat*}
\begin{tikzpicture}[>=stealth]
\begin{axis}[
	ymin=-5,ymax=5,
	x=1cm,
	y=1cm,
	axis x line=middle,
	axis y line=middle,
	axis line style=->,
	xlabel={$\Re$},
	ylabel={$\Im$},
	xmin=-5,xmax=5
	]

\node at (axis cs: -1,-1) {$z_{1}$};	
\node at (axis cs: 1,1) {$z_{2}$};
\node at (axis cs: 0,-2) {$z_{3}$};
\node at (axis cs: -1,1) {$z_{4}$};
\end{axis}
\end{tikzpicture}
\subsection{} %d
\subsubsection{} %i
In dieser Teilmenge sind alle komplexen Zahlen enthalten, die sich auf der Geraden befinden, die durch die Punkte (1,1) und (2,0) geht.
\subsubsection{} %ii
In der Teilmenge sind alle komplexen Zahlen enthalten, die sich auf der Kreislinie eines Kreises mit dem Radius 1 um den Punkt (1,1) befinden.
\section{} %3
\begin{alignat*}{2}
f(x,y) &=& -\frac{1}{5}x^{2} - xy - \frac{25}{10}y^{2} + 48x + 235y - 88 \\
g(x,y) &=& \frac{1}{5}x + y = 40
\end{alignat*}
\subsection{} %a
\begin{alignat*}{2}
g_{x} &=& \frac{1}{5} \\
g_{y} &=& 1 \\
\begin{pmatrix}\frac{\partial g}{\partial x} (x,y) & \frac{\partial g}{\partial y} (x,y)\end{pmatrix} &=& \begin{pmatrix}\frac{1}{5} & 1 \end{pmatrix} \\
\intertext{Der Rang dieser Matrix ist für alle $x,y$ gleich 2. Damit ist die Regularitätsbedingung erfüllt.}
L(x,y,\lambda) &=& -\frac{1}{5}x^{2} - xy - \frac{25}{10}y^{2} + 48x + 235y - 88 + \lambda(\frac{1}{5}x + y - 40) \\
I L_{x} &=& - \frac{2}{5}x - y + 48 + \lambda \cdot \frac{1}{5} = 0 \\
II L_{y} &=& - x - 5y + 235 + \lambda = 0 \\
III L_{\lambda} &=& \frac{1}{5}x + y - 40 = 0 \\
II &\Rightarrow & \lambda = x + 5y - 235 \\
I &\Rightarrow & - \frac{2}{5}x - y + 48 + (x + 5y - 235) \cdot \frac{1}{5} = 0\\
&\Leftrightarrow & - \frac{2}{5}x - y + 48 + \frac{1}{5}x + y - 47 = 0 \\
\intertext{Beachten der dritten Gleichung}
&\Leftrightarrow & - \frac{1}{5}x - \frac{1}{5}x - y + 48 - 7 = 0\\
&\Leftrightarrow & - \frac{1}{5}x + 1=\frac{1}{5}x + y - 40 \\
\intertext{Beachten der dritten Gleichung}
&\Leftrightarrow & - \frac{1}{5}x + 1 = 0  \\
&\Leftrightarrow & \frac{1}{5}x =  1 \\
&\Leftrightarrow & x = 5 \\
\intertext{Einsetzen in I}
&\Rightarrow & - \frac{1}{5} \cdot 5 - \frac{1}{5} \cdot 5 - y + 48 - 7 = 0 \\
&\Leftrightarrow & - 1 - 1 - y + 41 = 0 \\
&\Leftrightarrow & -y + 39 = 0 \\
&\Leftrightarrow & y = 39 \\
\intertext{Einsetzen in III}
\lambda &=& \frac{1}{5} \cdot 5 + 39 - 40 \\
&=& 1 + 39 - 40 \\
&=& 0
\intertext{Die einzige kritische Stelle befindet sich an (5, 39).}
\end{alignat*}
\subsection{} %b
\begin{alignat*}{2}
L_{xx} &=& - \frac{2}{5} \\
L_{yx} &=& - 1 \\
L_{xy} &=& - 1 \\
L_{yy} &=& - 5 \\
\overline{H} &=& \begin{pmatrix}0 & \frac{1}{5} & 1 \\
\frac{1}{5} & -\frac{2}{5} & -1\\
1 & -1 & -5 \end{pmatrix} \\
det \, \overline{H} &=& -\frac{1}{5} - \frac{1}{5} + \frac{2}{5} + \frac{1}{5} \\
&=& \frac{1}{5} > 0 \\
\intertext{An der kritischen Stelle liegt ein Maximum vor.}
\end{alignat*}
\section{} %4
\begin{alignat*}{2}
f(x,y) &=& -\frac{1}{5}x^{2} - xy - \frac{25}{10}y^{2} + 48x + 235y - 88 \\
g(x,y) &=& \frac{1}{5}x + y -40 = 0 \\
&\Leftrightarrow & y = -\frac{1}{5}x + 40 \\
\intertext{Einsetzen in f(x,y)}
f(x) &=& -\frac{1}{5}x^{2} - x \cdot (-\frac{1}{5}x + 40) - \frac{25}{10} \cdot (-\frac{1}{5}x + 40)^{2} + 48x + 235 \cdot (-\frac{1}{5}x + 40) - 88 \\
&=& -\frac{1}{5}x^{2} + \frac{1}{5}x^{2} - 40x - \frac{25}{10} \cdot (\frac{1}{25}x^{2} -16x + 1600) + 48x -47x + 9400 - 88 \\
&=& - 40x - \frac{1}{10}x^{2} + 40x - 4000 + 48x -47x + 9400 - 88 \\
&=& - \frac{1}{10}x^{2} + x + 5312 \\
f'(x) &=& -\frac{1}{5}x + 1 = 0 \\
&\Leftrightarrow & \frac{1}{5}x = 1 \\
&\Leftrightarrow & x = 5 \\
f''(x) &=& -\frac{1}{5} < 0 \\
\intertext{Unter der Nebenbedingung g(x,y) gibt es ein lokales Maximum für x = 5.
Einsetzen von x in die Nebenbedingung:}
g(y) &=& \frac{1}{5} \cdot 5 + y -40 = 0 \\
&\Leftrightarrow & 1 + y - 40 = 0 \\
&\Leftrightarrow & y - 39 = 0 \\
&\Leftrightarrow & y = 39 \\
\intertext{An der Stelle (5, 39) ist der Gewinn maximal.}
\end{alignat*}
\end{document}
